%%%%%%%%%%%%%%%%%%%%%%%%%%%%%%%%%%%%%%%%%%%%%%%%%%%%%%%%%%%%%%%%%
% Contents: Main Input File of the LaTeX2e Introduction
% $Id$
%%%%%%%%%%%%%%%%%%%%%%%%%%%%%%%%%%%%%%%%%%%%%%%%%%%%%%%%%%%%%%%%%
% lshort.tex - The not so short introduction to LaTeX   
%                                                      by Tobias Oetiker
%                                                     oetiker@ee.ethz.ch
%
%                           based on LKURTZ.TEX Uni Graz & TU Wien, 1987
%-----------------------------------------------------------------------
%
% To compile lshort, you need TeX 3.x, LaTeX and makeindex
%
% The sources files of the Intro are:
%      lshort.tex (this file),
%      titel.tex, contrib.tex, biblio.tex
%      things.tex, typeset.tex, math.tex, lssym.tex, spec.tex,
%      lshort.sty, fancyheadings.sty
%
% Further the  verbatim.sty and the layout.sty 
% from the LaTeX Tools distribution is
% required.
%
%
% To print the AMS symbols you need the AMS fonts and the packages
% amsfonts, eufrak and eucal from (AMS LaTeX 1.2)
%
% ---------------------------------------------------------------------


\usepackage{ifpdf}
\ifpdf
\usepackage{thumbpdf}
\pdfcompresslevel=9
\RequirePackage[colorlinks,hyperindex,plainpages=false]{hyperref}
\def\pdfBorderAttrs{/Border [0 0 0] } % No border arround Links
\else
\RequirePackage[plainpages=true]{hyperref}
\usepackage{color}
\fi

\usepackage{lshort}
\usepackage{makeidx,shortvrb,latexsym}

%
% This document is ``public domain''. It may be printed and
% distributed free of charge in its original form (including the
% list of authors). If it is changed or if parts of it are used
% within another document, then the author list must include
% all the original authors AND that author (those authors) who
% has (have) made the changes.
%
% Original Copyright H.Partl, E.Schlegl, and I.Hyna (1987).
% English Version Copyright by Tobias Oetiker (1994,1995),
% 
% ---------------------------------------------------------------------
%
%
% Formats also with\textt{letterpaper} option, but the pagebreaks might not
% fall as nicely.
%
% To produce a A5 booklet, use a tool like  pstops or dvitodvi
% to  past them together in the right order. Most dvi printer drivers
% can shrink the resulting output to fit on a A4 sheet.
%
\makeindex
\typeout{Copyright T.Oetiker, H.Partl, E.Schlegl, I.Hyna}

\DeclareMathOperator{\argh}{argh}
\DeclareMathOperator*{\nut}{Nut}

 
\begin{document}
\selectlanguage{english}
\frontmatter
%%%%%%%%%%%%%%%%%%%%%%%%%%%%%%%%%%%%%%%%%%%%%%%%%%%%%%%%%%%%%%%%%
% Contents: The title page
% $Id$
%%%%%%%%%%%%%%%%%%%%%%%%%%%%%%%%%%%%%%%%%%%%%%%%%%%%%%%%%%%%%%%%%

\ifpdf
  \pdfbookmark{Title Page}{title}
\fi
\newlength{\centeroffset}
\setlength{\centeroffset}{-0.5\oddsidemargin}
\addtolength{\centeroffset}{0.5\evensidemargin}
%\addtolength{\textwidth}{-\centeroffset}
\thispagestyle{empty}
\vspace*{\stretch{1}}
\noindent\hspace*{\centeroffset}\makebox[0pt][l]{\begin{minipage}{\textwidth}
\flushright
{\Huge\bfseries The Not So Short\\ 
Introduction to \LaTeXe

}
\noindent\rule[-1ex]{\textwidth}{5pt}\\[2.5ex]
\hfill\emph{\Large Or \LaTeXe{} in \pageref{LastPage} minutes}
\end{minipage}}

\vspace{\stretch{1}}
\noindent\hspace*{\centeroffset}\makebox[0pt][l]{\begin{minipage}{\textwidth}
\flushright
{\bfseries 
by Tobias Oetiker\\[1.5ex]
Hubert Partl, Irene Hyna and  Elisabeth Schlegl\\[3ex]} 
Version~5.00, December 14, 2010
\end{minipage}}

%\addtolength{\textwidth}{\centeroffset}
\vspace{\stretch{2}}


\pagebreak
\begin{small} 
  Copyright \copyright 1995-2010 Tobias Oetiker and Contributors.  All rights reserved.
 
  This document is free; you can redistribute it and/or modify it
  under the terms of the GNU General Public License as published by
  the Free Software Foundation; either version 2 of the License, or
  (at your option) any later version.
  
  This document is distributed in the hope that it will be useful, but
  \emph{without any warranty}; without even the implied warranty of
  \emph{merchantability} or \emph{fittness for a particular purpose}\@.  See the GNU
  General Public License for more details.
  
  You should have received a copy of the GNU General Public License
  along with this document; if not, write to the Free Software
  Foundation, Inc., 675 Mass Ave, Cambridge, MA 02139, USA.

\end{small}


\endinput

%

% Local Variables:
% TeX-master: "lshort2e"
% mode: latex
% mode: flyspell
% End:

%%%%%%%%%%%%%%%%%%%%%%%%%%%%%%%%%%%%%%%%%%%%%%%%%%%%%%%%%%%%%%%%%
% Contents: Who contributed to this Document
% $Id$
%%%%%%%%%%%%%%%%%%%%%%%%%%%%%%%%%%%%%%%%%%%%%%%%%%%%%%%%%%%%%%%%%
\chapter{Thank you!}
\noindent Much of the material used in this introduction comes from an
Austrian introduction to \LaTeX\ 2.09 written in German by:
\begin{verse}
\contrib{Hubert Partl}{partl@mail.boku.ac.at}%
{Zentraler Informatikdienst der Universit\"at f\"ur Bodenkultur Wien}
\contrib{Irene Hyna}{Irene.Hyna@bmwf.ac.at}%
   {Bundesministerium f\"ur Wissenschaft und Forschung Wien}
\contrib{Elisabeth Schlegl}{no email}%
   {in Graz}
\end{verse}

If you are interested in the German document, you can find a version
updated for \LaTeXe{} by J\"org Knappen at\\
\CTAN|info/lshort/german|

\newpage \noindent The
following individuals helped with corrections, suggestions and
material to improve this paper. They put in a big effort to help me
get this document into its present shape. I would like to
sincerely thank all of them. Naturally, all the mistakes you'll find
in this book are mine. If you ever find a word that is spelled
correctly, it must have been one of the people below dropping me a
line.

{ \flushleft\small
Eric~Abrahamsen,        %eric@ericabrahamsen.net
Rosemary~Bailey,        %r.a.bailey@qmw.ac.uk 0.2
Marc~Bevand,            % <bevand_m@epita.fr>
Friedemann~Brauer,      %fbrauer@is.dal.ca 3.4
Barbara~Beeton,         %bnb@ams.org
Salvatore~Bonaccorso,   % bonaccos@ee.ethz.ch
Jan~Busa,               % <busaj@ccsun.tuke.sk>
Markus~Br\"uhwiler,     % <m.br@switzerland.org>
Pietro~Braione,         % <braione@elet.polimi.it>
David~Carlisle,         %GONE carlisle@cs.man.ac.uk 1.0
Jos\'e~Carlos~Santos,   % <jcsantos@fc.up.pt>
Neil~Carter,            % N.Carter@Swansea.ac.uk
Mike~Chapman,           %chapman@eeh.ee.ethz.ch 3.16
Pierre~Chardaire,       % <pc@sys.uea.ac.uk
Christopher~Chin,       %chris.chin@rmit.edu.au 3.1
Carl~Cerecke,           %cdc@cosc.canterbury.ac.nz>
Chris~McCormack,        %GONE chrismc@eecs.umich.edu 0.1
Wim~van~Dam,            %GONE wimvdam@cs.kun.nl 2.2
Jan~Dittberner,         %jan@jan-dittberner.de 3.15
Michael~John~Downes,    %<mjd@ams.org> 14 Oct 1999
Matthias~Dreier,        %dreier@ostium.ch
David~Dureisseix,       %dureisse@lmt.ens-cachan.fr 1.1
Eilinger~August,        % <eaugust@student.ethz.ch>
Elliot,                 %GONE enh-a@minster.york.ac.uk 1.1
Hans~Ehrbar,            %ehrbar@econ.utah.edu
Daniel~Flipo,           %Daniel.Flipo@univ-lille1.fr
David~Frey,             %david@eos.lugs.ch 2.2
Hans~Fugal,             %hans@fugal.net
Robin~Fairbairns,       %Robin.Fairbairns@cl.cam.ac.uk 0.2 1.0
J\"org~Fischer,        %j.fischer@xpoint.at 3.16
Erik~Frisk,             %frisk@isy.liu.se 3.4
Mic~Milic~Frederickx,   % <mic.milic@web.de>
Frank,                  %frank@freezone.co.uk 11 Feb 2000
Kasper~B.~Graversen,    % <kbg@dkik.dk>
Arlo~Griffiths,         % <A.Griffiths@let.leidenuniv.nl>
Alexandre~Guimond,      %guimond@IRO.UMontreal.CA 0.9
Andy~Goth,              % <unununium@openverse.com>
Cyril~Goutte,           %goutte@ei.dtu.dk 2.1 2.2
Greg~Gamble,            %gregg@maths.uwa.edu.au 2.2
Frank~Fischli,          % <fischlifaenger@gmx.ch>
Morten~H�gholm,		% morten.hoegholm@latex-project.org
Neil~Hammond,           %nfh@dmu.ac.uk 0.3
Rasmus~Borup~Hansen,    %GONE rbhfamos@math.ku.dk 0.2 0.9 0.91 0.92 1.9.9
Joseph~Hilferty,        % <hilferty@fil.ub.es>
Bj\"orn Hvittfeldt,     %bjorn@hvittfeldt.com 3.13
Martien~Hulsen,         %M.A.Hulsen@WbMt.TUDelft.NL 1.0 1.1
Werner~Icking,          %<Werner.Icking@gmd.de> 3.1
Jakob,                  %diness@get2net.dk
Eric~Jacoboni,          %GONE jacoboni@enseeiht.fr 0.1 0.9
Alan~Jeffrey,           %alanje@cogs.sussex.ac.uk 0.2
Byron~Jones,            %bj@dmu.ac.uk 1.1
David~Jones,            %GONE djones@CA.McMaster.dcss.insight 1.1
Nils~Kanning,           % <nils@kanning.de>
Tobias~Krewer,          %tobias.krewer@googlemail.com
Johannes-Maria~Kaltenbach, %<kaltenbach@zeiss.de> 3.01
Michael~Koundouros,     % <mkoundouros@hotmail.com>
Andrzej~Kawalec,        %GONE akawalec@prz.rzeszow.pl 1.9.9
Sander~de~Kievit,       %Skievit@ucu.uu.nl
Alain~Kessi,            %ALAIN_KESSI@HOTMAIL.COM 2.2
Christian~Kern,         %ck@unixen.hrz.uni-oldenburg.de 2.1
Tobias~Klauser,		%tklauser@access.unizh.ch 4.17
J\"org~Knappen,         %knappen@vkpmzd.kph.uni-mainz.de 0.1
Kjetil~Kjernsmo,        %<kjetil.kjernsmo@astro.uio.no> 3.2
Maik~Lehradt,           %greek@uni-paderborn.de 0.1
R\'emi~Letot,           % <r_letot@yahoo.com>
Flori~Lambrechts,       % <f.lambrechts@softhome.net>
Axel~Liljencrantz,	% <Axel.Liljencrantz@byv.kth.se>
Johan~Lundberg,         %p99jlu@physto.se
Alexander~Mai,          %Alexander.Mai@physik.tu-darmstadt.de 3.8
Hendrik~Maryns,         %hendrik.maryns@ugent.be
Martin~Maechler,        %<maechler@stat.math.ethz.ch> 2.2
Aleksandar~S~Milosevic, % <aleksandar.milosevic@yale.edu>
Henrik~Mitsch,          % <Henrik.Mitsch@gmx.at>
Claus~Malten,           %GONE <ASI138%BITNET.DJUKFA11@BITNET.CEARN> 1.1
Kevin~Van~Maren,        % <vanmaren@fast.cs.utah.edu>  24 Nov 1999
Stefan~M.~Moser,        % stefan.moser@ieee.org 
Richard~Nagy,           % r.nagy@nameshield.net
Philipp~Nagele,         % Philipp.Nagele@t-systems.com
Lenimar~Nunes~de~Andrade, % <lenimar@mat.ufpb.br> Fri, 12 Nov 1999
I.~J.~Vera~ Mar\'un,    % I.J.Veramarun@ewi.utwente.nl
Manuel~Oetiker,         % manuel@oetiker.ch
Urs~Oswald,             % osurs@bluewin.ch
Lan~Thuy~Pham,          %<lan.thuy.pham@gmail.com>
Martin~Pfister,		% m@rtinpfister.ch
Demerson~Andre~Polli,   % polli@linux.ime.usp.br
Nikos~Pothitos,		% <n.pothitos@di.uoa.gr>
Maksym~Polyakov         % <polyama@myrealbox.com>
Hubert~Partl,           %partl@mail.boku.ac.at 0.2 1.1
John~Refling,           %refling@sierra.lbl.gov 0.1 0.9
Mike~Ressler,           %ressler@cougar.jpl.nasa.gov 0.1 0.2 0.9 1.0 1.9.9
Brian~Ripley,           %ripley@stats.ox.ac.uk 2.1
Young~U.~Ryu,           %ryoung@utdallas.edu 2.1
Bernd~Rosenlecher,      %9rosenle@informatik.uni-hamburg.de 10 Feb 2000
Kurt~Rosenfeld,		%kurt@isis.poly.edu
Chris~Rowley,           %C.A.Rowley@open.ac.uk 0.91
Risto~Saarelma,         %risto.saarelma@cs.helsinki.fi
Hanspeter~Schmid,       %schmid@isi.ee.ethz.ch
Craig~Schlenter,        %cschle@lucy.ee.und.ac.za 0.1 0.2 0.9
Gilles~Schintgen,       %gschintgen@internet.lu
Baron~Schwartz,         % <bps7j@cs.virginia.edu>      
Christopher~Sawtell,    %<csawtell@xtra.co.nz> 1 Sep 1999
Miles~Spielberg,        %zeibach@hotmail.com
Matthieu~Stigler,       % table row height
Geoffrey~Swindale,      % <geofftswin@ntlworld.com>
Laszlo~Szathmary,       % <szathml@delfin.klte.hu>
Boris~Tobotras,         % <tobotras@jet.msk.su>
Josef~Tkadlec,          %tkadlec@math.feld.cvut.cz 2.0 2.2
Scott~Veirs,            %scottv@ocean.washington.edu
Didier~Verna,           %verna@inf.enst.fr 2.2
Fabian~Wernli,          %wernli@iap.fr 3.2
Carl-Gustav~Werner,     % <Carl-Gustav.Werner@math.lu.se> 11 Oct 1999,3.16
David~Woodhouse,        % <dwmw2@infradead.org> 3.16
Chris~York,             % <c.s.york@Cummins.com>  21 Nov 1999
Fritz~Zaucker,          %zaucker@ee.ethz.ch 3.0
Rick~Zaccone,           %zaccone@bucknell.edu 2.2
and Mikhail~Zotov.      %zotov@eas.npi.msu.su 3.1

}

\vspace*{\stretch{1}}



\pagebreak
\endinput
%

% Local Variables:
% TeX-master: "lshort2e"
% mode: latex
% mode: flyspell
% End:

%%%%%%%%%%%%%%%%%%%%%%%%%%%%%%%%%%%%%%%%%%%%%%%%%%%%%%%%%%%%%%%%%
% Contents: Who contributed to this Document
% $Id$
%%%%%%%%%%%%%%%%%%%%%%%%%%%%%%%%%%%%%%%%%%%%%%%%%%%%%%%%%%%%%%%%%

% Because this introduction is the reader's first impression, I have
% edited very heavily to try to clarify and economize the language.
% I hope you do not mind! I always try to ask "is this word needed?"
% in my own writing but I don't want to impose my style on you... 
% but here I think it may be more important than the rest of the book.
% --baron

\chapter{Preface}

\LaTeX{} \cite{manual} is a typesetting system that is very 
suitable for producing scientific and mathematical documents of high
typographical quality. It is also suitable for producing all
sorts of other documents, from simple letters to complete books.
\LaTeX{} uses \TeX{} \cite{texbook} as its formatting engine.

This short introduction describes \LaTeXe{} and should be sufficient
for most applications of \LaTeX. Refer to~\cite{manual,companion} for
a complete description of the \LaTeX{} system.

\bigskip
\noindent This introduction is split into 6 chapters:
\begin{description}
\item[Chapter 1] tells you about the basic structure of \LaTeXe{}
  documents. You will also learn a bit about the history of \LaTeX{}.
  After reading this chapter, you should have a rough understanding how
  \LaTeX{} works.
\item[Chapter 2] goes into the details of typesetting your
  documents. It explains most of the essential \LaTeX{} commands and
  environments. After reading this chapter, you will be able to write
  your first documents. 
\item[Chapter 3] explains how to typeset formulae with \LaTeX. Many
  examples demonstrate how to use one of \LaTeX{}'s
  main strengths. At the end of the chapter are tables listing
  all mathematical symbols available in \LaTeX{}.
\item[Chapter 4] explains indexes,  bibliography generation and
  inclusion of EPS graphics. It introduces creation of PDF documents with pdf\LaTeX{}
  and presents some handy extension packages.
\item[Chapter 5] shows how to use \LaTeX{} for creating graphics. Instead
  of drawing a picture with some graphics program, saving it to a file and
  then including it into \LaTeX{} you describe the picture and have \LaTeX{}
  draw it for you.
\item[Chapter 6] contains some potentially dangerous information about
  how to alter the
  standard document layout produced by \LaTeX{}. It will tell you how  to
  change things such that the beautiful output of \LaTeX{}
  turns ugly or stunning, depending on your abilities.
\end{description}
\bigskip
\noindent It is important to read the chapters in order---the book is
not that big, after all. Be sure to carefully read the examples,
because a lot of the information is in the
examples placed throughout the book.

\bigskip
\noindent \LaTeX{} is available for most computers, from the PC and Mac to large
UNIX and VMS systems. On many university computer clusters you will
find that a \LaTeX{} installation is available, ready to use.
Information on how to access
the local \LaTeX{} installation should be provided in the \guide. If
you have problems getting started, ask the person who gave you this
booklet. The scope of this document is \emph{not} to tell you how to
install and set up a \LaTeX{} system, but to teach you how to write
your documents so that they can be processed by~\LaTeX{}.

\bigskip
\noindent If you need to get hold of any \LaTeX{} related material, 
have a look at one of the Comprehensive \TeX{} Archive Network
(CTAN) sites. The homepage is at
\url{http://www.ctan.org}.

You will find other references to CTAN throughout the book, especially
pointers to software and documents you might want to download. Instead
of writing down complete urls, I just wrote \texttt{CTAN:} followed by
whatever location within the CTAN tree you should go to.

If you want to run \LaTeX{} on your own computer, take a look at what
is available from \CTAN|systems|.

\vspace{\stretch{1}}
\noindent If you have ideas for something to be
added, removed or altered in this document, please let me know. I am
especially interested in feedback from \LaTeX{} novices about which
bits of this intro are easy to understand and which could be explained
better.

\bigskip
\begin{verse}
\contrib{Tobias Oetiker}{tobi@oetiker.ch}%
\noindent{OETIKER+PARTNER AG\\Aarweg 15\\4600 Olten\\Switzerland}
\end{verse}
\vspace{\stretch{1}}
\noindent The current version of this document is available on\\
\CTAN|info/lshort|

\endinput



%

% Local Variables:
% TeX-master: "lshort2e"
% mode: latex
% mode: flyspell
% End:

\tableofcontents
\listoffigures
\listoftables
\enlargethispage{\baselineskip}
\mainmatter
%%%%%%%%%%%%%%%%%%%%%%%%%%%%%%%%%%%%%%%%%%%%%%%%%%%%%%%%%%%%%%%%%
% Contents: Things you need to know
% $Id$
%%%%%%%%%%%%%%%%%%%%%%%%%%%%%%%%%%%%%%%%%%%%%%%%%%%%%%%%%%%%%%%%%
 
\chapter{Things You Need to Know}
\begin{intro}
The first part of this chapter presents a short 
overview of the philosophy and history of \LaTeXe. The second part
focuses on the basic structures of a \LaTeX{} document. 
After reading this chapter, you should have a rough knowledge
of how \LaTeX{} works, which you will need to understand the rest
of this book.  
\end{intro}

\section{The Name of the Game}
\subsection{\TeX}
 
\TeX{} is a computer program created by \index{Knuth, Donald E.}Donald
E. Knuth \cite{texbook}. It is aimed at typesetting text and
mathematical formulae. Knuth started writing the \TeX{} typesetting
engine in 1977 to explore the potential of the digital printing
equipment that was beginning to infiltrate the publishing industry at
that time, especially in the hope that he could reverse the trend of
deteriorating typographical quality that he saw affecting his own
books and articles. \TeX{} as we use it today was released in 1982,
with some slight enhancements added in 1989 to better support 8-bit
characters and multiple languages. \TeX{} is renowned for being
extremely stable, for running on many different kinds of computers,
and for being virtually bug free. The version number of \TeX{} is
converging to $\pi$ and is now at $3.141592$.
                                                                       
\TeX{} is pronounced ``Tech,'' with a ``ch'' as in the German word
``Ach''\footnote{In german there are actually two pronounciations for ``ch''
and one might assume that the soft ``ch'' sound from ``Pech'' would be a
more appropriate. Asked about this, Knuth wrote in the German Wikipedia:
\emph{I do not get angry when people pronounce \TeX{} in their favorite way
\ldots{} and in Germany many use a soft ch because the X follows the vowel
e, not the harder ch that follows the vowel a. In Russia, `tex' is a very
common word, pronounced `tyekh'. But I believe the most proper pronunciation
is heard in Greece, where you have the harsher ch of ach and Loch.}}
or in the Scottish ``Loch.'' The ``ch'' originates from the Greek
alphabet where X is the letter ``ch'' or ``chi''. \TeX{} is also the first syllable
of the Greek word texnologia (technology). In an ASCII environment, \TeX{}
becomes \texttt{TeX}.

\subsection{\LaTeX}
 
\LaTeX{} enables authors to typeset and print their work at the highest
typographical quality, using a predefined, professional layout. \LaTeX{} was
originally written by \index{Lamport, Leslie}Leslie Lamport~\cite{manual}.
It uses the \TeX{} formatter as its typesetting engine. These days \LaTeX{}
is maintained by \index{Mittelbach, Frank}Frank Mittelbach.

%In 1994 the \LaTeX{} package was updated by the \index{LaTeX3@\LaTeX
%  3}\LaTeX 3 team, led by \index{Mittelbach, Frank}Frank Mittelbach,
%to include some long-requested improvements, and to re\-unify all the
%patched versions which had cropped up since the release of
%\index{LaTeX 2.09@\LaTeX{} 2.09}\LaTeX{} 2.09 some years earlier. To
%distinguish the new version from the old, it is called \index{LaTeX
%2e@\LaTeXe}\LaTeXe. This documentation deals with \LaTeXe. These days you
%might be hard pressed to find the venerable \LaTeX{} 2.09 installed
%anywhere.

\LaTeX{} is pronounced ``Lay-tech'' or ``Lah-tech.'' If you refer to
\LaTeX{} in an ASCII environment, you type \texttt{LaTeX}.
\LaTeXe{} is pronounced ``Lay-tech two e'' and typed \texttt{LaTeX2e}.

%Figure~\ref{components} above % on page \pageref{components}
%shows how \TeX{} and \LaTeXe{} work together. This figure is taken from
%\texttt{wots.tex} by Kees van der Laan.

%\begin{figure}[btp]
%\begin{lined}{0.8\textwidth}
%\begin{center}
%\input{kees.fig}
%\end{center}
%\end{lined}
%\caption{Components of a \TeX{} System.} \label{components}
%\end{figure}

\section{Basics}
 
\subsection{Author, Book Designer, and Typesetter}

To publish something, authors give their typed manuscript to a
publishing company. One of their book designers then
decides the layout of the document (column width, fonts, space before
and after headings,~\ldots). The book designer writes his instructions
into the manuscript and then gives it to a typesetter, who typesets the
book according to these instructions.

A human book designer tries to find out what the author had in mind
while writing the manuscript. He decides on chapter headings,
citations, examples, formulae, etc.\ based on his professional
knowledge and from the contents of the manuscript.

In a \LaTeX{} environment, \LaTeX{} takes the role of the book
designer and uses \TeX{} as its typesetter. But \LaTeX{} is ``only'' a
program and therefore needs more guidance. The author has to provide
additional information to describe the logical structure of his
work. This information is written into the text as ``\LaTeX{}
commands.''

This is quite different from the \wi{WYSIWYG}\footnote{What you see is
  what you get.} approach that most modern word processors, such as
\emph{MS Word} or \emph{LibreOffice}, take. With these
applications, authors specify the document layout interactively while
typing text into the computer. They can see on the
screen how the final work will look when it is printed.

When using \LaTeX{} it is not normally possible to see the final output
while typing the text, but the final output can be previewed on the
screen after processing the file with \LaTeX. Then corrections can be
made before actually sending the document to the printer.

\subsection{Layout Design}

Typographical design is a craft. Unskilled authors often commit
serious formatting errors by assuming that book design is mostly a
question of aesthetics---``If a document looks good artistically,
it is well designed.'' But as a document has to be read and not hung
up in a picture gallery, the readability and understandability is 
much more important than the beautiful look of it.
Examples: 
\begin{itemize}
\item The font size and the numbering of headings have to be chosen to make
  the structure of chapters and sections clear to the reader.
\item The line length has to be short enough not to strain
  the eyes of the reader, while long enough to fill the page
  beautifully.
\end{itemize}

With \wi{WYSIWYG} systems, authors often generate aesthetically
pleasing documents with very little or inconsistent structure.
\LaTeX{} prevents such formatting errors by forcing the author to
declare the \emph{logical} structure of his document. \LaTeX{} then
chooses the most suitable layout.

\subsection{Advantages and Disadvantages}

When people from the \wi{WYSIWYG} world meet people who use \LaTeX{},
they often discuss ``the \wi{advantages of \LaTeX{}} over a normal
word processor'' or the opposite.  The best thing to do when such
a discussion starts is to keep a low profile, since such discussions
often get out of hand. But sometimes there is no escaping \ldots

\medskip\noindent So here is some ammunition. The main advantages
of \LaTeX{} over normal word processors are the following:

\begin{itemize}

\item Professionally crafted layouts are available, which make a
  document really look as if ``printed.''
\item The typesetting of mathematical formulae is supported in a
  convenient way.
\item Users only need to learn a few easy-to-understand commands
  that specify the logical structure of a document. They almost never
  need to tinker with the actual layout of the document.
\item Even complex structures such as footnotes, references, table of
  contents, and bibliographies can be generated easily.
\item Free add-on packages exist for many typographical tasks not directly supported by basic
  \LaTeX. For example, packages are
  available to include \PSi{} graphics or to typeset
  bibliographies conforming to exact standards. Many of these add-on
  packages are described in \companion.
\item \LaTeX{} encourages authors to write well-structured texts,
  because this is how \LaTeX{} works---by specifying structure.
\item \TeX, the formatting engine of \LaTeXe, is highly portable and free.
  Therefore the system runs on almost any hardware platform
  available. 

%
% Add examples ...
%
\end{itemize}

\medskip

\noindent\LaTeX{} also has some disadvantages, and I guess it's a bit
difficult for me to find any sensible ones, though I am sure other people
can tell you hundreds \texttt{;-)}

\begin{itemize}
\item \LaTeX{} does not work well for people who have sold their
  souls \ldots
\item Although some parameters can be adjusted within a predefined
  document layout, the design of a whole new layout is difficult and
  takes a lot of time.\footnote{Rumour says that this is one of the
    key elements that will be addressed in the upcoming \LaTeX 3
    system.}\index{LaTeX3@\LaTeX 3}
\item It is very hard to write unstructured and disorganized documents.
\item Your hamster might, despite some encouraging first steps, never be
able to fully grasp the concept of Logical Markup.
\end{itemize}

\section{\LaTeX{} Input Files}

The input for \LaTeX{} is a plain text file. On Unix/Linux text files are
pretty common. On windows, one would use Notepad to create a text file. It
contains the text of the document, as well as the commands that tell
\LaTeX{} how to typeset the text. If you are working with a LaTeX IDE, it will contain a program for creating
\LaTeX{} input files in text format.

\subsection{Spaces}

``Whitespace'' characters, such as blank or tab, are
treated uniformly as ``\wi{space}'' by \LaTeX{}. \emph{Several
  consecutive} \wi{whitespace} characters are treated as \emph{one}
``space.''  Whitespace at the start of a line is generally ignored, and
a single line break is treated as ``whitespace.''
\index{whitespace!at the start of a line}

An empty line between two lines of text defines the end of a
paragraph. \emph{Several} empty lines are treated the same as
\emph{one} empty line. The text below is an example. On the left hand
side is the text from the input file, and on the right hand side is the
formatted output.

\begin{example}
It does not matter whether you
enter one or several     spaces
after a word.

An empty line starts a new 
paragraph.
\end{example}
 
\subsection{Special Characters}

The following symbols are \wi{reserved characters} that either have a
special meaning under \LaTeX{} or are not available in all the fonts.
If you enter them directly in your text, they will normally not print,
but rather coerce \LaTeX{} to do things you did not intend. 
\begin{code}
\verb.#  $  %  ^  &  _  {  }  ~  \ . %$
\end{code}

As you will see, these characters can be used in your documents all
the same by using a prefix backslash:

\begin{example}
\# \$ \% \^{} \& \_ \{ \} \~{}
\textbackslash
\end{example}

The other symbols and many more can be printed with special commands
in mathematical formulae or as accents. The backslash character
\textbackslash{} can \emph{not} be entered by adding another backslash
in front of it (\verb|\\|); this sequence is used for
line breaking. Use the \ci{textbackslash} command instead.

\subsection{\LaTeX{} Commands}

\LaTeX{} \wi{commands} are case sensitive, and take one of the following
two formats:

\begin{itemize}
\item They start with a \wi{backslash} \verb|\| and then have a name
 consisting of letters only. Command names are terminated by a
 space, a number or any other `non-letter.'
\item They consist of a backslash and exactly one non-letter.
\end{itemize}

%
% \\* doesn't comply !
%

%
% Can \3 be a valid command ? (jacoboni)
%
\label{whitespace}

\LaTeX{} ignores whitespace after commands. If you want to get a
\index{whitespace!after commands}space after a command, you have to
put either \verb|{}| and a blank or a special spacing command after the
command name. The \verb|{}| stops \LaTeX{} from eating up all the space after
the command name. 

\begin{example}
I read that Knuth divides the 
people working with \TeX{} into 
\TeX{}nicians and \TeX perts.\\
Today is \today.
\end{example}

Some commands need a \wi{parameter}, which has to be given between
\wi{curly braces} \verb|{ }| after the command name. Some commands support
\wi{optional parameters}, which are added after the command name in
\wi{square brackets}~\verb|[ ]|. The next examples use some \LaTeX{}
commands. Don't worry about them; they will be explained later.

\begin{example}
You can \textsl{lean} on me!
\end{example}
\begin{example}
Please, start a new line
right here!\newline
Thank you!
\end{example}

\subsection{Comments}
\index{comments}

When \LaTeX{} encounters a \verb|%| character while processing an input file,
it ignores the rest of the present line, the line break, and all
whitespace at the beginning of the next line.

This can be used to write notes into the input file, which will not show up
in the printed version.

\begin{example}
This is an % stupid
% Better: instructive <----
example: Supercal%
              ifragilist%
    icexpialidocious
\end{example}

The \texttt{\%} character can also be used to split long input lines where no
whitespace or line breaks are allowed.

For longer comments you could use the \ei{comment} environment
provided by the \pai{verbatim} package. Add the
line \verb|\usepackage{verbatim}| to the preamble of your document as
explained below to use this command.

\begin{example}
This is another
\begin{comment}
rather stupid,
but helpful
\end{comment}
example for embedding
comments in your document.
\end{example}

Note that this won't work inside complex environments, like math for example.

\section{Input File Structure}

When \LaTeXe{} processes an input file, it expects it to follow a
certain \wi{structure}. Thus every input file must start with the
command
\begin{code}
\verb|\documentclass{...}|
\end{code}
This specifies what sort of document you intend to write. After that,
add commands to influence the style of the whole
document, or load \wi{package}s that add new
features to the \LaTeX{} system. To load such a package you use the
command
\begin{code}
\verb|\usepackage{...}|
\end{code}

When all the setup work is done,\footnote{The area between \texttt{\bs
    documentclass} and \texttt{\bs
    begin$\mathtt{\{}$document$\mathtt{\}}$} is called the
  \emph{\wi{preamble}}.} you start the body of the text with the
command

\begin{code}
\verb|\begin{document}|
\end{code}

Now you enter the text mixed with some useful \LaTeX{} commands.  At
the end of the document you add the
\begin{code}
\verb|\end{document}|
\end{code}
command, which tells \LaTeX{} to call it a day. Anything that
follows this command will be ignored by \LaTeX.

Figure~\ref{mini} shows the contents of a minimal \LaTeXe{} file. A
slightly more complicated \wi{input file} is given in
Figure~\ref{document}.

\begin{figure}[!bp]
\begin{lined}{6cm}
\begin{verbatim}
\documentclass{article}
\begin{document}
Small is beautiful.
\end{document}
\end{verbatim}
\end{lined}
\caption{A Minimal \LaTeX{} File.} \label{mini}
\end{figure}
 
\begin{figure}[!bp]
\begin{lined}{10cm}
\begin{verbatim}
\documentclass[a4paper,11pt]{article}
% define the title
\author{H.~Partl}
\title{Minimalism}
\begin{document}
% generates the title
\maketitle 
% insert the table of contents
\tableofcontents
\section{Some Interesting Words}
Well, and here begins my lovely article.
\section{Good Bye World}
\ldots{} and here it ends.
\end{document}
\end{verbatim}
\end{lined}
\caption[Example of a Realistic Journal Article.]{Example of a Realistic
Journal Article. Note that all the commands you see in this example will be
explained later in the introduction.} \label{document}

\end{figure}

\section{A Typical Command Line Session}

I bet you must be dying to try out the neat small \LaTeX{} input file
shown on page \pageref{mini}. Here is some help:
\LaTeX{} itself comes without a GUI or
fancy buttons to press. It is just a program that crunches away at your
input file. Some \LaTeX{} installations feature a graphical front-end where
there is a \LaTeX{} button to start compiling your input file. On other systems
there might be some typing involved, so here is how to coax \LaTeX{} into
compiling your input file on a text based system. Please note: this
description assumes that a working \LaTeX{} installation already sits on
your computer.\footnote{This is the case with most well groomed Unix
Systems, and \ldots{} Real Men use Unix, so \ldots{} \texttt{;-)}}

\begin{enumerate}
\item 
  
  Edit/Create your \LaTeX{} input file. This file must be plain ASCII
  text.  On Unix all the editors will create just that. On Windows you
  might want to make sure that you save the file in ASCII or
  \emph{Plain Text} format.  When picking a name for your file, make
  sure it bears the extension \eei{.tex}.

\item 

Open a shell or cmd window, cd to the directory where your input file is located and run \LaTeX{} on your input file. If successful you will end up with a
\texttt{.dvi} file. It may be necessary to run \LaTeX{} several times to get
the table of contents and all internal references right. When your input
file has a bug \LaTeX{} will tell you about it and stop processing your
input file. Type \texttt{ctrl-D} to get back to the command line.
\begin{lscommand}
\verb+latex foo.tex+
\end{lscommand}

\item 
Now you may view the DVI file. There are several ways to do that. Look at the file on screen with
\begin{lscommand}
\verb+xdvi foo.dvi &+
\end{lscommand}
This only works on Unix with X11. If you are on Windows you might want to try \texttt{yap} (yet another previewer).

Convert the dvi file to \PSi{} for printing or viewing with \wi{GhostScript}.
\begin{lscommand}
\verb+dvips -Pcmz foo.dvi -o foo.ps+
\end{lscommand}

If you are lucky your \LaTeX{} system even comes with the \texttt{dvipdf} tool, which allows
you to convert your \texttt{.dvi} files straight into pdf.
\begin{lscommand}
\verb+dvipdf foo.dvi+
\end{lscommand}

\end{enumerate}

 
\section{The Layout of the Document}
 
\subsection {Document Classes}\label{sec:documentclass}

The first information \LaTeX{} needs to know when processing an
input file is the type of document the author wants to create. This
is specified with the \ci{documentclass} command.
\begin{lscommand}
\ci{documentclass}\verb|[|\emph{options}\verb|]{|\emph{class}\verb|}|
\end{lscommand}
\noindent Here \emph{class} specifies the type of document to be created.
Table~\ref{documentclasses} lists the document classes explained in
this introduction. The \LaTeXe{} distribution provides additional
classes for other documents, including letters and slides.  The
\emph{\wi{option}s} parameter customises the behaviour of the document
class. The options have to be separated by commas. The most common options for the standard document
classes are listed in
Table~\ref{options}.


\begin{table}[!bp]
\caption{Document Classes.} \label{documentclasses}
\begin{lined}{\textwidth}
\begin{description}
 
\item [\normalfont\texttt{article}] for articles in scientific journals, presentations,
  short reports, program documentation, invitations, \ldots
  \index{article class}
\item [\normalfont\texttt{proc}] a class for proceedings based on the article class.
  \index{proc class}
\item [\normalfont\texttt{minimal}] is as small as it can get.
It only sets a page size and a base font. It is mainly used for debugging
purposes.
  \index{minimal class}
\item [\normalfont\texttt{report}] for longer reports containing several chapters, small
  books, PhD theses, \ldots \index{report class}
\item [\normalfont\texttt{book}] for real books \index{book class}
\item [\normalfont\texttt{slides}] for slides. The class uses big sans serif
  letters. You might want to consider using the Beamer class instead.
        \index{slides class}
\end{description}
\end{lined}
\end{table}

\begin{table}[!bp]
\caption{Document Class Options.} \label{options}
\begin{lined}{\textwidth}
\begin{flushleft}
\begin{description}
\item[\normalfont\texttt{10pt}, \texttt{11pt}, \texttt{12pt}] \quad Sets the size
  of the main font in the document. If no option is specified,
  \texttt{10pt} is assumed.  \index{document font size}\index{base
    font size}
\item[\normalfont\texttt{a4paper}, \texttt{letterpaper}, \ldots] \quad Defines
  the paper size. The default size is \texttt{letterpaper}. Besides
  that, \texttt{a5paper}, \texttt{b5paper}, \texttt{executivepaper},
  and \texttt{legalpaper} can be specified.  \index{legal paper}
  \index{paper size}\index{A4 paper}\index{letter paper} \index{A5
    paper}\index{B5 paper}\index{executive paper}

\item[\normalfont\texttt{fleqn}] \quad Typesets displayed formulae left-aligned
  instead of centred.

\item[\normalfont\texttt{leqno}] \quad Places the numbering of formulae on the
  left hand side instead of the right.

\item[\normalfont\texttt{titlepage}, \texttt{notitlepage}] \quad Specifies
  whether a new page should be started after the \wi{document title}
  or not. The \texttt{article} class does not start a new page by
  default, while \texttt{report} and \texttt{book} do.  \index{title}

\item[\normalfont\texttt{onecolumn}, \texttt{twocolumn}] \quad Instructs \LaTeX{} to typeset the
  document in \wi{one column} or \wi{two column}s.

\item[\normalfont\texttt{twoside, oneside}] \quad Specifies whether double or
  single sided output should be generated. The classes
  \texttt{article} and \texttt{report} are \wi{single sided} and the
  \texttt{book} class is \wi{double sided} by default. Note that this
  option concerns the style of the document only. The option
  \texttt{twoside} does \emph{not} tell the printer you use that it
  should actually make a two-sided printout.
\item[\normalfont\texttt{landscape}] \quad Changes the layout of the document to print in landscape mode.
\item[\normalfont\texttt{openright, openany}] \quad Makes chapters begin either
  only on right hand pages or on the next page available. This does
  not work with the \texttt{article} class, as it does not know about
  chapters. The \texttt{report} class by default starts chapters on
  the next page available and the \texttt{book} class starts them on
  right hand pages.

\end{description}
\end{flushleft}
\end{lined}
\end{table}

Example: An input file for a \LaTeX{} document could start with the
line
\begin{code}
\ci{documentclass}\verb|[11pt,twoside,a4paper]{article}|
\end{code}
which instructs \LaTeX{} to typeset the document as an \emph{article}
with a base font size of \emph{eleven points}, and to produce a
layout suitable for \emph{double sided} printing on \emph{A4 paper}.
\pagebreak[2]

\subsection{Packages}
\index{package} While writing your document, you will probably find
that there are some areas where basic \LaTeX{} cannot solve your
problem. If you want to include \wi{graphics}, \wi{coloured text} or
source code from a file into your document, you need to enhance the
capabilities of \LaTeX.  Such enhancements are called packages.
Packages are activated with the
\begin{lscommand}
\ci{usepackage}\verb|[|\emph{options}\verb|]{|\emph{package}\verb|}|
\end{lscommand}
\noindent command, where \emph{package} is the name of the package and
\emph{options} is a list of keywords that trigger special features in
the package. Some packages come with the \LaTeXe{} base distribution
(See Table~\ref{packages}). Others are provided separately. You may
find more information on the packages installed at your site in your
\guide. The prime source for information about \LaTeX{} packages is \companion.
It contains descriptions on hundreds of packages, along with
information of how to write your own extensions to \LaTeXe.

Modern \TeX{} distributions come with a large number of packages
preinstalled. If you are working on a Unix system, use the command
\texttt{texdoc} for accessing package documentation.

\begin{table}[btp]
\caption{Some of the Packages Distributed with \LaTeX.} \label{packages}
\begin{lined}{\textwidth}
\begin{description}
\item[\normalfont\pai{doc}] Allows the documentation of \LaTeX{} programs.\\
 Described in \texttt{doc.dtx}\footnote{This file should be installed
   on your system, and you should be able to get a \texttt{dvi} file
   by typing \texttt{latex doc.dtx} in any directory where you have
   write permission. The same is true for all the
   other files mentioned in this table.}  and in \companion.

\item[\normalfont\pai{exscale}] Provides scaled versions of the
  math extension  font.\\ 
  Described in \texttt{ltexscale.dtx}.

\item[\normalfont\pai{fontenc}] Specifies which \wi{font encoding}
  \LaTeX{} should use.\\
  Described in \texttt{ltoutenc.dtx}.

\item[\normalfont\pai{ifthen}] Provides commands of the form\\ 
  `if\ldots then do\ldots otherwise do\ldots.'\\ Described in
  \texttt{ifthen.dtx} and \companion.

\item[\normalfont\pai{latexsym}] To access the \LaTeX{} symbol
  font, you should use the \texttt{latexsym} package. Described in
  \texttt{latexsym.dtx} and in \companion.
 
\item[\normalfont\pai{makeidx}] Provides commands for producing
  indexes.  Described in section~\ref{sec:indexing} and in \companion.

\item[\normalfont\pai{syntonly}] Processes a document without
  typesetting it.
  
\item[\normalfont\pai{inputenc}] Allows the specification of an
  input encoding such as ASCII, ISO Latin-1, ISO Latin-2, 437/850 IBM
  code pages,  Apple Macintosh, Next, ANSI-Windows or user-defined one.
  Described in \texttt{inputenc.dtx}. 
\end{description}
\end{lined}
\end{table}


\subsection{Page Styles}
 
\LaTeX{} supports three predefined \wi{header}/\wi{footer}
combinations---so-called \wi{page style}s. The \emph{style} parameter
of the \index{page style!plain@\texttt{plain}}\index{plain@\texttt{plain}}
\index{page style!headings@\texttt{headings}}\index{headings@texttt{headings}}
\index{page style!empty@\texttt{empty}}\index{empty@\texttt{empty}}
\begin{lscommand}
\ci{pagestyle}\verb|{|\emph{style}\verb|}|
\end{lscommand}
\noindent command defines which one to use. 
Table~\ref{pagestyle}
lists the predefined page styles.

\begin{table}[!hbp]
\caption{The Predefined Page Styles of \LaTeX.} \label{pagestyle}
\begin{lined}{\textwidth}
\begin{description}

\item[\normalfont\texttt{plain}] prints the page numbers on the bottom
  of the page, in the middle of the footer. This is the default page
  style.

\item[\normalfont\texttt{headings}] prints the current chapter heading
  and the page number in the header on each page, while the footer
  remains empty.  (This is the style used in this document)
\item[\normalfont\texttt{empty}] sets both the header and the footer
  to be empty.

\end{description}
\end{lined}
\end{table}

It is possible to change the page style of the current page
with the command
\begin{lscommand}
\ci{thispagestyle}\verb|{|\emph{style}\verb|}|
\end{lscommand}
A description how to create your own
headers and footers can be found in \companion{} and in section~\ref{sec:fancy} on page~\pageref{sec:fancy}.
%
% Pointer to the Fancy headings Package description !
%

\section{Files You Might Encounter}

When you work with \LaTeX{} you will soon find yourself in a maze of
files with various \wi{extension}s and probably no clue. The following
list explains the various \wi{file types} you might encounter when
working with \TeX{}. Please note that this table does not claim to be
a complete list of extensions, but if you find one missing that you
think is important, please drop me a line.

\begin{description}
  
\item[\eei{.tex}] \LaTeX{} or \TeX{} input file. Can be compiled with
  \texttt{latex}.
\item[\eei{.sty}] \LaTeX{} Macro package. Load this
  into your \LaTeX{} document using the \ci{usepackage} command.
\item[\eei{.dtx}] Documented \TeX{}. This is the main distribution
  format for \LaTeX{} style files. If you process a .dtx file you get
  documented macro code of the \LaTeX{} package contained in the .dtx
  file.
\item[\eei{.ins}] The installer for the files contained in the
  matching .dtx file. If you download a \LaTeX{} package from the net,
  you will normally get a .dtx and a .ins file. Run \LaTeX{} on the
  .ins file to unpack the .dtx file.
\item[\eei{.cls}] Class files define what your document looks
  like. They are selected with the \ci{documentclass} command.
\item[\eei{.fd}] Font description file telling  \LaTeX{} about new fonts.
\end{description}
The following files are generated when you run \LaTeX{} on your input
file:

\begin{description}
\item[\eei{.dvi}] Device Independent File. This is the main result of a \LaTeX{}
  compile run. Look at its content with a DVI previewer
  program or send it to a printer with \texttt{dvips} or a
  similar application.
\item[\eei{.log}] Gives a detailed account of what happened during the
  last compiler run.
\item[\eei{.toc}] Stores all your section headers. It gets read in for the
  next compiler run and is used to produce the table of content.
\item[\eei{.lof}] This is like .toc but for the list of figures.
\item[\eei{.lot}] And again the same for the list of tables.
\item[\eei{.aux}] Another file that transports information from one
  compiler run to the next. Among other things, the .aux file is used
  to store information associated with cross-references.
\item[\eei{.idx}] If your document contains an index. \LaTeX{} stores all
  the words that go into the index in this file. Process this file with
  \texttt{makeindex}. Refer to section \ref{sec:indexing} on
  page \pageref{sec:indexing} for more information on indexing.
\item[\eei{.ind}] The processed .idx file, ready for inclusion into your
  document on the next compile cycle.
\item[\eei{.ilg}] Logfile telling what \texttt{makeindex} did.
\end{description}


% Package Info pointer
%
%



%
% Add Info on page-numbering, ...
% \pagenumbering

\section{Big Projects}
When working on big documents, you might want to split the input file
into several parts. \LaTeX{} has two commands that help you to do
that.

\begin{lscommand}
\ci{include}\verb|{|\emph{filename}\verb|}|
\end{lscommand}
\noindent Use this command in the document body to insert the
contents of another file named \emph{filename.tex}. Note that \LaTeX{}
will start a new page
before processing the material input from \emph{filename.tex}.

The second command can be used in the preamble. It allows you to
instruct \LaTeX{} to only input some of the \verb|\include|d files.
\begin{lscommand}
\ci{includeonly}\verb|{|\emph{filename}\verb|,|\emph{filename}%
\verb|,|\ldots\verb|}|
\end{lscommand}
After this command is executed in the preamble of the document, only
\ci{include} commands for the filenames that are listed in the
argument of the \ci{includeonly} command will be executed. Note that
there must be no spaces between the filenames and the commas.

The \ci{include} command starts typesetting the included text on a new
page. This is helpful when you use \ci{includeonly}, because the
page breaks will not move, even when some include files are omitted.
Sometimes this might not be desirable. In this case, use the
\begin{lscommand}
\ci{input}\verb|{|\emph{filename}\verb|}|
\end{lscommand}
\noindent command. It simply includes the file specified. 
No flashy suits, no strings attached.


To make \LaTeX{} quickly check your document use the \pai{syntonly}
package. This makes \LaTeX{} skim through your document only checking for
proper syntax and usage of the commands, but doesn't produce any (DVI) output.
As \LaTeX{} runs faster in this mode you may save yourself valuable time.
Usage is very simple:

\begin{verbatim}
\usepackage{syntonly}
\syntaxonly
\end{verbatim}
When you want to produce pages, just comment out the second line
(by adding a percent sign).


%

% Local Variables:
% TeX-master: "lshort2e"
% mode: latex
% mode: flyspell
% End:

%%%%%%%%%%%%%%%%%%%%%%%%%%%%%%%%%%%%%%%%%%%%%%%%%%%%%%%%%%%%%%%%%
% Contents: Typesetting Part of LaTeX2e Introduction
% $Id$
%%%%%%%%%%%%%%%%%%%%%%%%%%%%%%%%%%%%%%%%%%%%%%%%%%%%%%%%%%%%%%%%%
\chapter{Typesetting Text}

\begin{intro}
  After reading the previous chapter, you should know about the basic
  stuff of which a \LaTeXe{} document is made. In this chapter I
  will fill in the remaining structure you will need to know in order
  to produce real world material.
\end{intro}

\section{The Structure of Text and Language}
\secby{Hanspeter Schmid}{hanspi@schmid-werren.ch}
The main point of writing a text (some modern DAAC\footnote{Different
  At All Cost, a translation of the Swiss German UVA (Um's Verrecken
  Anders).} literature excluded), is to convey ideas, information, or
knowledge to the reader.  The reader will understand the text better
if these ideas are well-structured, and will see and feel this
structure much better if the typographical form reflects the logical
and semantical structure of the content.

\LaTeX{} is different from other typesetting systems in that you just
have to tell it the logical and semantical structure of a text.  It
then derives the typographical form of the text according to the
``rules'' given in the document class file and in various style files.

The most important text unit in \LaTeX{} (and in typography) is the
\wi{paragraph}.  We call it ``text unit'' because a paragraph is the
typographical form that should reflect one coherent thought, or one
idea.  You will learn in the following sections how to force
line breaks with e.g.{} \texttt{\bs\bs}, and paragraph breaks with e.g.{} 
leaving an empty line in the source code.  Therefore, if a new thought
begins, a new paragraph should begin, and if not, only line breaks
should be used.  If in doubt about paragraph breaks, think about your
text as a conveyor of ideas and thoughts.  If you have a paragraph
break, but the old thought continues, it should be removed.  If some
totally new line of thought occurs in the same paragraph, then it
should be broken.

Most people completely underestimate the importance of well-placed
paragraph breaks.  Many people do not even know what the meaning of
a paragraph break is, or, especially in \LaTeX, introduce paragraph
breaks without knowing it.  The latter mistake is especially easy to
make if equations are used in the text.  Look at the following
examples, and figure out why sometimes empty lines (paragraph breaks)
are used before and after the equation, and sometimes not.  (If you
don't yet understand all commands well enough to understand these
examples, please read this and the following chapter, and then read
this section again.)

\begin{code}
\begin{verbatim}
% Example 1
\ldots when Einstein introduced his formula 
\begin{equation} 
  e = m \cdot c^2 \; , 
\end{equation} 
which is at the same time the most widely known 
and the least well understood physical formula. 


% Example 2
\ldots from which follows Kirchhoff's current law:
\begin{equation} 
  \sum_{k=1}^{n} I_k = 0 \; .
\end{equation} 

Kirchhoff's voltage law can be derived \ldots


% Example 3
\ldots which has several advantages.

\begin{equation} 
  I_D = I_F - I_R
\end{equation} 
is the core of a very different transistor model. \ldots
\end{verbatim}
\end{code} 

The next smaller text unit is a sentence.  In English texts, there is
a larger space after a period that ends a sentence than after one
that ends an abbreviation.  \LaTeX{} tries to figure out which one
you wanted to have.  If \LaTeX{} gets it wrong, you must tell it what
you want.  This is explained later in this chapter.

The structuring of text even extends to parts of sentences.  Most
languages have very complicated punctuation rules, but in many
languages (including German and English), you will get almost every
comma right if you remember what it represents: a short stop in the
flow of language.  If you are not sure about where to put a comma,
read the sentence aloud and take a short breath at every comma.  If
this feels awkward at some place, delete that comma; if you feel the
urge to breathe (or make a short stop) at some other place, insert a
comma.

Finally, the paragraphs of a text should also be structured logically
at a higher level, by putting them into chapters, sections,
subsections, and so on.  However, the typographical effect of writing
e.g.{} \verb|\section{The| \texttt{Structure of Text and Language}\verb|}| is
so obvious that it is almost self-evident how these high-level
structures should be used.

\section{Line Breaking and Page Breaking}
 
\subsection{Justified Paragraphs}

Books are often typeset with each line having the same length.
\LaTeX{} inserts the necessary \wi{line break}s and spaces between words
by optimizing the contents of a whole paragraph. If necessary, it
also hyphenates words that would not fit comfortably on a line.
How the paragraphs are typeset depends on the document class.
Normally the first line of a paragraph is indented, and there is no
additional space between two paragraphs. Refer to section~\ref{parsp}
for more information.

In special cases it might be necessary to order \LaTeX{} to break a
line: 
\begin{lscommand}
\ci{\bs} or \ci{newline} 
\end{lscommand}
\noindent starts a new line without starting a new paragraph. 

\begin{lscommand}
\ci{\bs*}
\end{lscommand}
\noindent additionally prohibits a page break after the forced
line break. 

\begin{lscommand}
\ci{newpage}
\end{lscommand}
\noindent starts a new page. 

\begin{lscommand}
\ci{linebreak}\verb|[|\emph{n}\verb|]|,
\ci{nolinebreak}\verb|[|\emph{n}\verb|]|, 
\ci{pagebreak}\verb|[|\emph{n}\verb|]|,
\ci{nopagebreak}\verb|[|\emph{n}\verb|]|
\end{lscommand}
\noindent suggest places where a break may (or may not happen). They enable the author to influence their
actions with the optional argument \emph{n}, which can be set to a number
between zero and four. By setting \emph{n} to a value below 4, you leave
\LaTeX{} the option of ignoring your command if the result would look very
bad. Do not confuse these ``break'' commands with the ``new'' commands. Even
when you give a ``break'' command, \LaTeX{} still tries to even out the
right border of the line and the total length of the page, as described in
the next section; this can lead to unpleasant gaps in your text.
If you really want to start a ``new line'' or a ``new page'', then use the
corresponding command. Guess their names!


\LaTeX{} always tries to produce the best line breaks possible. If it
cannot find a way to break the lines in a manner that meets its high
standards, it lets one line stick out on the right of the paragraph.
\LaTeX{} then complains (``\wi{overfull hbox}'') while processing the
input file. This happens most often when \LaTeX{} cannot find a
suitable place to hyphenate a word.\footnote{Although \LaTeX{} gives
  you a warning when that happens (\texttt{Overfull \bs{}hbox}) and displays the
  offending line, such lines are not always easy to find. If you use
  the option \texttt{draft} in the \ci{documentclass} command, these
  lines will be marked with a thick black line on the right margin.}
Instruct \LaTeX{} to lower its standards a little by giving
the \ci{sloppy} command. It prevents such over-long lines by
increasing the inter-word spacing---even if the final output is not
optimal.  In this case a warning (``\wi{underfull hbox}'') is given to
the user.  In most such cases the result doesn't look very good. The
command \ci{fussy} brings \LaTeX{} back to its default behaviour.

\subsection{Hyphenation} \label{hyph}

\LaTeX{} hyphenates words whenever necessary. If the hyphenation
algorithm does not find the correct hyphenation points,
remedy the situation by using the following commands to tell \TeX{}
about the exception.

The command
\begin{lscommand}
\ci{hyphenation}\verb|{|\emph{word list}\verb|}|
\end{lscommand}
\noindent causes the words listed in the argument to be hyphenated only at
the points marked by ``\verb|-|''.  The argument of the command should only
contain words built from normal letters, or rather signs that are considered
to be normal letters by \LaTeX{}. The hyphenation hints are
stored for the language that is active when the hyphenation command
occurs. This means that if you place a hyphenation command into the preamble
of your document it will influence the English language hyphenation. If you
place the command after the \verb|\begin{document}| and you are using some
package for national language support like \pai{babel}, then the hyphenation
hints will be active in the language activated through \pai{babel}.

The example below will allow ``hyphenation'' to be hyphenated as well as
``Hyphenation'', and it prevents ``FORTRAN'', ``Fortran'' and ``fortran''
from being hyphenated at all.  No special characters or symbols are allowed
in the argument.

Example:
\begin{code}
\verb|\hyphenation{FORTRAN Hy-phen-a-tion}|
\end{code}

The command \ci{-} inserts a discretionary hyphen into a word. This
also becomes the only point hyphenation is allowed in this word. This
command is especially useful for words containing special characters
(e.g.{} accented characters), because \LaTeX{} does not automatically
hyphenate words containing special characters.
%\footnote{Unless you are using the new
%\wi{DC fonts}.}.

\begin{example}
I think this is: su\-per\-cal\-%
i\-frag\-i\-lis\-tic\-ex\-pi\-%
al\-i\-do\-cious
\end{example}

Several words can be kept together on one line with the command
\begin{lscommand}
\ci{mbox}\verb|{|\emph{text}\verb|}|
\end{lscommand}
\noindent It causes its argument to be kept together under all circumstances.

\begin{example}
My phone number will change soon.
It will be \mbox{0116 291 2319}.

The parameter 
\mbox{\emph{filename}} should 
contain the name of the file.
\end{example}

\ci{fbox} is similar to \ci{mbox}, but in addition there will
be a visible box drawn around the content.


\section{Ready-Made Strings}

In some of the examples on the previous pages, you have seen
some very simple \LaTeX{} commands for typesetting special
text strings:

\vspace{2ex}

\noindent
\begin{tabular}{@{}lll@{}}
Command&Example&Description\\
\hline
\ci{today} & \today   & Current date\\
\ci{TeX} & \TeX       & Your favorite typesetter\\
\ci{LaTeX} & \LaTeX   & The Name of the Game\\
\ci{LaTeXe} & \LaTeXe & The current incarnation\\
\end{tabular}

\section{Special Characters and Symbols}
 
\subsection{Quotation Marks}

You should \emph{not} use the \verb|"| for \wi{quotation marks}
\index{""@\texttt{""}} as you would on a typewriter.  In publishing
there are special opening and closing quotation marks.  In \LaTeX{},
use two~\textasciigrave~(grave accent) for opening quotation marks and
two~\textquotesingle~(vertical quote) for closing quotation marks. For single
quotes you use just one of each.
\begin{example}
``Please press the `x' key.''
\end{example}
Yes I know the rendering is not ideal, it's really a back-tick or grave accent
(\textasciigrave) for
opening quotes and vertical quote (\textquotesingle) for closing, despite what the font chosen might suggest.

\subsection{Dashes and Hyphens}

\LaTeX{} knows four kinds of \wi{dash}es. Access three of
them with different number of consecutive dashes. The fourth sign
is actually not a dash at all---it is the mathematical minus sign: \index{-}
\index{--} \index{---} \index{-@$-$} \index{mathematical!minus}

\begin{example}
daughter-in-law, X-rated\\
pages 13--67\\
yes---or no? \\
$0$, $1$ and $-1$
\end{example}
The names for these dashes are: 
`-' \wi{hyphen}, `--' \wi{en-dash}, `---' \wi{em-dash} and
`$-$' \wi{minus sign}.

\subsection{Tilde ($\sim$)}
\index{www}\index{URL}\index{tilde}
A character often seen in web addresses is the tilde. To generate
this in \LaTeX{} use \verb|\~| but the result: \~{} is not really
what you want. Try this instead:

\begin{example}
http://www.rich.edu/\~{}bush \\
http://www.clever.edu/$\sim$demo
\end{example}  
 
\subsection{Degree Symbol \texorpdfstring{($\circ$)}{}}

Printing the \wi{degree symbol} in pure \LaTeX{}.

\begin{example}
It's $-30\,^{\circ}\mathrm{C}$.
I will soon start to
super-conduct.
\end{example}

The \pai{textcomp} package makes the degree symbol also available as
\ci{textdegree} or in combination with the C by using the \ci{textcelsius}.

\begin{example}
30 \textcelsius{} is
86 \textdegree{}F.
\end{example}

\subsection{The Euro Currency Symbol \texorpdfstring{(\officialeuro)}{}}

When writing about money these days, you need the Euro symbol. Many current
fonts contain a Euro symbol. After loading the \pai{textcomp} package in the preamble of your document
\begin{lscommand}
\ci{usepackage}\verb|{textcomp}| 
\end{lscommand}
use the command
\begin{lscommand}
\ci{texteuro}
\end{lscommand}
to access it.

If your font does not provide its own Euro symbol or if you do not like the
font's Euro symbol, you have two more choices:

First the \pai{eurosym} package. It provides the official Euro symbol:
\begin{lscommand}
\ci{usepackage}\verb|[|\emph{official}\verb|]{eurosym}|
\end{lscommand}
If you prefer a Euro symbol that matches your font, use the option
\texttt{gen} in place of the \texttt{official} option.

%If the Adobe Eurofonts are installed on your system (they are available for
%free from \url{ftp://ftp.adobe.com/pub/adobe/type/win/all}) you can use
%either the package \pai{europs} and the command \ci{EUR} (for a Euro symbol
%that matches the current font).
% does not work
% or the package
% \pai{eurosans} and the command \ci{euro} (for the ``official Euro'').

%The \pai{marvosym} package also provides many different symbols, including a
%Euro, under the name \ci{EURtm}. Its disadvantage is that it does not provide
%slanted and bold variants of the Euro symbol.

\begin{table}[!htbp]
\caption{A bag full of Euro symbols} \label{eurosymb}
\begin{lined}{10cm}
\begin{tabular}{llccc}
LM+textcomp  &\verb+\texteuro+ & \huge\texteuro &\huge\sffamily\texteuro
                                                &\huge\ttfamily\texteuro\\
eurosym      &\verb+\euro+ & \huge\officialeuro &\huge\sffamily\officialeuro
                                                &\huge\ttfamily\officialeuro\\
$[$gen$]$eurosym &\verb+\euro+ & \huge\geneuro  &\huge\sffamily\geneuro
                                                &\huge\ttfamily\geneuro\\
%europs       &\verb+\EUR + & \huge\EURtm        &\huge\EURhv
%                                                &\huge\EURcr\\
%eurosans     &\verb+\euro+ & \huge\EUROSANS  &\huge\sffamily\EUROSANS
%                                             & \huge\ttfamily\EUROSANS \\
%marvosym     &\verb+\EURtm+  & \huge\mvchr101  &\huge\mvchr101
%                                               &\huge\mvchr101
\end{tabular}
\medskip
\end{lined}
\end{table}

\subsection{Ellipsis (\texorpdfstring{\ldots}{...})}

On a typewriter, a \wi{comma} or a \wi{period} takes the same amount of
space as any other letter. In book printing, these characters occupy
only a little space and are set very close to the preceding letter.
Therefore, entering `\wi{ellipsis}' by just typing three
dots would be wrong produce the wrong result. Instead, there is a special
command for these dots. It is called

\begin{lscommand}
\ci{ldots}
\end{lscommand}
\index{...@\ldots}


\begin{example}
Not like this ... but like this:\\
New York, Tokyo, Budapest, \ldots
\end{example}
 
\subsection{Ligatures}

Some letter combinations are typeset not just by setting the
different letters one after the other, but by actually using special
symbols.
\begin{code}
{\large ff fi fl ffi\ldots}\quad
instead of\quad {\large f{}f f{}i f{}l f{}f{}i \ldots}
\end{code}
These so-called \wi{ligature}s can be prohibited by inserting an \ci{mbox}\verb|{}|
between the two letters in question. This might be necessary with
words built from two words.

\begin{example}
\Large Not shelfful\\
but shelf\mbox{}ful
\end{example}
 
\subsection{Accents and Special Characters}
 
\LaTeX{} supports the use of \wi{accent}s and \wi{special character}s
from many languages. Table~\ref{accents} shows all sorts of accents
being applied to the letter o. Naturally other letters work too.

To place an accent on top of an i or a j, its dots have to be
removed. This is accomplished by typing \verb|\i| and \verb|\j|.

\begin{example}
H\^otel, na\"\i ve, \'el\`eve,\\ 
sm\o rrebr\o d, !`Se\~norita!,\\
Sch\"onbrunner Schlo\ss{} 
Stra\ss e
\end{example}

\begin{table}[!hbp]
\caption{Accents and Special Characters.} \label{accents}
\begin{lined}{10cm}
\begin{tabular}{*4{cl}}
\A{\`o} & \A{\'o} & \A{\^o} & \A{\~o} \\
\A{\=o} & \A{\.o} & \A{\"o} & \B{\c}{c}\\[6pt]
\B{\u}{o} & \B{\v}{o} & \B{\H}{o} & \B{\c}{o} \\
\B{\d}{o} & \B{\b}{o} & \B{\t}{oo} \\[6pt]
\A{\oe}  &  \A{\OE} & \A{\ae} & \A{\AE} \\
\A{\aa} &  \A{\AA} \\[6pt]
\A{\o}  & \A{\O} & \A{\l} & \A{\L} \\
\A{\i}  & \A{\j} & !` & \verb|!`| & ?` & \verb|?`| 
\end{tabular}
\index{dotless \i{} and \j}\index{Scandinavian letters}
\index{ae@\ae}\index{umlaut}\index{grave}\index{acute}
\index{oe@\oe}\index{aa@\aa}

\bigskip
\end{lined}
\end{table}

\section{International Language Support}
\index{international} When you write documents in \wi{language}s
other than English, there are three areas where \LaTeX{} has to be
configured appropriately:

\begin{enumerate}
\item All automatically generated text strings\footnote{Table of
    Contents, List of Figures, \ldots} have to be adapted to the new
  language.  For many languages, these changes can be accomplished by
  using the \pai{babel} package by Johannes Braams.
\item \LaTeX{} needs to know the hyphenation rules for the new
  language. Getting hyphenation rules into \LaTeX{} is a bit more
  tricky. It means rebuilding the format file with different
  hyphenation patterns enabled. Your \guide{} should give more
  information on this.
\item Language specific typographic rules. In French for example, there is a
  mandatory space before each colon character (:).
\end{enumerate}

If your system is already configured appropriately, activate
the \pai{babel} package by adding the command
\begin{lscommand}
\ci{usepackage}\verb|[|\emph{language}\verb|]{babel}|
\end{lscommand}
\noindent after the \verb|\documentclass| command. A list of the
\emph{language}s built into your \LaTeX{} system will be displayed
every time the compiler is started. Babel will
automatically activate the appropriate hyphenation rules for the
language you choose. If your \LaTeX{} format does not support
hyphenation in the language of your choice, babel will still work but
will disable hyphenation, which has quite a negative effect on the
appearance of the typeset document.

\textsf{Babel} also specifies new commands for some languages, which
simplify the input of special characters. The \wi{German} language, for
example, contains a lot of umlauts (\"a\"o\"u).  With \textsf{babel} loaded,
enter an \"o by typing \verb|"o| instead of~\verb|\"o|.

If you call babel with multiple languages
\begin{lscommand}
\ci{usepackage}\verb|[|\emph{languageA}\verb|,|\emph{languageB}\verb|]{babel}| 
\end{lscommand}
\noindent then the last language in the option list will be active (i.e.
languageB). Use the command
\begin{lscommand}
\ci{selectlanguage}\verb|{|\emph{languageA}\verb|}|
\end{lscommand}
\noindent to change the active language.

%Input Encoding
\newcommand{\ieih}[1]{%
\index{encodings!input!#1@\texttt{#1}}%
\index{input encodings!#1@\texttt{#1}}%
\index{#1@\texttt{#1}}}
\newcommand{\iei}[1]{%
\ieih{#1}\texttt{#1}}
%Font Encoding
\newcommand{\feih}[1]{%
\index{encodings!font!#1@\texttt{#1}}%
\index{font encodings!#1@\texttt{#1}}%
\index{#1@\texttt{#1}}}
\newcommand{\fei}[1]{%
\feih{#1}\texttt{#1}}

Most of the modern computer systems allow you to input letter of
national alphabets  directly from the keyboard. In order to 
handle variety of input encoding used for different groups of 
languages and/or on different computer platforms \LaTeX{} employs the
\pai{inputenc} package:
\begin{lscommand}
\ci{usepackage}\verb|[|\emph{encoding}\verb|]{inputenc}|
\end{lscommand}

When using this package, you should consider that other people might not
be able to display your input files on their computer, because they use
a different encoding. For example, the German umlaut \"a on OS/2 is
encoded as 132, on Unix systems using ISO-LATIN~1 it is encoded as 228,
while in Cyrillic encoding cp1251 for Windows this letter does not exist
at all; therefore you should use this feature with care. The following
encodings may come in handy, depending on the type of system you are
working on\footnote{To learn more about supported  input
encodings for Latin-based and Cyrillic-based languages, read the
documentation for \texttt{inputenc.dtx} and \texttt{cyinpenc.dtx}
respectively. Section~\ref{sec:Packages} tells how to produce package
documentation.}

\begin{center}
\begin{tabular}{l | r | r }
Operating & \multicolumn{2}{c}{encodings}\\
system  & western Latin      & Cyrillic\\
\hline
Mac     &  \iei{applemac} & \iei{macukr}  \\
Unix    &  \iei{latin1}   & \iei{koi8-ru}  \\ 
Windows &  \iei{ansinew}  & \iei{cp1251}    \\
DOS, OS/2  &  \iei{cp850} & \iei{cp866nav}
\end{tabular}                
\end{center}                 

If you have a multilingual document with conflicting input encodings,
you might want to switch to unicode, using the \pai{ucs} package.


\begin{lscommand}
\ci{usepackage}\verb|{ucs}|\\ 
\ci{usepackage}\verb|[|\iei{utf8x}\verb|]{inputenc}| 
\end{lscommand}
\noindent will enable you to create \LaTeX{} input files in 
\iei{utf8x}, a multi-byte encoding in which each character can be encoded in
as little as one byte and as many as four bytes. 

Font encoding is a different matter. It defines at which position inside
a \TeX-font each letter is stored. Multiple input encodings could be mapped into 
one font encoding, which reduces number of required font sets.
Font encodings are handled through 
\pai{fontenc} package: \label{fontenc}
\begin{lscommand}
\ci{usepackage}\verb|[|\emph{encoding}\verb|]{fontenc}| \index{font encodings}
\end{lscommand}
\noindent where \emph{encoding} is font encoding. It is possible to load several
encodings simultaneously.

The default \LaTeX{} font encoding is \label{OT1} \fei{OT1}, the encoding of the
original Computer Modern \TeX{} font. It contains only the 128
characters of the 7-bit ASCII character set. When accented characters
are required, \TeX{} creates them by combining a normal character with
an accent. While the resulting output looks perfect, this approach stops
the automatic hyphenation from working inside words containing accented
characters. Besides, some of Latin letters could not be created by
combining a normal character with an accent, to say nothing about letters of
non-Latin alphabets, such as Greek or Cyrillic.

To overcome these shortcomings, several 8-bit CM-like font sets were created.
\emph{Extended Cork} (EC) fonts in \fei{T1} encoding contains 
letters and punctuation characters for most of the European
languages based on Latin script. The LH font set contains letters necessary
to typeset documents in languages using Cyrillic script. Because of the large
number of Cyrillic glyphs, they are arranged into four font
encodings---\fei{T2A}, \fei{T2B}, \fei{T2C},
and~\fei{X2}.\footnote{The list of languages supported by each of these
encodings could be found in \cite{cyrguide}.} The CB bundle contains fonts
in \fei{LGR} encoding for the composition of Greek text.

Improve/enable hyphenation in non-English
documents by using these fonts. Another advantage of using new CM-like fonts is that they 
provide fonts of CM families in all weights, shapes, and optically
scaled font sizes. 

\subsection{Support for Portuguese}

\secby{Demerson Andre Polli}{polli@linux.ime.usp.br}
To enable hyphenation and change all automatic text to \wi{Portuguese},
\index{Portugu\^es} use the command:
\begin{lscommand}
\verb|\usepackage[portuguese]{babel}|
\end{lscommand}
Or if you are in Brazil, substitute the language for \texttt{\wi{brazilian}}.

As there are a lot of accents in Portuguese you might want to use
\begin{lscommand}
\verb|\usepackage[latin1]{inputenc}|
\end{lscommand}
to be able to input them correctly as well as 
\begin{lscommand}
\verb|\usepackage[T1]{fontenc}|
\end{lscommand}
to get the hyphenation right.

See table~\ref{portuguese} for the preamble you need to write in the
Portuguese language. Note that we are using the latin1 input encoding here,
so this will not work on a Mac or on DOS. Just use
the appropriate encoding for your system.

\begin{table}[btp]
\caption{Preamble for Portuguese documents.} \label{portuguese}
\begin{lined}{5cm}
\begin{verbatim}
\usepackage[portuguese]{babel}
\usepackage[latin1]{inputenc}
\usepackage[T1]{fontenc}
\end{verbatim}
\bigskip
\end{lined}
\end{table}


\subsection{Support for French}

\secby{Daniel Flipo}{daniel.flipo@univ-lille1.fr}
Some hints for those creating \wi{French} documents with \LaTeX{}:
load French language support with the following command:

\begin{lscommand}
\verb|\usepackage[francais]{babel}|
\end{lscommand}

This enables French hyphenation, if you have configured your
\LaTeX{} system accordingly. It also changes all automatic text into
French: \verb+\chapter+ prints Chapitre, \verb+\today+ prints the current
date in French and so on. A set of new commands also
becomes available, which allows you to write French input files more
easily. Check out table \ref{cmd-french} for inspiration. 

\begin{table}[!htbp]
\caption{Special commands for French.} \label{cmd-french}
\begin{lined}{9cm}
\selectlanguage{french}
\begin{tabular}{ll}
\verb+\og guillemets \fg{}+         \quad &\og guillemets \fg \\[1ex]
\verb+M\up{me}, D\up{r}+            \quad &M\up{me}, D\up{r}  \\[1ex]
\verb+1\ier{}, 1\iere{}, 1\ieres{}+ \quad &1\ier{}, 1\iere{}, 1\ieres{}\\[1ex]
\verb+2\ieme{} 4\iemes{}+           \quad &2\ieme{} 4\iemes{}\\[1ex]
\verb+\No 1, \no 2+                 \quad &\No 1, \no 2   \\[1ex]
\verb+20~\degres C, 45\degres+      \quad &20~\degres C, 45\degres \\[1ex]
\verb+\bsc{M. Durand}+              \quad &\bsc{M.~Durand} \\[1ex]
\verb+\nombre{1234,56789}+          \quad &\nombre{1234,56789}
\end{tabular}
\selectlanguage{english}
\bigskip
\end{lined}
\end{table}

You will also notice that the layout of lists changes when switching to the
French language. For more information on what the \texttt{francais}
option of \textsf{babel} does and how to customize its behaviour, run
\LaTeX{} on file \texttt{frenchb.dtx} and read the produced file
\texttt{frenchb.dvi}.

Recent versions of \pai{frenchb} rely on \pai{numprint} to implement the \ci{nombre} command.

\subsection{Support for German}

Some hints for those creating \wi{German}\index{Deutsch}
documents with \LaTeX{}: load German language support with the following
command:

\begin{lscommand}
\verb|\usepackage[german]{babel}|
\end{lscommand}

This enables German hyphenation, if you have configured your
\LaTeX{} system accordingly. It also changes all automatic text into
German. Eg. ``Chapter'' becomes ``Kapitel.'' A set of new commands also
becomes available, which allows you to write German input files more quickly
even when you don't use the inputenc package. Check out table
\ref{german} for inspiration. With inputenc, all this becomes moot, but your 
text also is locked in a particular encoding world.

\begin{table}[!htbp]
\caption{German Special Characters.} \label{german}
\begin{lined}{8cm}
\selectlanguage{german}
\begin{tabular}{*2{ll}}
\verb|"a| & "a \hspace*{1ex} & \verb|"s| & "s \\[1ex]
\verb|"`| & "` & \verb|"'| & "' \\[1ex]
\verb|"<| or \ci{flqq} & "<  & \verb|">| or \ci{frqq} & "> \\[1ex]
\ci{flq} & \flq & \ci{frq} & \frq \\[1ex]
\ci{dq} & " \\
\end{tabular}
\selectlanguage{english}
\bigskip
\end{lined}
\end{table}

In German books you often find French quotation marks (\flqq guil\-le\-mets\frqq).
German typesetters, however, use them differently. A quote in a German book
would look like \frqq this\flqq. In the German speaking part of Switzerland,
typesetters use \flqq guillemets\frqq~the same way the French do.

A major problem arises from the use of commands
like \verb+\flq+: If you use the OT1 font (which is the default font) the  
guillemets will look like the math symbol ``$\ll$'', which turns a typesetter's stomach.
T1 encoded fonts, on the other hand, do contain the required symbols. So if you are using this type
of quote, make sure you use the T1 encoding. (\verb|\usepackage[T1]{fontenc}|)

\subsection[Support for Korean]{Support for Korean\footnotemark}\label{support_korean}%
\footnotetext{%
Considering a number of issues  Korean \LaTeX{} users
have to cope with.
This section was written by Karnes KIM on behalf of the
Korean lshort translation team. It  was translated into English
by SHIN Jungshik and shortened by Tobi Oetiker.}

To use \LaTeX{} for typesetting  \wi{Korean}, 
we need to solve three problems: 

\begin{enumerate}
\item 
We must be able to 
edit \wi{Korean input files}.
Korean input files must be in plain text format, but because Korean
uses its own character set outside the
repertoire of US-ASCII, they will look rather strange with a normal ASCII editor.  The two most widely used encodings for
Korean text files are  EUC-KR and its upward compatible
extension used in Korean MS-Windows, CP949/Windows-949/UHC.
In these encodings each US-ASCII character represents its normal ASCII
character similar to other ASCII compatible encodings such as
ISO-8859-\textit{x}, EUC-JP, Big5, or Shift\_JIS. On the other hand, Hangul
syllables, Hanjas (Chinese characters as used in Korea), Hangul Jamos,
Hiraganas, Katakanas, Greek and Cyrillic characters and other
symbols and letters drawn from KS~X~1001 are represented by two
consecutive octets. The first has its MSB set.
Until the mid-1990's, it took a considerable amount of time and effort to
set up a Korean-capable environment under a non-localized (non-Korean)
operating system. 
Skim through the now much-outdated \url{http://jshin.net/faq} to get 
a glimpse of what it was like to use Korean under non-Korean OS in mid-1990's.
These days all three major operating systems (Mac OS, Unix, Windows) come equipped
with pretty decent multilingual support and internationalization features
so that editing Korean text file is not so much of a problem anymore, even
on non-Korean operating systems.

\item \TeX{} and \LaTeX{} were originally written for
scripts with no more than 256 characters in their alphabet.
To make them work for languages with considerably 
more characters such as
Korean%,
 \footnote{Korean Hangul is an alphabetic script with 14 basic consonants
 and 10 basic vowels (Jamos). Unlike Latin or Cyrillic scripts, the
 individual characters have to be arranged in rectangular
 clusters about the same size as Chinese characters. Each cluster
 represents a syllable. An unlimited number of syllables can be
 formed out of this finite set of vowels and consonants. Modern Korean
 orthographic standards (both in South Korea and  North Korea), however,
 put some restriction on the formation of these clusters.
 Therefore only a finite number of  orthographically correct syllables exist.
 The Korean Character encoding defines individual code points for each of these syllables (KS~X~1001:1998 and KS~X~1002:1992). So Hangul, albeit alphabetic, is
 treated like the Chinese and Japanese writing systems with tens of thousands of
 ideographic/logographic characters.  ISO~10646/Unicode offers both ways of
 representing Hangul used for \emph{modern} Korean by encoding Conjoining
 Hangul Jamos (alphabets: \url{http://www.unicode.org/charts/PDF/U1100.pdf})
 in addition to encoding all the orthographically allowed Hangul syllables in
 \emph{modern} Korean (\url{http://www.unicode.org/charts/PDF/UAC00.pdf}).
 One of the most daunting challenges in Korean typesetting with
 \LaTeX{} and related typesetting system is supporting Middle Korean---and possibly future Korean---syllables that can be only represented
 by conjoining Jamos in Unicode. It is hoped that future \TeX{} engines like $\Omega$ and
 $\Lambda$ will eventually provide solutions to this
 so that some Korean linguists and historians
 will defect from MS Word that already has  a pretty good support 
 for Middle Korean.}
or Chinese, a subfont mechanism was developed.
It divides a single CJK font with  thousands or tens of thousands of
glyphs into a set of subfonts with 256 glyphs each. 
For Korean, there are three widely used packages;  \wi{H\LaTeX}
by UN~Koaunghi, \wi{h\LaTeX{}p} by CHA~Jaechoon and the \wi{CJK package}
by Werner~Lemberg.\footnote{% 
They can be obtained at \CTANref|language/korean/HLaTeX/|\\
   \CTANref|language/korean/CJK/| and
   \texttt{http://knot.kaist.ac.kr/htex/}}
H\LaTeX{} and h\LaTeX{}p are specific to Korean and provide
Korean localization on top of the font support.
They both can process Korean input text files encoded in EUC-KR. H\LaTeX{} can
even process input files encoded in CP949/Windows-949/UHC and UTF-8
when used along with $\Lambda$, $\Omega$.

The CJK package is not specific to Korean. It can
process input files in UTF-8 as well as in various CJK encodings
including EUC-KR and CP949/Windows-949/UHC, it can be used to typeset documents with
multilingual content (especially Chinese, Japanese and Korean).
The CJK package has no Korean localization such as the one offered by H\LaTeX{} and it
does not come with as many special Korean fonts as H\LaTeX.

\item The ultimate purpose of using typesetting programs like \TeX{}
and \LaTeX{} is to get documents typeset in an `aesthetically' satisfying way.
Arguably the most important element in typesetting is  a set of
well-designed fonts. The H\LaTeX{} distribution
includes \index{Korean font!UHC font}UHC \PSi{} fonts 
of 10
different families and
Munhwabu\footnote{Korean Ministry of Culture.}
fonts (TrueType) of 5 different families.
The CJK package works with a set of fonts used by earlier versions
of H\LaTeX{} and it can use Bitstream's cyberbit TrueType
font.
\end{enumerate}

To use the  H\LaTeX{} package for typesetting your Korean text, put the following
declaration into the preamble of your document:
\begin{lscommand}
\verb+\usepackage{hangul}+
\end{lscommand}

This command turns the Korean localization on. The headings
of chapters, sections, subsections, table of content and table of
figures are all translated into Korean and the formatting of the document
is changed to follow Korean conventions. 
The package also provides automatic ``particle selection.''
In Korean, there are pairs of post-fix particles 
grammatically equivalent but different in form. Which 
of any given pair is correct depends on 
whether the preceding syllable ends with a  vowel or a consonant.
(It is a bit more complex than this, but this should give you
a good picture.)
Native Korean speakers have no problem picking the right particle, but
it cannot be determined which particle to use for references and other automatic
text that will change while you edit the document.
It 
takes a painstaking effort to place appropriate particles manually
every time you add/remove references or simply shuffle  parts
of your document around.
H\LaTeX{} relieves its users from this boring and error-prone process.

In case you don't need Korean localization features
but just want 
to  typeset Korean text, put the following line in the 
preamble, instead.
\begin{lscommand}
\verb+\usepackage{hfont}+
\end{lscommand}

For more details on typesetting  Korean with H\LaTeX{}, refer to
the \emph{H\LaTeX{} Guide}.  Check out the web site of the Korean
\TeX{} User Group (KTUG) at  \url{http://www.ktug.or.kr/}.
There is also a Korean translation
of this manual available.

\subsection{Writing in Greek}
\secby{Nikolaos Pothitos}{pothitos@di.uoa.gr}
See table~\ref{preamble-greek} for the preamble you need to write in the
\wi{Greek} \index{Greek} language.  This preamble enables hyphenation and
changes all automatic text to Greek.\footnote{If you select the
\texttt{utf8x}
option for the package \texttt{inputenc}, \LaTeX{} will understand Greek and polytonic
Greek
unicode characters.}

\begin{table}[btp]
\caption{Preamble for Greek documents.} \label{preamble-greek}
\begin{lined}{7cm}
\begin{verbatim}
\usepackage[english,greek]{babel}
\usepackage[iso-8859-7]{inputenc}
\end{verbatim}
\bigskip
\end{lined}
\end{table}

A set of new commands also becomes available, which allows you to write
Greek input files more easily.  In order to temporarily switch to English
and vice versa, one can use the commands \verb|\textlatin{|\emph{english
text}\verb|}| and \verb|\textgreek{|\emph{greek text}\verb|}| that both take
one argument which is then typeset using the requested font encoding. 
Otherwise use the command \verb|\selectlanguage{...}| described in a
previous section.  Check out table~\ref{sym-greek} for some Greek
punctuation characters.  Use \verb|\euro| for the Euro symbol.

\begin{table}[!htbp]
\caption{Greek Special Characters.} \label{sym-greek}
\begin{lined}{4cm}
\selectlanguage{french}
\begin{tabular}{*2{ll}}
\verb|;| \hspace*{1ex}  &  $\cdot$ \hspace*{1ex}  &  \verb|?| \hspace*{1ex}&  ;   \\[1ex]
\verb|((|               &  \og                    &  \verb|))|&  \fg \\[1ex]
\verb|``|               &  `                      &  \verb|''| &  '   \\
\end{tabular}
\selectlanguage{english}
\bigskip
\end{lined}
\end{table}


\subsection{Support for Cyrillic}

\secby{Maksym Polyakov}{polyama@myrealbox.com}
Version~3.7h of \pai{babel} includes support for the
\fei{T2*}~encodings and for typesetting Bulgarian, Russian and
Ukrainian texts using Cyrillic letters.  

Support for Cyrillic is based on standard \LaTeX{} mechanisms plus
the \pai{fontenc} and \pai{inputenc} packages. But, if you are going to
use Cyrillics in math mode, you need to load \pai{mathtext} package
before \pai{fontenc}:\footnote{If you use \AmS-\LaTeX{} packages, 
load them before \pai{fontenc} and \pai{babel} as well.}
\begin{lscommand}
\verb+\usepackage{mathtext}+\\
\verb+\usepackage[+\fei{T1}\verb+,+\fei{T2A}\verb+]{fontenc}+\\
\verb+\usepackage[+\iei{koi8-ru}\verb+]{inputenc}+\\
\verb+\usepackage[english,bulgarian,russian,ukranian]{babel}+
\end{lscommand}

Generally, \pai{babel} will authomatically choose the default font encoding,
for the above three languages this is \fei{T2A}.  However, documents are not
restricted to a single font encoding. For multi-lingual documents using
Cyrillic and Latin-based languages it makes sense to include Latin font
encoding explicitly. \pai{babel} will take care of switching to the appropriate
font encoding when a different language is selected within the document.

In addition to enabling hyphenations, translating automatically
generated text strings, and activating some language specific 
typographic rules (like \ci{frenchspacing}), \pai{babel} provides some 
commands allowing typesetting according to the standards of 
Bulgarian, Russian, or Ukrainian languages. 


For all three languages, language specific punctuation is provided:
The Cyrillic dash for the text (it is little narrower than Latin dash and
surrounded by tiny spaces), a dash for direct speech, quotes, and
commands to facilitate hyphenation, see Table~\ref{Cyrillic}.

% Table borrowed from Ukrainian.dtx
\begin{table}[htb]
  \begin{center}
  \index{""-@\texttt{""}\texttt{-}} 
  \index{""---@\texttt{""}\texttt{-}\texttt{-}\texttt{-}} 
  \index{""=@\texttt{""}\texttt{=}} 
  \index{""`@\texttt{""}\texttt{`}} 
  \index{""'@\texttt{""}\texttt{'}} 
  \index{"">@\texttt{""}\texttt{>}} 
  \index{""<@\texttt{""}\texttt{<}} 
  \caption[Bulgarian, Russian, and Ukrainian]{The extra definitions made
           by Bulgarian, Russian, and Ukrainian options of \pai{babel}}\label{Cyrillic}
  \begin{tabular}{@{}p{.1\hsize}@{}p{.9\hsize}@{}}
   \hline
   \verb="|= & disable ligature at this position.               \\
   \verb|"-| & an explicit hyphen sign, allowing hyphenation
               in the rest of the word.                         \\
   \verb|"---| & Cyrillic emdash in plain text.                      \\
   \verb|"--~| & Cyrillic emdash in compound names (surnames).       \\
   \verb|"--*| & Cyrillic emdash for denoting direct speech.         \\
   \verb|""| & like \verb|"-|, but producing no hyphen sign
               (for compound words with hyphen, e.g.\verb|x-""y|
               or some other signs  as ``disable/enable'').     \\
   \verb|"~| & for a compound word mark without a breakpoint.        \\
   \verb|"=| & for a compound word mark with a breakpoint, allowing
          hyphenation in the composing words.                   \\
   \verb|",| & thinspace for initials with a breakpoint
           in following surname.                                \\
   \verb|"`| & for German left double quotes
               (looks like ,\kern-0.08em,).                     \\
   \verb|"'| & for German right double quotes (looks like ``).       \\%''
   \verb|"<| & for French left double quotes (looks like $<\!\!<$).  \\
   \verb|">| & for French right double quotes (looks like $>\!\!>$). \\
   \hline
  \end{tabular}
  \end{center}
\end{table}


The Russian and Ukrainian options of \pai{babel} define the commands \ci{Asbuk}
and \ci{asbuk}, which act like \ci{Alph} and \ci{alph}, but produce capital
and small letters of Russian or Ukrainian alphabets (whichever is the
active language of the document). The Bulgarian option of \pai{babel} 
provides the commands \ci{enumBul} and \ci{enumLat} (\ci{enumEng}), which
make \ci{Alph} and \ci{alph} produce letters of either
Bulgarian or Latin (English) alphabets. The default behaviour of
\ci{Alph}  and \ci{alph} for the Bulgarian language option is to
produce letters from the Bulgarian alphabet. 

%Finally, math alphabets are redefined and  as well as the commands for math
%operators according to Cyrillic typesetting traditions. 

\subsection{Support for Mongolian}

To use \LaTeX{} for typesetting Mongolian you have a choice between two packages:
Multilingual Babel and Mon\TeX{} by Oliver Corff.

Mon\TeX{} includes support for both Cyrillic and traditional
Mongolian Script. In order to access the commands of Mon\TeX{}, add:
\begin{lscommand}
\ci{usepackage}\verb|[|\emph{language},\emph{encoding}\verb|]{mls}|
\end{lscommand}
\noindent to the preamble. Choose the \emph{language} option \pai{xalx} to generate
captions and the dates in Modern Mongolian. To write a complete document in the traditional Mongolian script
you have to choose \pai{bicig} for the \emph{language} option. The document
language option \pai{bicig} enables the ``Simplified Transliteration'' input method.

Enable and disable Latin Transliteration Mode with
\begin{lscommand}
\verb|\SetDocumentEncodingLMC|
\end{lscommand}
and
\begin{lscommand}
\verb|\SetDocumentEncodingNeutral|
\end{lscommand}

More information about Mon\TeX{} is available from
\CTAN|language/mongolian/montex/doc|.

Mongolian Cyrillic script is supported by \pai{babel}. Activate
Mongolian language support with the following commands:

\begin{lscommand}
\verb|\usepackage[T2A]{fontenc}|\\
\verb|\usepackage[mn]{inputenc}|\\
\verb|\usepackage[mongolian]{babel}|
\end{lscommand}

\noindent where \iei{mn} is the \iei{cp1251} input encoding. For a more modern approach
invoke \iei{utf8} instead.

\section{The Space Between Words}

To get a straight right margin in the output, \LaTeX{} inserts varying
amounts of space between the words. It inserts slightly more space at
the end of a sentence, as this makes the text more readable.  \LaTeX{}
assumes that sentences end with periods, question marks or exclamation
marks. If a period follows an uppercase letter, this is not taken as a
sentence ending, since periods after uppercase letters normally occur in
abbreviations.

Any exception from these assumptions has to be specified by the
author. A backslash in front of a space generates a space that will
not be enlarged. A tilde~`\verb|~|' character generates a space that cannot be
enlarged and additionally prohibits a line break. The command
\verb|\@| in front of a period specifies that this period terminates a
sentence even when it follows an uppercase letter.
\cih{"@} \index{~@ \verb.~.} \index{tilde@tilde ( \verb.~.)}
\index{., space after}

\begin{example}
Mr.~Smith was happy to see her\\
cf.~Fig.~5\\
I like BASIC\@. What about you?
\end{example}

The additional space after periods can be disabled with the command
\begin{lscommand}
\ci{frenchspacing}
\end{lscommand}
\noindent which tells \LaTeX{} \emph{not} to insert more space after a
period than after ordinary character. This is very common in
non-English languages, except bibliographies. If you use
\ci{frenchspacing}, the command \verb|\@| is not necessary.

\section{Titles, Chapters, and Sections}

To help the reader find his or her way through your work, you should
divide it into chapters, sections, and subsections.  \LaTeX{} supports
this with special commands that take the section title as their
argument.  It is up to you to use them in the correct order.

The following sectioning commands are available for the
\texttt{article} class: \nopagebreak

\begin{lscommand}
\ci{section}\verb|{...}|\\
\ci{subsection}\verb|{...}|\\
\ci{subsubsection}\verb|{...}|\\
\ci{paragraph}\verb|{...}|\\
\ci{subparagraph}\verb|{...}|
\end{lscommand}

If you want to split your document in parts without influencing the
section or chapter numbering use
\begin{lscommand}
\ci{part}\verb|{...}|
\end{lscommand}

When you work with the \texttt{report} or \texttt{book} class,
an additional top-level sectioning command becomes available
\begin{lscommand}
\ci{chapter}\verb|{...}|
\end{lscommand}

As the \texttt{article} class does not know about chapters, it is quite easy
to add articles as chapters to a book.
The spacing between sections, the numbering and the font size of the
titles will be set automatically by \LaTeX. 

Two of the sectioning commands are a bit special: 
\begin{itemize}
\item The \ci{part} command does
  not influence the numbering sequence of chapters.  
\item The \ci{appendix} command does not take an argument. It just
  changes the chapter numbering to letters.\footnote{For the article
    style it changes the section numbering.}
\end{itemize}



\LaTeX{} creates a table of contents by taking the section headings
and page numbers from the last compile cycle of the document. The command 
\begin{lscommand} 
\ci{tableofcontents}
\end{lscommand} 
\noindent expands to a table of contents at the place it
is issued. A new
document has to be compiled (``\LaTeX ed'') twice to get a
correct \wi{table of contents}. Sometimes it might be
necessary to compile the document a third time. \LaTeX{} will tell you
when this is necessary.

All sectioning commands listed above also exist as ``starred''
versions.  A ``starred'' version of a command is built by adding a
star \verb|*| after the command name.  This generates section headings
that do not show up in the table of contents and are not
numbered. The command \verb|\section{Help}|, for example, would become
\verb|\section*{Help}|.

Normally the section headings show up in the table of contents exactly
as they are entered in the text. Sometimes this is not possible,
because the heading is too long to fit into the table of contents. The
entry for the table of contents can then be specified as an
optional argument in front of the actual heading.

\begin{code}
\verb|\chapter[Title for the table of contents]{A long|\\
\verb|    and especially boring title, shown in the text}|
\end{code} 

The \wi{title} of the whole document is generated by issuing a 
\begin{lscommand}
\ci{maketitle}
\end{lscommand}
\noindent command. The contents of the title have to be defined by the commands
\begin{lscommand}
\ci{title}\verb|{...}|, \ci{author}\verb|{...}| 
and optionally \ci{date}\verb|{...}| 
\end{lscommand}
\noindent before calling \verb|\maketitle|. In the argument to \ci{author}, you can supply several names separated by \ci{and} commands. 

An example of some of the commands mentioned above can be found in
Figure~\ref{document} on page~\pageref{document}.

Apart from the sectioning commands explained above, \LaTeXe{}
introduced three additional commands for use with the \verb|book| class. 
They are useful for dividing your publication. The commands alter
chapter headings and page numbering to work as you would expect it in
a book:
\begin{description}
\item[\ci{frontmatter}] should be the very first command after
  the start of the document body (\verb|\begin{document}|). It will switch page numbering to Roman
    numerals and sections be non-enumerated. As if you were using 
    the starred sectioning commands (eg \verb|\chapter*{Preface}|)
    but the sections will still show up in the table of contents.
\item[\ci{mainmatter}] comes right before the first chapter of
  the book. It turns on Arabic page numbering and restarts the page
  counter.
\item[\ci{appendix}] marks the start of additional material in your
  book. After this command chapters will be numbered with letters.
\item[\ci{backmatter}] should be inserted before the very last items
  in your book, such as the bibliography and the index. In the standard
  document classes, this has no visual effect.
\end{description}


\section{Cross References}

In books, reports and articles, there are often 
\wi{cross-references} to figures, tables and special segments of text.
\LaTeX{} provides the following commands for cross referencing
\begin{lscommand}
\ci{label}\verb|{|\emph{marker}\verb|}|, \ci{ref}\verb|{|\emph{marker}\verb|}| 
and \ci{pageref}\verb|{|\emph{marker}\verb|}|
\end{lscommand}
\noindent where \emph{marker} is an identifier chosen by the user. \LaTeX{}
replaces \verb|\ref| by the number of the section, subsection, figure,
table, or theorem after which the corresponding \verb|\label| command
was issued. \verb|\pageref| prints the page number of the
page where the \verb|\label| command occurred.\footnote{Note that these commands
  are not aware of what they refer to. \ci{label} just saves the last
  automatically generated number.} As with the section titles, the
numbers from the previous run are used.

\begin{example}
A reference to this subsection
\label{sec:this} looks like:
``see section~\ref{sec:this} on 
page~\pageref{sec:this}.''
\end{example}
 
\section{Footnotes}
With the command
\begin{lscommand}
\ci{footnote}\verb|{|\emph{footnote text}\verb|}|
\end{lscommand}
\noindent a footnote is printed at the foot of the current page.  Footnotes
should always be put\footnote{``put'' is one of the most common
  English words.} after the word or sentence they refer to. Footnotes
referring to a sentence or part of it should therefore be put after
the comma or period.\footnote{Note that footnotes
  distract the reader from the main body of your document. After all,
  everybody reads the footnotes---we are a curious species, so why not
  just integrate everything you want to say into the body of the
  document?\footnotemark}
\footnotetext{A guidepost doesn't necessarily go where it's pointing to :-).}

\begin{example}
Footnotes\footnote{This is 
  a footnote.} are often used 
by people using \LaTeX.
\end{example}
 
\section{Emphasized Words}

If a text is typed using a typewriter, important words are
  \texttt{emphasized by \underline{underlining} them.}
\begin{lscommand}
\ci{underline}\verb|{|\emph{text}\verb|}|
\end{lscommand}
In printed books,
however, words are emphasized by typesetting them in an \emph{italic}
font.  \LaTeX{} provides the command
\begin{lscommand}
\ci{emph}\verb|{|\emph{text}\verb|}|
\end{lscommand}
\noindent to emphasize text.  What the command actually does with 
its argument depends on the context:

\begin{example}
\emph{If you use 
  emphasizing inside a piece
  of emphasized text, then 
  \LaTeX{} uses the
  \emph{normal} font for 
  emphasizing.}
\end{example}

Please note the difference between telling \LaTeX{} to
\emph{emphasize} something and telling it to use a different
\emph{font}:

\begin{example}
\textit{You can also
  \emph{emphasize} text if 
  it is set in italics,} 
\textsf{in a 
  \emph{sans-serif} font,}
\texttt{or in 
  \emph{typewriter} style.}
\end{example}

\section{Environments} \label{env}

% To typeset special purpose text, \LaTeX{} defines many different
% \wi{environment}s for all sorts of formatting:
\begin{lscommand}
\ci{begin}\verb|{|\emph{environment}\verb|}|\quad
   \emph{text}\quad
\ci{end}\verb|{|\emph{environment}\verb|}|
\end{lscommand}
\noindent Where \emph{environment} is the name of the environment. Environments can be
nested within each other as long as the correct nesting order is
maintained.
\begin{code}
\verb|\begin{aaa}...\begin{bbb}...\end{bbb}...\end{aaa}|
\end{code}

\noindent In the following sections all important environments are explained.

\subsection{Itemize, Enumerate, and Description}

The \ei{itemize} environment is suitable for simple lists, the
\ei{enumerate} environment for enumerated lists, and the
\ei{description} environment for descriptions.
\cih{item}

\begin{example}
\flushleft
\begin{enumerate}
\item You can mix the list
environments to your taste:
\begin{itemize}
\item But it might start to
look silly. 
\item[-] With a dash.
\end{itemize}
\item Therefore remember:
\begin{description}
\item[Stupid] things will not
become smart because they are
in a list.
\item[Smart] things, though,
can be presented beautifully
in a list.
\end{description}
\end{enumerate}
\end{example}
 
\subsection{Flushleft, Flushright, and Center}

The environments \ei{flushleft} and \ei{flushright} generate
paragraphs that are either left- or \wi{right-aligned}. \index{left
  aligned} The \ei{center} environment generates centred text. If you
do not issue \ci{\bs} to specify line breaks, \LaTeX{} will
automatically determine line breaks.

\begin{example}
\begin{flushleft}
This text is\\ left-aligned. 
\LaTeX{} is not trying to make 
each line the same length.
\end{flushleft}
\end{example}

\begin{example}
\begin{flushright}
This text is right-\\aligned. 
\LaTeX{} is not trying to make
each line the same length.
\end{flushright}
\end{example}

\begin{example}
\begin{center}
At the centre\\of the earth
\end{center}
\end{example}

\subsection{Quote, Quotation, and Verse}

The \ei{quote} environment is useful for quotes, important phrases and
examples.

\begin{example}
A typographical rule of thumb
for the line length is:
\begin{quote}
On average, no line should
be longer than 66 characters.
\end{quote}
This is why \LaTeX{} pages have 
such large borders by default
and also why multicolumn print
is used in newspapers.
\end{example}

There are two similar environments: the \ei{quotation} and the
\ei{verse} environments. The \texttt{quotation} environment is useful
for longer quotes going over several paragraphs, because it indents the
first line of each paragraph. The \texttt{verse} environment is useful for poems
where the line breaks are important. The lines are separated by
issuing a \ci{\bs} at the end of a line and an empty line after each
verse.


\begin{example}
I know only one English poem by 
heart. It is about Humpty Dumpty.
\begin{flushleft}
\begin{verse}
Humpty Dumpty sat on a wall:\\
Humpty Dumpty had a great fall.\\ 
All the King's horses and all
the King's men\\
Couldn't put Humpty together
again.
\end{verse}
\end{flushleft}
\end{example}

\subsection{Abstract}

In scientific publications it is customary to start with an abstract which
gives the reader a quick overview of what to expect. \LaTeX{} provides the
\ei{abstract} environment for this purpose. Normally \ei{abstract} is used
in documents typeset with the article document class.

\newenvironment{abstract}%
        {\begin{center}\begin{small}\begin{minipage}{0.8\textwidth}}%
        {\end{minipage}\end{small}\end{center}}
\begin{example}
\begin{abstract}
The abstract abstract.
\end{abstract}
\end{example}

\subsection{Printing Verbatim}

Text that is enclosed between \verb|\begin{|\ei{verbatim}\verb|}| and
\verb|\end{verbatim}| will be directly printed, as if typed on a
typewriter, with all line breaks and spaces, without any \LaTeX{}
command being executed.

Within a paragraph, similar behavior can be accessed with
\begin{lscommand}
\ci{verb}\verb|+|\emph{text}\verb|+|
\end{lscommand}
\noindent The \verb|+| is just an example of a delimiter character. Use any
character except letters, \verb|*| or space. Many \LaTeX{} examples in this
booklet are typeset with this command.

\begin{example}
The \verb|\ldots| command \ldots

\begin{verbatim}
10 PRINT "HELLO WORLD ";
20 GOTO 10
\end{verbatim}
\end{example}

\begin{example}
\begin{verbatim*}
the starred version of
the      verbatim   
environment emphasizes
the spaces   in the text
\end{verbatim*}
\end{example}

The \ci{verb} command can be used in a similar fashion with a star:

\begin{example}
\verb*|like   this :-) |
\end{example}

The \texttt{verbatim} environment and the \verb|\verb| command may not be used
within parameters of other commands.

 
\subsection{Tabular}

\newcommand{\mfr}[1]{\framebox{\rule{0pt}{0.7em}\texttt{#1}}}

The \ei{tabular} environment can be used to typeset beautiful
\wi{table}s with optional horizontal and vertical lines. \LaTeX{}
determines the width of the columns automatically.

The \emph{table spec} argument of the 
\begin{lscommand}
\verb|\begin{tabular}[|\emph{pos}\verb|]{|\emph{table spec}\verb|}|
\end{lscommand} 
\noindent command defines the format of the table. Use an \mfr{l} for a column of
left-aligned text, \mfr{r} for right-aligned text, and \mfr{c} for
centred text; \mfr{p\{\emph{width}\}} for a column containing justified
text with line breaks, and \mfr{|} for a vertical line.

If the text in a column is too wide for the page, \LaTeX{} won't
automatically wrap it. Using \mfr{p\{\emph{width}\}} you can define
a special type of column which will wrap-around the text as in a normal paragraph.

The \emph{pos} argument specifies the vertical position of the table
relative to the baseline of the surrounding text.  Use either of the
letters \mfr{t}, \mfr{b} and \mfr{c} to specify table
alignment at the top, bottom or center.
 
Within a \texttt{tabular} environment, \texttt{\&} jumps to the next
column, \ci{\bs} starts a new line and \ci{hline} inserts a horizontal
line.  Add partial lines by using the \ci{cline}\texttt{\{}\emph{j}\texttt{-}\emph{i}\texttt{\}},
where j and i are the column numbers the line should extend over.

\index{"|@ \verb."|.}

\begin{example}
\begin{tabular}{|r|l|}
\hline
7C0 & hexadecimal \\
3700 & octal \\ \cline{2-2}
11111000000 & binary \\
\hline \hline
1984 & decimal \\
\hline
\end{tabular}
\end{example}

\begin{example}
\begin{tabular}{|p{4.7cm}|}
\hline
Welcome to Boxy's paragraph.
We sincerely hope you'll 
all enjoy the show.\\
\hline 
\end{tabular}
\end{example}

The column separator can be specified with the \mfr{@\{...\}}
construct. This command kills the inter-column space and replaces it
with whatever is between the curly braces.  One common use for
this command is explained below in the decimal alignment problem.
Another possible application is to suppress leading space in a table with
\mfr{@\{\}}.

\begin{example}
\begin{tabular}{@{} l @{}}
\hline 
no leading space\\
\hline
\end{tabular}
\end{example}

\begin{example}
\begin{tabular}{l}
\hline
leading space left and right\\
\hline
\end{tabular}
\end{example}

%
% This part by Mike Ressler
%

\index{decimal alignment} Since there is no built-in way to align
numeric columns to a decimal point,\footnote{If the `tools' bundle is
  installed on your system, have a look at the \pai{dcolumn} package.}
we can ``cheat'' and do it by using two columns: a right-aligned
integer and a left-aligned fraction. The \verb|@{.}| command in the
\verb|\begin{tabular}| line replaces the normal inter-column spacing with
just a ``.'', giving the appearance of a single,
decimal-point-justified column.  Don't forget to replace the decimal
point in your numbers with a column separator (\verb|&|)! A column label
can be placed above our numeric ``column'' by using the
\ci{multicolumn} command.
 
\begin{example}
\begin{tabular}{c r @{.} l}
Pi expression       &
\multicolumn{2}{c}{Value} \\
\hline
$\pi$               & 3&1416  \\
$\pi^{\pi}$         & 36&46   \\
$(\pi^{\pi})^{\pi}$ & 80662&7 \\
\end{tabular}
\end{example}

\begin{example}
\begin{tabular}{|c|c|}
\hline
\multicolumn{2}{|c|}{Ene} \\
\hline
Mene & Muh! \\
\hline
\end{tabular}
\end{example}

Material typeset with the tabular environment always stays together on one
page. If you want to typeset long tables, you might want to use the
\pai{longtable} environments.

Sometimes the default \LaTeX{} tables to feel a bit cramped. So you may want
to give them a bit more breathing space by setting a higher
\ci{arraystretch} and \ci{tabcolsep} value.

\begin{example}
\begin{tabular}{|l|}
\hline
These lines\\\hline
are tight\\\hline
\end{tabular}

{\renewcommand{\arraystretch}{1.5}
\renewcommand{\tabcolsep}{0.2cm}
\begin{tabular}{|l|}
\hline
less cramped\\\hline
table layout\\\hline
\end{tabular}}

\end{example}

If you just want to grow the height of a single row in your table add an invisible vertical bar\footnote{In professional typesetting,
this is called a \wi{strut}.} Use a zero width \ci{rule} to implement this trick.

\begin{example}
\begin{tabular}{|c|}
\hline
\rule{1pt}{4ex}Pitprop \ldots\\
\hline
\rule{0pt}{4ex}Strut\\
\hline
\end{tabular}
\end{example}

\section{Floating Bodies}
Today most publications contain a lot of figures and tables. These
elements need special treatment, because they cannot be broken across
pages.  One method would be to start a new page every time a figure or
a table is too large to fit on the present page. This approach would
leave pages partially empty, which looks very bad.

The solution to this problem is to `float' any figure or table that
does not fit on the current page to a later page, while filling the
current page with body text. \LaTeX{} offers two environments for
\wi{floating bodies}; one for tables and  one for figures.  To
take full advantage of these two environments it is important to
understand approximately how \LaTeX{} handles floats internally.
Otherwise floats may become a major source of frustration, because
\LaTeX{} never puts them where you want them to be.

\bigskip
Let's first have a look at the commands \LaTeX{} supplies
for floats:

Any material enclosed in a \ei{figure} or \ei{table} environment will
be treated as floating matter. Both float environments support an optional
parameter
\begin{lscommand}
\verb|\begin{figure}[|\emph{placement specifier}\verb|]| or
\verb|\begin{table}[|\ldots\verb|]|
\end{lscommand}
\noindent called the \emph{placement specifier}. This parameter
is used to tell \LaTeX{} about the locations to which the float
is allowed to be moved.  A \emph{placement specifier} is constructed by building a string
of \emph{float-placing permissions}. See Table~\ref{tab:permiss}.

\begin{table}[!bp]
\caption{Float Placing Permissions.}\label{tab:permiss}
\noindent \begin{minipage}{\textwidth}
\medskip
\begin{center}
\begin{tabular}{@{}cp{8cm}@{}}
Spec&Permission to place the float \ldots\\
\hline
\rule{0pt}{1.05em}\texttt{h} & \emph{here} at the very place in the text
  where it occurred.  This is useful mainly for small floats.\\[0.3ex]
\texttt{t} & at the \emph{top} of a page\\[0.3ex]
\texttt{b} & at the \emph{bottom} of a page\\[0.3ex]
\texttt{p} & on a special \emph{page} containing only floats.\\[0.3ex]
\texttt{!} & without considering most of the  internal parameters\footnote{Such as the
    maximum number of floats allowed  on one page.}, which could stop this
  float from being placed.
\end{tabular}
\end{center}
Note that \texttt{pt} and \texttt{em} are \TeX{} units. Read more
on this in table \ref{units} on page \pageref{units}.
\end{minipage}
\end{table}

A table could be started with the following line e.g.{}
\begin{code}
\verb|\begin{table}[!hbp]|
\end{code}
\noindent The \wi{placement specifier} \verb|[!hbp]| allows \LaTeX{} to 
place the table right here (\texttt{h}) or at the bottom (\texttt{b}) 
of some page
or on a special floats page (\texttt{p}), and all this even if it does not
look that good (\texttt{!}). If no placement specifier is given, the standard
classes assume \verb|[tbp]|.

\LaTeX{} will place every float it encounters according to the
placement specifier supplied by the author. If a float cannot be
placed on the current page it is deferred either to the
\emph{figures} or the \emph{tables} queue.\footnote{These are FIFO---`first in first out'---queues!}  When a new page is started,
\LaTeX{} first checks if it is possible to fill a special `float'
page with floats from the queues. If this is not possible, the first
float on each queue is treated as if it had just occurred in the
text: \LaTeX{} tries again to place it according to its
respective placement specifiers (except `h,' which is no longer
possible).  Any new floats occurring in the text get placed into the
appropriate queues. \LaTeX{} strictly maintains the original order of
appearance for each type of float. That's why a figure that cannot
be placed pushes all further figures to the end of the document.
Therefore:

\begin{quote}
If \LaTeX{} is not placing the floats as you expected,
it is often only one float jamming one of the two float queues.
\end{quote}                 

While it is possible to give \LaTeX{}  single-location placement
specifiers, this causes problems.  If the float does not fit in the
location specified it becomes stuck, blocking subsequent floats.
In particular, you should never, ever use the [h] option---it is so bad
that in more recent versions of \LaTeX, it is automatically replaced by
[ht].

\bigskip
\noindent Having explained the difficult bit, there are some more things to
mention about the \ei{table} and \ei{figure} environments.
Use the 

\begin{lscommand}
\ci{caption}\verb|{|\emph{caption text}\verb|}|
\end{lscommand}

\noindent command to define a caption for the float. A running number and
the string ``Figure'' or ``Table'' will be added by \LaTeX.

The two commands

\begin{lscommand}
\ci{listoffigures} and \ci{listoftables} 
\end{lscommand}

\noindent operate analogously to the \verb|\tableofcontents| command,
printing a list of figures or tables, respectively.  These lists will
display the whole caption, so if you tend to use long captions
you must have a shorter version of the caption for the lists.
This is accomplished by entering the short version in brackets after
the \verb|\caption| command.
\begin{code}
\verb|\caption[Short]{LLLLLoooooonnnnnggggg}| 
\end{code}

Use \ci{label} and \ci{ref},to create a reference to a float within
your text. Note that the \ci{label} command must come \emph{after} the
\ci{caption} command since you want it to reference the number of the
caption.

The following example draws a square and inserts it into the
document. You could use this if you wanted to reserve space for images
you are going to paste into the finished document.

\begin{code}
\begin{verbatim}
Figure~\ref{white} is an example of Pop-Art.
\begin{figure}[!hbtp]
\makebox[\textwidth]{\framebox[5cm]{\rule{0pt}{5cm}}}
\caption{Five by Five in Centimetres.\label{white}}
\end{figure}
\end{verbatim}
\end{code}

\noindent In the example above, 
\LaTeX{} will try \emph{really hard}~(\texttt{!})\ to place the figure
right \emph{here}~(\texttt{h}).\footnote{assuming the figure queue is
  empty.} If this is not possible, it tries to place the figure at the
\emph{bottom}~(\texttt{b}) of the page.  Failing to place the figure
on the current page, it determines whether it is possible to create a float
page containing this figure and maybe some tables from the tables
queue. If there is not enough material for a special float page,
\LaTeX{} starts a new page, and once more treats the figure as if it
had just occurred in the text.

Under certain circumstances it might be necessary to use the 

\begin{lscommand}
\ci{clearpage} or even the \ci{cleardoublepage} 
\end{lscommand}

\noindent command. It orders \LaTeX{} to immediately place all 
floats remaining in the queues and then start a new
page. \ci{cleardoublepage} even goes to a new right-hand page.

You will learn how to include \PSi{}
drawings into your \LaTeXe{} documents later in this introduction.

\section{Protecting Fragile Commands}

Text given as arguments of commands like \ci{caption} or \ci{section} may
show up more than once in the document (e.g. in the table of contents as
well as in the body of the document). Some commands will break when used in
the argument of \ci{section}-like commands. Compilation of your document
will fail. These commands are called \wi{fragile commands}---for example,
\ci{footnote} or \ci{phantom}. These fragile commands need protection (don't
we all?). Protect them by putting the \ci{protect} command in front
of them.

\ci{protect} only refers to the command that follows right behind, not even
to its arguments. In most cases a superfluous \ci{protect} won't hurt.

\begin{code}
\verb|\section{I am considerate|\\
\verb|      \protect\footnote{and protect my footnotes}}|
\end{code}

% Local Variables:
% TeX-master: "lshort2e"
% mode: latex
% mode: flyspell
% End:

%%%%%%%%%%%%%%%%%%%%%%%%%%%%%%%%%%%%%%%%%%%%%%%%%%%%%%%%%%%%%%%%
% Contents: Math typesetting with LaTeX
% $Id$
%
% Changes by Stefan M. Moser: 2008/10/22
%
% -Section 2: "Single Equations": added comment about preference of
%  equation* over \[
% -Replaced (almost) all examples with \[ by equation*
% -New section 4: "Single Equations that are Too Long: multline"
% -New section 5: "Multiple Equations"
% -Section 6: "Arrays and Matrices": made a full section and added
%  some material
% -Section 9: "Theorems, Lemmas, ...": added a subsection about proofs
%  with new material
%
% Other Changes:
% -in lshort.sty: 
%    *example environment adapted: changed in three places
%     \textwidth by \linewidth. This is necessary for
%     example-environment within a itemize-list.
%    *added \RequirePackage[retainorgcmds]{IEEEtrantools}
%
% THINGS TO DO:
% -adapt typesetting of new sections to rest of lshort, including all
%  the usual commands used so far. In particular, I guess we have to
%  get rid of the \verb-commands everywhere
% -include index-commands
%%%%%%%%%%%%%%%%%%%%%%%%%%%%%%%%%%%%%%%%%%%%%%%%%%%%%%%%%%%%%%%%%
 
\chapter{Typesetting Mathematical Formulae}

\begin{intro}
  Now you are ready! In this chapter, we will attack the main strength
  of \TeX{}: mathematical typesetting. But be warned, this chapter
  only scratches the surface. While the things explained here are
  sufficient for many people, don't despair if you can't find a
  solution to your mathematical typesetting needs here. It is highly likely
  that your problem is addressed in \AmS-\LaTeX{}.
\end{intro}
  


\section{The \texorpdfstring{\AmS}{AMS}-\LaTeX{} bundle}

If you want to typeset (advanced) \wi{mathematics}, you should
use \AmS-\LaTeX{}. The \AmS-\LaTeX{} bundle is a collection of packages and classes for
mathematical typesetting. We will mostly deal with the \pai{amsmath} package
which is a part of the bundle. \AmS-\LaTeX{} is produced by The \emph{\wi{American Mathematical Society}} 
and it is used extensively for mathematical typesetting. \LaTeX{} itself does provide
some basic features and environments for mathematics, but they are limited (or
maybe it's the other way around: \AmS-\LaTeX{} is \emph{unlimited}!) and
in some cases inconsistent. 

\AmS-\LaTeX{} is a part of the required distribution and is provided
with all recent \LaTeX{} distributions.\footnote{If yours is missing it, go to
  \CTAN|macros/latex/required/amslatex|.} In this chapter, we assume
  \pai{amsmath} is loaded in the preamble; \verb|\usepackage{amsmath}|.

\section{Single Equations}
  
There are two ways to typeset mathematical \wi{formulae}: in-line within a paragraph
(\emph{\wi{text style}}), or the paragraph can be broken to 
typeset it separately (\textit{\wi{display style}}). Mathematical \wi{equation}s 
\emph{within} a paragraph are entered \index{$@\texttt{\$}} %$
between \texttt{\$} and \texttt{\$}:
\begin{example}
Add $a$ squared and $b$ squared
to get $c$ squared. Or, using 
a more mathematical approach:
$a^2 + b^2 = c^2$
\end{example}
\begin{example}
\TeX{} is pronounced as 
$\tau\epsilon\chi$\\[5pt]
100~m$^{3}$ of water\\[5pt]
This comes from my $\heartsuit$
\end{example}

If you want your larger equations to be set apart
from the rest of the paragraph, it is preferable to \emph{display} them
rather than to break the paragraph apart.
To do this, you enclose them between \verb|\begin{|\ei{equation}\verb|}| and
\verb|\end{equation}|.\footnote{This is an \textsf{amsmath} command. If you don't
have access to the package for some obscure reason, you can use \LaTeX's own
\ei{displaymath} environment instead.} You can then \ci{label} an equation number and refer to
it somewhere else in the text by using the \ci{eqref} command. If you want to
name the equation something specific, you \ci{tag} it instead.
\begin{example}
Add $a$ squared and $b$ squared
to get $c$ squared. Or, using
a more mathematical approach
 \begin{equation}
   a^2 + b^2 = c^2
 \end{equation}
Einstein says
 \begin{equation}
   E = mc^2 \label{clever}
 \end{equation}
He didn't say
 \begin{equation}
  1 + 1 = 3 \tag{dumb}
 \end{equation}
This is a reference to 
\eqref{clever}. 
\end{example}

If you don't want \LaTeX{} to number the equations, use the starred
version of \texttt{equation} using an asterisk, \ei{equation*}, or even easier, enclose the
equation in \ci{[} and \ci{]}:\footnote{\index{equation!\textsf{amsmath}}
  \index{equation!\LaTeX{}}This is again from \textsf{amsmath}. Standard \LaTeX{}'s has only the \texttt{equation} environment without the star.}
\begin{example}
Add $a$ squared and $b$ squared
to get $c$ squared. Or, using
a more mathematical approach
 \begin{equation*}
   a^2 + b^2 = c^2
 \end{equation*}
or you can type less for the
same effect:
 \[ a^2 + b^2 = c^2 \]
\end{example}
While \ci{[} is short and sweet, it does not allow to switch between numbered and not numbered style as easily as
nicely \ei{equation} and \ei{equation*}.

Note the difference in typesetting style between \wi{text style} and \wi{display style}
equations: 
\begin{example}
This is text style: 
$\lim_{n \to \infty} 
 \sum_{k=1}^n \frac{1}{k^2} 
 = \frac{\pi^2}{6}$.
And this is display style:
 \begin{equation}
  \lim_{n \to \infty} 
  \sum_{k=1}^n \frac{1}{k^2} 
  = \frac{\pi^2}{6}
 \end{equation}
\end{example}

In text style, enclose tall or deep math expressions or sub
expressions in \ci{smash}. This makes \LaTeX{} ignore the height of
these expressions. This keeps the line spacing even.

\begin{example}
A $d_{e_{e_p}}$ mathematical
expression  followed by a
$h^{i^{g^h}}$ expression. As
opposed to a smashed 
\smash{$d_{e_{e_p}}$} expression 
followed by a
\smash{$h^{i^{g^h}}$} expression.
\end{example}

\subsection{Math Mode}

There are also differences between \emph{\wi{math mode}} and \emph{text mode}. For
example, in \emph{math mode}: 

\begin{enumerate}

\item \index{spacing!math mode} Most spaces and line breaks do not have any significance, as all spaces
are either derived logically from the mathematical expressions, or
have to be specified with special commands such as \ci{,}, \ci{quad} or
\ci{qquad} (we'll get back to that later, see section~\ref{sec:math-spacing}).
 
\item Empty lines are not allowed. Only one paragraph per formula.

\item Each letter is considered to be the name of a variable and will be
typeset as such. If you want to typeset normal text within a formula
(normal upright font and normal spacing) then you have to enter the
text using the \verb|\text{...}| command (see also section \ref{sec:fontsz} on
page \pageref{sec:fontsz}).

\end{enumerate}
\begin{example}
$\forall x \in \mathbf{R}:
 \qquad x^{2} \geq 0$
\end{example}
\begin{example}
$x^{2} \geq 0\qquad
 \text{for all }x\in\mathbf{R}$
\end{example}
 
Mathematicians can be very fussy about which symbols are used:
it would be conventional here to use the `\wi{blackboard bold}' font,
\index{bold symbols} which is obtained using \ci{mathbb} from the
package \pai{amssymb}.\footnote{\pai{amssymb} is not a part
  of the \AmS-\LaTeX{} bundle, but it is perhaps still a part of your \LaTeX{}
  distribution. Check your distribution
  or go to \texttt{CTAN:/fonts/amsfonts/latex/} to obtain it.}
\ifx\mathbb\undefined\else
The last example becomes
\begin{example}
$x^{2} \geq 0\qquad
 \text{for all } x 
 \in \mathbb{R}$
\end{example}
\fi
See Table~\ref{mathalpha} on page~\pageref{mathalpha} and
Table~\ref{mathfonts} on page~\pageref{mathfonts} for more math fonts.



\section{Building Blocks of a Mathematical Formula}

In this section, we describe the most important commands used in mathematical
typesetting. Most of the commands in this section will not require
\textsf{amsmath} (if they do, it will be stated clearly), but load it anyway.


\textbf{Lowercase \wi{Greek letters}} are entered as \verb|\alpha|,
 \verb|\beta|, \verb|\gamma|, \ldots, uppercase letters
are entered as \verb|\Gamma|, \verb|\Delta|, \ldots\footnote{There is no
  uppercase Alpha, Beta etc. defined in \LaTeXe{} because it looks the same as a 
  normal roman A, B\ldots{} Once the new math coding is done, things will
  change.} 

Take a look at Table~\ref{greekletters} on page~\pageref{greekletters} for a
list of Greek letters.
\begin{example}
$\lambda,\xi,\pi,\theta,
 \mu,\Phi,\Omega,\Delta$
\end{example}


\textbf{Exponents and Subscripts} can be specified using\index{exponent}\index{subscript}
the \verb|^|\index{^@\verb"|^"|} and the \verb|_|\index{_@\verb"|_"|} character.
Most math mode commands act only on the next character, so if you
want a command to affect several characters, you have to group them
together using curly braces: \verb|{...}|.

Table~\ref{binaryrel} on page \pageref{binaryrel} lists a lot of other binary
relations like $\subseteq$ and $\perp$.

\begin{example}
$p^3_{ij} \qquad
 m_\text{Knuth} \\[5pt]
 a^x+y \neq a^{x+y}\qquad 
 e^{x^2} \neq {e^x}^2$
\end{example}


The \textbf{\wi{square root}} is entered as \ci{sqrt}; the
$n^\text{th}$ root is generated with \verb|\sqrt[|$n$\verb|]|. The size of
the root sign is determined automatically by \LaTeX. If just the sign
is needed, use \verb|\surd|.

See other kinds of arrows like $\hookrightarrow$ and $\rightleftharpoons$ on
Table~\ref{tab:arrows} on page \pageref{tab:arrows}. 
\begin{example}
$\sqrt{x} \Leftrightarrow x^{1/2}
 \quad \sqrt[3]{2}
 \quad \sqrt{x^{2} + \sqrt{y}}
 \quad \surd[x^2 + y^2]$
\end{example}


\index{dots!three}
\index{vertical!dots}
\index{horizontal!dots}
Usually you don't typeset an explicit \textbf{\wi{dot}} sign to indicate
the multiplication operation when handling symbols; however sometimes it is written
to help the reader's eyes in grouping a formula.
You should use \ci{cdot} which typesets a single dot centered. \ci{cdots} is
three centered \textbf{\wi{dots}} while \ci{ldots} sets the dots on the
baseline. Besides that, there are \ci{vdots} for 
vertical and \ci{ddots} for \wi{diagonal dots}. You can find another example in
section~\ref{sec:arraymat}.
\begin{example}
$\Psi = v_1 \cdot v_2
 \cdot \ldots \qquad 
 n! = 1 \cdot 2 
 \cdots (n-1) \cdot n$
\end{example}

The commands \ci{overline} and \ci{underline} create
\textbf{horizontal lines} directly over or under an expression:
\index{horizontal!line} \index{line!horizontal}
\begin{example}
$0.\overline{3} = 
 \underline{\underline{1/3}}$
\end{example}

The commands \ci{overbrace} and \ci{underbrace} create
long \textbf{horizontal braces} over or under an expression:
\index{horizontal!brace} \index{brace!horizontal} 
\begin{example}
$\underbrace{\overbrace{a+b+c}^6 
 \cdot \overbrace{d+e+f}^9}
 _\text{meaning of life} = 42$
\end{example}

\index{mathematical!accents} To add mathematical accents such as \textbf{small
arrows} or \textbf{\wi{tilde}} signs to variables, the commands
given in Table~\ref{mathacc} on page~\pageref{mathacc} might be useful.  Wide hats and
tildes covering several characters are generated with \ci{widetilde}
and \ci{widehat}. Notice the difference between \ci{hat} and \ci{widehat} and the placement of
\ci{bar} for a variable with subscript. The \wi{apostrophe} mark
\verb|'|\index{'@\verb"|'"|} gives a \wi{prime}:
% a dash is --
\begin{example}
$f(x) = x^2 \qquad f'(x) 
 = 2x \qquad f''(x) = 2\\[5pt]
 \hat{XY} \quad \widehat{XY}
 \quad \bar{x_0} \quad \bar{x}_0$
\end{example}


\textbf{Vectors}\index{vectors} are often specified by adding small
\wi{arrow symbols} on top of a variable. This is done with the
\ci{vec} command. The two commands \ci{overrightarrow} and
\ci{overleftarrow} are useful to denote the vector from $A$ to $B$:
\begin{example}
$\vec{a} \qquad
 \vec{AB} \qquad
 \overrightarrow{AB}$
\end{example}


Names of log-like functions are often typeset in an upright
font, and not in italics as variables are, so \LaTeX{} supplies the
following commands to typeset the most important function names:
\index{mathematical!functions}

\begin{tabular}{llllll}
\ci{arccos} &  \ci{cos}  &  \ci{csc} &  \ci{exp} &  \ci{ker}    & \ci{limsup} \\
\ci{arcsin} &  \ci{cosh} &  \ci{deg} &  \ci{gcd} &  \ci{lg}     & \ci{ln}     \\
\ci{arctan} &  \ci{cot}  &  \ci{det} &  \ci{hom} &  \ci{lim}    & \ci{log}    \\
\ci{arg}    &  \ci{coth} &  \ci{dim} &  \ci{inf} &  \ci{liminf} & \ci{max}    \\
\ci{sinh}   & \ci{sup}   &  \ci{tan}  & \ci{tanh}&  \ci{min}    & \ci{Pr}     \\
\ci{sec}    & \ci{sin} \\
\end{tabular}

\begin{example}
\begin{equation*}
  \lim_{x \rightarrow 0}
  \frac{\sin x}{x}=1
\end{equation*}
\end{example}

For functions missing from the list, use the \ci{DeclareMathOperator}
command. There is even a starred version for functions with limits.
This command works only in the preamble so the commented lines in the
example below must be put into the preamble.

\begin{example}
%\DeclareMathOperator{\argh}{argh}
%\DeclareMathOperator*{\nut}{Nut}
\begin{equation*}
  3\argh = 2\nut_{x=1}    
\end{equation*}
\end{example}

For the \wi{modulo function}, there are two commands: \ci{bmod} for the
binary operator ``$a \bmod b$'' and \ci{pmod}
for expressions
such as ``$x\equiv a \pmod{b}$:''
\begin{example}
$a\bmod b \\
 x\equiv a \pmod{b}$
\end{example}

A built-up \textbf{\wi{fraction}} is typeset with the
\ci{frac}\verb|{...}{...}| command. In in-line equations, the fraction is shrunk to
fit the line. This style is obtainable in display style with \ci{tfrac}. The
reverse, i.e.\ display style fraction in text, is made with \ci{dfrac}.
Often the slashed form $1/2$ is preferable, because it looks better
for small amounts of `fraction material:'
\begin{example}
In display style:
\begin{equation*}
  3/8 \qquad \frac{3}{8} 
  \qquad \tfrac{3}{8}
\end{equation*}
\end{example}

\begin{example}
In text style:
$1\frac{1}{2}$~hours \qquad
$1\dfrac{1}{2}$~hours
\end{example}
 
Here the \ci{partial} command for \wi{partial derivative}s is used:
\begin{example}
\begin{equation*} 
  \sqrt{\frac{x^2}{k+1}}\qquad
  x^\frac{2}{k+1}\qquad
  \frac{\partial^2f}
  {\partial x^2} 
\end{equation*}
\end{example}

To typeset \wi{binomial coefficient}s or similar structures, use
the command \ci{binom} from \pai{amsmath}:
\begin{example}
Pascal's rule is
\begin{equation*}
 \binom{n}{k} =\binom{n-1}{k}
 + \binom{n-1}{k-1}
\end{equation*}
\end{example}

For \wi{binary relations} it may be useful to stack symbols over each other.
\ci{stackrel}\verb|{#1}{#2}| puts the symbol given
in \verb|#1| in superscript-like size over \verb|#2| which
is set in its usual position.
\begin{example}
\begin{equation*}
 f_n(x) \stackrel{*}{\approx} 1
\end{equation*}
\end{example}

The \textbf{\wi{integral operator}} is generated with \ci{int}, the
\textbf{\wi{sum operator}} with \ci{sum}, and the \textbf{\wi{product operator}}
with \ci{prod}. The upper and lower limits are specified with~\verb|^|
and~\verb|_| like subscripts and superscripts:
\begin{example}
\begin{equation*}
\sum_{i=1}^n \qquad
\int_0^{\frac{\pi}{2}} \qquad
\prod_\epsilon
\end{equation*}
\end{example}

To get more control over the placement of indices in complex
expressions, \pai{amsmath} provides the \ci{substack} command:
\begin{example}
\begin{equation*}
\sum^n_{\substack{0<i<n \\ 
        j\subseteq i}}
   P(i,j) = Q(i,j)
\end{equation*}
\end{example}



\LaTeX{} provides all sorts of symbols for \textbf{\wi{braces}} and other
\textbf{\wi{delimiters}} (e.g.~$[\;\langle\;\|\;\updownarrow$).
Round and square braces can be entered with the corresponding keys and
curly braces with \verb|\{|, but all other delimiters are generated with
special commands (e.g.~\verb|\updownarrow|).
\begin{example}
\begin{equation*}
{a,b,c} \neq \{a,b,c\}
\end{equation*}
\end{example}

If you put \ci{left} in front of an opening delimiter and
\ci{right} in front of a closing delimiter, \LaTeX{} will automatically
determine the correct size of the delimiter. Note that you must close
every \ci{left} with a corresponding \ci{right}. If you
don't want anything on the right, use the invisible ``\ci{right.}'':
\begin{example}
\begin{equation*}
1 + \left(\frac{1}{1-x^{2}}
    \right)^3 \qquad 
\left. \ddagger \frac{~}{~}\right)
\end{equation*}
\end{example}

In some cases it is necessary to specify the correct size of a
mathematical delimiter\index{mathematical!delimiter} by hand,
which can be done using the commands \ci{big}, \ci{Big}, \ci{bigg} and
\ci{Bigg} as prefixes to most delimiter commands:
\begin{example}
$\Big((x+1)(x-1)\Big)^{2}$\\
$\big( \Big( \bigg( \Bigg( \quad
\big\} \Big\} \bigg\} \Bigg\} \quad
\big\| \Big\| \bigg\| \Bigg\| \quad
\big\Downarrow \Big\Downarrow 
\bigg\Downarrow \Bigg\Downarrow$
\end{example}
 For a list of all delimiters available, see Table~\ref{tab:delimiters} on page
\pageref{tab:delimiters}. 


\section{Single Equations that are Too Long: multline}
\index{long equations}
\label{sec:multline}

If an equation is too long, we have to wrap it somehow. Unfortunately,
wrapped equations are usually less easy to read than not wrapped
ones. To improve the readability, there are certain rules on how to do
the wrapping:
\begin{enumerate}
\item In general one should always wrap an equation \textbf{before} an
  equality sign or an operator.
\item A wrap before an equality sign is preferable to a wrap before
  any operator.
\item A wrap before a plus- or minus-operator is preferable to a wrap
  before a multiplication-operator.
\item Any other type of wrap should be avoided if ever possible.
\end{enumerate}
The easiest way to achieve such a wrapping is the use of the
\ei{multline} en\-vi\-ron\-ment:\footnote{The
  \texttt{multline}-environment is from \texttt{amsmath}.}
\begin{example}
\begin{multline}
  a + b + c + d + e + f 
  + g + h + i  
  \\
  = j + k + l + m + n 
\end{multline}
\end{example}
\noindent
The difference to the \ei{equation} environment is that an arbitrary
line-break (or also multiple line-breaks) can be introduced. This is
done by putting a \verb+\\+ on those places where the equation needs
to be wrapped. Similarly to \ei{equation*} there also exists a
\ei{multline*} version for preventing an equation number.

However, in spite of its ease in use, often the
\ei{IEEEeqnarray} environment (see section~\ref{sec:IEEEeqnarray})
will yield better results.  Particularly, consider the following
common situation:
\begin{example}
\begin{equation}
  a = b + c + d + e + f 
  + g + h + i + j 
  + k + l + m + n + o + p  
  \label{eq:equation_too_long}
\end{equation}
\end{example}
\noindent
Here it is actually the RHS that is too long to fit on one line. The
\ei{multline} environment will now yield the following:
\begin{example}
\begin{multline}
  a = b + c + d + e + f 
  + g + h + i + j \\
  + k + l + m + n + o + p
\end{multline}
\end{example}

This is of course much better than \eqref{eq:equation_too_long}, but
it has the disadvantage that the equality sign loses its natural
stronger importance with respect to the plus operator in front of
$k$. The better solution is provided by the
\ei{IEEEeqnarray} environment that will be discussed in detail in
Section~\ref{sec:IEEEeqnarray}: 
\begin{example}
\begin{IEEEeqnarray}{rCl}
  a & = & b + c + d + e + f 
  + g + h + i + j \nonumber\\
  && +\: k + l + m + n + o + p 
  \label{eq:dont_use_multline}
\end{IEEEeqnarray}
\end{example}
In this case the second line is vertically aligned to the first line:
the $+$ in front of $k$ is exactly below $b$, \emph{i.e.}, the RHS is
clearly visible as contrast to the LHS of the equation.




\section{Multiple Equations}
\index{equation!multiple}
\label{sec:IEEEeqnarray}

In the most general situation we have a sequence of several
equalities that do not fit onto one line. Here we need to work with
vertical alignment in order to keep the array of equations in a nice
and readable structure.

Before we offer our suggestions on how to do this, we start with a few
bad examples that show the biggest drawbacks of some common solutions.


\subsection{Problems with Traditional Commands}
\label{sec:problems_traditional}

To group multiple equations the
\ei{align} environment\footnote{The \texttt{align}-environment can
  also be used to group several blocks of equations beside each other.
  However, for this rather rare situation we also recommend to use the
  \ei{IEEEeqnarray} environment with an argument like,
  \texttt{\{rCl+rCl\}}.} could be used:
\begin{example}
\begin{align}
  a & = b + c \\
  & = d + e
\end{align}
\end{example}

However, this approach does not work once a single line is too long:
\begin{example}
\begin{align}
  a & = b + c \\
  & = d + e + f + g + h + i 
  + j + k + l \nonumber \\
  & + m + n + o \\
  & = p + q + r + s
\end{align}
\end{example}
\noindent
Here $+\:m$ should be below $d$ and not below the equality sign. Of
course, one could add some space (\verb+\hspace{...}+),
but this will never yield a precise arrangement (and is bad
style\ldots).

A better solution is offered by the \ei{eqnarray} environment:
\begin{example}
\begin{eqnarray}
  a & = & b + c \\
  & = & d + e + f + g + h + i 
  + j + k + l \nonumber \\
  && +\: m + n + o \\
  & = & p + q + r + s
\end{eqnarray}
\end{example}

The \ei{eqnarray} environment, however, has a few very severe
disadvantages:
\begin{itemize}
\item The spaces around the equality signs are too big.
  Particularly, they are \textbf{not} the same as in the
  \ei{multline} and \ei{equation} environments:
\begin{example}
\begin{eqnarray}
  a & = & a = a
\end{eqnarray}
\end{example}

\item The expression sometimes overlaps with the equation number even
  though there would be enough room on the left:
\begin{example}
\begin{eqnarray}
  a & = & b + c 
  \\
  & = & d + e + f + g + h^2 
  + i^2 + j 
  \label{eq:faultyeqnarray}
\end{eqnarray}
\end{example}

\item The environment offers a command \ci{lefteqn} that can
  be used when the LHS is too long:
\begin{example}
\begin{eqnarray}
  \lefteqn{a + b + c + d 
    + e + f + g + h}\nonumber\\
  & = & i + j + k + l + m 
  \\
  & = & n + o + p + q + r + s
\end{eqnarray}
\end{example}
Unfortunately, this command is faulty: if the RHS is too short, the array is
not properly centered:
\begin{example}
\begin{eqnarray}
  \lefteqn{a + b + c + d 
    + e + f + g + h} 
  \nonumber \\
  & = & i + j 
\end{eqnarray}
\end{example}
Moreover, it is very complicated to change the vertical alignment of
the equality sign on the second line.
\end{itemize}

Fortunately there is a better way\ldots.


\subsection{IEEEeqnarray-Environment}
\label{sec:IEEEeqnarray_intro}

The \ei{IEEEeqnarray} environment is a very powerful command with
many options. Here, we will only introduce its basic
functionalities. For more information we refer to the
manual.\footnote{The official manual is called
  \texttt{IEEEtran\_HOWTO.pdf}. The part about \texttt{IEEEeqnarray}
  can be found in Appendix~F.}

First of all, in order to be able to use the
\ei{IEEEeqnarray} environment one needs to load the
package\footnote{This package can be found on CTAN.}
\pai{IEEEtrantools}. Include the following line in the header of
your document: \small
\begin{verbatim}
\usepackage[retainorgcmds]{IEEEtrantools}
\end{verbatim}
\normalsize

The strength of \ei{IEEEeqnarray} is the possibility of specifying
the number of \emph{columns} in the equation array. Usually, this
specification will be \verb+{rCl}+, \emph{i.e.}, three columns, the
first column right-justified, the middle one centered with a little
more space around it (therefore we specify capital \texttt{C} instead of
lower-case \texttt{c}) and the third column left-justified:
\begin{example}
\begin{IEEEeqnarray}{rCl}
  a & = & b + c 
  \\
  & = & d + e + f + g + h 
  + i + j + k \nonumber\\
  && +\: l + m + n + o 
  \\
  & = & p + q + r + s
\end{IEEEeqnarray}
\end{example}

However, any number of columns can be specified:
\verb+{c}+ will give only one column with all entries centered, or
\verb+{rCll}+ would add a fourth, left-justified column to use
for comments. Moreover, beside \texttt{l}, \texttt{c}, \texttt{r}, \texttt{L},
\texttt{C}, \texttt{R} for math mode entries there are also \texttt{s},
\texttt{t}, \texttt{u} for left, centered, and right text mode entries.
Additional space can be added with \texttt{.} and
\texttt{/} and \texttt{?} in increasing order.\footnote{For more spacing
  types refer to Section~\ref{sec:putting-qed-right}.}

In contrast to \texttt{eqnarry} the spaces around the equality
signs are correct!

\subsection{Common Usage}
\label{sec:common-usage}

In the following we will describe how we use \texttt{IEEEeqnarray} to
solve the most common situations.
\begin{itemize}
\item If a line overlaps with the equation number as in
  \eqref{eq:faultyeqnarray}, the command 
\small
\begin{verbatim}
\IEEEeqnarraynumspace
\end{verbatim} 
\normalsize
  can be used: it has to be added in the corresponding line and makes
  sure that the whole equation array is shifted by the size of the
  equation numbers (the shift depends on the size of the number!):
  instead of
\begin{example}
\begin{IEEEeqnarray}{rCl}
  a & = & b + c 
  \\
  & = & d + e + f + g + h 
  + i + j + k 
  \\
  & = & l + m + n
\end{IEEEeqnarray}
\end{example}
  we get
\begin{example}
\begin{IEEEeqnarray}{rCl}
  a & = & b + c 
  \\
  & = & d + e + f + g + h 
  + i + j + k 
  \IEEEeqnarraynumspace\\
  & = & l + m + n.
\end{IEEEeqnarray}
\end{example}

\item If the LHS is too long, as a replacement for the faulty
  \ci{lefteqn} command, \texttt{IEEEeqnarray} offers the
  \ci{IEEEeqnarraymulticol} command which works in all situations:
\begin{example}
\begin{IEEEeqnarray}{rCl}
  \IEEEeqnarraymulticol{3}{l}{
    a + b + c + d + e + f 
    + g + h
  }\nonumber\\ \quad
  & = & i + j 
  \\
  & = & k + l + m
\end{IEEEeqnarray}
\end{example}
The usage is identical to the \ci{multicolumns} command in the
\texttt{tabular}-en\-vi\-ron\-ment. The first argument \verb+{3}+
specifies that three columns shall be combined to one which will be
left-justified \verb+{l}+.

Note that by adapting the \ci{quad} command one can easily adapt
the depth of the equation signs,\footnote{I think that one quad is the
  distance that looks good for most cases.} \emph{e.g.},
\begin{example}
\begin{IEEEeqnarray}{rCl}
  \IEEEeqnarraymulticol{3}{l}{
    a + b + c + d + e + f 
    + g + h
  }\nonumber\\ \qquad\qquad
  & = & i + j
  \\
  & = & k + l + m
\end{IEEEeqnarray}
\end{example}

\item If an equation is split into two or more lines, \LaTeX\
  interprets the first $+$ or $-$ as sign instead of operator.
  Therefore, it is necessary to add an additional space \ci{:}
  between the operator and the term: instead of
\begin{example}
\begin{IEEEeqnarray}{rCl}
  a & = & b + c 
  \\
  & = & d + e + f + g + h 
  + i + j + k \nonumber\\
  && + l + m + n + o 
  \\
  & = & p + q + r + s
\end{IEEEeqnarray}
\end{example}
  we should write
\begin{example}
\begin{IEEEeqnarray}{rCl}
  a & = & b + c 
  \\
  & = & d + e + f + g + h 
  + i + j + k \nonumber\\
  && +\: l + m + n + o 
  \\
  & = & p + q + r + s
\end{IEEEeqnarray}
\end{example}
  (Compare the space between $+$ and $l$!)
  
  \textbf{Attention:} \LaTeX\ is not completely silly: in certain
  situations like, \emph{e.g.}, in front of
  \begin{itemize}
  \item an operator name like \ci{log}, \ci{sin}, \ci{det},
    \ci{max}, \emph{etc.},
  \item an integral \ci{int} or sum \ci{sum},
  \item a bracket with adaptive size using \ci{left} and
      \ci{right} (this is in contrast to normal brackets or
    brackets with fixed size like \ci{big(} ),
  \end{itemize}
  a $+$ or $-$ cannot be a sign, but must be an operator. In those
  situations \LaTeX\ will add the correct spacing and no additional
  space is needed.
  \begin{itemize}
  \item[$\rhd$] \it Whenever you wrap a line, quickly check the result
    and verify that the spacing is correct!
  \end{itemize}

\item If a particular line should not have an equation number, the
  number can be suppressed using \ci{nonumber} (or
  \ci{IEEEnonumber}). If on such a line a label
  \verb+\label{eq:...}+ is defined, then this label is passed on
  further to the next equation number that is not suppressed. However,
  it is recommended to put the labels right before the line-break
  \verb+\\+ or the end of the equation it belongs to. Apart from
  improving the readability of the source code this prevents a
  compilation error in the situation of a \ci{IEEEmulticol} command
  after the label-definition.
  
\item There also exists a *-version where all equation numbers are
  suppressed. In this case an equation number can be made to appear
  using the command \ci{IEEEyesnumber}:
\begin{example}
\begin{IEEEeqnarray*}{rCl}
  a & = & b + c \\
  & = & d + e \IEEEyesnumber\\
  & = & f + g
\end{IEEEeqnarray*}
\end{example}

\item Sub-numbers are also easily possible using 
  \ci{IEEEyessubnumber}:
\begin{example}
\begin{IEEEeqnarray}{rCl}
  a & = & b + c 
  \IEEEyessubnumber\\
  & = & d + e 
  \nonumber\\
  & = & f + g 
  \IEEEyessubnumber  
\end{IEEEeqnarray}
\end{example}
  
\end{itemize}



\section{Arrays and Matrices} \label{sec:arraymat}

To typeset \textbf{arrays}, use the \ei{array} environment. It works
somewhat similar to the \texttt{tabular} environment. The \verb|\\| command is
used to break the lines:
\begin{example}
  \begin{equation*}
    \mathbf{X} = \left( 
      \begin{array}{ccc}
        x_1 & x_2 & \ldots \\
        x_3 & x_4 & \ldots \\
        \vdots & \vdots & \ddots
      \end{array} \right)
  \end{equation*}
\end{example}

The \ei{array} environment can also be used to typeset \wi{piecewise function}s by
using a ``\verb|.|'' as an invisible \ci{right} delimiter:
\begin{example}
\begin{equation*}
  |x| = \left\{
    \begin{array}{rl}
      -x & \text{if } x < 0,\\
      0 & \text{if } x = 0,\\
      x & \text{if } x > 0.
    \end{array} \right.
\end{equation*}
\end{example}
However the \ei{cases} environment from \textsf{amsmath} simplifies
the syntax, so it is worth a look:
\begin{example}
  \begin{equation*}
    |x| = 
    \begin{cases}
      -x & \text{if } x < 0,\\
      0 & \text{if } x = 0,\\
      x & \text{if } x > 0.
    \end{cases} 
\end{equation*}
\end{example}


Matrices\index{matrix} can also be typeset by \ei{array}, but
\pai{amsmath} provides a better solution using the different \ei{matrix}
environments. There are six versions with different delimiters: \ei{matrix}
(none), \ei{pmatrix} $($, \ei{bmatrix} $[$, \ei{Bmatrix} $\{$, \ei{vmatrix} $\vert$ and
\ei{Vmatrix} $\Vert$. You don't have to specify the number of columns as with
\ei{array}. The maximum number is 10, but it is customisable (though it is not
very often you need 10 columns!):
\begin{example}
\begin{equation*}
  \begin{matrix} 
    1 & 2 \\
    3 & 4 
  \end{matrix} \qquad
  \begin{bmatrix} 
    p_{11} & p_{12} & \ldots 
    & p_{1n} \\
    p_{21} & p_{22} & \ldots 
    & p_{2n} \\
    \vdots & \vdots & \ddots 
    & \vdots \\
    p_{m1} & p_{m2} & \ldots 
    & p_{mn} 
  \end{bmatrix}
\end{equation*}
\end{example}



\section{Spacing in Math Mode} \label{sec:math-spacing}

\index{math spacing} If the spacing within formulae chosen by \LaTeX{}
is not satisfactory, it can be adjusted by inserting special spacing
commands: \ci{,} for $\frac{3}{18}\:\textrm{quad}$
(\demowidth{0.166em}), \ci{:} for $\frac{4}{18}\: \textrm{quad}$
(\demowidth{0.222em}) and \ci{;} for $\frac{5}{18}\: \textrm{quad}$
(\demowidth{0.277em}).  The escaped space character \verb*|\ |
generates a medium sized space comparable to the interword spacing and
\ci{quad} (\demowidth{1em}) and \ci{qquad} (\demowidth{2em}) produce
large spaces. The size of a \ci{quad} corresponds to the width of the
character `M' of the current font. \verb|\!|\cih{"!} produces a
negative space of $-\frac{3}{18}\:\textrm{quad}$
($-$\demowidth{0.166em}).

Note that `d' in the differential is conventionally set in roman:
\begin{example}
\begin{equation*}
  \int_1^2 \ln x \mathrm{d}x 
  \qquad
  \int_1^2 \ln x \,\mathrm{d}x
\end{equation*}
\end{example}


In the next example, we define a new command \ci{ud} which produces
``$\,\mathrm{d}$'' (notice the spacing \demowidth{0.166em} before the
$\text{d}$), so we don't have to write it every time. The \ci{newcommand} is
placed in the preamble. %  More on
% \ci{newcommand} in section~\ref{} on page \pageref{}. To Do: Add label and
% reference to "Customising LaTeX" -> "New Commands, Environments and Packages"
% -> "New Commands".
\begin{example}
\newcommand{\ud}{\,\mathrm{d}}

\begin{equation*}
 \int_a^b f(x)\ud x 
\end{equation*}
\end{example}

If you want to typeset multiple integrals, you'll discover that the spacing
between the integrals is too wide. You can correct it using \ci{!}, but
\pai{amsmath} provides an easier way for fine-tuning
the spacing, namely the \ci{iint}, \ci{iiint}, \ci{iiiint}, and \ci{idotsint}
commands.

\begin{example}
\newcommand{\ud}{\,\mathrm{d}}

\begin{IEEEeqnarray*}{c}
  \int\int f(x)g(y) 
                  \ud x \ud y \\
  \int\!\!\!\int 
         f(x)g(y) \ud x \ud y \\
  \iint f(x)g(y)  \ud x \ud y 
\end{IEEEeqnarray*}
\end{example}

See the electronic document \texttt{testmath.tex} (distributed with
\AmS-\LaTeX) or Chapter 8 of \companion{} for further details.

\subsection{Phantoms}

When vertically aligning text using \verb|^| and \verb|_| \LaTeX{} is sometimes
just a little too helpful. Using the \ci{phantom} command you can
reserve space for characters that do not show up in the final output.
The easiest way to understand this is to look at an example:
\begin{example}
\begin{equation*}
{}^{14}_{6}\text{C}
\qquad \text{versus} \qquad
{}^{14}_{\phantom{1}6}\text{C}
\end{equation*}
\end{example}
If you want to typeset a lot of isotopes as in the example, the \pai{mhchem}
package is very useful for typesetting isotopes and chemical formulae too.


\section{Fiddling with the Math Fonts}\label{sec:fontsz}
Different math fonts are listed on Table~\ref{mathalpha} on page
\pageref{mathalpha}.
\begin{example}
 $\Re \qquad
  \mathcal{R} \qquad
  \mathfrak{R} \qquad
  \mathbb{R} \qquad $  
\end{example}
The last two require \pai{amssymb} or \pai{amsfonts}.

Sometimes you need to tell \LaTeX{} the correct font
size. In math mode, this is set with the following four commands:
\begin{flushleft}
\ci{displaystyle}~($\displaystyle 123$),
 \ci{textstyle}~($\textstyle 123$), 
\ci{scriptstyle}~($\scriptstyle 123$) and
\ci{scriptscriptstyle}~($\scriptscriptstyle 123$).
\end{flushleft}

If $\sum$ is placed in a fraction, it'll be typeset in text style unless you tell
\LaTeX{} otherwise:
\begin{example}
\begin{equation*}
 P = \frac{\displaystyle{ 
   \sum_{i=1}^n (x_i- x)
   (y_i- y)}} 
   {\displaystyle{\left[
   \sum_{i=1}^n(x_i-x)^2
   \sum_{i=1}^n(y_i- y)^2
   \right]^{1/2}}}
\end{equation*}    
\end{example}
Changing styles generally affects the way big operators and limits are displayed.

% This is not a math accent, and no maths book would be set this way.
% mathop gets the spacing right.


\subsection{Bold Symbols}
\index{bold symbols}

It is quite difficult to get bold symbols in \LaTeX{}; this is
probably intentional as amateur typesetters tend to overuse them.  The
font change command \verb|\mathbf| gives bold letters, but these are
roman (upright) whereas mathematical symbols are normally italic, and
furthermore it doesn't work on lower case Greek letters.
There is a \ci{boldmath} command, but \emph{this can only be used
outside math mode}. It works for symbols too, though:
\begin{example}
$\mu, M \qquad 
\mathbf{\mu}, \mathbf{M}$
\qquad \boldmath{$\mu, M$}
\end{example}

The package \pai{amsbsy} (included by \pai{amsmath}) as well as the
\pai{bm} from the \texttt{tools} bundle make this much easier as they include
a \ci{boldsymbol} command:

\begin{example}
$\mu, M \qquad
\boldsymbol{\mu}, \boldsymbol{M}$
\end{example}


\section{Theorems, Lemmas, \ldots}

When writing mathematical documents, you probably need a way to
typeset ``Lemmas'', ``Definitions'', ``Axioms'' and similar
structures.
\begin{lscommand}
\ci{newtheorem}\verb|{|\emph{name}\verb|}[|\emph{counter}\verb|]{|%
         \emph{text}\verb|}[|\emph{section}\verb|]|
\end{lscommand}
The \emph{name} argument is a short keyword used to identify the
``theorem''. With the \emph{text} argument you define the actual name
of the ``theorem'', which will be printed in the final document.

The arguments in square brackets are optional. They are both used to
specify the numbering used on the ``theorem''. Use  the \emph{counter}
argument to specify the \emph{name} of a previously declared
``theorem''. The new ``theorem'' will then be numbered in the same
sequence.  The \emph{section} argument allows you to specify the
sectional unit within which the ``theorem'' should get its numbers.

After executing the \ci{newtheorem} command in the preamble of your
document, you can use the following command within the document.
\begin{code}
\verb|\begin{|\emph{name}\verb|}[|\emph{text}\verb|]|\\
This is my interesting theorem\\
\verb|\end{|\emph{name}\verb|}|     
\end{code}

The \pai{amsthm} package (part of \AmS-\LaTeX) provides the 
\ci{theoremstyle}\verb|{|\emph{style}\verb|}|
command which lets you define what the theorem is all about by picking
from three predefined styles: \texttt{definition} (fat title, roman body),
\texttt{plain} (fat title, italic body) or \texttt{remark} (italic
title, roman body).

This should be enough theory. The following examples should
remove any remaining doubt, and make it clear that the
\verb|\newtheorem| environment is way too complex to understand.

% actually define things
\theoremstyle{definition} \newtheorem{law}{Law}
\theoremstyle{plain}      \newtheorem{jury}[law]{Jury}
\theoremstyle{remark}     \newtheorem*{marg}{Margaret}

First define the theorems:

\begin{verbatim}
\theoremstyle{definition} \newtheorem{law}{Law}
\theoremstyle{plain}      \newtheorem{jury}[law]{Jury}
\theoremstyle{remark}     \newtheorem*{marg}{Margaret}
\end{verbatim}

\begin{example}
\begin{law} \label{law:box}
Don't hide in the witness box
\end{law}
\begin{jury}[The Twelve]
It could be you! So beware and
see law~\ref{law:box}.\end{jury}
\begin{marg}No, No, No\end{marg}
\end{example}

The ``Jury'' theorem uses the same counter as the ``Law''
theorem, so it gets a number that is in sequence with
the other ``Laws''. The argument in square brackets is used to specify 
a title or something similar for the theorem.
\begin{example}
\newtheorem{mur}{Murphy}[section]

\begin{mur} If there are two or 
more ways to do something, and 
one of those ways can result in
a catastrophe, then someone 
will do it.\end{mur}
\end{example}

The ``Murphy'' theorem gets a number that is linked to the number of
the current section. You could also use another unit, for example chapter or
subsection.

If you want to customize your theorems down to the last dot, the
\pai{ntheorem} package offers a plethora of options.


\subsection{Proofs and End-of-Proof Symbol}
\label{sec:putting-qed-right}

The \pai{amsthm} package also provides the \ei{proof} environment.

\begin{example}
\begin{proof}
 Trivial, use
 \begin{equation*}
   E=mc^2.
 \end{equation*}
\end{proof}
\end{example}

With the command \ci{qedhere} you can move the `end of proof' symbol
around for situations where it would end up alone on a line.

\begin{example}
\begin{proof}
 Trivial, use
 \begin{equation*}
   E=mc^2. \qedhere
 \end{equation*}
\end{proof}
\end{example}

Unfortunately, this correction does not work for \texttt{IEEEeqnarray}:
\begin{example}
\begin{proof}
  This is a proof that ends
  with an equation array:
  \begin{IEEEeqnarray*}{rCl}
    a & = & b + c \\
    & = & d + e. \qedhere
  \end{IEEEeqnarray*}  
\end{proof}
\end{example}
\noindent
The reason for this is the internal structure of \texttt{IEEEeqnarray}:
it always puts two invisible columns at both sides of the array that
only contain a stretchable space. By this \texttt{IEEEeqnarray} ensures
that the equation array is horizontally centered. The
\ci{qedhere} command should actually be put \emph{outside} this
stretchable space, but this does not happen as these columns are
invisible to the user.

There is, however, a very simple remedy: we define these stretching
columns ourselves!
\begin{example}
\begin{proof}
  This is a proof that ends
  with an equation array:
  \begin{IEEEeqnarray*}{+rCl+x*}
    a & = & b + c \\
    & = & d + e. & \qedhere
  \end{IEEEeqnarray*}  
\end{proof}
\end{example}
\noindent
Note that the \verb=+= in \verb={+rCl+x*}= denotes stretchable spaces, one
on the left of the equations (which, if not specified, will be done
automatically by \texttt{IEEEeqnarray}!) and one on the right of the
equations. But now on the right, \emph{after} the stretching column,
we add an empty column \verb=x=. This column will be only needed on
the last line when we will put the \ci{qedhere} command
there. Finally, we specify a \verb=*=. This is a null-space that
prevents \texttt{IEEEeqnarray} to add another unwanted \verb=+=-space!

In case of equation numbering, we have a similar problem. If you compare
\begin{example}
\begin{proof}
  This is a proof that ends
  with a numbered equation:
  \begin{equation}
    a = b + c.
  \end{equation}
\end{proof}
\end{example}
\noindent
with
\begin{example}
\begin{proof}
  This is a proof that ends
  with a numbered equation:
  \begin{equation}
    a = b + c. \qedhere
  \end{equation}
\end{proof}
\end{example}
\noindent
you notice that in the (correct) second version the $\Box$ is much
closer to the equation than in the first version.

Similarly, the correct way of putting the QED-symbol at the end of an
equation array is as follows:
\begin{example}
\begin{proof}
  This is a proof that ends
  with an equation array:
  \begin{IEEEeqnarray}{+rCl+x*}
    a & = & b + c \\
    & = & d + e. \\
    &&& \qedhere\nonumber
  \end{IEEEeqnarray}  
\end{proof}
\end{example}
\noindent
which contrasts with
\begin{example}
\begin{proof}
  This is a proof that ends
  with an equation array:
  \begin{IEEEeqnarray}{rCl}
    a & = & b + c \\
    & = & d + e.
  \end{IEEEeqnarray}  
\end{proof}
\end{example}


%

% Local Variables:
% TeX-master: "lshort"
% mode: latex
% mode: flyspell
% End:
 
%%%%%%%%%%%%%%%%%%%%%%%%%%%%%%%%%%%%%%%%%%%%%%%%%%%%%%%%%%%%%%%%%
% Contents: TeX and LaTeX and AMS symbols for Maths
% $Id$
%%%%%%%%%%%%%%%%%%%%%%%%%%%%%%%%%%%%%%%%%%%%%%%%%%%%%%%%%%%%%%%%%


\section{List of Mathematical Symbols}  \label{symbols}
 
The following tables demonstrate all the symbols normally accessible
from \emph{math mode}.  

%
% Conditional Text in case the AMS Fonts are installed
%
Note that some tables show symbols only accessible after loading the \pai{amssymb}
package in the preamble of your document\footnote{The tables were derived
  from \texttt{symbols.tex} by David~Carlisle and subsequently changed
extensively as suggested by Josef~Tkadlec.}. If the \AmS{} package and
fonts are not installed on your system, have a look at
\CTANref|CTAN:pkg/amslatex|. An even more comprehensive list of
symbols can be found at \CTANref|CTAN:info/symbols/comprehensive|.
 
\begin{table}[!h]
\caption{Math Mode Accents.}  \label{mathacc}
\begin{symbols}{*3{cl}}
\W{\hat}{a}   & \W{\check}{a} & \W{\tilde}{a}       \\
\W{\grave}{a} & \W{\dot}{a}   & \W{\ddot}{a}        \\
\W{\bar}{a}   & \W{\vec}{a}   & \W{\widehat}{AAA}   \\  
\W{\acute}{a} & \W{\breve}{a} & \W{\widetilde}{AAA} \\
\W{\mathring}{a}
\end{symbols}
\end{table}
 

\begin{table}[!h]
\caption{Greek Letters.} \label{greekletters}
\bigskip
There is no uppercase of some of the letters like \ci{Alpha}, \ci{Beta} and so
on, because they look the same as normal roman letters: A, B\ldots
\begin{symbols}{*4{cl}}
 \X{\alpha}     & \X{\theta}     & \X{o}          & \X{\upsilon}  \\
 \X{\beta}      & \X{\vartheta}  & \X{\pi}        & \X{\phi}      \\
 \X{\gamma}     & \X{\iota}      & \X{\varpi}     & \X{\varphi}   \\
 \X{\delta}     & \X{\kappa}     & \X{\rho}       & \X{\chi}      \\
 \X{\epsilon}   & \X{\lambda}    & \X{\varrho}    & \X{\psi}      \\
 \X{\varepsilon}& \X{\mu}        & \X{\sigma}     & \X{\omega}    \\
 \X{\zeta}      & \X{\nu}        & \X{\varsigma}  &               \\
 \X{\eta}       & \X{\xi}        & \X{\tau} & \\
 \X{\Gamma}     & \X{\Lambda}    & \X{\Sigma}     & \X{\Psi}      \\
 \X{\Delta}     & \X{\Xi}        & \X{\Upsilon}   & \X{\Omega}    \\
 \X{\Theta}     & \X{\Pi}        & \X{\Phi} 
\end{symbols}
\end{table}



\begin{table}[!tbp]
\caption{Binary Relations.} \label{binaryrel}
\bigskip
You can negate the following symbols by prefixing them with a \ci{not} command.
\begin{symbols}{*3{cl}}
 \X{<}           & \X{>}           & \X{=}          \\
 \X{\leq}or \verb|\le|   & \X{\geq}or \verb|\ge|   & \X{\equiv}     \\
 \X{\ll}         & \X{\gg}         & \X{\doteq}     \\
 \X{\prec}       & \X{\succ}       & \X{\sim}       \\
 \X{\preceq}     & \X{\succeq}     & \X{\simeq}     \\
 \X{\subset}     & \X{\supset}     & \X{\approx}    \\
 \X{\subseteq}   & \X{\supseteq}   & \X{\cong}      \\
 \X{\sqsubset}$^a$ & \X{\sqsupset}$^a$ & \X{\Join}$^a$    \\
 \X{\sqsubseteq} & \X{\sqsupseteq} & \X{\bowtie}    \\
 \X{\in}         & \X{\ni}, \verb|\owns|  & \X{\propto}    \\
 \X{\vdash}      & \X{\dashv}      & \X{\models}    \\
 \X{\mid}        & \X{\parallel}   & \X{\perp}      \\
 \X{\smile}      & \X{\frown}      & \X{\asymp}     \\
 \X{:}           & \X{\notin}      & \X{\neq}or \verb|\ne|
\end{symbols}
\centerline{\footnotesize $^a$Use the \textsf{latexsym} package to access this symbol}
\end{table}

\begin{table}[!tbp]
\caption{Binary Operators.}
\begin{symbols}{*3{cl}}
 \X{+}              & \X{-}              & &                 \\
 \X{\pm}            & \X{\mp}            & \X{\triangleleft} \\
 \X{\cdot}          & \X{\div}           & \X{\triangleright}\\
 \X{\times}         & \X{\setminus}      & \X{\star}         \\
 \X{\cup}           & \X{\cap}           & \X{\ast}          \\
 \X{\sqcup}         & \X{\sqcap}         & \X{\circ}         \\
 \X{\vee}, \verb|\lor|     & \X{\wedge}, \verb|\land|  & \X{\bullet}       \\
 \X{\oplus}         & \X{\ominus}        & \X{\diamond}      \\
 \X{\odot}          & \X{\oslash}        & \X{\uplus}        \\
 \X{\otimes}        & \X{\bigcirc}       & \X{\amalg}        \\
 \X{\bigtriangleup} &\X{\bigtriangledown}& \X{\dagger}       \\
 \X{\lhd}$^a$         & \X{\rhd}$^a$         & \X{\ddagger}      \\
 \X{\unlhd}$^a$       & \X{\unrhd}$^a$       & \X{\wr}
\end{symbols}
 
\end{table}

\begin{table}[!tbp]
\caption{BIG Operators.}
\begin{symbols}{*4{cl}}
 \X{\sum}      & \X{\bigcup}   & \X{\bigvee}  \\
 \X{\prod}     & \X{\bigcap}   & \X{\bigwedge} \\
 \X{\coprod}   & \X{\bigsqcup} & \X{\biguplus} \\
 \X{\int}      & \X{\oint}     & \X{\bigodot} \\
 \X{\bigoplus} & \X{\bigotimes} & \\
\end{symbols}
 
\end{table}


\begin{table}[!tbp]
\caption{Arrows.} \label{tab:arrows}
\begin{symbols}{*2{cl}}
 \X{\leftarrow}or \verb|\gets|& \X{\longleftarrow} \\
 \X{\rightarrow}or \verb|\to|& \X{\longrightarrow} \\
 \X{\leftrightarrow}    & \X{\longleftrightarrow} \\
 \X{\Leftarrow}         & \X{\Longleftarrow}     \\
 \X{\Rightarrow}        & \X{\Longrightarrow}    \\
 \X{\Leftrightarrow}    & \X{\Longleftrightarrow}\\
 \X{\mapsto}            & \X{\longmapsto}        \\
 \X{\hookleftarrow}     & \X{\hookrightarrow}    \\
 \X{\leftharpoonup}     & \X{\rightharpoonup}    \\
 \X{\leftharpoondown}   & \X{\rightharpoondown}  \\
 \X{\rightleftharpoons} & \X{\iff}(bigger spaces) \\
 \X{\uparrow}   & \X{\downarrow} \\
 \X{\updownarrow} & \X{\Uparrow} \\
 \X{\Downarrow} &  \X{\Updownarrow} \\
 \X{\nearrow} &  \X{\searrow} \\
  \X{\swarrow} & \X{\nwarrow} \\
 \X{\leadsto}$^a$
\end{symbols}
\centerline{\footnotesize $^a$Use the \textsf{latexsym} package to access this symbol}
\end{table}

\begin{table}[!tbp]
\caption{Arrows as Accents.}  \label{arrowacc}
\begin{symbols}{*2{cl}}
\W{\overrightarrow}{AB}     & \W{\underrightarrow}{AB}     \\
\W{\overleftarrow}{AB}      & \W{\underleftarrow}{AB}      \\
\W{\overleftrightarrow}{AB} & \W{\underleftrightarrow}{AB} \\
\end{symbols}
\end{table}

\begin{table}[!tbp]
\caption{Delimiters.}\label{tab:delimiters}
\begin{symbols}{*3{cl}}
 \X{(}            & \X{)}            & \X{\uparrow} \\
 \X{[}or \verb|\lbrack|   & \X{]}or \verb|\rbrack|  & \X{\downarrow}   \\
 \X{\{}or \verb|\lbrace|  & \X{\}}or \verb|\rbrace|  & \X{\updownarrow} \\
 \X{\langle}      & \X{\rangle}      &  \X{\Uparrow} \\
 \X{|}or \verb|\vert| & \X{\|}or \verb|\Vert| & \X{\Downarrow} \\
  \X{/}            & \X{\backslash}   &   \X{\Updownarrow}  \\ 
 \X{\lfloor}      & \X{\rfloor}      &  \\
 \X{\rceil}       &  \X{\lceil}  &&\\
\end{symbols}
\end{table}

\begin{table}[!tbp]
\caption{Large Delimiters.}
\begin{symbols}{*3{cl}}
 \Y{\lgroup}      & \Y{\rgroup}      & \Y{\lmoustache}  \\
 \Y{\arrowvert}   & \Y{\Arrowvert}   & \Y{\bracevert} \\
 \Y{\rmoustache} \\
\end{symbols}
\end{table}


\begin{table}[!tbp]
\caption{Miscellaneous Symbols.}
\begin{symbols}{*4{cl}}
 \X{\dots}       & \X{\cdots}      & \X{\vdots}      & \X{\ddots}     \\
 \X{\hbar}       & \X{\imath}      & \X{\jmath}      & \X{\ell}       \\
 \X{\Re}         & \X{\Im}         & \X{\aleph}      & \X{\wp}        \\
 \X{\forall}     & \X{\exists}     & \X{\mho}$^a$      & \X{\partial}   \\
 \X{'}           & \X{\prime}      & \X{\emptyset}   & \X{\infty}     \\
 \X{\nabla}      & \X{\triangle}   & \X{\Box}$^a$     & \X{\Diamond}$^a$ \\
 \X{\bot}        & \X{\top}        & \X{\angle}      & \X{\surd}      \\
\X{\diamondsuit} & \X{\heartsuit}  & \X{\clubsuit}   & \X{\spadesuit} \\
 \X{\neg}or \verb|\lnot| & \X{\flat}       & \X{\natural}    & \X{\sharp}
\end{symbols}
\centerline{\footnotesize $^a$Use the \textsf{latexsym} package to access this symbol}
\end{table}


\begin{table}[!tbp]
\caption{Non-Mathematical Symbols.}
\bigskip
These symbols can also be used in text mode.
\begin{symbols}{*4{cl}}
 \SC{\dag}  &  \SC{\S}  &  \SC{\copyright} &  \SC{\textregistered}  \\
 \SC{\ddag} &  \SC{\P}  &  \SC{\pounds}    &  \SC{\%}               \\
\end{symbols}
\end{table}

\clearpage

%
%
% If the AMS Stuff is not available, we drop out right here :-)
%

\begin{table}[!tbp]
\caption{\AmS{} Delimiters.}\label{AMSD}
\bigskip
\begin{symbols}{*4{cl}}
\X{\ulcorner}&\X{\urcorner}&\X{\llcorner}&\X{\lrcorner}\\
\X{\lvert}&\X{\rvert}&\X{\lVert}&\X{\rVert}
\end{symbols}
\end{table}

\begin{table}[!tbp]
\caption{\AmS{} Greek and Hebrew.}
\begin{symbols}{*5{cl}}
\X{\digamma}     &\X{\varkappa} & \X{\beth} &\X{\gimel} & \X{\daleth}    
\end{symbols}
\end{table}

\begin{table}[tbp]
  \caption{Math Alphabets.} \label{mathalpha}
\bigskip See Table~\ref{mathfonts} on \pageref{mathfonts} for other math fonts.
\begin{symbols}{@{}*3l@{}}
Example& Command &Required package\\
\hline
\rule{0pt}{1.05em}$\mathrm{ABCDE abcde 1234}$
        & \verb|\mathrm{ABCDE abcde 1234}|
        &       \\
$\mathit{ABCDE abcde 1234}$
        & \verb|\mathit{ABCDE abcde 1234}|
        &       \\
$\mathnormal{ABCDE abcde 1234}$
        & \verb|\mathnormal{ABCDE abcde 1234}|
        &  \\
$\mathcal{ABCDE abcde 1234}$
        & \verb|\mathcal{ABCDE abcde 1234}|
        &  \\
$\mathscr{ABCDE abcde 1234}$
        &\verb|\mathscr{ABCDE abcde 1234}|
        &\pai{mathrsfs}\\
$\mathfrak{ABCDE abcde 1234}$
        & \verb|\mathfrak{ABCDE abcde 1234}|
        &\pai{amsfonts}  or \textsf{amssymb}  \\
$\mathbb{ABCDE abcde 1234}$
        & \verb|\mathbb{ABCDE abcde 1234}|
        &\pai{amsfonts}  or \textsf{amssymb} \\
\end{symbols}
\end{table}

\begin{table}[!tbp]
\caption{\AmS{} Binary Operators.}
\begin{symbols}{*3{cl}}
 \X{\dotplus}        & \X{\centerdot}      &       \\
 \X{\ltimes}         & \X{\rtimes}         & \X{\divideontimes} \\
 \X{\doublecup}      & \X{\doublecap}	   & \X{\smallsetminus} \\
 \X{\veebar}         & \X{\barwedge}       & \X{\doublebarwedge}\\
 \X{\boxplus}        & \X{\boxminus}       & \X{\circleddash}   \\
 \X{\boxtimes}       & \X{\boxdot}         & \X{\circledcirc}   \\
 \X{\intercal}       & \X{\circledast}     & \X{\rightthreetimes} \\
 \X{\curlyvee}       & \X{\curlywedge}     & \X{\leftthreetimes}
\end{symbols}
\end{table}

\begin{table}[!tbp]
\caption{\AmS{} Binary Relations.}
\begin{symbols}{*3{cl}}
 \X{\lessdot}           & \X{\gtrdot}            & \X{\doteqdot} \\
 \X{\leqslant}          & \X{\geqslant}          & \X{\risingdotseq}     \\
 \X{\eqslantless}       & \X{\eqslantgtr}        & \X{\fallingdotseq}    \\
 \X{\leqq}              & \X{\geqq}              & \X{\eqcirc}           \\
 \X{\lll}or \verb|\llless| & \X{\ggg}            & \X{\circeq}  \\
 \X{\lesssim}           & \X{\gtrsim}            & \X{\triangleq}        \\
 \X{\lessapprox}        & \X{\gtrapprox}         & \X{\bumpeq}           \\
 \X{\lessgtr}           & \X{\gtrless}           & \X{\Bumpeq}           \\
 \X{\lesseqgtr}         & \X{\gtreqless}         & \X{\thicksim}         \\
 \X{\lesseqqgtr}        & \X{\gtreqqless}        & \X{\thickapprox}      \\
 \X{\preccurlyeq}       & \X{\succcurlyeq}       & \X{\approxeq}         \\
 \X{\curlyeqprec}       & \X{\curlyeqsucc}       & \X{\backsim}          \\
 \X{\precsim}           & \X{\succsim}           & \X{\backsimeq}        \\
 \X{\precapprox}        & \X{\succapprox}        & \X{\vDash}            \\
 \X{\subseteqq}         & \X{\supseteqq}         & \X{\Vdash}            \\
 \X{\shortparallel}     & \X{\Supset}            & \X{\Vvdash}           \\
 \X{\blacktriangleleft} & \X{\sqsupset}          & \X{\backepsilon}      \\
 \X{\vartriangleright}  & \X{\because}           & \X{\varpropto}        \\
 \X{\blacktriangleright}& \X{\Subset}            & \X{\between}          \\
 \X{\trianglerighteq}   & \X{\smallfrown}        & \X{\pitchfork}        \\
 \X{\vartriangleleft}   & \X{\shortmid} 	 & \X{\smallsmile} 	\\
 \X{\trianglelefteq}    & \X{\therefore} 	 & \X{\sqsubset}  
\end{symbols}
\end{table}

\begin{table}[!tbp]
\caption{\AmS{} Arrows.}
\begin{symbols}{*2{cl}}
 \X{\dashleftarrow}      & \X{\dashrightarrow}     \\
 \X{\leftleftarrows}     & \X{\rightrightarrows}   \\
 \X{\leftrightarrows}    & \X{\rightleftarrows}    \\
 \X{\Lleftarrow}         & \X{\Rrightarrow}        \\
 \X{\twoheadleftarrow}   & \X{\twoheadrightarrow}  \\
 \X{\leftarrowtail}      & \X{\rightarrowtail}     \\
 \X{\leftrightharpoons}  & \X{\rightleftharpoons}  \\
 \X{\Lsh}                & \X{\Rsh}                \\
 \X{\looparrowleft}      & \X{\looparrowright}     \\
 \X{\curvearrowleft}     & \X{\curvearrowright}    \\
 \X{\circlearrowleft}    & \X{\circlearrowright}   \\
 \X{\multimap}  &  \X{\upuparrows}  \\
 \X{\downdownarrows} & \X{\upharpoonleft} \\
 \X{\upharpoonright} & \X{\downharpoonright} \\
 \X{\rightsquigarrow} & \X{\leftrightsquigarrow} \\
\end{symbols}
\end{table}

\begin{table}[!tbp]
\caption{\AmS{} Negated Binary Relations and Arrows.}\label{AMSNBR}
\begin{symbols}{*3{cl}}
 \X{\nless}           & \X{\ngtr}            & \X{\varsubsetneqq}  \\
 \X{\lneq}            & \X{\gneq}            & \X{\varsupsetneqq}  \\
 \X{\nleq}            & \X{\ngeq}            & \X{\nsubseteqq}     \\
 \X{\nleqslant}       & \X{\ngeqslant}       & \X{\nsupseteqq}     \\
 \X{\lneqq}           & \X{\gneqq}           & \X{\nmid}           \\
 \X{\lvertneqq}       & \X{\gvertneqq}       & \X{\nparallel}      \\
 \X{\nleqq}           & \X{\ngeqq}           & \X{\nshortmid}      \\
 \X{\lnsim}           & \X{\gnsim}           & \X{\nshortparallel} \\
 \X{\lnapprox}        & \X{\gnapprox}        & \X{\nsim}           \\
 \X{\nprec}           & \X{\nsucc}           & \X{\ncong}          \\
 \X{\npreceq}         & \X{\nsucceq}         & \X{\nvdash}         \\
 \X{\precneqq}        & \X{\succneqq}        & \X{\nvDash}         \\
 \X{\precnsim}        & \X{\succnsim}        & \X{\nVdash}         \\
 \X{\precnapprox}     & \X{\succnapprox}     & \X{\nVDash}         \\
 \X{\subsetneq}       & \X{\supsetneq}       & \X{\ntriangleleft}  \\
 \X{\varsubsetneq}    & \X{\varsupsetneq}    & \X{\ntriangleright} \\
 \X{\nsubseteq}       & \X{\nsupseteq}       & \X{\ntrianglelefteq}\\
 \X{\subsetneqq}      & \X{\supsetneqq}      &\X{\ntrianglerighteq}\\[0.5ex]
 \X{\nleftarrow}      & \X{\nrightarrow}     & \X{\nleftrightarrow}\\
 \X{\nLeftarrow}      & \X{\nRightarrow}     & \X{\nLeftrightarrow}

\end{symbols}
\end{table}

\begin{table}[!tbp] \label{AMSmisc}
\caption{\AmS{} Miscellaneous.}
\begin{symbols}{*3{cl}}
 \X{\hbar}             & \X{\hslash}           & \X{\Bbbk}            \\
 \X{\square}           & \X{\blacksquare}      & \X{\circledS}        \\
 \X{\vartriangle}      & \X{\blacktriangle}    & \X{\complement}      \\
 \X{\triangledown}     &\X{\blacktriangledown} & \X{\Game}            \\
 \X{\lozenge}          & \X{\blacklozenge}     & \X{\bigstar}         \\
 \X{\angle}            & \X{\measuredangle}    & \\
 \X{\diagup}           & \X{\diagdown}         & \X{\backprime}       \\
 \X{\nexists}          & \X{\Finv}             & \X{\varnothing}      \\
 \X{\eth}              & \X{\sphericalangle}   & \X{\mho}              
\end{symbols}
\end{table}





\endinput

%

% Local Variables:
% TeX-master: "lshort2e"
% mode: latex
% mode: flyspell
% End:

% The Not So Short Introduction to LaTeX
%
% Copyright (C) 1995--2022 Tobias Oetiker, Marcin Serwin, Hubert Partl,
% Irene Hyna, Elisabeth Schlegl and Contributors.
%
% This document is free software: you can redistribute it and/or modify it
% under the terms of the GNU General Public License as published by the Free
% Software Foundation, either version 3 of the License, or (at your option) any
% later version.
%
% This document is distributed in the hope that it will be useful, but WITHOUT
% ANY WARRANTY; without even the implied warranty of MERCHANTABILITY or FITNESS
% FOR A PARTICULAR PURPOSE.  See the GNU General Public License for more
% details.
%
% You should have received a copy of the GNU General Public License along with
% this document.  If not, see <https://www.gnu.org/licenses/>.

% !TEX root = ./lshort.tex
%%%%%%%%%%%%%%%%%%%%%%%%%%%%%%%%%%%%%%%%%%%%%%%%%%%%%%%%%%%%%%%%%
% Contents: Specialities of the LaTeX system
% $Id$
%%%%%%%%%%%%%%%%%%%%%%%%%%%%%%%%%%%%%%%%%%%%%%%%%%%%%%%%%%%%%%%%%

\chapter{Specialities}\label{specialities}
\begin{intro}
  When putting together a large document, \LaTeX{} will help with some special
  features like index generation, automatic linking to relevant pages and other
  things. A much more complete description of specialities and enhancements
  possible with \LaTeX{} can be found in the {\normalfont\manual{}} and
    {\normalfont\companion}.
\end{intro}

\section{Indexing}\label{sec:indexing}
A very useful feature of many books is their \wi{index}. With \LaTeX{}
and the support program \texttt{makeindex},\footnote{On systems not
  necessarily supporting
  filenames longer than 8~characters, the program may be called
  \texttt{makeidx}.} an index can be generated quite easily.  This
introduction will only explain the basic index generation commands.
For a more in-depth view, please refer to \companion.\index{makeindex
  program}\index{makeidx package}

To enable the indexing feature of \LaTeX{}, the \pai{makeidx} package
must be loaded in the preamble with
\begin{lscommand}
  \mintinline{latex}|\usepackage{makeidx}|
\end{lscommand}
\noindent and the special indexing commands must be enabled by putting
the
\begin{lscommand}
  \csi{makeindex}
\end{lscommand}
\noindent command in the preamble.

The content of the index is specified with
\begin{lscommand}
  \csi{index}[{{«\bs carg{key}»@«\bs carg{formatted entry}»}}: c]
\end{lscommand}
\noindent commands, where \carg{formatted entry} will appear in the index
and \carg{key} will be used for sorting.  The \carg{formatted entry} is
optional. If it is missing the \carg{key} will be used. You enter the index
commands at the points in the text that you want the final index entries to
point to. \autoref{index} explains the syntax with several examples.

\begin{table}[!tp]
  \centering
  \caption{Index Key Syntax Examples.}\label{index}
  \begin{tabular}{@{}lll@{}}
    \toprule
    Code                                   & Entry                 & Explanation            \\
    \midrule
    \rule{0pt}{1.05em}\verb|\index{hello}| & hello, 1              & Plain entry            \\
    \verb|\index{hello!Peter}|             & \hspace*{2ex}Peter, 3 & Subentry under `hello' \\
    \verb|\index{Sam@\emph{Sam}}|          & \emph{Sam}, 2         & Formatted entry        \\
    \verb|\index{Kaese@\emph{K\"ase}}|     & \emph{K\"ase}, 33     & Formatted entry        \\
    \verb.\index{ecole@\'ecole}.           & \'ecole, 4            & Formatted entry        \\
    \verb.\index{Jenny|emph}.              & Jenny, \emph{3}       & Formatted page number  \\
    \verb.\index{Joe@\emph{Joe}|emph}.     & \emph{Joe}, \emph{5}  & Formatted page number  \\
    \bottomrule
  \end{tabular}
\end{table}

When the input file is processed with \LaTeX{}, each \verb|\index|
command writes an appropriate index entry, together with the current
page number, to a special file. The file has the same name as the
\LaTeX{} input file, but a different extension (\verb|.idx|). This
\eei{.idx} file can then be processed with the \texttt{makeindex}
program:
\begin{lscommand}
  \texttt{makeindex} \emph{filename}
\end{lscommand}

The \texttt{makeindex} program generates a sorted index with the same base
file name, but this time with the extension \eei{.ind}. If now the
\LaTeX{} input file is processed again, this sorted index gets
included into the document at the point where \LaTeX{} finds
\begin{lscommand}
  \csi{printindex}
\end{lscommand}

The \pai{showidx} package that comes with \LaTeX{} prints out all
index entries in the left margin of the text. This is quite useful for
proofreading a document and verifying the index. Make sure to load the package
afer the \pai{hyperref} package.

Note that the \csi{index} command can affect your layout if not used carefully.

\begin{chktexignore}
  \begin{example}
My Word \index{Word}. As opposed
to Word\index{Word}. Note the
position of the full stop.
\end{example}
\end{chktexignore}

Note that the texttt{makeindex} has no clue about characters outside the ASCII range. To
get the sorting correct, use the \verb|@| character as shown in the K\"ase
and \'ecole examples above.

\section{Fancy Headers}\label{sec:fancy}

\subsection{Basic commands}

%$$ Not sure if the overall conversational tone of this subsection is in keeping
%$$ with the bulk of the rest of the document.

The \pai*{fancyhdr} package provides a few simple commands that allow you to
customise the header and footer lines of your document. It defines an
additional page style \cargv{fancy} and a set of commands to customise it to
your liking. By default, it only adds a line separating the header from the page
body.
\begin{example}[standalone, paperheight=3cm]
\geometry{includefoot, includehead, headsep=.5em, footskip=1em} %!hide
%!showbegin %!hide
\documentclass{article}

%!showend %!hide
\usepackage{fancyhdr}
\pagestyle{fancy}

\begin{document}
\noindent %!hide
This statement is false.
\end{document}
\end{example}

The primary command of the package is
\begin{lscommand}
  \csi{fancyhf}[places: o, field: m]
\end{lscommand}
The \carg{places} is a comma separated list of places where the \carg{field} should be displayed. There are total of 12 different places and each is identified by a
combination of three letters:
\begin{itemize}
  \item The first specifies whether the location is in the header (\cargv{H}) or
        in the footer (\cargv{F}).
  \item The second letter specifies the location within the header or footer:
        \cargv{L} for left, \cargv{C} for centre, and \cargv{R} for right.
  \item The third letter defines whether the field should be printed on even
        (\cargv{E}) or odd (\cargv{O}) pages. If the document is not two sided,
        then all pages are treated as odd.
\end{itemize}
For example, the combination \cargv{FCE} identifies the centre part of the
footer on even pages. If any of the letters is omitted, then the identifier
points toward all positions specifiable by the omitted letter. For example,
\cargv{HR} is right side of the header on both odd and even pages.
\begin{example}[template=empty, standalone, paperheight=2.3cm, paperwidth=6cm, to_page=2,vertical_pages]
\documentclass[twoside]{article}

%!hidebegin
\usepackage[paperheight=\height,
    paperwidth=\width,
    margin=0.3cm,
    includefoot,
    includehead,
    headsep=.5em,
    footskip=1em
]{geometry}
%!hideend
\usepackage{fancyhdr}
\pagestyle{fancy}

\fancyhf[HCE]{A}
\fancyhf[L]{\emph{B}}
\fancyhf[FR]{\textbf{C}}
\fancyhf[HCO, HRE]{\textsl{D}}

\begin{document}
\noindent %!hide
The next statement is false.
The previous statement is true.
\end{document}
\end{example}

There are two additional commands, \csi{fancyhead} and \csi{fancyfoot}, that
work in the same way, except that they respectively assume \cargv{H} and
\cargv{F} in their \carg{places} argument, unless otherwise specified.
\begin{example}[standalone, paperheight=3.5cm]
\geometry{includefoot, includehead, headsep=.5em, footskip=1em} %!hide
\sloppy %!hide
\usepackage{csquotes} %!hide
\usepackage{fancyhdr}%!hide
\pagestyle{fancy}%!hide

\fancyhead[L]{A}
\fancyfoot[R]{B}

\begin{document}%!hide
\noindent %!hide
\enquote{Yields a falsehood when
appended to its own quotation}
yields a falsehood when appended
to its own quotation.
\end{document}%!hide
\end{example}

The lines drawn by the \pai{fancyhdr} package may also be customised. To change
%$$ Inserted comma after 'thickness'.
their thickness, redefine the
\begin{lscommand}
  \csi{headrulewidth} \\
  \csi{footrulewidth}
\end{lscommand}
macros to the desired size.
\begin{example}[standalone, paperheight=3cm, paperwidth=3cm]
\geometry{includefoot, includehead, headsep=.5em, footskip=1em} %!hide
\sloppy %!hide
\usepackage{fancyhdr} %!hide
\pagestyle{fancy} %!hide
\RenewDocumentCommand{\headrulewidth}{}{.2cm}
\RenewDocumentCommand{\footrulewidth}{}{.5cm}

\begin{document}%!hide
\noindent %!hide
Do not read this sentence.
\end{document}%!hide
\end{example}

%$$ Inserted comma after 'default'.
By default, headers and footers are as long as the text on the page. If you want
%$$ Changed 'extend\slash{}shorten' to 'extend or shorten'.
%$$ Added comma after 'them'.
to extend\slash{}shorten them, use the
\begin{lscommand}
  \csi{fancyhfoffset}[places: o, offset: m]
\end{lscommand}
%$$ Added comma after \csi{fancyhf}.
command. The \carg{places} argument is the same as in \csi{fancyhf}, except that
it cannot contain \cargv{C}.
\begin{example}[standalone, paperheight=3cm]
\geometry{includehead, includefoot, headsep=.5em, footskip=1em} %!hide
\sloppy %!hide
\usepackage{fancyhdr}%!hide
\pagestyle{fancy}%!hide
\fancyhfoffset[L]{-1cm}
\fancyhfoffset[R]{.2cm}

\begin{document}%!hide
\noindent %!hide
If this sentence is true,
then \(2 + 2 = 5\).
\end{document}%!hide
\end{example}
The command \csi{fancyheadoffset} and \csi{fancyfootoffset} are
used the same way as \csi{fancyf}, but they only modify header or footer respectively.

\subsection{Contents of the headers}

%$$ Not sure if the overall conversational tone of this subsection is in keeping
%$$ with the bulk of the rest of the document.

The default footer of the article class contains the current page number. To
use it inside the fancy header, simply use the command \csi{thepage}.
\begin{example}[standalone, paperheight=2.5cm, to_page=2, vertical_pages]
\geometry{includehead, includefoot, headsep=.5em, footskip=1em} %!hide
\sloppy %!hide
\usepackage{fancyhdr}%!hide
\pagestyle{fancy}%!hide
\fancyhf{Page~\thepage}

\begin{document}%!hide
\noindent %!hide
This statement is dedicated to
all statements that are not
dedicated to themselves.
\end{document}%!hide
\end{example}

It is often useful to have the header and footer contain information based on
the content of the page. These are called \enquote{marks} in \LaTeX{}
terminology. Before we talk about the default ones, let's consider how you can
define your own using the \pai{extramarks}\footnote{\pai{extramarks} is part
  of \pai*{fancyhdr}.} package.
\begin{lscommand}
  \csi{extramarks}[left:m, right:m] \\
  \csi{firstleftxmark} \\
  \csi{firstrightxmark} \\
  \csi{lastleftxmark} \\
  \csi{lastrightxmark}
\end{lscommand}
The \csi{extramarks} command sets the contents of \carg{left} and \carg{right}
marks. Then you can access these marks inside the headers by using the appropriate
command. The \texttt{first}- commands refer to the first mark occurring on the
page, while the \texttt{last}- refer to the last one.
\begin{example}[standalone, paperheight=4cm]
\geometry{includehead, includefoot, headsep=.5em, footskip=1em} %!hide
\sloppy %!hide
\usepackage{fancyhdr}%!hide
\usepackage{extramarks}
\pagestyle{fancy}%!hide

\fancyhead[L]{\firstleftxmark}
\fancyhead[R]{\lastleftxmark}
\fancyfoot[L]{\firstrightxmark}
\fancyfoot[R]{\lastrightxmark}

\begin{document} %!hide
\noindent %!hide
The second statement is false.
\extramarks{One}{2 is false}
The third statement is false.
\extramarks{Two}{3 is false}
The first statement is false.
\extramarks{Three}{1 is false}
\end{document} %!hide
\end{example}

Let us now look at the default marks defined by \LaTeX{}. After loading
\pai{extramarks}, these may be set and accessed similarly:
\begin{lscommand}
  \csi{markboth}[left:m, right:m] \\
  \csi{firstleftmark} \\
  \csi{firstrightmark} \\
  \csi{lastleftmark} \\
  \csi{lastrightmark}
\end{lscommand}
The only difference between these and the extra marks is that \LaTeX{} classes
automatically fill them. Hence, if you are not careful, you may lose your
content.\footnote{You may prevent this by setting the pagestyle to
  \cargv{myheadings} and then redefining it to use it with \pai{fancyhdr} as
  described later.}
For example, in the article class, \carg{left} is set by the \csi{section}
command, while \carg{right} is set by the \csi{subsection} command.
\begin{example}[standalone, paperheight=5cm]
\geometry{includehead, includefoot, headsep=.5em, footskip=1em} %!hide
\sloppy %!hide
\usepackage{fancyhdr}%!hide
\usepackage{extramarks}%!hide
\pagestyle{fancy}%!hide
\fancyhead[L]{\firstleftmark}
\fancyhead[R]{\lastleftmark}
\fancyfoot[L]{\firstrightmark}
\fancyfoot[R]{\lastrightmark}

\begin{document}%!hide
\section{First}
\subsection{Sub}
\section{Second}
\subsection{Sub}
\end{document}%!hide
\end{example}
Note that \csi{firstrightmark} is empty in the above example. This is caused by
the fact that \csi{section} commands set both marks, leaving the right one
empty.

You may have noticed that section titles are typeset in uppercase when provided
by -\texttt{mark} commands. To disable this use the \csi{nouppercase} command inside the
\csi{fancyhf} command.
\begin{example}[standalone, paperheight=3cm]
\geometry{includehead, includefoot, headsep=.5em, footskip=1em} %!hide
\sloppy %!hide
\usepackage{fancyhdr}%!hide
\usepackage{extramarks}%!hide
\pagestyle{fancy}%!hide
\fancyhead[R]{%
  \nouppercase{\firstleftmark}%
}

\begin{document}%!hide
\section{Today}
is opposite day.
\end{document}%!hide
\end{example}

\subsection{Advanced commands}

If you want even more control over section titles, you can redefine the
\csi{sectionmark}.\footnote{Or \csi{chaptermark}, \csi{subsectionmark} \ldots} The
command receives the section title as its first argument, while the section
number is available as \csi{thesection}.
\begin{example}[standalone, paperheight=4.5cm, paperwidth=4cm]
\geometry{includehead, includefoot, headsep=.5em, footskip=1em} %!hide
\sloppy %!hide
\usepackage{fancyhdr}%!hide
\usepackage{extramarks}%!hide
\pagestyle{fancy}%!hide
\fancyhead[R]{\firstleftmark}
\RenewDocumentCommand{\sectionmark}{m}{%
  \markboth{%
    Section no.\,\thesection: #1}{}%
}

\begin{document}%!hide
\section{Person}
There exists a person such that
if they are reading this then
everybody is reading this.
\end{document}%!hide
\end{example}
If you only want to change the right mark, use the \csi{markright} command.

By default, the field in the centre of the header or footer will expand equally
to the left and right. This is usually the desired behaviour, but in some
cases it may overlap with either left or right text, while still having some
space on the other side.
\begin{example}[standalone, paperheight=3cm]
\geometry{includehead, includefoot, headsep=.5em, footskip=1em} %!hide
\sloppy %!hide
\usepackage{fancyhdr}%!hide
\usepackage{extramarks}%!hide
\pagestyle{fancy}%!hide
\fancyhead[L]{\thepage}
\fancyhead[C]{\firstleftmark}
\fancyhead[R]{Jane Doe}

\begin{document}%!hide
\section{Section}
Section by Jane Doe
\end{document}%!hide
\end{example}
In this situation you may want to use the
\begin{lscommand}
  \csi{fancycenter}[distance:o, stretch:o, left:m, centre:m, right:m]
\end{lscommand}
command, which automatically shifts the centre toward the shorter text. The
\carg{distance} is the minimum distance by which the elements are always
surrounded (\cargv{1em}, by default). The \carg{stretch} controls the preference
for shifting the \carg{centre}; \cargv{1} means shift only when and as much
as necessary. Higher numbers will start shifting sooner and more aggressively.
The default is \cargv{3}. This command writes over the whole header\slash{}footer
space so it should only be put in one place (typically \cargv{C}) and other
places (\cargv{L,R}) should be empty.
\begin{example}[standalone, paperheight=3cm]
\geometry{includehead, includefoot, headsep=.5em, footskip=1em} %!hide
\sloppy %!hide
\usepackage{fancyhdr}%!hide
\usepackage{extramarks}%!hide
\pagestyle{fancy}%!hide
\fancyhead[L,R]{}
\fancyhead[C]{%
  \fancycenter%
    {\thepage}%
    {\firstleftmark}%
    {Jane Doe}%
}

\begin{document}%!hide
\section{Section}
Section by Jane Doe
\end{document}%!hide
\end{example}

You may want to present different headers or footers when the corresponding
page starts with a float or ends with a footnote. The \pai{fancyhdr} package
defines four commands that let you achieve this.
\begin{lscommand}
  \csi{iftopfloat}[true branch:m, false branch:m] \\
  \csi{ifbotfloat}[true branch:m, false branch:m] \\
  \csi{iffloatpage}[true branch:m, false branch:m] \\
  \csi{iffootnote}[true branch:m, false branch:m]
\end{lscommand}
Commands \csi{iftopfloat} and \csi{ifbotfloat} execute their \carg{true branch}
if a float sits at the top or bottom of the page (respectively).

Similarly, \csi{iffloatpage} checks whether
the page is a special float-only page, while \csi{iffootnote} checks for a
footnote at the bottom of the page.
\begin{example}[standalone, paperheight=4cm, to_page=2, vertical_pages]
\geometry{includehead, includefoot, headsep=.5em, footskip=1em} %!hide
\sloppy %!hide
\usepackage{fancyhdr}%!hide
\usepackage{extramarks}%!hide
\pagestyle{fancy}%!hide

\fancyhead[C]{%
  \iftopfloat{%
    A float is below me%
  } {}
}
\fancyfoot[C]{%
  \iffootnote{%
    A footnote is above me%
  } {%
    \thepage%
  }%
}

\begin{document}%!hide
\noindent %!hide
Ignore this footnote.%
\footnote{Read this.}
\begin{figure}[t]
  \centering
  A floating float.
  \caption{Hmm}
\end{figure}
\end{document}%!hide
\end{example}

The ruled lines in headers and footers are created by invoking \csi{headrule}
and \csi{footrule} commands. If you want finer control over the lines, consider
redefining these macros. Use \csi{headruleskip} and \csi{footruleskip} to raise
or lower them, if necessary.
\begin{example}[standalone, paperheight=4cm, paperwidth=3.2cm]
\geometry{includehead, includefoot, headsep=.5em, footskip=1em} %!hide
\sloppy %!hide
\usepackage{fancyhdr}%!hide
\usepackage{extramarks}%!hide
\pagestyle{fancy}%!hide
\RenewDocumentCommand{\headrule}{}{
  \rule{0.05\headwidth}{0.2cm}%
  \rule[0.1cm]{0.9\headwidth}{\headrulewidth}%
  \rule{0.05\headwidth}{0.2cm}%
}

\RenewDocumentCommand{\headruleskip}{}{-0.2cm}

\begin{document}%!hide
\noindent %!hide
My age is the first number not
nameable in under twenty words.
\end{document} %!hide
\end{example}

While working on a longer document, you may want to have several page styles for
different occasions. You may also notice that some commands (\csi{chapter} and
\csi{maketitle}, for example) change the page style to \cargv{plain}.
The solution to both of these problems is the
\begin{lscommand}
  \csi{fancypagestyle}[name:m, code:m]
\end{lscommand}
command. The \carg{name} is the name of page style to (re)define, while the %chktex 36
\carg{code} is the code to set the page style. All page styles declared this
way use the \cargv{fancy} page style as their basis, so if empty \carg{code} is
given they will match \cargv{fancy} exactly.
\begin{example}[standalone, paperheight=3cm]
\geometry{includehead, includefoot, headsep=.5em, footskip=1em} %!hide
\sloppy %!hide
\usepackage{fancyhdr}%!hide
\usepackage{extramarks}%!hide
\fancypagestyle{mine}{
  \fancyhead[L]{My style}
}
\pagestyle{mine}

\begin{document}%!hide
%!showbegin %!hide
Were you expecting a paradox here?
%!showend !hidebegin
\noindent
You may be disappointed.
\end{document}
\end{example}

\section{Installing Extra Packages}\label{sec:Packages}

Most \LaTeX{} installations come with a large set of pre-installed
style packages, but many more are available on the net. The main
place to look for style packages on the Internet is CTAN (\url{http://www.ctan.org/}).

Packages such as \pai{geometry}, \pai{hyphenat}, and many
others are typically made up of two files: a file with the extension
\texttt{.ins} and another with the extension \texttt{.dtx}. There
will often be a \texttt{readme.txt} with a brief description of the
package. You should of course read this file first.

In any event, once you have copied the package files onto your
machine, you still have to process them in a way that (a) tells your
\TeX\ distribution about the new style package, and (b) gives you
the documentation.  Here's how you do the first part:

\begin{enumerate}
  \item Run \LaTeX{} on the \texttt{.ins} file. This will
        extract a \eei{.sty} file.
  \item Move the \eei{.sty} file to a place where your distribution
        can find it. Usually this is in your \texttt{\ldots/\emph{localtexmf}/tex/latex}
        subdirectory (Windows or OS/2 users should feel free to change the
        direction of the slashes).
  \item Refresh your distribution's file-name database. The command
        depends on the \LaTeX{} distribution you use:
        \TeXLive{} --- \texttt{texhash}; web2c --- \texttt{maktexlsr};
        MiK\TeX{} --- \texttt{initexmf -{}-update-fndb} or use the GUI\@.
\end{enumerate}

\noindent Now extract the documentation from the
\texttt{.dtx} file:

\begin{enumerate}
  \item Run \hologo{XeLaTeX} on the \texttt{.dtx} file.  This will generate a
        \texttt{.pdf} file. Note that you may have to run \hologo{XeLaTeX}
        several times before it gets the cross-references right.
  \item Check to see if \LaTeX\ has produced a \texttt{.idx} file
        among the various files you now have.
        If you do not see this file, then the documentation has no index. Continue
        with step~\ref{step:final}.
  \item In order to generate the index, type the following:\\
        \fbox{\texttt{makeindex -s gind.ist \textit{name}}}\\
        (where \textit{name} stands for the main-file name without any
        extension).
  \item Run \LaTeX\ on the \texttt{.dtx} file once again.\label{step:next}

  \item Last but not least, make a \texttt{.ps} or \texttt{.pdf}
        file to increase your reading pleasure.\label{step:final}

\end{enumerate}

Sometimes you will see that a \texttt{.glo}
(glossary) file has been produced. Run the following
command between
step~\ref{step:next} and~\ref{step:final}:

\noindent\texttt{makeindex -s gglo.ist -o \textit{name}.gls \textit{name}.glo}

\noindent Be sure to run \LaTeX\ on the \texttt{.dtx} one last
time before moving on to step~\ref{step:final}.

%%%%%%%%%%%%%%%%%%%%%%%%%%%%%%%%%%%%%%%%%%%%%%%%%%%%%%%%%%%%%%%%%
% Contents: Chapter on pdfLaTeX
% French original by Daniel Flipo 14/07/2004
%%%%%%%%%%%%%%%%%%%%%%%%%%%%%%%%%%%%%%%%%%%%%%%%%%%%%%%%%%%%%%%%%

\section{\LaTeX{} and PDF}\label{sec:pdftex}\index{PDF}
\begingroup
\LshortExampleSetup{hide_hyperref=false}

The initial release of \TeX{} predated the PDF format by nearly 16 years. The
original output files it produced---\eei{.dvi}s---were meant to be only
printed. Today many documents are never or seldom printed, we read them
directly on a screen. PDF format contains many improvements for viewing
documents like this but they are not implemented in core \LaTeX{}. These are
accessible via the \pai*{hyperref} package.

\subsection{Hypertext Links}\label{hyperlinks}

Hyperlinks are used to quickly jump around the document. The prime example of
using them is the table of contents, you don't have to manually scroll to a
given page---just click on a given chapter and you will be immediately
transported there. You already know that table of contents can be typeset using
the \csi{tableofcontents} command, but it doesn't contain any hyperlinks.

Luckily the process of updating your document is extremely easy: just add
\mintinline{latex}|\usepackage{hyperref}| as the \emph{last} package loaded in
your preamble. Doing so redefines internal \LaTeX{} commands to produce
hyperlinks.
\begin{example}
%!hidebegin
\hypersetup{
  hidelinks,
  pdfborder=0 0 1,
}
%!showbegin !hideend
% In preamble
\usepackage{hyperref}
%!showend !hide
% ...
The reference to this section
now looks like that:\footnote{
  Footnotes are also
  made into hyperlinks.}
Section~\ref{hyperlinks} is
on page~\pageref{hyperlinks}.
The hyperref bibliographic
entry is~\cite{pack:hyperref}.
\end{example}
By default links are marked by a red box around them. This box
is only visible when viewing the document on screen and will not be printed.
The boxes are, however, rather ugly, so you may want to add the \cargv{colorlinks}
option while loading the package.
\begin{example}
%!showbegin !hide
% In preamble
\usepackage[
  colorlinks
]{hyperref}
%!showend !hide
% ...
The reference to this section
now looks like that:
Section~\ref{hyperlinks} is
on page~\pageref{hyperlinks}.
\end{example}
This is the option used throughout this booklet and assumed in further
examples. While this makes the links more visually appealing, it has the
disadvantage that the coloured links will be printed. To marry the best of both
worlds you can use the \pai*{ocgx2} package with the option
\cargv{ocgcolorlinks} like this
\begin{code}
\begin{minted}{latex}
\usepackage{hyperref}
\usepackage[ocglinks]{ocgx2}
\end{minted}
\end{code}
Be warned though that it is not well supported by popular PDF viewers. Another
option is to use option \cargv{hidelinks} that makes the links clickable but
does not distinguish them visually.

In addition to redefining internal \LaTeX{} command, \pai{hyperref} defines
some additional ones. To typeset URLs you can now use the \csi{url} command.
\begin{example}
\url{https://www.ctan.org/}
\end{example}
If you want to display different text for the clickable link you can use the
\begin{lscommand}
  \csi{href}[URL: m, text: m]
\end{lscommand}
command. Note that if you intend for the document to be useful when printed,
you have to provide the full URLs anyway.
\begin{example}
Packages can be found on
\href{https://www.ctan.org/}{
  CTAN}.
\end{example}

A similar command is
\begin{lscommand}
  \csi{hyperref}[marker: o, text: m]
\end{lscommand}
that allows to create a hyperlink within the document with a different text.
\begin{example}
\hyperref[hyperlinks]{
  Hyperlinks section}
describes the usage of
hyperlinks.
\end{example}
To avoid nesting hyperlinks inside one another, \pai{hyperref} provides starred
versions of \csi{ref} and \csi{pageref} commands that produce text without
hyperlinking them.
\begin{example}
\hyperref[hyperlinks]{
  Section~\ref*{hyperlinks}}
describes the usage of
hyperlinks.

This ref~\ref*{hyperlinks}
is not hyperlinked.
\end{example}

Throughout this booklet you have seen references such as
\enquote*{\autoref{hyperlinks} on \autopageref{hyperlinks}}. While this could
be achieved using the aforementioned \csi{hyperref} command, this usecase is so
common that \pai{hyperref} provides two commands that make it much easier:
\begin{lscommand}
  \csi{autoref}[marker: m] \\
  \csi{autopageref}[marker: m]
\end{lscommand}
Their usage is identical to the \csi{ref} and \csi{pageref} commands, but they
produce additional text based on the counter the \carg{marker} refers to.
\begin{example}
\RenewDocumentCommand{\chapterautorefname}{}{chapter} %!hide
\autoref{hyperlinks} in
\autoref{specialities} on
\autopageref{hyperlinks}.
\end{example}
The names are controlled by commands such as \cs{autorefchaptername} or
\cs{autorefsectionname}. See the \pai*{hyperref} documentation for a full list.

So far we have used the default colours for URLs, hyperlinks. These can be
changed using the \csi{hypersetup} command. It accepts a key value list
customising the appearance of links. Colours may be specified using
\cargv{linkcolor}, \cargv{citecolor} and \cargv{urlcolor}. If you have the
\pai{xcolor} package loaded, you may specify colours the same way as described
in \autoref{sec:colors}.
\begin{chktexignore}
\begin{example}
\hypersetup{
  urlcolor = pink,
  citecolor = purple,
  linkcolor = teal!50!yellow,
}
\url{https://www.ctan.org/} \\
\cite{pack:hyperref} \\
\autoref{hyperlinks}
\end{example}
\end{chktexignore}

You can also adjust the borders around the links. The basic key is
\cargv{pdfborder}. It accepts three numbers: horizontal corner radius, vertical
corner radius and border width.
\begin{example}
\hypersetup{pdfborder = 0 0 1}
\url{https://www.ctan.org/} \\
\hypersetup{pdfborder = 10 10 3}
\url{https://www.ctan.org/} \\
\hypersetup{pdfborder = 10 5 2}
\url{https://www.ctan.org/} \\
\hypersetup{pdfborder = 2 7 5}
\url{https://www.ctan.org/}
\end{example}
The colour of the boxes may be adjusted with the
\cargv{linkbordercolor}, \cargv{citebordercolor} and \cargv{urlbordercolor}
keys, assuming the \pai{xcolor} package is loaded.
\begin{chktexignore}
  \begin{example}
\hypersetup{
  pdfborder = 0 0 2,
  urlbordercolor = violet,
  citebordercolor = pink,
  linkbordercolor = teal,
}
\url{https://www.ctan.org/} \\
\cite{pack:hyperref} \\
\autoref{hyperlinks}
\end{example}
\end{chktexignore}

\subsection{Document Metadata}\label{sec:pdfmeta}

Another thing that may be adjusted with the \pai{hyperref} package is document
metadata. These are information about your document that are not visible in the
document itself, but may be used by your PDF viewer in various ways. For
example, the title of the document may be shown in its top window bar.

Additional information about your document may be set using \cargv{pdfinfo}
key. This key itself accepts a key value list of document properties.
\begin{minted}{latex}
\hypersetup{
  pdfinfo = {
    Title = Title of the Book,
    Author = {Us, Ourselves and We},
    Subject = Book creation with LaTeX,
    Creator = Our House,
    Keywords = {LaTeX, typesetting},
    Producer = LuaTeX,
  }
}
\end{minted}

You can also control the way your document presents itself when opening. For
example you can choose whether the bookmarks should be shown
(\cargv{bookmarksopen}), whether external links should be opened in new windows
(\cargv{pdfnewwindow}) whether the pages should initially fit the window
(\cargv{pdffitwindow}).

If you want to set the metadata without redefining internal \LaTeX{} commands
to produce links, you may pass \mintinline{latex}{implicit=false} to
\pai{hyperref} package options.
\endgroup

\subsection{Problems with Outline}

The \pai{hyperref} package automatically uses table of contents generated by
the document as a document outline for easier navigation. This may lead to some
problems if your section titles contain some non-text content (for example,
\enquote{\LaTeX}). If this is the case, then it will be ignored by the
\pai{hyperref}
package and the following warning will be reported
\begin{code}
\begin{verbatim}
Package hyperref Warning:
Token not allowed in a PDFDocEncoded string:
\end{verbatim}
\end{code}
To get around this problem you can use the
\begin{lscommand}
  \csi{texorpdfstring}[\bs TeX{} text: m, outline text: m]
\end{lscommand}
command. Its first argument, \carg{\TeX{} text}, is the text to be displayed
inside the document, while the second is the fallback for \pai{hyperref} to
use. An example would be to change
\begin{minted}{latex}
\section{\LaTeX{} is awesome!}
\end{minted}
to
\begin{minted}{latex}
\section{\texorpdfstring{\LaTeX}{LaTeX} is awesome!}
\end{minted}

While the above method may be necessary for some complicated math formulae
that need to be nicely printed in the outline, usually the replacement texts
are rather obvious for a given command. In this case you can use the
\begin{lscommand}
  \csi{pdfstringdefDisableCommands}[commands: m]
\end{lscommand}
command, which allows you to define general fallbacks for some commands. The
\carg{commands} argument is a list of redefinitions to be done when evaluating
outline titles. Commands may be redefined using the
\csi{RenewExpandableDocumentCommand} command. For a description of how to
redefine the commands see \autoref{sec:new_commands}. An example of using this
command would be
\begin{minted}{latex}
\pdfstringdefDisableCommands {
  \RenewExpandableDocumentCommand{\ldots}{}{...}
  \RenewExpandableDocumentCommand{\LaTeX}{}{LaTeX}
  \RenewExpandableDocumentCommand{\emph}{m}{*#1*}
}
\end{minted}

\section{Creating Presentations}\label{sec:beamer}
\begingroup
\LshortExampleSetup{
  paperheight=4.5cm,
  paperwidth=6cm,
  template=beamer,
  standalone,
}

\TeX{} and \LaTeX{} were primarily designed for creating text documents, and
that is where they shine. Still, it is also possible to use them for creating
presentations. These lack fancy animations and transitions---they are just
%$ 'lack fancy animations' - not entirely true, with some effort and JavaScript.
%$ animation package: https://ctan.org/pkg/animate
%$ and: https://tex.stackexchange.com/questions/22412/beamer-transition-effect
cleverly set up PDF files, after all. But on the plus side, you will be able to
tap into the strengths of \LaTeX{} such as logical markup, mathematical
typesetting and all the typesetting magic provided by the various \LaTeX{}
extension packages.

Historically, there have been several ways to create presentations; for example,
the standard \LaTeX{} \cli{slides} class, or the \pai{powerdot} package. While
these are still supported, this booklet will focus on the \pai*{beamer} package,
which is the most popular option these days. This section only scratches the
surface of \pai{beamer}'s capabilities. For a more in-depth tutorial please see
the User Guide~\cite{pack:beamer} distributed with \pai{beamer}.

\subsection{Basic Usage}

The \pai{beamer} package provides the \cli{beamer} class, which loads all the
necessary packages. The most basic environment is the \ei{frame} environment
%$ consider 'fundamental' instead of 'most basic', and the 'environment' after
%$ 'frame' may be redundant, since the beginning of the sentence has already
%$ identified 'frame' as an environment.
which adds a single page to your presentation.\footnote{Other presentation
  software often uses the term \enquote{slide} instead of frame, but slides are
  a slightly different concept in \pai{beamer}.}
\begin{example}
%!showbegin !hide
\documentclass{beamer}
%!showend !hide

\begin{document}
\begin{frame}
  Small is beautiful.
\end{frame}
\end{document}
\end{example}
The full syntax of the \ei{frame} is
\begin{lscommand}
  \ei*{frame}[options: o, title: m, subtitle: m]
\end{lscommand}
where \carg{options} is a key-value list applied to the frame, while
\carg{title} and \carg{subtitle} are displayed at the top of the slide. Despite
the curly brackets, these are optional arguments and have been
omitted in the previous example.
\begin{example}
\begin{document} %!hide
\begin{frame}{%
  Interesting title}{%
  Even more interesting
  subtitle}

  Not so interesting contents.
\end{frame}
\end{document} %!hide
\end{example}
The title and subtitle can also be specified using the \csi{frametitle} and
\csi{framesubtitle} commands.

As in the standard classes, you can specify a title, author and date in the preamble
by using the \csi{title}, \csi{author} and \csi{date} commands. The
\csi{maketitle} command uses these settings to create a title frame.
\begin{example}
\author{Jane Doe}
\title{Interesting Title}
\date{\today}
% ...

\begin{document} %!hide
\maketitle
\end{document} %!hide
\end{example}
Additionally, \csi{subtitle}, \csi{institute} and \csi{titlegraphic} are
available to add extra information to the title frame. The \csi{date} is often
repurposed to hold the conference name, since it is more informative than the
exact date. Some themes (discussed later) display these fields on each slide.
If you find that they are too long for such use, you may specify shorter
versions via an optional argument, like this:
\begin{minted}{latex}
  \title[Short Title]{A Title That Is Too Long}
\end{minted}

Sectioning commands, such as \csi{section}, are meant to be used between frames
and define the structure of the presentation. Using them has no visible impact
on the frames themselves, but they allow for a  nice \csi{tableofcontents} with
properly hyperlinked entries.
\begin{example}
\begin{document} %!hide
\begin{frame}{Outline}
  \tableofcontents
\end{frame}

\section{A Section}
\begin{frame}
  % ...
\end{frame}
\section{A Longer Section}
% ...
\subsection{A Subsection}
% ...
%!hidebegin
\begin{frame}
  a dummy frame so the table of contents renders properly
\end{frame}
\end{document} %!hide
\end{example}

%$ Should this be \pai{beamer}?
\cli{beamer} provides special commands for emphasising information on the slide.
The \csi{alert} command typesets its argument using a bright red colour. The
\ei{block} environment displays its contents with a title and separation from
the rest of the text.
\begin{example}
\begin{document} %!hide
\begin{frame} %!hide
Here we will talk about
\alert{emphasis}. It is a
\emph{very} interesting topic.
\begin{block}{Font Shape}
  \emph{Italic type} is often
  used.
\end{block}
\begin{block}{Colour}
  A distinct \alert{colour}
  also works.
\end{block}
\end{frame} %!hide
\end{document} %!hide
\end{example}
%$ Previously just said 'Beamer' here. Changed for consistency with section.
\pai{beamer} redefines many \pai{amsmath}\footnote{\pai{amsmath} is loaded
  automatically by \cli{beamer}.} environments, including \ei{theorem} and
\ei{proof}, to also produce blocks.

You will find that most of the \LaTeX{} commands and environments, such as
\cs{ref}s or \env{itemize}, work just as expected with \cli{beamer}. A notable %$ \pai instead of \cli?
exception is the verbatim input and similar constructs described in
\autoref{sec:code_listings}. Using them requires you to pass the
\cargv{fragile} option to the frame options.
\begin{example}
\begin{document} %!hide
\begin{frame}[fragile]
  Inside \verb|fragile| frames
  you can use the verbatim
  freely.
  \begin{verbatim}
Hello!
  \end{verbatim}
\end{frame}
\end{document} %!hide
\end{example}

\subsection{Overlay Specification}

So far, our frames have been static---they show all of their contents
at once. It is, however, possible to show a frame piece-wise, in order to not
overwhelm an audience with too much information.

The simplest way to do this is the \csi{pause} command. It splits the frame so
that the content before the pause is presented on the current page, while the
rest of the frame is presented on the next page. These partial frames are called
slides.
\begin{example}[
  vertical_mode,
  to_page=3,
  paperheight=2.8cm,
  paperwidth=3.7cm,
  examplewidth=.95\linewidth,
]
\begin{document} %!hide
\begin{frame}
  This will be shown first. \pause Then this. \pause
  And finally this.
\end{frame}
\end{document} %!hide
\end{example}
While presenting, this has the effect of revealing sentences one by one.
%$ Not, strictly speaking, sentences. Statements?

%$ Consider simply 'A more powerful splitting command is', or similar.
If the \csi{pause} command is not enough for you, a more powerful variant
exists:
\begin{lscommand}
  \csi{onslide}[<«\bs carg{overlay specification}»>: c]
\end{lscommand}
The \carg{overlay specification} argument specifies which slides the contents
should appear on. It may be a single number (slides start at 1), a range of
numbers (separated by \cargv{-}) or a comma delimited list of the two.
\begin{chktexignore}
\begin{example}[
  vertical_mode,
  to_page=3,
  paperheight=2.8cm,
  paperwidth=3.7cm,
  examplewidth=.95\linewidth,
]
\begin{document} %!hide
\begin{frame}
  \onslide<1> One. \onslide<2> Two. \onslide<1,3> One and
  three. \onslide<2-> Two onwards. \onslide<1-2> One to two.
\end{frame}
\end{document} %!hide
\end{example}
\end{chktexignore}
%$ 'together with a counter' superfluous.
%$ Alt. 'Internally, the \csi{pause} command uses \csi{onslide} with a counter'
The \csi{pause} command uses \csi{onslide} internally together with a counter,
so these can be mixed together as desired (although care is required to set the
slide numbers correctly).

A similar command is \csi{uncover}. It accepts the same optional \carg{overlay
  specification} argument, but only applies it to its mandatory argument.
\begin{chktexignore}
\begin{example}[
  vertical_mode,
  to_page=3,
  paperheight=2.8cm,
  paperwidth=3.7cm,
  examplewidth=.95\linewidth,
]
\begin{document} %!hide
\begin{frame}
  \uncover<1>{One.} \uncover<2>{Two.} \uncover<1,3>{One
    and three.} All.
\end{frame}
\end{document} %!hide
\end{example}
\end{chktexignore}
There are more commands that accept \carg{overlay specification} argument, and
many more preexisting commands and environments that are extended by
\pai{beamer} to support them. Refer to its User Guide~\cite{pack:beamer} for
more examples.

%$ 'not shown at all' untrue; does get shown eventually.
By default, content remains invisible until uncovered. If you
don't want a lot of empty space on your slides, or if you prefer not to
surprise your audience, you may adjust this behaviour to typeset the covered
text as transparent. To do so, use the \csi{setbeamercovered} with the
\cargv{transparent} option.
\begin{example}[
  vertical_mode,
  to_page=3,
  paperheight=2.8cm,
  paperwidth=3.7cm,
  examplewidth=.95\linewidth,
]
\setbeamercovered{transparent}
% ...

\begin{document} %!hide
\begin{frame}
  This will be shown first. \pause Then this. \pause
  And finally this.
\end{frame}
\end{document} %!hide
\end{example}
%$ No mention any of 'the other possible arguments' besides 'dynamic'.
Among the other possible arguments to these commands, \cargv{dynamic} makes the
text increasingly transparent the later it is
shown.\footnote{The effect is amplified a bit in the below example, so it is
  more visible. The default variant works better when there are more
  \csi{pause}s.}
\begin{example}[
  vertical_mode,
  to_page=3,
  paperheight=2.8cm,
  paperwidth=3.7cm,
  examplewidth=.95\linewidth,
]
\setbeamercovered{dynamic}
\setbeamercovered{still covered={\opaqueness<1>{40}\opaqueness<2>{10}}} %!hide
% ...

\begin{document} %!hide
\begin{frame}
  This will be shown first. \pause Then this. \pause
  And finally this.
\end{frame}
\end{document} %!hide
\end{example}

\subsection{Customisation}

If the default appearance of the presentation is not to your liking,
\pai{beamer} provides a lot of options to customise it. The most basic commands
are \csi{usetheme} and \csi{usecolortheme}. These allow you to choose from a
predefined set of themes that alter the style and colours of the presentation.

\begin{example}
\usetheme{Madrid}
\usecolortheme{wolverine}
\author{Jane Doe}
\title{Title}
\date{Yesterday}
% ...

\begin{document} %!hide
\begin{frame}{Title}
  Normal text.
  \alert{Alerted text}.
  \begin{block}{Block}
    A block.
  \end{block}
\end{frame}
\end{document} %!hide
\end{example}
The full list of available themes can be viewed
at reference~\cite{AnotherBeamerThemeMatrix}.

The \csi{usefonttheme} command can be used in a similar fashion. By default the
text in frames is typeset using a sans serif font. This choice makes sense,
because it is easier to read on lower resolution projectors, but is less
relevant when using a high-resolution display. You can pass the \cargv{serif}
option to \csi{usefonttheme} to switch to the default \LaTeX{} font.
\begin{example}
\usefonttheme{serif}
% ...

\begin{document} %!hide
\begin{frame}{Is serif better?}
  Often repeated assertion is
  that serif fonts are easier
  to read on paper, however,
  this is not scientifically
  confirmed.
\end{frame}
\end{document} %!hide
\end{example}

%$ Use 'fine grained' or 'coarse grained'; 'more granular' is ambiguous.
Greater customisation is possible using the
\begin{lscommand}
  \csi{setbeamerfont}[element: m, attributes: m] \\
  \csi{setbeamercolor}[element: m, attributes: m]
\end{lscommand}
commands. The \carg{element} argument is the element to which the customised
font or colour should be applied, while the \carg{attributes} specifies font
attributes, such as \cargv{size}, \cargv{shape} and \cargv{family}, and the
foreground and background colours, \cargv{fg} (and \cargv{bg}. Fonts are
further discussed in \autoref{sec:fontspec}, while colours are covered in
\autoref{sec:colors}.
\begin{example}
\setbeamercolor{frametitle}{
  fg=red,
  bg=lime,
}
\setbeamerfont{block title}{
  series=\bfseries,
  family=\ttfamily,
}
% ...

\begin{document} %!hide
\begin{frame}{Frame Title}
  Normal text.
  \begin{block}{Block title}
    More text.
  \end{block}
\end{frame}
\end{document} %!hide
\end{example}

By default, \LaTeX{} typesets mathematics in serif fonts.
When \pai{beamer} attempts the same typesetting in a sans serif font, not all
symbols may exist (Greek letters, for example).
If you presentation is mathematically heavy, it may be best to switch to a
sans serif math font and matching text font.
Switching math fonts is described further in \autoref{sec:math_fonts}.
\begin{example}
\usepackage{unicode-math}
\setsansfont{Fira Sans}
\setmathfont{Fira Math}
\setoperatorfont{\mathsf}
% ...

\begin{document} %!hide
\begin{frame} %!hide
With these fonts Greek
variables don't look silly
next to the Latin ones.
\[ x + b = \chi + \beta \]
\[ \lim_{x \to 0}
   \frac{\sin(x)}{x} = 1 \]
\end{frame} %!hide
\end{document} %!hide
\end{example}

%$ earlier: 'slides are a slightly different concept in beamer'.
%$ Slides are rendered onto 4 : 3 ..., but not just slides.
By default, \pai{beamer} typesets onto \(4 : 3\) aspect-ratio pages. If the
%$ not 'it'; could refer to changing the projector
projector uses some other aspect ratio, you may prefer to match. You
may do so, by passing an \cargv{aspectratio} to the \cs{documentclass} options.
It accepts a single integer that encodes the desired ratio. For example,
\cargv{169} is interpreted as \(16:9\), while \cargv{54} is interpreted as
\(5:4\). Wider ratios are especially useful if you want to include a sidebar
with the table of contents on your slides. %$ 'include' and 'on', or 'incorporate' and 'into'
\begin{example}[paperwidth=8cm]
%!showbegin !hide
\documentclass[
  aspectratio=169
]{beamer}
%!showend !hide

\begin{document}
\begin{frame}
  Wide is beautiful.
\end{frame}
\end{document}
\end{example}

\subsection{Handouts}

%$ As much as I like this feature of Beamer, increasingly other software _is_
%$ able to generate handouts. Maybe not as well, but in time...
A feature of \pai{beamer} is the ability to easily create handouts.
The simplest way to do this is to add the \cargv{handout} option to the
\cs{documentclass} command. Doing so will make the \cli{beamer} class ignore
all of the \csi{pause}s, and similar commands, to produce documents with fewer
pages.
\begin{example}[template=empty]
\documentclass[handout]{beamer}

% !hidebegin
\geometry{
  paperheight=\height,
  paperwidth=\width,
}

\setbeamersize{
  text margin left=0.6cm,
  text margin right=0.6cm,
}
% !hideend
\begin{document}
\begin{frame}
  All of the \pause text is on
  \pause a single slide, \pause
  even though \pause pauses
  are present.
\end{frame}
\end{document}
\end{example}
This mode is also useful while creating a presentation, as it previews the
frames as a whole without uncovering effects.

%$ Suggest a review of this paragraph, for clarity.
Internally, handout generation is accomplished by using overlay specifications
with modes. %$ We haven't really talked about 'modes', is this a formal term?
The default mode is \cargv{beamer}, while the \cargv{handout} option
switches to \cargv{handout} mode.
You can specify modes explicitly within an
overlay specification by passing \cargv{\carg{mode}:\carg{spec}}. Several mode
specifications can be specified by separating them with \cargv{|}~(vertical
bar). If no mode is specified,
\cargv{beamer} mode is assumed.
\begin{example}[
  template=empty,
  vertical_mode,
  to_page=2,
  paperheight=3cm,
  paperwidth=4cm,
  examplewidth=.7\linewidth,
]
\documentclass[handout]{beamer}

% !hidebegin
\geometry{
  paperheight=\height,
  paperwidth=\width,
}

\setbeamersize{
  text margin left=0.6cm,
  text margin right=0.6cm,
}
% !hideend
\begin{document}
\begin{frame}
  Some text \pause with pauses.
  \onslide<beamer: 3-| handout: 2-> This text
  will be on a separate slide, even in handout.
\end{frame}
\end{document}
\end{example}
Often you will want to show some parts only in one mode---to provide
additional commentary in handouts, or to hide things that only make sense
during the presentation. This is easily achieved by using a special zeroth
slide that is not included in the rendered document. An example of doing so
is presented in \autoref{lst:zero_slides}.
%$ Careful here: the slide is not included in the compiled (should that be
%$ rendered?) document, but it _is_ included when _compiling_ the document.
%$ In this example, the slide is omitted from the final PDF, but the graphic
%$ file still needs to exist because the frame is still processed by beamer.
%$\begin{frame}<0>
%$\begin{tikzpicture}[remember picture,overlay]
%$\node[at=(current page.center)]
%$	{\includegraphics[width=0.95\paperwidth]{this_still_needs_to_exist.pdf}};
%$\end{tikzpicture}
%$\end{frame}
%$ (end of example)
\begin{listing}
  \begin{example}[
    template=empty,
    vertical_pages,
    to_page=2,
    paperheight=3cm,
    paperwidth=4cm,
    examplewidth=.7\linewidth,
  ]
\documentclass[handout]{beamer}

% !hidebegin
\geometry{
  paperheight=\height,
  paperwidth=\width,
}

\setbeamersize{
  text margin left=0.6cm,
  text margin right=0.6cm,
}
% !hideend
\begin{document}
\begin{frame}
  A frame.
\end{frame}
\begin{frame}<handout: 0>
  A frame only visible in presentation.
\end{frame}
\begin{frame}<0>
  A frame only visible in handout.
\end{frame}
\end{document}
\end{example}
  \caption{An example of using zeroth slides to hide content in presentation and
    in handout.}\label{lst:zero_slides}
\end{listing}
\endgroup

Handouts created in this way still use the slide structure---titles, lots of
empty space, and bright colours.
As an alternative, the \cargv{article} mode will typeset the document in a
fashion similar to a regular article. Due to the more pronounced differences
between layouts, this may require more tweaking than the \cargv{handout} mode,
but the result will often be much more suitable for a printed handout.

%$ Strictly speaking, this is a LaTeX article with some beamer definitions.
%$ Don't know that it's accurate to call this a 'beamer mode'.
Switching to the \cargv{article} mode differs from switching to other
\pai{beamer} modes. Rather than passing options to the class, the class is
replaced by \cli{article} and the \pai{beamerarticle} is included in the
preamble. See \autoref{lst:article_mode} for an example. The
\pai{beamerarticle} package defines all of the \pai{beamer} commands and
environments, so that they will produce sensible output in the article.
\begin{listing}
  \begin{lined}{\textwidth}
    \begin{example}[standalone]
%!showbegin !hide
% \documentclass{beamer}
\documentclass{article}
%!showend !hide

\usepackage{beamerarticle}

\title{Handouts}
\author{Me and Myself}

\begin{document}
\maketitle

\begin{frame}
  Handouts are \alert{very}
  important.
\end{frame}

\begin{frame}
  Almost as important as the
  presentations.
\end{frame}
\end{document}
\end{example}
  \end{lined}
  \caption{An example of using \pai{beamer} in article
    mode.}\label{lst:article_mode}
\end{listing}

Because the concept of slides is not present in an article,\footnote{To be more
  precise, only contents of the first slide are typeset.} the overlay
specifications will be useless. However the special zeroth slide
works normally and can be used to hide content.
\begin{example}[standalone, paperheight=3cm]
%!hidebegin
\usepackage{beamerarticle}

\begin{document}
%!hideend
\begin{frame}<0>
  Additional information for
  the article.
  \uncover<article: 2->{This
    won't print.}
\end{frame}

\begin{frame}<article: 0>
  Additional information for
  the presentation.
\end{frame}
\end{document} %!hide
\end{example}

Because the overlay specification is often used to provide contents for a
single mode, there exists a short cut: f you
provide only the mode name, this will be the same as setting all other modes to
zeroth slide and leaving the first slide for the specified mode. This usually
results in a more readable code.
\begin{example}[standalone, paperheight=3cm]
%!hidebegin
\usepackage{beamerarticle}

\begin{document}
%!hideend
\begin{frame}<article>
  Content for the article.
\end{frame}

\begin{frame}<beamer>
  Content for the presentation.
\end{frame}
\end{document} %!hide
\end{example}


%$ Realising this is not the comprehensive guide to LaTeX, here's a few titbits
%$ that may or may not be useful and within scope:

%$ To use the default theme, but without navigation symbols (in the preamble):
%$ \beamertemplatenavigationsymbolsempty
%$ No mention of how to disable these in the section.

%$ The default font, Computer Modern, font does not support italicised sans-
%$ serif fonts (at least, not in bold or semi-bold anyway, but I think in
%$ general). This means all your emphasised text renders in Roman upright,
%$ same as everything else. The Latin Modern font does support emphasised
%$ sans serif. In the preamble again:
%$ \usepackage{lmodern}

%$ Latin Modern's font weight is a little light for my liking, while its bold
%$ is very bold indeed (technically, it's a bold extended (wide) font). What we
%$ want, is something in between. In the case of Latin Modern, we have a
%$ semi-bold condensed (sbc) variant, which we now set as the default style for
%$ the structure element. This is inherited by everything else; normal text,
%$ altered text, block titles, enumerators, and so on. In the preamble:
%$ \setbeamerfont{structure}{size=\normalsize, series=\fontseries{sbc}}
%$ \AtBeginDocument{\usebeamerfont{structure}}

%$ To include external graphics files with TikZ, you may need to enable the
%$ 'remember picture' and 'overlay' options, and process the document twice
%$ (pictures will be incorrectly placed after the first run).
%$ \begin{frame}[plain]
%$ \begin{tikzpicture}[remember picture,overlay]
%$ \node[at=(current page.center)]
%$   {\includegraphics[width=1.01\paperwidth]{Art/picture.pdf}};
%$ \node[text=blue,text width=0.5\paperwidth,align=right] at
%$   ([xshift=0.25\paperwidth,yshift=-0.20\paperheight]current page.north)
%$   {\large{\textbf{Overlaid Title}}};
%$ \end{tikzpicture}
%$ \end{frame}

%$ Changing the frame title to a smaller but bold font, with an underline.
%$ Again, the preamble.
%$ \setbeamertemplate{frametitle}
%$   {\bfseries\large\insertframetitle\par\vskip-6pt\color{black}\leavemode}
%$ Adding \leaders\hrule height 0.6pt\hfill\kern0pt after \leavemode (and before
%$ the } adds an underline. There are subtleties between LaTeX vs pdfTeX.

%$ Changing the bullet-point vertical spacing, the markers, and their colour
%$ can be accomplished by redefining Beamer's itemize environment, but it's
%$ easier to create a new environment from scratch. Too long and specialised a
%$ discussion for this booklet, I would think.

%%%%%%%%%%%%%%%%%%%%%%%%%%%%%%%%%%%%%%%%%%%%%%%%%%%%%%%%%%%%%%%%%
%%%%%%%%%%%%%%%%%%%%%%%%%%%%%%%%%%%%%%%%%%%%%%%%%%%%%%%%%%%%%%%%%
\setcounter{chapter}{4}
\newcommand{\graphicscompanion}{\emph{The \LaTeX{} Graphics Companion}~\cite{graphicscompanion}} 
\newcommand{\hobby}{\emph{A User's Manual for \MP{}}~\cite{metapost}}
\newcommand{\hoenig}{\emph{\TeX{} Unbound}~\cite{unbound}}
\newcommand{\graphicsinlatex}{\emph{Graphics in \LaTeXe{}}~\cite{ursoswald}}

\chapter{Producing Mathematical Graphics}
\label{chap:graphics}

\begin{intro}
Most people use \LaTeX\ for typesetting their text. But as the non content and
structure oriented approach to authoring is so convenient, \LaTeX\ also offers a,
if somewhat restricted, possibility for producing graphical output from textual 
descriptions. Furthermore, quite a number of \LaTeX\ extensions have been created 
in order to overcome these restrictions. In this section, you will learn about a 
few of them.
\end{intro}

\section{Overview}

The \ei{picture} environment allows programming pictures directly in
\LaTeX. A detailed
description can be found in the \manual. On the one hand, there are rather
severe constraints, as the slopes of line segments as well as the radii of
circles are restricted to a narrow choice of values.  On the other hand, the
\ei{picture} environment of \LaTeXe\ brings with it the \ci{qbezier}
command, ``\texttt{q}'' meaning ``quadratic''.  Many frequently used curves
such as circles, ellipses, or catenaries can be satisfactorily approximated
by quadratic B\'ezier curves, although this may require some mathematical
toil. If, in addition, a programming language like Java is used to generate
\ci{qbezier} blocks of \LaTeX\ input files, the \ei{picture} environment
becomes quite powerful.

Although programming pictures directly in \LaTeX\ is severely restricted,
and often rather tiresome, there are still reasons for doing so. The documents
thus produced are ``small'' with respect to bytes, and there are no additional
graphics files to be dragged along.

Packages like \pai{epic} and \pai{eepic} (described, for instance, in \companion), or
\pai{pstricks} help to eliminate the restrictions hampering the original \ei{picture} 
environment, and greatly strengthen the graphical power of \LaTeX.

While the former two packages just enhance the \ei{picture} environment, the \pai{pstricks}
package has its own drawing environment, \ei{pspicture}. The power of \pai{pstricks} stems
from the fact that this package makes extensive use of \PSi{} possibilities.
In addition, numerous packages have been written for specific purposes. One of them is
\texorpdfstring{\Xy}{Xy}-pic, described at the end of this chapter. A wide variety of these
packages is described in detail in \graphicscompanion{} (not to be confused with \companion).

Perhaps the most powerful graphical tool related with \LaTeX\ is \MP, the twin of
Donald E. Knuth's \MF. \MP{} has the very powerful and 
mathematically sophisticated programming language of \MF. Contrary to \MF,
which generates bitmaps, \MP{} generates encapsulated \PSi{} files, 
which can be imported in \LaTeX. For an introduction, see \hobby, or the tutorial on \cite{ursoswald}.

A very thorough discussion of \LaTeX{} and \TeX{} strategies for graphics (and fonts) can 
be found in \hoenig.

\section{The \texttt{picture} Environment}
\secby{Urs Oswald}{osurs@bluewin.ch}

\subsection{Basic Commands}

A \ei{picture} environment\footnote{Believe it or not, the picture environment works out of the
box, with standard \LaTeXe{} no package loading necessary.} is created with one of the two commands
\begin{lscommand}
\ci{begin}\verb|{picture}(|$x,y$\verb|)|\ldots\ci{end}\verb|{picture}|
\end{lscommand}
\noindent or
\begin{lscommand}
\ci{begin}\verb|{picture}(|$x,y$\verb|)(|$x_0,y_0$\verb|)|\ldots\ci{end}\verb|{picture}|
\end{lscommand}
The numbers $x,\,y,\,x_0,\,y_0$ refer to \ci{unitlength}, which can be reset any time
(but not within a \ei{picture} environment) with a command such as
\begin{lscommand}
\ci{setlength}\verb|{|\ci{unitlength}\verb|}{1.2cm}|
\end{lscommand}
The default value of \ci{unitlength} is \texttt{1pt}. The first pair, $(x,y)$, effects
the reservation, within the document, of rectangular space for the picture. The optional
second pair, $(x_0,y_0)$, assigns arbitrary coordinates to the bottom left corner of the
reserved rectangle. 

Most drawing commands have one of the two forms
\begin{lscommand}
\ci{put}\verb|(|$x,y$\verb|){|\emph{object}\verb|}|
\end{lscommand}
\noindent or
\begin{lscommand}
\ci{multiput}\verb|(|$x,y$\verb|)(|$\Delta x,\Delta y$\verb|){|$n$\verb|}{|\emph{object}\verb|}|\end{lscommand}
B\'ezier curves are an exception. They are drawn with the command
\begin{lscommand}
\ci{qbezier}\verb|(|$x_1,y_1$\verb|)(|$x_2,y_2$\verb|)(|$x_3,y_3$\verb|)|
\end{lscommand}
\newpage

\subsection{Line Segments}
\begin{example}
\setlength{\unitlength}{5cm}
\begin{picture}(1,1)
  \put(0,0){\line(0,1){1}}
  \put(0,0){\line(1,0){1}}  
  \put(0,0){\line(1,1){1}}  
  \put(0,0){\line(1,2){.5}}
  \put(0,0){\line(1,3){.3333}}
  \put(0,0){\line(1,4){.25}}  
  \put(0,0){\line(1,5){.2}}
  \put(0,0){\line(1,6){.1667}}
  \put(0,0){\line(2,1){1}}
  \put(0,0){\line(2,3){.6667}}
  \put(0,0){\line(2,5){.4}}
  \put(0,0){\line(3,1){1}}  
  \put(0,0){\line(3,2){1}}
  \put(0,0){\line(3,4){.75}}
  \put(0,0){\line(3,5){.6}}
  \put(0,0){\line(4,1){1}}
  \put(0,0){\line(4,3){1}}  
  \put(0,0){\line(4,5){.8}}
  \put(0,0){\line(5,1){1}}
  \put(0,0){\line(5,2){1}}
  \put(0,0){\line(5,3){1}}
  \put(0,0){\line(5,4){1}}
  \put(0,0){\line(5,6){.8333}}
  \put(0,0){\line(6,1){1}}
  \put(0,0){\line(6,5){1}}
\end{picture}
\end{example}
Line segments are drawn with the command
\begin{lscommand}
\ci{put}\verb|(|$x,y$\verb|){|\ci{line}\verb|(|$x_1,y_1$\verb|){|$length$\verb|}}|
\end{lscommand}
The \ci{line} command has two arguments:
\begin{enumerate}
  \item a direction vector,
  \item a length.
\end{enumerate}
The components of the direction vector are restricted to the integers
\[
  -6,\,-5,\,\ldots,\,5,\,6,
\]
and they have to be coprime (no common divisor except 1). The figure illustrates all
25 possible slope values in the first quadrant. The length is relative to \ci{unitlength}.
The length argument is the vertical coordinate in the case of a vertical line segment, the
horizontal coordinate in all other cases.

\subsection{Arrows}

\begin{example}
\setlength{\unitlength}{0.75mm}
\begin{picture}(60,40)
  \put(30,20){\vector(1,0){30}}
  \put(30,20){\vector(4,1){20}}
  \put(30,20){\vector(3,1){25}}
  \put(30,20){\vector(2,1){30}}
  \put(30,20){\vector(1,2){10}}
  \thicklines
  \put(30,20){\vector(-4,1){30}}
  \put(30,20){\vector(-1,4){5}}
  \thinlines
  \put(30,20){\vector(-1,-1){5}}
  \put(30,20){\vector(-1,-4){5}}
\end{picture}
\end{example}
Arrows are drawn with the command
\begin{lscommand}
\ci{put}\verb|(|$x,y$\verb|){|\ci{vector}\verb|(|$x_1,y_1$\verb|){|$length$\verb|}}|
\end{lscommand}
For arrows, the components of the direction vector are even more narrowly restricted than
for line segments, namely to the integers
\[
  -4,\,-3,\,\ldots,\,3,\,4.
\]
Components also have to be coprime (no common divisor except 1). Notice the effect  of the
\ci{thicklines} command on the two arrows pointing to the upper left.

\subsection{Circles}

\begin{example}
\setlength{\unitlength}{1mm}
\begin{picture}(60, 40)
  \put(20,30){\circle{1}}
  \put(20,30){\circle{2}}
  \put(20,30){\circle{4}}
  \put(20,30){\circle{8}}
  \put(20,30){\circle{16}}
  \put(20,30){\circle{32}}
  
  \put(40,30){\circle{1}}
  \put(40,30){\circle{2}}
  \put(40,30){\circle{3}}
  \put(40,30){\circle{4}}
  \put(40,30){\circle{5}}
  \put(40,30){\circle{6}}
  \put(40,30){\circle{7}}
  \put(40,30){\circle{8}}
  \put(40,30){\circle{9}}
  \put(40,30){\circle{10}}
  \put(40,30){\circle{11}}
  \put(40,30){\circle{12}}
  \put(40,30){\circle{13}}
  \put(40,30){\circle{14}}
  
  \put(15,10){\circle*{1}}
  \put(20,10){\circle*{2}}
  \put(25,10){\circle*{3}}
  \put(30,10){\circle*{4}}
  \put(35,10){\circle*{5}}
\end{picture}
\end{example}
The command
\begin{lscommand}
  \ci{put}\verb|(|$x,y$\verb|){|\ci{circle}\verb|{|\emph{diameter}\verb|}}|
\end{lscommand}
\noindent draws a circle with center $(x,y)$ and diameter (not radius) \emph{diameter}.
The \ei{picture} environment only admits diameters up to approximately 14\,mm,
and even below this limit, not all diameters are possible. The \ci{circle*}
command produces disks (filled circles).

As in the case of line segments, one may have to resort to additional packages, 
such as \pai{eepic} or \pai{pstricks}. 
For a thorough description of these packages, see \graphicscompanion.

There is also a possibility within the
\ei{picture} environment. If one is not afraid of doing the necessary calculations
(or leaving them to a program), arbitrary circles and ellipses can be patched
together from quadratic B\'ezier curves. 
See \graphicsinlatex\ for examples and Java source files.

\subsection{Text and Formulas}

\begin{example}
\setlength{\unitlength}{0.8cm}
\begin{picture}(6,5)
  \thicklines
  \put(1,0.5){\line(2,1){3}}
  \put(4,2){\line(-2,1){2}}
  \put(2,3){\line(-2,-5){1}}
  \put(0.7,0.3){$A$}
  \put(4.05,1.9){$B$}
  \put(1.7,2.95){$C$}
  \put(3.1,2.5){$a$}
  \put(1.3,1.7){$b$}
  \put(2.5,1.05){$c$}
  \put(0.3,4){$F=
    \sqrt{s(s-a)(s-b)(s-c)}$}  
  \put(3.5,0.4){$\displaystyle
    s:=\frac{a+b+c}{2}$}
\end{picture}
\end{example}
As this example shows, text and formulas can be written into a \ei{picture} environment with
the \ci{put} command in the usual way.

\subsection{\ci{multiput} and \ci{linethickness}}

\begin{example}
\setlength{\unitlength}{2mm}
\begin{picture}(30,20)
  \linethickness{0.075mm}
  \multiput(0,0)(1,0){26}%
    {\line(0,1){20}}
  \multiput(0,0)(0,1){21}%
    {\line(1,0){25}}
  \linethickness{0.15mm}    
  \multiput(0,0)(5,0){6}%
    {\line(0,1){20}}
  \multiput(0,0)(0,5){5}%
    {\line(1,0){25}}
  \linethickness{0.3mm}    
  \multiput(5,0)(10,0){2}%
    {\line(0,1){20}}
  \multiput(0,5)(0,10){2}%
    {\line(1,0){25}}
\end{picture}
\end{example}
The command
\begin{lscommand}
  \ci{multiput}\verb|(|$x,y$\verb|)(|$\Delta x,\Delta y$\verb|){|$n$\verb|}{|\emph{object}\verb|}|
\end{lscommand}
\noindent has 4 arguments: the starting point, the translation vector from one object to the next, 
the number of objects, and the object to be drawn. The \ci{linethickness} command applies to 
horizontal and vertical line segments, but neither to oblique line segments, nor to circles. 
It does, however, apply to quadratic B\'ezier curves!

\subsection{Ovals}

\begin{example}
\setlength{\unitlength}{0.75cm}
\begin{picture}(6,4)
  \linethickness{0.075mm}
  \multiput(0,0)(1,0){7}%
    {\line(0,1){4}}
  \multiput(0,0)(0,1){5}%
    {\line(1,0){6}}
  \thicklines
  \put(2,3){\oval(3,1.8)} 
  \thinlines
  \put(3,2){\oval(3,1.8)} 
  \thicklines
  \put(2,1){\oval(3,1.8)[tl]} 
  \put(4,1){\oval(3,1.8)[b]} 
  \put(4,3){\oval(3,1.8)[r]} 
  \put(3,1.5){\oval(1.8,0.4)}     
\end{picture}
\end{example}
The command
\begin{lscommand}
  \ci{put}\verb|(|$x,y$\verb|){|\ci{oval}\verb|(|$w,h$\verb|)}|
\end{lscommand}
\noindent or
\begin{lscommand}
  \ci{put}\verb|(|$x,y$\verb|){|\ci{oval}\verb|(|$w,h$\verb|)[|\emph{position}\verb|]}|
\end{lscommand}
\noindent produces an oval centered at $(x,y)$ and having width $w$ and height $h$. The optional 
\emph{position} arguments \texttt{b}, \texttt{t}, \texttt{l}, \texttt{r} refer to 
``top'', ``bottom'', ``left'', ``right'', and can be combined, as the example illustrates. 

Line thickness can be controlled by two kinds of commands: \\ 
\ci{linethickness}\verb|{|\emph{length}\verb|}|
on the one hand, \ci{thinlines} and \ci{thicklines} on the other. While \ci{linethickness}\verb|{|\emph{length}\verb|}|
applies only to horizontal and vertical lines (and quadratic B\'ezier curves), \ci{thinlines} and \ci{thicklines}
apply to oblique line segments as well as to circles and ovals. 


\subsection{Multiple Use of Predefined Picture Boxes}

\begin{example}
\setlength{\unitlength}{0.5mm}
\begin{picture}(120,168)
\newsavebox{\foldera}
\savebox{\foldera}
  (40,32)[bl]{% definition 
  \multiput(0,0)(0,28){2}
    {\line(1,0){40}}
  \multiput(0,0)(40,0){2}
    {\line(0,1){28}}
  \put(1,28){\oval(2,2)[tl]}
  \put(1,29){\line(1,0){5}}
  \put(9,29){\oval(6,6)[tl]}
  \put(9,32){\line(1,0){8}}
  \put(17,29){\oval(6,6)[tr]}
  \put(20,29){\line(1,0){19}}
  \put(39,28){\oval(2,2)[tr]}  
}
\newsavebox{\folderb}
\savebox{\folderb}
  (40,32)[l]{%         definition 
  \put(0,14){\line(1,0){8}}
  \put(8,0){\usebox{\foldera}}
}
\put(34,26){\line(0,1){102}} 
\put(14,128){\usebox{\foldera}}
\multiput(34,86)(0,-37){3}
  {\usebox{\folderb}} 
\end{picture}
\end{example}
A picture box can be \emph{declared} by the command
\begin{lscommand}
  \ci{newsavebox}\verb|{|\emph{name}\verb|}|
\end{lscommand}
\noindent then \emph{defined} by  
\begin{lscommand}
  \ci{savebox}\verb|{|\emph{name}\verb|}(|\emph{width,height}\verb|)[|\emph{position}\verb|]{|\emph{content}\verb|}|
\end{lscommand}
\noindent and finally arbitrarily often be \emph{drawn} by
\begin{lscommand}
  \ci{put}\verb|(|$x,y$\verb|)|\ci{usebox}\verb|{|\emph{name}\verb|}|
\end{lscommand}

The optional \emph{position} parameter has the effect of defining the
`anchor point' of the savebox. In the example it is set to \texttt{bl} which
puts the anchor point into the bottom left corner of the savebox. The other
position specifiers are \texttt{t}op and \texttt{r}ight.

The \emph{name} argument refers to a \LaTeX{} storage bin and therefore is
of a command nature (which accounts for the backslashes in the current
example). Boxed pictures can be nested: In this example, \ci{foldera} is
used within the definition of \ci{folderb}.

The \ci{oval} command had to be used as the \ci{line} command does not work if the segment length is less than 
about 3\,mm.

\subsection{Quadratic B\'ezier Curves}

\begin{example}
\setlength{\unitlength}{0.8cm}
\begin{picture}(6,4)
  \linethickness{0.075mm}
  \multiput(0,0)(1,0){7}
    {\line(0,1){4}}
  \multiput(0,0)(0,1){5}
    {\line(1,0){6}}
  \thicklines
  \put(0.5,0.5){\line(1,5){0.5}}    
  \put(1,3){\line(4,1){2}} 
  \qbezier(0.5,0.5)(1,3)(3,3.5)
  \thinlines   
  \put(2.5,2){\line(2,-1){3}}
  \put(5.5,0.5){\line(-1,5){0.5}}
  \linethickness{1mm}
  \qbezier(2.5,2)(5.5,0.5)(5,3)
  \thinlines
  \qbezier(4,2)(4,3)(3,3)
  \qbezier(3,3)(2,3)(2,2)
  \qbezier(2,2)(2,1)(3,1)
  \qbezier(3,1)(4,1)(4,2)
\end{picture}
\end{example}
As this example illustrates, splitting up a circle into 4 quadratic B\'ezier curves
is not satisfactory. At least 8 are needed. The figure again shows the effect of
the \ci{linethickness} command on horizontal or vertical lines, and of the 
\ci{thinlines} and the \ci{thicklines} commands on oblique line segments. It also 
shows that both kinds of commands affect quadratic B\'ezier curves, each command
overriding all previous ones.

Let $P_1=(x_1,\,y_1),\,P_2=(x_2,\,y_2)$ denote the end points, and $m_1,\,m_2$ the
respective slopes, of a quadratic B\'ezier curve. The intermediate control point 
$S=(x,\,y)$ is then given by the equations
\begin{equation} \label{zwischenpunkt}
  \left\{
    \begin{array}{rcl}
      x & = & \displaystyle \frac{m_2 x_2-m_1x_1-(y_2-y_1)}{m_2-m_1}, \\
      y & = & y_i+m_i(x-x_i)\qquad (i=1,\,2).
    \end{array}
  \right.
\end{equation}
\noindent See \graphicsinlatex\ for a Java program which generates
the necessary \ci{qbezier} command line.

\subsection{Catenary}

\begin{example}
\setlength{\unitlength}{1cm}
\begin{picture}(4.3,3.6)(-2.5,-0.25)
\put(-2,0){\vector(1,0){4.4}}
\put(2.45,-.05){$x$}
\put(0,0){\vector(0,1){3.2}}
\put(0,3.35){\makebox(0,0){$y$}}
\qbezier(0.0,0.0)(1.2384,0.0)
  (2.0,2.7622) 
\qbezier(0.0,0.0)(-1.2384,0.0)
  (-2.0,2.7622)
\linethickness{.075mm}
\multiput(-2,0)(1,0){5}
  {\line(0,1){3}}
\multiput(-2,0)(0,1){4}
  {\line(1,0){4}}
\linethickness{.2mm}
\put( .3,.12763){\line(1,0){.4}}
\put(.5,-.07237){\line(0,1){.4}}
\put(-.7,.12763){\line(1,0){.4}}
\put(-.5,-.07237){\line(0,1){.4}}
\put(.8,.54308){\line(1,0){.4}}
\put(1,.34308){\line(0,1){.4}}
\put(-1.2,.54308){\line(1,0){.4}}
\put(-1,.34308){\line(0,1){.4}}
\put(1.3,1.35241){\line(1,0){.4}}
\put(1.5,1.15241){\line(0,1){.4}}
\put(-1.7,1.35241){\line(1,0){.4}}
\put(-1.5,1.15241){\line(0,1){.4}}
\put(-2.5,-0.25){\circle*{0.2}}
\end{picture}
\end{example}

In this figure, each symmetric half of the catenary $y=\cosh x -1$ is approximated by a quadratic
B\'ezier curve. The right half of the curve ends in the point \((2,\,2.7622)\), the slope there having the value 
\(m=3.6269\). Using again equation (\ref{zwischenpunkt}), we can 
calculate the intermediate control points. They turn out to be $(1.2384,\,0)$ and $(-1.2384,\,0)$. 
The crosses indicate points of the \emph{real} catenary. The error is barely noticeable, being less 
than one percent.

This example points out the use of the optional argument of the \\
\verb|\begin{picture}| command.
The picture is defined in convenient ``mathematical'' coordinates, whereas by the command
\begin{lscommand} 
  \ci{begin}\verb|{picture}(4.3,3.6)(-2.5,-0.25)|
\end{lscommand}
\noindent its lower left corner (marked by the black disk) is assigned the coordinates $(-2.5,-0.25)$. 

\subsection{Rapidity in the Special Theory of Relativity}

\begin{example}
\setlength{\unitlength}{0.8cm}
\begin{picture}(6,4)(-3,-2)
  \put(-2.5,0){\vector(1,0){5}}
  \put(2.7,-0.1){$\chi$}
  \put(0,-1.5){\vector(0,1){3}}
  \multiput(-2.5,1)(0.4,0){13}
    {\line(1,0){0.2}}
  \multiput(-2.5,-1)(0.4,0){13}
    {\line(1,0){0.2}}
  \put(0.2,1.4)
    {$\beta=v/c=\tanh\chi$}
  \qbezier(0,0)(0.8853,0.8853)
    (2,0.9640)
  \qbezier(0,0)(-0.8853,-0.8853)
    (-2,-0.9640)
  \put(-3,-2){\circle*{0.2}}
\end{picture}
\end{example}
The control points of the two B\'ezier curves were calculated with formulas (\ref{zwischenpunkt}).
The positive branch is determined by $P_1=(0,\,0),\,m_1=1$ and $P_2=(2,\,\tanh 2),\,m_2=1/\cosh^2 2$.
Again, the picture is defined in mathematically convenient coordinates, and the lower left corner
is assigned the mathematical coordinates $(-3,-2)$ (black disk).

\section{The TikZ \& PGF Graphics Package}

Today every \LaTeX output generation system can create nice vector graphics,
its just the interfaces that are rather diverse. The PGF package provides an
abstraction layer over these interface and lets you use simple commands to
conveniently create complex vector graphics right from inside your document. The
PGF package comes with its own 500+ page documentation
\cite{pfgplots}. So we are only going to scratch the surface of the package with this little
section.

For high level access to 
the PGF functions you should load the \pai{tikz} package.
With the \pai{tikz} package you can use highly efficient commands to
draw graphics right from inside your document Use the \ei{tikzpicture}
environment to wrap your instructions.
\begin{example}
\begin{tikzpicture}[scale=3]
  \clip (-0.1,-0.2)
     rectangle (1.95,1.2);
  \draw[step=.25cm,gray,very thin]
       (-1.4,-1.4) grid (3.4,3.4);
  \draw (-1.5,0) -- (2.5,0);
  \draw (0,-1.5) -- (0,1.5);
  \draw (0,0) circle (1cm);
  \filldraw[fill=green!20!white,
            draw=green!50!black]
    (0,0) -- (3mm,0mm) 
         arc (0:30:3mm) -- cycle;
\end{tikzpicture}
\end{example}
If  you know other programming languages you
may notice the familiar semicolon (\texttt{;}) character that is used
to separate the different commands. With the \ci{usetikzlibrary}
command in the preamble you can enable a wide variety of additional
features for drawing special shapes, like this box which is slightly bent.
\begin{example}
\usetikzlibrary{%
  decorations.pathmorphing}
\begin{tikzpicture}[
     decoration={bent,aspect=.3}]
 \draw [decorate,fill=lightgray]
        (0,0) rectangle (5.5,2);
 \node[circle,draw] 
        (A) at (.5,.5) {A};
 \node[circle,draw] 
        (B) at (5,1.5) {B};
 \draw[->,decorate] (A) -- (B);
 \draw[->,decorate] (B) -- (A);
\end{tikzpicture}
\end{example}

You can even draw diagrams that look as if they came straight from a book pascal
programming spruced up with some fancy backgrounds. The code is a bit
more daunting than the example above, so I will just show you the
result of this one.

\begin{center}
\begin{tikzpicture}[point/.style={coordinate},thick,draw=black!50,>=stealth',
                    tip/.style={->,shorten >=1pt},every join/.style={rounded corners},
                    skip loop/.style={to path={-- ++(0,#1) -| (\tikztotarget)}},
                    hv path/.style={to path={-| (\tikztotarget)}},
                    vh path/.style={to path={|- (\tikztotarget)}},
                 terminal/.style={
            rounded rectangle,
            minimum size=6mm,
            very thick,draw=black!50,
            top color=white,bottom color=black!20,
            font=\ttfamily\scriptsize},
                nonterminal/.style={
                       rectangle,
                       minimum size=6mm,
                       very thick,
                       draw=red!50!black!50,         % 50% red and 50% black,
                       top color=white,              % a shading that is white at the top...
                       bottom color=red!50!black!20, % and something else at the bottom
                       font=\itshape\scriptsize}]
\matrix[column sep=4mm] {
  % First row:
  & & & & & & & & & & & \node (plus) [terminal] {+};\\
  % Second row:
  \node (p1) [point] {}; &     \node (ui1)    [nonterminal] {unsigned integer}; &
  \node (p2) [point] {}; &     \node (dot)    [terminal]    {.};                &
  \node (p3) [point] {}; &     \node (digit) [terminal]     {digit};            &
  \node (p4) [point] {}; &     \node (p5)     [point] {};                       &
  \node (p6) [point] {}; &     \node (e)      [terminal]    {E};                &
  \node (p7) [point] {}; &                                                      &
  \node (p8) [point] {}; &     \node (ui2)    [nonterminal] {unsigned integer}; &
  \node (p9) [point] {}; &     \node (p10)    [point]       {};\\
  % Third row:
  & & & & & & & & & & & \node (minus)[terminal] {-};\\
};
{ [start chain]
  \chainin (p1);
  \chainin (ui1)   [join=by tip];
  \chainin (p2)    [join];
  \chainin (dot)   [join=by tip];
  \chainin (p3)    [join];
  \chainin (digit) [join=by tip];
  \chainin (p4)    [join];
  { [start branch=digit loop]
    \chainin (p3) [join=by {skip loop=-6mm,tip}];
  }
  \chainin (p5)    [join,join=with p2 by {skip loop=6mm,tip}];
  \chainin (p6)    [join];
  \chainin (e)     [join=by tip];
  \chainin (p7)    [join];
  { [start branch=plus]
    \chainin (plus) [join=by {vh path,tip}];
    \chainin (p8)    [join=by {hv path,tip}];
  }
  { [start branch=minus]
    \chainin (minus) [join=by {vh path,tip}];
    \chainin (p8)    [join=by {hv path,tip}];
  }
  \chainin (p8)    [join];
  \chainin (ui2)   [join=by tip];
  \chainin (p9)    [join,join=with p6 by {skip loop=-11mm,tip}];
  \chainin (p10)   [join=by tip];
}
\end{tikzpicture}
\end{center}

\pagebreak
And there is more, if you have to draw plots of numerical data or
functions, you should have a closer look at the  \pai{pgfplot}
package. It provides everything you need to draw plots. It can even
call the external \texttt{gnuplot} command to evaluate actual
functions you wrote into the graph.

%%% Local Variables:
%%% TeX-master: "lshort.tex"
%%% mode: flyspell
%%% TeX-PDF-mode: t
%%% End:

% The Not So Short Introduction to LaTeX
%
% Copyright (C) 1995--2022 Tobias Oetiker, Marcin Serwin, Hubert Partl,
% Irene Hyna, Elisabeth Schlegl and Contributors.
%
% This document is free software: you can redistribute it and/or modify it
% under the terms of the GNU General Public License as published by the Free
% Software Foundation, either version 3 of the License, or (at your option) any
% later version.
%
% This document is distributed in the hope that it will be useful, but WITHOUT
% ANY WARRANTY; without even the implied warranty of MERCHANTABILITY or FITNESS
% FOR A PARTICULAR PURPOSE.  See the GNU General Public License for more
% details.
%
% You should have received a copy of the GNU General Public License along with
% this document.  If not, see <https://www.gnu.org/licenses/>.

% !TEX root = ./lshort.tex
%%%%%%%%%%%%%%%%%%%%%%%%%%%%%%%%%%%%%%%%%%%%%%%%%%%%%%%%%%%%%%%%%
% Contents: Customising LaTeX output
% $Id$
%%%%%%%%%%%%%%%%%%%%%%%%%%%%%%%%%%%%%%%%%%%%%%%%%%%%%%%%%%%%%%%%%
\chapter{Customising \LaTeX}\label{chap:custom}

\begin{intro}
  Documents produced with the commands you have learned up to this
  point will look acceptable to a large audience. While they are not
  fancy-looking, they obey all the established rules of good
  typesetting, which will make them easy to read and pleasant to look at.

  However, there are situations where \LaTeX{} does not provide a
  command or environment that matches your needs, or the output
  produced by some existing command may not meet your requirements.

  In this chapter, I will try to give some hints on
  how to teach \LaTeX{} new tricks and how to make it produce output
  that looks different from what is provided by default.
\end{intro}

\section{New Commands, Environments and Packages}

At the beginning of this book we have mentioned that \LaTeX{} allows us to
write documents using logical markup, with commands like \csi{emph} or \csi{section}. There
may be however situations where \LaTeX{} does not provide an appropriate command for the content you want to write about.
You may have noticed that all the
commands I introduce in this book are typeset in a box, and that they show up
in the index at the end of the book. There is no \LaTeX{} markup to format example code or commands, but \LaTeX{} allows me to define my own commands for this purpose. I
can write:

\begin{example}
\begin{lscommand}
  \csi{dum}
\end{lscommand}
\end{example}

In this example, I am using both a new environment called
\ei{lscommand}, which is responsible for drawing the box around the
command, and a new command named \csi{csi}, which typesets the command
name and makes a corresponding entry in the index. Check
this out by looking up the \csi{dum} command in the index at the back
of this book, where you'll find an entry for \csi{dum}, pointing to
every page where I mentioned the \csi{dum} command.

If I ever decide that I do not like having the commands typeset in
a box any more, I can simply change the definition of the
\ei{lscommand} environment to create a new look. This is much
easier than going through the whole document to hunt down all the
places where I have used some generic \LaTeX{} commands to draw a
box around some word.

\subsection{New Commands}\label{sec:new_commands}

You have already learned some basic command creation in
\autoref{sec:simple_commands}. The main command is the
\begin{lscommand}
  \csi{NewDocumentCommand}[name:M, argspec:m, definition:m]
\end{lscommand}
It requires three arguments: the \carg{name} of the command you want to create,
the \carg{argspec} (argument specification) and the \emph{definition} of the
command.

The \carg{argspec} argument specifies the number and types of arguments the
command receives. The two most important types are \cargv{m}, for
\emph{mandatory} and \cargv{o} for \emph{optional}. To create a command that
takes two optional arguments, then two mandatory, then again one optional and
finally three mandatory you would write \cargv{oommommm}. If the \carg{argspec}
argument is empty then the command will take no arguments, as you have already
seen.

This example defines a new command called \csi{tnss}. This is
short for \enquote{The Not So Short Introduction to \LaTeX}. Such a command
could come in handy if you had to write the title of this book over
and over again.

\begin{example}
\NewDocumentCommand{\tnss}{}{%
  The not so Short Introduction
    to \LaTeX}
This is \enquote{\tnss} \ldots{}
\enquote{\tnss}
\end{example}

The next example illustrates how to define a new command that takes two
arguments. In order to refer to the received arguments you use
\mintinline{latex}|#1| for the first argument, \mintinline{latex}|#2| for the
second, and so on.

\begin{example}
\NewDocumentCommand{\txsit}{mm}
 {This is the \emph{#1}
  #2 Introduction to \LaTeX}

% in the document body:
\txsit{not so}{short}

\txsit{very}{long}
\end{example}

If your command accepts an optional argument, but the user does not supply one,
a special marker \cargv{-NoValue-} will be inserted instead.
\begin{example}
\NewDocumentCommand{\txsit}{om}
{This is the \emph{#1}
  #2 Introduction to \LaTeX}

% in the document body:
\txsit{definitive}

\txsit[very]{long}
\end{example}
In order to test whether the user supplied a value, use
\begin{lscommand}
  \csi{IfValueTF}[argument:m, value version:m, no value version:m]
\end{lscommand}
macro.
\begin{example}
\NewDocumentCommand{\MyCommand}{o}{
  \IfValueTF {#1} {
    Optional argument: #1.
  } {
    No optional argument given.
  }%
}

\MyCommand\\
\MyCommand[hello]
\end{example}

There are two variations of it: \csi{IfValueT} and \csi{IfValueF} which may
be used if you only need output for one of the branches. The example with
\cargv{-NoValue-} in the output could be fixed by writing
\begin{example}
\NewDocumentCommand{\txsit}{om}
{This is the
  \IfValueT{#1}{\emph{#1} }%
  #2 Introduction to \LaTeX}

% in the document body:
\txsit{definitive}

\txsit[very]{long}
\end{example}
The commands \csi{IfNoValueTF}, \csi{IfNoValueT} and \csi{IfNoValueF}
work exactly the same, but the value\slash{}no-value branches are swapped.

Often you will want to use optional arguments when present but use some default
values when the user does not provide them. This could be achieved by
\csi{IfValueTF}, but with the \cargv{O} argument a default value can be set directly.
It works like \cargv{o} but allows setting a default
if no value is supplied. Write

\begin{example}
\NewDocumentCommand{\txsit}{O{not so}m}
{This is the \emph{#1}
  #2 Introduction to \LaTeX}

% in the document body:
\txsit{definitive}

\txsit[very]{long}
\end{example}

Another useful argument specification is \cargv{s}, short for star. This
argument allows providing different definitions based on whether the
starred or non-starred version of command was issued by the user. It uses
\csi{IfBooleanTF} command (and its variations\csih{IfBooleanT}\csih{IfBooleanF})
that works like the \csi{IfValueTF} command.

\begin{chktexignore}
  \begin{example}
\NewDocumentCommand{\txsit}{sO{not so}m}
{This is the \emph{#2}
  #3 Introduction to \LaTeX%
  \IfBooleanT{#1}{%
    : Superstar Edition%
  }%
}

% in the document body:
\txsit{long}\\
\txsit*{long}
\end{example}
\end{chktexignore}

These are just the most common argument specifications. For a full description,
take a look at the~\cite{usrguide3}.

As we have discussed in \autoref{sec:simple_commands}, \LaTeX{} will not allow
you to create a new command that would overwrite an existing one, you can do so
using \csi{RenewDocumentCommand}. It uses the same argument specification
syntax as the \csi{NewDocumentCommand} command.

In some cases you might want to use the \csi{ProvideDocumentCommand} command.
It works like \csi{NewDocumentCommand}, but if the command is already defined,
\LaTeX{} will silently ignore the new definition. Yet another variant is the
\csi{DeclareDocumentCommand}. It always creates the given command, overwriting
old definition if it exists.

\subsection{New Environments}
The \csi{NewDocumentEnvironment} command lets you create your own environments. It has the
following syntax:

\begin{lscommand}
  \small
  \csi{NewDocumentEnvironment}[name:m, argspec:m, at begin:m, at end:m]
\end{lscommand}
The \carg{argspec} argument is the same as in the
\csi{NewDocumentCommand} command. The contents of \carg{at begin} and \carg{at
  end} arguments will be inserted respectively when the commands
\csi{begin}[name: m] and \csi{end}[name: m] is encountered. The example
presented in \autoref{lst:kingenv} illustrates the usage of this command.
\begin{listing}
  \begin{example}
\NewDocumentEnvironment{king}{}{%
  \emph{Listen! For the
    king made a statement:}%
  \\[1em]%
} {%
  \\[1em]%
  \emph{This concludes
    the king's statement.}%
}

\begin{king}
My humble subjects \ldots
\end{king}
\end{example}
  \caption{An example of using \csi{NewDocumentEnvironment}
    command.}\label{lst:kingenv}
\end{listing}

Note that when environment argument are read, they are read \emph{after} the
\csi{begin}[name: m] command. This may be especially counterintuitive when we
consider the \cargv{s} specification. \autoref{lst:kingsenv} illustrates this.
\begin{listing}
  \begin{example}
\NewDocumentEnvironment{king}{s}{%
  \IfBooleanTF{#1}{
    \begin{center}
      \emph{Thus spoke Charles I:}
      \\[1em]%
    \end{center}%
  } {
    \emph{Listen! For the
      king made a statement:}%
      \\[1em]%
  }%
} {%
  \\[1em]%
  \emph{This concludes
    the king's statement.}%
}

\begin{king}*
My humble subjects \ldots
\end{king}
\end{example}
  \caption{An example of using the \cargv{s} specifier when defining a new
    environment.}\label{lst:kingsenv}
\end{listing}
If you want to create a starred version of an environment (similar to the
\hologo{AmSLaTeX} environments) you have to define it separately.
\begin{minted}{latex}
\NewDocumentEnvironment{king}{moo} { ... } { ... }
\NewDocumentEnvironment{king*}{moo} { ... } { ... }
\end{minted}
Obviously you can use the same internal commands to define them, for example
renaming the previous implementation to \cargv{kinginternal} making the
\cargv{king} and \cargv{king*} a thin wrapper around it.

The \csi{NewDocumentEnvironment} also introduces a special argument
specification: \cargv{+b}, short for body.\footnote{The \cargv{+} indicates
  that it may contain multiple paragraphs.} It is only allowed as the last
argument in the \carg{argspec}. It allows you to receive the body of the
environment as an argument.
\begin{example}
\NewDocumentEnvironment{twice}{+b} {%
  First time:\\ #1

  Second time:\\ #1
} {}

\begin{twice}
This will be printed twice!
\end{twice}
\end{example}
While this makes one of the \carg{at begin}, \carg{at end} arguments redundant,
they are still required. (In the example above we provided an empty \carg{at
  end}.)

Do not overuse \cargv{+b} as this will both add limitations to the environments
like \csi{verb} not being allowed inside and it will slow down the
typesetting.

Similar to the \csi{NewDocumentCommand}, \LaTeX{} makes sure that you do not
define an environment that already exists. If you ever want to change an
existing environment, use the \csi{RenewDocumentEnvironment} command. Its
arguments are the same as the \csi{NewDocumentEnvironment} command.

\subsection{Copying commands}\label{sec:copyingcommands}

When redefining commands you may want to use the original version of the
command. Your initial code may look like this
\begin{minted}{latex}
\RenewDocumentCommand{\emph}{m}{%
  \emph{#1}~(\enquote{#1} is emphasised)%
}
\end{minted}
but when you try to compile the document you will get the error message
\begin{verbatim}
! TeX capacity exceeded, sorry [input stack size=5000].
\end{verbatim}

To understand why this happens it is instructive to consider how \TeX{} expands
the defined commands. The above \csi{RenewDocumentCommand} tells the \TeX{}
engine that whenever \csi{emph}[foo:vm] is seen it must replace it with
\mintinline{latex}|\emph{foo}~(\enquote{foo} is emphasised)|. You may already see the
problem here. In the next stage it will again replace the \csi{emph}[foo:vm]
yielding
\begin{minted}[breaklines]{latex}
\emph{foo}~(\enquote{foo} is emphasised)~(\enquote{foo} is emphasised)
\end{minted}
This process will never end and at some point \TeX{}
simply gives up.

Note that
\begin{minted}{latex}
\NewDocumentCommand{\oldemph}{m}{\emph{#1}}
\RenewDocumentCommand{\emph}{m}{%
  \oldemph{#1}~(\enquote{#1} is emphasised)%
}
\end{minted}
will suffer the same fate since \mintinline{latex}|\oldemph{...}| %chktex 11 
will  be replaced by \TeX{} with \mintinline{latex}|\emph{...}| and %chktex 11
the  cycle repeats.

In order to avoid this problem a special command exists
\begin{lscommand}
  \csi{NewCommandCopy}[name:M, command:M]
\end{lscommand}
It makes the \carg{name} the exact copy of the \carg{command}. The following example shows how this works
\begin{example}[examplewidth=0.35\linewidth]
\NewDocumentCommand{\foo}{}{Batman!}

\NewDocumentCommand{\newfoo}{}{\foo}
\NewCommandCopy{\copiedfoo}{\foo}

\RenewDocumentCommand{\foo}{}{Na Na Na}

\foo{} \newfoo{} \copiedfoo{}
\end{example}

This is precisely the behaviour we need in order to redefine the \csi{emph}
command as we have tried to do earlier.
\begin{example}[examplewidth=0.35\linewidth]
\NewCommandCopy{\oldemph}{\emph}
\RenewDocumentCommand{\emph}{m}{%
  \oldemph{#1}~(\enquote{#1}
  is emphasised)%
}

And here it \emph{comes}.
\end{example}

\subsection{Command-line \LaTeX}

If you work on a Unix-like OS, you might be using Makefiles to build your
\LaTeX{} projects. In that connection it might be interesting to produce
different versions of the same document by calling \LaTeX{} with command-line
parameters. If you add the following structure to your document:

\begin{minted}{latex}
\IfBooleanTF{\blackandwhite} {
  % "black and white" mode; do something...
} {
  % "color" mode; do something different...
}
\end{minted}

Now compile document like this:
\begin{verbatim}
xelatex '\NewCommandCopy{\blackandwhite}{\BooleanTrue}
    \input{test.tex}'
\end{verbatim}
First the command \verb|\blackandwhite| is defined as the \csi{BooleanTrue} macro which
holds a special value used in \csi{IfBooleanTF} checks. Then the actual file is
read with input. By setting \verb|\blackandwhite| to \csi{BooleanFalse} the
colour version of the document would be produced.

\subsection{Your Own Package}

If you define a lot of new environments and commands, the preamble of
your document will get quite long. In this situation, it is a good
idea to create a \LaTeX{} package containing all your command and
environment definitions. Use the \csi{usepackage}
command to make the package available in your document.

\begin{listing}
  \begin{lined}{\textwidth}
    \begin{minted}{latex}
\ProvidesExplPackage{demopack}{2022-05-05}{0.1}{%
  Package by Tobias Oetiker
}

\NewDocumentCommand{\tnss}{} {
  The~not~so~Short~Introduction~to~\LaTeX
}
\NewDocumentCommand{\txsit}{O{not~so}} {
  The~\emph{#1}~Short~Introduction~to~\LaTeX
}

\NewDocumentEnvironment{king}{} {
  \begin{quote}
} {
  \end{quote}
}
\end{minted}
  \end{lined}
  \caption{Example Package.}\label{package}
\end{listing}

Writing a package basically consists of copying the contents of your document
preamble (with minor adjustments) into a separate file with a name ending in
\texttt{.sty}. There is one special command,
\begin{lscommand}
  \csi{ProvidesExplPackage}[name:m, date:m, version:m, description:m]
\end{lscommand}
for use at the very beginning of your package file. This command tells the
\LaTeX{} to process the  file in \emph{expl mode}. The most visible effect of
this is that all whitespace is ignored. You may have noticed that in many of
the examples above we had to end most lines with \ai{\%} to get correct spacing
in the output. In \emph{expl mode}, spaces have to be added explicitly if
needed at all. They are usually quite rare when writing a package. To insert
spaces, use the \ai{\~} character, which normally denotes non-breaking
space.\footnote{If you want to insert non-breaking space in \emph{expl mode},
  use \csi{nobreakspace}.} Paragraphs can be started with the \csi{par} command.

The arguments are used to provide information about package in the log file. If
you use this package and look at the log file you will find
\begin{verbatim}
Package: demopack 2022-05-05 v0.1 Package by Tobias Oetiker
\end{verbatim}
in the \eei{.log} file.

\csi{ProvidesExplPackage} will also issue a sensible error message when you try
to include a package twice. \autoref{package} shows a small example package
that contains the commands defined in the examples above.

\section{Fonts and Sizes}\label{sec:fontsize}

\subsection{Font Changing Commands}\index{font}\index{font size}

\LaTeX{} fonts are influenced by four parameters
\begin{description}
  \item[family] The collection of fonts. For example, \enquote*{Latin Modern
      Roman} or \enquote*{Source Code Pro}.
  \item[series] The weight of the font. For example, \enquote*{bold} or
    \enquote*{medium}.
  \item[shape] The shape of glyphs within a font family. For example,
    \enquote*{small caps} or \enquote*{italics}.
  \item[size] The size of the glyphs. For example, \enquote*{\qty{10}{pt}} or
    \enquote*{\qty{12}{pt}}.
\end{description}
\LaTeX{} automatically chooses the appropriate font family, series, shape and
size based on the logical structure of the document (sections, footnotes,
emphasis, \ldots). It is possible however to instruct \LaTeX{} manually which
font to use. It is important to note that not every combination of
family\slash{}series\slash{}shape exists as an actual font. \LaTeX{} will
complain if you try for something that does not exist.

\LaTeX{} predefines three font families to use throughout the document: the
upright or roman family accessible via \csi{textrm}, the \textsf{sans serif
  family} accessible via \csi{textsf} and monospace or typewriter family
accessible via \csi{texttt}.
\begin{example}
\textrm{Roman is the default
  in articles.} \\
\textsf{Sans serif is used in
  presentations.} \\
\texttt{Monospace is used in
  verbatim code blocks.}
\end{example}

There are only two predefined \LaTeX{} series: medium (\csi{textmd}) and bold
(\csi{textbf}).
\begin{example}
\textmd{The default.} \\
\textbf{Bold font.}
\end{example}

Shapes are a bit more complicated. The three basic shapes are: italics
(\csi{textit}), oblique or slanted\footnote{Oblique shape differs from the
  italics in that italic shape uses different glyphs while oblique shape uses
  the same glyphs but slanted. The difference is really obvious
  \fontspec{cmunui.otf} when you look at the unslanted italic
  font.} (\csi{textsl}) and small capitals (\csi{textsc}).
\begin{example}
\textit{Italic shape.} \\
\textsl{Slanted shape.} \\
\textsc{Small Capitals.}
\end{example}
However there are two additional shapes that are not provided by default
\LaTeX{} fonts: swash (\csi{textsw}), for decorative fonts and spaced caps and
small caps (\csi{textssc}). These are rarely used but may come in handy when
using custom fonts as described in \autoref{sec:fontspec}.
\begin{example}
\setmainfont{EB Garamond} %!hide
Question vs. \textsw{Question}
\end{example}
In addition two virtual shapes are provided: upright (\csi{textup}) and
upper-lowercase (\csi{textulc}). These are not actually shapes but utility
commands. The former one switches back to upright font while the latter
disables small capitals. The command \csi{textnormal} is just the combination
of the two.
\begin{example}
\textsl{\textsc{Back to
  \textup{upright.}}} \\
\textsl{\textsc{Back to
  \textulc{lowercase.}}} \\
\textsl{\textsc{Back to
  \textnormal{normal.}}}
\end{example}

All of the commands described above also exist in their switch version. Instead
of receiving the text via argument, they change the font permanently until it is
changed again. For example, the switch version of \csi{textit} and \csi{textrm}
are \csi{itshape} and \csi{rmfamily}, respectively. While the argument versions
are useful for defining commands, switch versions are especially useful when
defining your own environments.
\begin{example}
Only \textit{argument} is
affected. After \itshape
everything is in italics
until \upshape is encountered.
\end{example}
Both argument and switch versions of the described commands are presented in
\autoref{tbl:fonts}.
\begin{table}
  \caption{Default font changing commands of \LaTeX.}\label{tbl:fonts}
  \begin{tabular}{@{}lll@{}}
    \toprule
    Argument Command         & Switch           & Example                      \\
    \midrule
    \csi{textrm}[text:m]     & \csi{rmfamily}   & \textrm{\wi{roman}}          \\
    \csi{textsf}[text:m]     & \csi{sffamily}   & \textsf{\wi{sans serif}}     \\
    \csi{texttt}[text:m]     & \csi{ttfamily}   & \texttt{typewriter}          \\[6pt]
    \csi{textmd}[text:m]     & \csi{mdseries}   & \textmd{medium}              \\
    \csi{textbf}[text:m]     & \csi{bfseries}   & \textbf{\wi{bold face}}      \\[6pt]
    \csi{textup}[text:m]     & \csi{upshape}    & \textup{\wi{upright}}        \\
    \csi{textit}[text:m]     & \csi{itshape}    & \textit{\wi{italic}}         \\
    \csi{textsl}[text:m]     & \csi{slshape}    & \textsl{\wi{slanted}}        \\
    \csi{textsc}[text:m]     & \csi{scshape}    & \textsc{\wi{Small Caps}}     \\
    \csi{textsw}[text:m]     & \csi{swshape}    & \textsw{\wi{Queen of Swash}} \\[6pt]
    \csi{textnormal}[text:m] & \csi{normalfont} & \textnormal{document} font   \\
    \bottomrule
  \end{tabular}
\end{table}

When working with switch versions of fonts that are slanted right it is
important to remember about \wi{italic correction}. This is a small space after
the end of right slanting text that is sometimes necessary to avoid overlapping
letters. It is inserted using \csi{/} command.
\begin{chktexignore}  
\begin{example}
Without: {\itshape oof}bar \\
With: {\itshape oof\/}bar
\end{example}
\end{chktexignore}
The italic correction is handled automatically by the argument versions of the
commands.

In contrast to the previous font changing commands, the size of font can only
be controlled via switch versions. \LaTeX{} predefines some switches for
changing font size, see \autoref{sizes} and \autoref{tab:pointsizes} for their
description.
\begin{table}
  \caption{Commands changing font size.}\label{sizes}\index{font size}
  \begin{tabular}{@{}ll@{\qquad}ll@{}}
    \toprule
    Command                     & Size                             &
    Command                     & Size                               \\
    \midrule
    \csi{tiny}                  & \tiny tiny                       &
    \csi{Large}                 & \Large larger                      \\
    \csi{scriptsize}            & \scriptsize very small           &
    \csi{LARGE}                 & \LARGE very large                  \\
    \csi{footnotesize}          & \footnotesize  quite small       &
    \multirow{2}{*}{\csi{huge}} & \multirow{2}{*}{\huge huge}        \\
    \csi{small}                 & \small small                     &
                                &                                    \\
    \csi{normalsize}            & \normalsize  normal              &
    \multirow{2}{*}{\csi{Huge}} & \multirow{2.2}{*}{\Huge largest}   \\
    \csi{large}                 & \large large                     &
                                &                                    \\
    \bottomrule
  \end{tabular}
\end{table}

\begin{table}
  \caption[Absolute point sizes in standard classes.]{Absolute point sizes in
    standard classes depending on the class option. The default class option is
    \cargv{10pt}.}\label{tab:pointsizes}\label{tab:sizes}
  \sisetup{table-format=2.2}
  \begin{tabular}{@{}lSSS@{}}
    \toprule
                       & \multicolumn{3}{c}{Size (\unit{pt})}                                   \\
    \cmidrule(l){2-4}
    Command            & {\cargv{10pt}}                       & {\cargv{11pt}} & {\cargv{12pt}} \\
    \midrule
    \csi{tiny}         & 5                                    & 6              & 6              \\
    \csi{scriptsize}   & 7                                    & 8              & 8              \\
    \csi{footnotesize} & 8                                    & 9              & 10             \\
    \csi{small}        & 9                                    & 10             & 10.95          \\
    \csi{normalsize}   & 10                                   & 10.95          & 12             \\
    \csi{large}        & 12                                   & 12             & 14.4           \\
    \csi{Large}        & 14.4                                 & 14.4           & 17.28          \\
    \csi{LARGE}        & 17.28                                & 17.28          & 20.74          \\
    \csi{huge}         & 20.74                                & 20.74          & 24.88          \\
    \csi{Huge}         & 24.88                                & 24.88          & 24.88          \\
    \bottomrule
  \end{tabular}
\end{table}

When using these commands it is important to remember that the
line spacing is only updated after the paragraph ends. To avoid putting empty
lines before the closing curly brace you may use the \csi{par} command.
\begin{example}
{\Large Here the line spacing
  is not updated. A bit tight!}

{\Large Much better! I can
  breathe freely again!\par}
\end{example}

An arbitrary font size can be specified using the
\begin{lscommand}
  \csi{fontsize}[size: m, line skip: m]\csi{selectfont}
\end{lscommand}
command combo. The \carg{line skip} determines the height of the text line and
should be usually around \(1.2\) times larger than the \carg{size}.
\begin{example}[vertical_mode, examplewidth=0.7\linewidth]
\fontsize{2cm}{2.4cm}\selectfont A big one!
\end{example}

Fun fact: \LaTeX{} default font is a bit unusual in that it looks slightly
different depending on its size. The difference is presented in the table below
where the text written using different sizes was rescaled to the same height.
\begin{center}
  \begin{tabular}{@{}lll@{}}
    \toprule
    \csi{tiny}                           &
    \csi{normalsize}                     &
    \csi{Huge}                             \\
    \midrule
    \resizebox{!}{2em}{\tiny Text}       &
    \resizebox{!}{2em}{\normalsize Text} &
    \resizebox{!}{2em}{\Huge Text}         \\
    \bottomrule
  \end{tabular}
\end{center}

If you need to access even more font variants and shapes\footnote{For example
  the aforementioned \fontspec{cmunui.otf} upright italic shape.} check out
\citetitle{fntguide}~\cite{fntguide}.

\subsection{Danger, Will Robinson, Danger}

Note! Using explicit font setting commands defies the basic idea of
\LaTeX{} described in \autoref{sec:logical_structure}, which is to separate the
logical and visual markup. The fonts should get switched automatically
according to the requirements of the context. A simple rule of thumb: If you
use the same font changing command in several places in order to typeset a
special kind of information, you should use \csi{NewDocumentCommand} to define
a \enquote{logical wrapper command} for the font changing command.

\begin{example}
\NewDocumentCommand{\oops}{m}{%
 \textbf{#1}}
Do not \oops{enter} this room,
it's occupied by \oops{machines}
of unknown origin and purpose.
\end{example}

This approach has the advantage that you can decide at some later
stage that you want to use a visual representation of danger other
than \csi{textbf}, without having to wade through your document,
identifying all the occurrences of \csi{textbf} and then figuring out
for each one whether it was used for pointing out danger or for some other
reason.

\subsection{Advice}

To conclude this journey into the land of fonts and font sizes,
here is a little word of advice:\nopagebreak

\begin{quote}
  \underline{\textbf{Reme\(\mathfrak{mber}\)\Huge!}} \textit{The}
  \textsf{M\textbf{\LARGE O} \(\mathcal{R}\)\textsl{E}} fonts \Huge you
  \tiny use \footnotesize \textbf{in} a \small \texttt{document},
  \large \textit{the} \normalsize more \textsc{readable} and
  \textsl{\textsf{beautiful} it bec\large o\Large m\LARGE e\huge s}.
\end{quote}

\section{Custom Fonts with \pai{fontspec}}\label{sec:fontspec}

In the following examples we use Adobe Source fonts~\cites{sourceserif,
  sourcesans, sourcecodepro}. These fonts are included with \TeXLive{} \LaTeX{}
distributions and should be available in the directory
\begin{code}
  \nolinkurl{.../texmf-dist/fonts/opentype/adobe}
\end{code}
where the \cargv{...} denotes the install-path of \TeXLive{}.
\hologo{LuaTeX} checks this directory automatically, so it should work fine,
but if you are using \hologo{XeTeX} you must first install these fonts in your
system. You can also download and install them manually from the links provided
in the bibliography. Alternatively swap out their respective names with some other
fonts installed in your system.

Many free OpenType fonts are available at \url{https://fontlibrary.org/}.

\subsection{Main Document Fonts}

If you are not pleased with the default Latin Modern font, you can change it to
any font installed in your system using the \pai*{fontspec} package. It provides
three main commands for changing document fonts:
\begin{lscommand}
  \csi{setmainfont}[options: o, font: m] \\
  \csi{setsansfont}[options: o, font: m] \\
  \csi{setmonofont}[options: o, font: m]
\end{lscommand}
This commands change, respectively, the main font of the document, \textsf{the
  sans serif font used in the document} and \texttt{the monospace font in the
  document}.
\begin{example}
Normal text.
  \emph{Emphasised.} \\
\textsf{Sans serif text.
  \emph{Emphasised}.} \\
\texttt{Monospace text.
  \emph{Emphasised}.} \\

\setmainfont{Source Serif Pro}
\setsansfont{Source Sans Pro}
\setmonofont{Source Code Pro}
Normal text.
  \emph{Emphasised.} \\
\textsf{Sans serif text.
  \emph{Emphasised}.} \\
\texttt{Monospace text.
  \emph{Emphasised}.}
\end{example}
Note that it is best to put these commands in the preamble of your document,
because some fonts are frozen when the body starts.

The optional \carg{options} argument accepts key value lists that allow to
customise the font features. For example, many fonts contain,
old style numerals that are not used by default. You can pass
\cargv{Number=OldStyle} if you want to use them in your document.
\begin{example}
\setmainfont{Source Serif Pro}
0123456789

\setmainfont[
  Numbers=OldStyle,
]{Source Serif Pro}
0123456789
\end{example}

Some fonts also provide special glyphs for a given language. For example the
Latin Modern Font provides a special
\enquote{\setmainfont[Language=Polish]{Latin Modern Roman}fk} ligature for the
Polish language. You can set the \cargv{Language} key to a given language to
enable these features.
\begin{example}
agrafka

\setmainfont[
  Language=Polish,
]{Latin Modern Roman}
agrafka
\end{example}
The \pai{polyglossia} package activates these features automatically so you
don't have to worry about them if you use it.

If your font supports it you may wish to enable automatic fractions insertion
with \cargv{Fractions=On} key.
\begin{example}
1/2 3/4 123/456

\setmainfont[
  Fractions=On,
]{Latin Modern Roman}
1/2 3/4 123/456

\setmainfont[
  Fractions=On,
]{Source Serif Pro}
1/2 3/4 123/456
\end{example}

The OpenType font format defines a lot of more font features, that may or may not
be supported by your font of choice. Consult with the \pai*{fontspec} package
documentation for a comprehensive description and examples.

\subsection{Specifying Fonts via Filenames}\label{ssec:fonts_filename}

If you do not want to install fonts in your system or you are working on a
collaborative project where not everybody has the necessary fonts installed on their system,
you can add font files to your project and specify the fonts directly via their filenames.
In this case you must specify
font variations manually. Because the filenames are usually very similar, it is
possible to enter them using \cargv{*} patterns, where \cargv{*} is replaced by
the main name defined. Extension may also be passed via the \cargv{Extension}
key to avoid repetition. See \autoref{lst:fontloading} for a comparison of font
loading techniques. If the font files are not present in the same directory as
the document you may have to specify it directly using the \cargv{Path} key.
\begin{listing}
  \begin{example}[vertical_mode, examplewidth=0.8\linewidth]
\setmainfont{Source Serif Pro}
Normal text. \textit{Italics.} \textbf{Bold.}
\textit{\textbf{Bold italics.}} \\

\setmainfont{SourceSerifPro-Regular.otf}
Normal text. \textit{Italics.} \textbf{Bold.}
\textit{\textbf{Bold italics.}} \\
 
\setmainfont[
  ItalicFont=SourceSerifPro-RegularIt.otf,
  BoldFont=SourceSerifPro-Bold.otf,
  BoldItalicFont=SourceSerifPro-BoldIt.otf,
]{SourceSerifPro-Regular.otf}
Normal text. \textit{Italics.} \textbf{Bold.}
\textit{\textbf{Bold italics.}} \\

\setmainfont[
  Extension=.otf,
  UprightFont=*-Regular,
  ItalicFont=*-RegularIt,
  BoldFont=*-Bold,
  BoldItalicFont=*-BoldIt,
]{SourceSerifPro}
Normal text. \textit{Italics.} \textbf{Bold.}
\textit{\textbf{Bold italics.}}
\end{example}
  \caption{Comparison of font loading with the \pai{fontspec}
    package.}\label{lst:fontloading}
\end{listing}

The default \LaTeX{} fonts are rather atypical in that they distinguish between
\textit{italics} and \textsl{slanted} font. Most fonts do not do this, so
\pai{fontspec} defines slanted font to be the same as italics. This may be
fixed by setting the \cargv{SlantedFont} key explicitly.
\begin{example}[vertical_mode, examplewidth=0.7\linewidth]
\setmainfont{Latin Modern Roman}
\textit{italics} vs. \textsl{slanted}

\setmainfont[
  SlantedFont=Latin Modern Roman Slanted,
]{Latin Modern Roman}
\textit{italics} vs. \textsl{slanted}
\end{example}

\subsection{Defining New Fonts}

So far we have only talked about changing the fonts for the whole document. It
is possible however to define new fonts that are used only sporadically
throughout the document, for emphasis or decorative purposes. It is possible to
do so using the
\begin{lscommand}
  \csi{newfontfamily}[command: M, options: o, font: m]
\end{lscommand}
It defines new \carg{command} that works like to the \csi{rmfamily} or
\csi{sffamily} commands.
\begin{example}
\newfontfamily{\sourcefamily}[
  Numbers=OldStyle,
]{Source Serif Pro}

Normal text when suddenly
\ldots{} \sourcefamily
a different font! 0123456789
\end{example}
This is especially useful when working with multiple languages as you have
already seen in \autoref{sec:polyglossia}.

The \csi{newfontfamily} checks whether the font family is already defined and
raises an error if it is. As in \autoref{sec:new_commands} the
\csi{renewfontfamily} and \csi{providefontfamily} are available if you want to
redefine existing font families.

\subsection{Math Fonts}\label{sec:math_fonts}

The package \pai{unicode-math}, introduced in \autoref{chap:math}, uses
\pai{fontspec} under the hood and already enables you to use any OpenType math
font within your document. The main command to do so is called
\csi{setmathfont}. It accepts either a font name or a filename. In contrast to
the text fonts that often consist of multiple files, math fonts typically
consist of a single file, thus specifying it via a filename is not as
complicated as presented in \autoref{ssec:fonts_filename}.
\begin{code}
  \begin{minted}{latex}
\setmathfont{STIX Two Math}
  \end{minted}
\end{code}
is equivalent to
\begin{code}
  \begin{minted}{latex}
\setmathfont{STIXTwoMath-Regular.otf}
  \end{minted}
\end{code}
and the latter works in both \hologo{LuaLaTeX} and \hologo{XeLaTeX}. While
changing math fonts throughout the document is possible, it may lead to some
problems; prefer to set them in the preamble for the whole document.
\begin{example}[standalone, paperheight=2.5cm, paperwidth=6cm]
\usepackage{unicode-math}%!hide
\setmainfont{EB Garamond}
\setmathfont{Garamond Math}
% ...

\begin{document} %!hide
\noindent %!hide
Now we are using Garamond fonts.
\[
  \symrm{e}^{\symrm{\pi}
    \symrm{i}} + 1 = 0 \quad
  \sum_{i=0}^\infty \iint_a^b
  \lim_{h\to0}\frac{\sqrt[3]{
    \symbb{A}}}{2^h}\,\symrm{d}x
\]
\end{document}%!hide
\end{example}

Not all math fonts have the same character coverage. For example, the default
font doesn't have lowercase script letters. If you don't want to switch the
fonts entirely but just use some characters from a different font, you
can use the \cargv{range} key in the options to the \csi{setmathfont} command.
\begin{example}[standalone, paperheight=1cm, paperwidth=6cm]
\usepackage{unicode-math}%!hide
\begin{document} %!hide
\noindent %!hide
\(xyz = \symscr{Hello}\) vs.\
\setmathfont[
  range=scr,
]{STIX Two Math}
\(xyz = \symscr{Hello}\)
\end{document}%!hide
\end{example}
You can also set it to exact Unicode ranges if you need more control over
replaced symbols.

Some fonts define two types of script font \emph{roundhand} and \emph{chancery}. These are
normally available as the first stylistic set feature of the font. You can map
\csi{symcal} which is normally a synonym for \csi{symscr}, to produce the
alternative script letters.
\begin{example}[standalone, paperheight=1cm, paperwidth=6cm]
\usepackage{unicode-math}%!hide
\begin{document} %!hide
% TODO: Waiting for unicode-math fix
\setmathfont{STIX Two Math}
\setmathfont[
  range={cal, bfcal},
  StylisticSet=1,
]{STIX Two Math}

\noindent %!hide
\(\symscr{ABCDabcd}\) vs.\
\(\symcal{ABCDabcd}\) 
\end{document}%!hide
\end{example}

The default behaviour of the \csi{not} command, is to combine the negating glyph
with the following symbol. This usually produces satisfactory results. If the
font defines a dedicated negated symbol it is probably better to use it in such
situations. The \csi{not} command is able to use a predefined mapping to use
such glyphs based on the negated symbol. If the default mapping does not
contain the combination, or if you prefer to use a different negation you can
create a new mapping by using \csi{NewNegationCommand}.
\begin{example}
\(\not\cong\) vs.\
\NewNegationCommand{%
    \cong}{\simneqq}%
\(\not\cong\)
\end{example}

\section{Colours}\label{sec:colors}\index{colours}

\subsection{Coloured Text}
In the \autoref{sec:logical_structure} we have used different text colours to
illustrate an example. These can be obtained with the \pai*{xcolor} package. It
provides three commands to change the colour of text:
\begin{lscommand}
  \csi{color}[model: o, color: m] \\
  \csi{textcolor}[model: o, color: m, text: m]
  \csi{mathcolor}[model: o, color: m, text: m]
\end{lscommand}
The \csi{color} is a switch version while \csi{textcolor} and \csi{mathcolor}
only apply to their argument. If no \carg{model} is specified, then
\carg{color} is specified as colour expression. The simplest colour expression
is just the name of the colour, for example \cargv{yellow} or \cargv{red}.
\begin{example}
\textcolor{yellow}{foo} \\
\color{red} baz
\[
  \mathcolor{blue}{
    \sum_{k=0}
  }^{10} i
\]
\end{example}
The list of predefined colours can be found in \autoref{tbl:basecolors}. You can
also pass \cargv{dvipsnames}, \cargv{svgnames} or \cargv{x11names} as a package
options to extend the predefined colours. Consult the package documentation for
a full list.
\begin{table}
  \ExplSyntaxOn
  \NewDocumentCommand{\DemoColor}{m}{
    \raisebox{0.15cm}{\cargv{#1}} &
    \fcolorbox{black}{#1}{\phantom{\rule{0.5cm}{0.5cm}}}
  }
  \ExplSyntaxOff
  \caption{Basic colours predefined by the \pai{xcolor}
    package.}\label{tbl:basecolors}
  \begin{tabular}{@{}lc*2{@{\qquad}lc}@{}}
    \toprule
    Name                 & Demo                  &
    Name                 & Demo                  &
    Name                 & Demo                                       \\
    \midrule
    \DemoColor{black}    & \DemoColor{lightgray} & \DemoColor{purple} \\
    \DemoColor{blue}     & \DemoColor{lime}      & \DemoColor{red}    \\
    \DemoColor{brown}    & \DemoColor{magenta}   & \DemoColor{teal}   \\
    \DemoColor{cyan}     & \DemoColor{olive}     & \DemoColor{violet} \\
    \DemoColor{darkgray} & \DemoColor{orange}    & \DemoColor{white}  \\
    \DemoColor{gray}     & \DemoColor{pink}      & \DemoColor{yellow} \\
    \DemoColor{green}    &                       &                    \\
    \bottomrule
  \end{tabular}
\end{table}

Another type of colour expression is a mix of two colours. The syntax is
\begin{code}
  \cargv{\carg{first color}!\carg{percentage}!\carg{second color}}.
\end{code}
The resulting colour will be the result of mixing
\qty[parse-numbers=false]{\carg{percentage}}{\percent} of the \carg{first
  color} and \qty[parse-numbers=false]{100-\carg{percentage}}{\percent} of the
\carg{second color}. If you omit the \carg{second color} it defaults to white.
\begin{example}
\textcolor{green!100!red}{C}%
\textcolor{green!80!red}{o}%
\textcolor{green!60!red}{l}%
\textcolor{green!40!red}{o}%
\textcolor{green!20!red}{r}%
\textcolor{green!0!red}{s} \\
\textcolor{blue!100}{B}%
\textcolor{blue!75}{l}%
\textcolor{blue!50}{u}%
\textcolor{blue!25}{e}
\end{example}
Colour mixing is left associative so
\begin{code}
  \cargv{\carg{A}!\carg{n}!\carg{B}!\carg{m}!\carg{C}}
\end{code}
means calculate the mixture of \carg{A} and \carg{B} and then mixture of the
result and \carg{C}. You can also use the minus sign before the expression to
get the complementary colour.
\begin{example}
\color{green!20!red!60!blue}
\LaTeX{} \\
\color{-green!20!red!60!blue}
\LaTeX{}
\end{example}

\subsection{Models}

While colour mixing via expression is useful for simple colour specification, it
is often the case that we want to use colour that is defined in terms of its RGB
or HSB values. Different input method colours can be specified using the optional
\carg{model} argument. Note that it is case-sensitive.

The simplest model is \cargv{Gray}. It accepts a single number from \(0\)to
\(15\) and produces a grey colour with the given brightness.
\begin{example}
\textcolor[Gray]{0}{Zero}    \\
\textcolor[Gray]{3}{Three}   \\
\textcolor[Gray]{7}{Seven}   \\
\textcolor[Gray]{11}{Eleven} \\
\textcolor[Gray]{15}{Fifteen}
\end{example}

You can input RGB values in three ways: \cargv{rgb}, \cargv{RGB} and
\cargv{HTML} models. The \cargv{HTML} model accepts a hexadecimal colour code.
The code may be either upper or lowercase.
\begin{example}
\textcolor[HTML]{e63946}{e63946}
\textcolor[HTML]{06D6A0}{06D6A0}
\end{example}
The \cargv{rgb} model accepts three decimal numbers, each between \(0\) and
\(1\), while the \cargv{RGB} model accepts three integers from \(0\) to \(255\).
\begin{example}
\textcolor[RGB]{255, 204, 102}{
  255, 204, 102
} \\
\textcolor[rgb]{0.4, 0.4, 1.0}{
  0.4, 0.4, 1.0
}
\end{example}

If you prefer the subtractive colour model, both \cargv{cmy} and \cargv{cmyk} are
available. They accept decimal numbers between \(0\) and \(1\) to specify
the amount of each colour.
\begin{example}
\textcolor[cmy]{0.7, 0.4, 0.3}{
  0.7, 0.4, 0.3
} \\
\textcolor[cmyk]{
  0.7, 0.4, 0.3, 0.5
}{
  0.7, 0.4, 0.3, 0.5
}
\end{example}

There are three models that enable defining colours by HSB\@: \cargv{hsb},
\cargv{Hsb} and \cargv{HSB}. The first two accept three decimal numbers for
each value, the difference being that the \cargv{Hsb} accepts hue as an angle
in degrees, that is a number between \(0\) and \(360\). The \cargv{hsb} accepts
it as a number between \(0\) and \(1\), while saturation and brightness
are passed the same way in both model---as a number between \(0\) and \(1\).
\begin{example}
\textcolor[hsb]{
  0.4, 0.8, 0.75
}{
  0.4, 0.8, 0.75
}\\
\textcolor[Hsb]{
  144, 0.8, 0.75
}{
  144, 0.8, 0.75
}
\end{example}
The \cargv{HSB} in turn accepts all three as integers---each between \(0\) and
\(240\).
\begin{example}
\textcolor[HSB]{
  144, 200, 120
}{
  144, 200, 120
}
\end{example}

If you are writing a paper about light you may also find that the \cargv{wave}
model comes in handy. It allows you to specify a colour by its wavelength. It
accepts a single decimal number that represents a wavelength in visible
spectrum in nanometres.
\begin{example}
\textcolor[wave]{452}{
  If a light has wavelength
  \qty{452}{\nm} it looks
  like this.  
} \\
\textcolor[wave]{700}{
  Light with wavelength above
  \qty{814}{\nm} is called
  infrared.
}
\end{example}

\subsection{Defining Your Own Colours}

If you want to use a given colour more than once it makes sense to define
it as a macro. While you could use the \csi{NewDocumentCommand} to define it,
the \pai{xcolor} package provides a better way via the
\begin{lscommand}
  \csi{definecolor}[name: m, model: m, value: m].
\end{lscommand}
command. Using it makes it possible to use the newly defined colour in colour
mixing and such.
\begin{example}
\definecolor{MyRed}{wave}{712}
\textcolor{MyRed}{MyRed is
  the perfect colour for you!}
\textcolor{MyRed!60}{Tints
  are also available!}
\end{example}
Be careful though, since it doesn't guard against redefinition. If you want to
check whether you haven't redefined some colour put \csi{tracingcolors} in your
preamble. This will produce warnings when redefinition happens.

If you want to make sure a colour is present but don't want to redefine it if it
already exists then \csi{providecolor} does exactly that. There is also
\csi{colorlet} that simply creates a copy of a given colour similar to the
\csi{NewCommandCopy} command.

Colours defined in different models may need to be converted when mixing them.
This may lead to a situation where \mintinline{latex}|\color{a!75!b}| will
result in different colour than \mintinline{latex}|\color{b!25!a}|. Keep that in
mind when mixing your own colours.

\subsection{Colourful Pages and Boxes}

So far we have only considered changing the text colour. It is however possible
to also change the background colour of the document page. To do this use the
\begin{lscommand}
  \csi{pagecolor}[model: o, color: m]
\end{lscommand}
command, which accepts the same arguments as the \csi{color} command. If you
want to revert to the default transparent background you may do so with the
\csi{nopagecolor} command.
\begin{example}[standalone, paperheight=1cm]
\usepackage{xcolor} %!hide
\begin{document} %!hide
\pagecolor{orange} \color{-orange}
Small is colourful \ldots?
\end{document} %!hide
\end{example}

If you only want to specify a background of some text instead of the whole page
you can use the
\begin{lscommand}
  \csi{colorbox}[model: o, color: m, text: m] \\
  \csi{fcolorbox}[model: o, color: m, model: o, color: m, text: m]
\end{lscommand}
commands. The first one only colours the background, while the second one
allows also drawing a frame (the \enquote*{f} stands for \enquote{framed}).
\begin{example}
It is \colorbox{gray}{curious}
how much a document can be
enhanced or ruined by
\fcolorbox{blue}{red}{
  colours.
}
\end{example}
Boxes are explored further in \autoref{sec:boxes}.

\section{Lengths and Spacing}

\subsection{\LaTeX{} Units}\index{units (\TeX)}\index{dimensions}%
\label{sec:dimensions}
\begingroup
\DeclareSIUnit{\in}{in}
\DeclareSIUnit{\pt}{pt}
\DeclareSIUnit{\bp}{bp}
\DeclareSIUnit{\sp}{sp}
\DeclareSIUnit{\dd}{dd}
\ExplSyntaxOn
\NewDocumentCommand{\DimVal}{m}{
  \num{\dim_to_decimal_in_sp:n {#1}}
}
\NewDocumentCommand{\fnum}{mm}{
  \num[
    number-mode=text,
    parse-numbers=false,
  ]{
    \sfrac{
      \num[number-mode=math,parse-numbers=true]{#1}
    }{
      \num[number-mode=math,parse-numbers=true]{#2}
    }
  }
}

\NewDocumentCommand{\fqty}{mmm}{
  \qty[
    number-mode=text,
    parse-numbers=false,
  ]{
    \sfrac{
      \num[number-mode=math,parse-numbers=true]{#1}
    }{
      \num[number-mode=math,parse-numbers=true]{#2}
    }
  }{#3}
}
\ExplSyntaxOff

Throughout this booklet we have often presented commands that accept length as
one of its parameters such as \csi{\bs} or \csi{fontsize}. When introducing
them we have used \unit{\cm} and \unit{\pt} which stand for centimetre and
point, but these are not the only units available in \LaTeX{}.

The most fundamental unit in \LaTeX{} is \unit{\sp} which stands for
\emph{scaled point}. Its width is equal to \fqty{1}{65536}{\pt}, where
\qty{1}{\pt} is equal to \fnum{1}{72.27} of an international inch which in turn
is defined as exactly \qty{25.4}{\mm}. All units in \TeX{} are ultimately
represented as a whole numbers of \unit{sp}. See \autoref{units} for the exact
values.
\begin{table}
  \caption{\LaTeX{} Units.}\label{units}\index{units}
  \begin{tabular}{@{}clrrl@{}}
    \toprule
    Unit     & Meaning                                                                                                 & Definition            & {Value (\unit{\sp})} & Demo            \\
    \midrule
    \ltx|cm| & centimetre                                                                                              & \qty{0.01}{\m}        & \DimVal{1cm}         & \demowidth{1cm} \\
    \ltx|mm| & millimetre                                                                                              & \qty{0.001}{\m}       & \DimVal{1mm}         & \demowidth{1mm} \\
    \ltx|in| & inch                                                                                                    & \qty{25.4}{\mm}       & \DimVal{1in}         & \demowidth{1in} \\
    \ltx|pt| & point                                                                                                   & \fqty{1}{72.27}{\in}  & \DimVal{1pt}         & \demowidth{1pt} \\
    \ltx|sp| & scaled point                                                                                            & \fqty{1}{65536}{\pt}  & \DimVal{1sp}         & \demowidth{1sp} \\
    \ltx|pc| & pica                                                                                                    & \qty{12}{\pt}         & \DimVal{1pc}         & \demowidth{1pc} \\
    \ltx|dd| & didot                                                                                                   & \qty{0.376065}{\mm}   & \DimVal{1dd}         & \demowidth{1dd} \\
    \ltx|cc| & cicero                                                                                                  & \qty{12}{\dd}         & \DimVal{1cc}         & \demowidth{1cc} \\
    \ltx|nd| & new didot                                                                                               & \qty{0.375}{\mm}      & \DimVal{1nd}         & \demowidth{1nd} \\
    \ltx|bp| & big point                                                                                               & \fqty{1}{72}{\in}     & \DimVal{1bp}         & \demowidth{1bp} \\[6pt]
    \ltx|em| & \multicolumn{3}{m{7cm}}{roughly width of an \enquote*{M} in the current font}                           & \demowidth{1em}                                                \\
    \ltx|ex| & \multicolumn{3}{m{7cm}}{roughly height of an \enquote*{x} in the current font}                          & \demowidth{1ex}                                                \\
    \ltx|mu| & \multicolumn{3}{m{7cm}}{equal to \fqty{1}{18}{em}, where \unit{em} is taken from the current math font} & \demowidth{0.05556em}                                          \\
    \bottomrule
  \end{tabular}
\end{table}

The last three units mentioned in the table are relative to the current font
used. Historically they were related to the \enquote*{M} and \enquote*{x}
glyphs in a given font but today they are arbitrarily set by fonts. These units
are useful if we want the length to scale proportionally when used with
different font sizes. The \mintinline{latex}|em| unit is usually used for
horizontal lengths, while the \mintinline{latex}|ex| is used for vertical
lengths. The \mintinline{latex}|mu| unit can only be used in math mode for math
spacing (see \autoref{sec:math-spacing}).
\begin{example}
\begin{minipage}[b]{0.5\linewidth} %!hide
foo\\[1ex] bar
\end{minipage}\begin{minipage}[b]{0.5\linewidth}%!hide

\tiny foo\\[1ex] bar
\end{minipage}%!hide
\end{example}

The desktop publishing point (DTP point) is the \emph{de facto} standard point
as used in most programs, and it is defined as \fqty{1}{72}{\in}. For
historical reasons the default \TeX{} points are a bit smaller, while the DTP
points are called \enquote{big points}. While this shouldn't be noticeable in
normal circumstances, remember to use \mintinline{latex}|bp| if exact point
values are required of you.
\begin{example}
\fontsize{12pt}{15pt}\selectfont
Text in 12 \TeX{} points.

\fontsize{12bp}{15bp}\selectfont
Text in 12 DTP points.
\end{example}
\endgroup

\subsection{Horizontal Space}\label{sec:hspace}

\LaTeX{} determines the spaces between words and sentences automatically.
However, similarly to commands described in \autoref{sec:math_spacing}, there are
ways to influence the spaces in normal text. For example, to add horizontal
space, you can use:\index{horizontal!space}
\begin{lscommand}
  \csi{hspace}[length: m]
\end{lscommand}
The \carg{length} argument can be specified using the units described in previous
section in the usual way. You can use decimal and even negative numbers as
values.
\begin{example}
This\hspace{1.5cm}is a space
of \qty{1.5}{\cm}. A bit too
cramped\hspace{-5pt}here.
\end{example}
The space added that way will disappear if it lands on the end of a line,
similarly to an interword spacing. If you want to retain the extra space, use the
starred version of the command.
\begin{example}
The gap here is\hspace{1cm}%
\linebreak missing.

Here the gap is\hspace*{1cm}%
\linebreak not missing.
\end{example}

So far all the lengths we have seen have been \emph{rigid}\index{rigid
  length}\index{length!rigid}, that is the length is exactly as specified. But
you probably noticed that the spaces between words are not rigid---they can
stretch and shrink, so that \TeX{} can make the right margin equal. Lengths
that can do that are called \emph{rubber} lengths\index{rubber
  length}\index{length!rubber}.

Such lengths can be specified using a special \ltx{plus} and \ltx{minus}
syntax. For example to specify that a space can stretch if a need arises you
can specify it by writing \ltx{plus} followed by the maximum allowed stretch.
\begin{example}
This small\hspace{1em plus 2cm}%
space may grow if a need arises.

Here\hspace{1em plus 2cm}the
need\linebreak very much arises.
\end{example}
The additional space can extend even beyond the specified maximum, but in such
cases \LaTeX{} will print a warning.

The shrinking can be specified in a similar way using the \ltx{minus} syntax.
\begin{example}
This\hspace{1em minus 2em}%
space may disappear if it gets
too crampy.
\end{example}

When there are multiple rubber spaces in text \TeX{} calculates the amount of
'stretch` proportionally to the specified maximum. Thus if \TeX{} needs
additional \qty{2}{\cm} of whitespace and one length has \ltx{plus 1cm} while
the other has \ltx{plus 3cm} modifier, it will result in the first one being
enlarged by \(\left(\frac{1}{1+3}\right) \times \qty{2}{\cm}\) while the second
one by \(\left(\frac{3}{1+3}\right) \times \qty{2}{\cm}\).
\begin{example}
This\hspace{0pt plus 3cm}is
stretched\hspace{0pt plus 1cm}%
three\linebreak times as much.
\end{example}

With these informations you can use the spaces to automatically centre the text
in a page by setting the allowed stretching to a high number.
\begin{example}
\hspace*{0pt plus 100cm}Hello
\hspace*{0pt plus 100cm}
\linebreak
\end{example}
This approach will, however, interfere with the spacing inside the centred
expression, since the spaces are still distributed proportionally. The effect
will be getting smaller if you set the allowed stretching to a larger number
but it will still be present. For situations like these, \TeX{} actually
supports a concept of infinitely stretchable space---by using the special
\cargv{fill} unit, allowed only as stretching and shrinking value, we can
ensure that all the other rubber lengths will not stretch.\footnote{\TeX{}
  actually recognises three orders of infinity: \cargv{fil}, \cargv{fill} and
  \cargv{filll}, but as a document author you should stick to using only the
  second one. The first order infinity---\cargv{fil}---is used by some of the
  internal \LaTeX{} such as \cs{\bs} or \cs{newpage}. The third one can be used
  to disallow stretching of the second order infinity when it's needed in some
  very rare circumstances.}
\begin{example}
\hspace*{0pt plus 1fill}%
No\hspace{0pt plus 100cm}
stretching allowed.%
\hspace*{0pt plus 1fill}%
\linebreak
\end{example}

Because the \cargv{fill} value is often used with zero width space \LaTeX{}
defines a macro that simplifies entering it---the
\begin{lscommand}
  \csi{stretch}[n: m]
\end{lscommand}
command.
\begin{example}
\hspace*{\stretch{1}}
is equivalent to
\hspace*{0pt plus 1fill}
\linebreak
\end{example}
The \carg{n} argument is the coefficient by which the \cargv{fill} is
multiplied. Recall that the spaces are distributed proportionally and this is
still the case when infinities are involved.
\begin{example}
x\hspace{\stretch{1}}%
x\hspace{\stretch{3}}x
\end{example}

Still, the most common value to use is \ltx{\hspace{0pt plus 1fill}} and so
\LaTeX{} defines \csi{hfill} that is equivalent to it. It's often used when you
want to flush the rest of the line right.
\begin{example}
Peter Pan\hfill Neverland

Dear Wendy, \ldots
\end{example}

\subsection{Vertical Space}

You have already seen that vertical space between lines can be inserted using
\csi{\bs} command. However, it does not work well when used for spacing between
paragraphs---the reason being that it always starts a new line, so if it's used
at the end of paragraph, it will end in an empty line.
\begin{example}
Paragraph.\\

There's an empty line above.
\end{example}
\LaTeX{} has a dedicated command for setting the space between paragraphs, the
\begin{lscommand}
  \csi{vspace}[length: m]
\end{lscommand}
command. It works similarly to the \cs{hspace} command, however when used
inside a line it will only produce the space after the line is ended.
\begin{example}
Some\vspace{1em} text that
spans multiple lines.
\end{example}
Since it does not produce empty lines when used, it is perfect for inserting a
space between paragraphs. Similarly to the \cs{hspace} command, the space will
be discarded if it lands at the end of a page---use the starred version if this
is not desirable.

The rubber lengths, \csi{stretch} and \csi{vfill} work for vertical space too. However, since manual vertical spacing is
much more common compared to manual horizontal spacing, \LaTeX{} also declares
three semantic commands for inserting them:
\begin{lscommand}
  \csi{bigskip} \\
  \csi{medskip} \\
  \csi{smallskip}
\end{lscommand}
The exact sizes of these skips is dependent on the class used. You can access
them as length by appending \enquote{amount} to their name, for example,
\csi{medskipamount}. These are useful when creating your own environments or to
indicate a thought break between paragraphs.\index{vertical space}

There also exist the \csi{addvspace command}, which works similarly to the
\cs{vspace} command, however when multiple such commands are entered one after
another, only the one with the biggest length will be used. Note that using it
will lead to an error if it is not used between paragraphs.
\begin{example}
Hello.

\addvspace{1pt}
\addvspace{1em}
\addvspace{1cm}
There is \qty{1}{\cm} space
above me.
\end{example}
This command is useful if you want to ensure that a vertical space is present
but avoid entering several of them accidentally.

\subsection{Length Variables}

Like many things in \LaTeX, lengths can be stored inside commands to allow
reuse.
\begin{example}
\NewDocumentCommand{%
  \mylength}{}{2em}
foo\hspace{\mylength}bar
\end{example}
However, \LaTeX{} provides dedicated length variables, which are much better
suited for the purpose. These are created using
\begin{lscommand}
  \csi{newlength}[variable: M]
\end{lscommand}
and set using
\begin{lscommand}
  \csi{setlength}[variable: M, length: m]
\end{lscommand}
It's important to note that \csi{newlength} declares the length globally, but
\csi{setlength} only affects the current group. If you declare a length
variable without setting it, it will have a default value of zero.

In contrast to the command approach, length variables are stored as numbers and
not as text. This allows you to to do simple arithmetic operations, for
example, to scale them by prepending them with a number.
\begin{example}
\newlength{\mylength}
\setlength{\mylength}{2em}
foo\hspace{0.5\mylength}bar%
\hspace{2\mylength}baz
\end{example}
In order to increase an existing length variable you can use \csi{addtolength}.
\begin{example}
\setlength{\mylength}{2em} %!hide
foo\hspace{\mylength}bar\\
\addtolength{\mylength}{2em}
foo\hspace{\mylength}bar
\end{example}

Since length variables are not stored as text macros, but as \TeX{} internal
numbers, trying to typeset them in a document results in an error. To translate
them back into their textual representation use the \csi{the} command. Note
that their value will always be printed using points as units.
\begin{chktexignore}  
\begin{example}
\setlength{\mylength}{1cm}
\the\mylength
\end{example}
\end{chktexignore}

Lengths can be also determined dynamically from the content. The following
commands allow you to determine the width, height and depth of a \LaTeX{} text
and assign it to length variable.
\begin{lscommand}
  \csi{settoheight}[variable: M, element: m]\\
  \csi{settodepth}[variable: M, element: m] \\
  \csi{settowidth}[variable: M, element: m]
\end{lscommand}
The element height is calculated by taking into account the part of the element
that extends above baseline, while the depth takes into account the part that
extends below baseline.
\begin{example}[vertical_mode, examplewidth=0.85\linewidth]
\newlength{\myheight} \settoheight{\myheight}{Major}
\newlength{\mydepth} \settodepth{\mydepth}{Major}
\newlength{\mywidth} \settowidth{\mywidth}{Major}
The word Major has width of \the\mywidth, height of
\the\myheight{} and depth of \the\mydepth.
\end{example}

These commands are especially useful when creating your own environments
requiring complicated alignment or spacing. An example of using them is
presented in \autoref{lst:fun_with_lengths}.

\begin{listing}
  \begin{example}[vertical_mode, examplewidth=0.85\linewidth]
\newlength{\vardescindent}
\NewDocumentEnvironment{vardesc}{m}{%
  \settowidth{\vardescindent}{#1:\ }%
  \RenewDocumentCommand{\item}{so}{%
    \IfBooleanTF{##1}{#1: }{\\\hspace*{\vardescindent}}%
    \IfValueT{##2}{##2 ---}%
  }%
  \ignorespaces
}{}

\[  a^2+b^2=c^2 \]
\begin{vardesc}{Where}
  \item*[\(a\), \(b\)] are adjacent to the right angle
    of a right-angled triangle.
  \item[\(c\)] is the hypotenuse of the triangle and
    feels lonely.
  \item[\(d\)] finally does not show up here at all.
    Isn't that puzzling?
\end{vardesc}
\end{example}
  \caption{An example of using \cs{settowidth} to align all of the
    definitions to a preceding phrase.}\label{lst:fun_with_lengths}
\end{listing}

As you will see \LaTeX{} uses length variables for many things. When
typesetting a document you can use some of them when setting lengths to make them
relative. For example, the \csi{linewidth} length holds the length of the
current line. You can use it when inserting a picture to make it fill the page
or scale it to take up half the space available.
\begin{example}
\includegraphics[
  width=0.5\linewidth,
]{example-image}
\end{example}

\section{The Layout of the Document}

\subsection{Document Class Options}\label{sec:documentclassoptions}

The easiest way to influence the layout of the document is to pass options to
the
\begin{lscommand}
  \csi{documentclass}[options: o, class: m]
\end{lscommand}
command at the beginning of your file. The available classes where already
described in \autoref{documentclasses} on \autopageref{documentclasses}. The
\carg{options} have to be separated by commas. The most common options for the
standard document classes are listed in \autoref{options}.

\begin{table}
  \RenewDocumentCommand{\arraystretch}{}{1.2}
  \caption{Document Class Options.}\label{options}
  \begin{tabular}{@{}>{\RaggedRight}p{2.5cm}p{9cm}@{}}
    \toprule
    Options                                  & Description                   \\
    \midrule
    \cargv{10pt}, \cargv{11pt}, \cargv{12pt} & Sets the size
    of the main font in the document. If no option is specified,
    \cargv{10pt} is assumed.\index{document font size}\index{base
    font size}                                                               \\
    \cargv{a4paper}, \cargv{letterpaper}     & Defines the paper size.
    The default size can be configured when installing \LaTeX. Besides that,
    \cargv{a5paper}, \cargv{b5paper}, \cargv{executivepaper}, and
    \cargv{legalpaper} can be specified. Note that it only affects the layout
    of margins, not the PDF paper size itself, see \autoref{sec:page_layout}
    for more details.\index{legal paper}\index{paper
      size}\index{A4 paper}\index{letter paper}\index{A5 paper}\index{B5
    paper}\index{executive paper}                                            \\
    \cargv{fleqn}                            & Typesets
    displayed formulae left-aligned instead of centred.                      \\
    \cargv{leqno}                            & Places
    the numbering of formulae on the left hand side instead of the right.    \\
    \cargv{titlepage}, \cargv{notitlepage}   & Specifies
    whether a new page should be started after the \wi{document title}
    or not. The \cli{article} class does not start a new page by
    default, while \cli{report} and \cli{book} do.\index{title}              \\
    \cargv{onecolumn}, \cargv{twocolumn}     & Instructs
    \LaTeX{} to typeset the document in \wi{one column} or \wi{two column}s. \\
    \cargv{twoside, oneside}                 & Specifies whether
    double or single sided output should be generated. The classes
    \cli{article} and \cli{report} are \wi{single sided} and the
    \cli{book} class is \wi{double sided} by default. Note that this
    option concerns the style of the document only. The option
    \cargv{twoside} does \emph{not} tell the printer you use that it
    should actually make a two-sided printout.                               \\
    \cargv{landscape}                        & Changes the
    layout of the document to print in landscape mode.                       \\
    \cargv{openright, openany}               & Makes chapters begin
    either only on right hand pages or on the next page available. This does
    not work with the \cli{article} class, as it does not know about
    chapters. The \cli{report} class by default starts chapters on
    the next page available and the \cli{book} class starts them on
    right hand pages.                                                        \\
    \bottomrule
  \end{tabular}
\end{table}

For example, if an input file for a \LaTeX{} document starts with the line
\begin{code}
\csi{documentclass}[{11pt, twoside, a4paper}: vo, article: vm]
\end{code}
then it instructs \LaTeX{} to typeset the document as an article with a base
font size of eleven points, and to produce a layout suitable for double sided
printing\footnote{Note that this only influences the appearance of the document
  to be adequate for double sided printing---you still have to pass proper
  instructions to your printer to print it on both sides.} on A4 paper.

If you do not like the appearance of the standard \LaTeX{} classes, the easiest
way to change it is to use some alternatives. For example, the
\pai*{koma-script} package provides alternatives that produce documents with European
typography traditions in mind. Another popular package is \pai*{memoir}, which
provides a single \cli{memoir} class with extensive customization options. On
\href{https://www.ctan.org/}{CTAN} you can find many more specialized classes
that will let you produce documents as prescribed by various universities or
typographical traditions.

\subsection{Page Styles}

\LaTeX{} supports three predefined \wi{header}\slash\wi{footer}
combinations---so-called \wi{page style}s. The \carg{style} parameter
of the\index{page style!plain@\texttt{plain}}\index{plain@\texttt{plain}}\index{page
  style!headings@\texttt{headings}}\index{headings@\texttt{headings}}\index{page
  style!empty@\texttt{empty}}\index{empty@\texttt{empty}}
\begin{lscommand}
  \csi{pagestyle}[style: m]
\end{lscommand}
command defines which one to use. \autoref{pagestyle} lists the predefined
page styles.

\begin{table}
  \caption{The Predefined Page Styles of \LaTeX.}\label{pagestyle}
  \begin{tabular}{lp{8cm}}
    \toprule
    Style              & Description                                      \\
    \midrule
    \cargv{plain}      & prints the page numbers on the bottom
    of the page, in the middle of the footer. This is the default page
    style.                                                                \\
    \cargv{headings}   & prints the current chapter heading
    and the page number in the header on each page, while the footer
    remains empty.  (This is the style used in this document)             \\
    \cargv{empty}      & sets both the header and the footer
    to be empty.                                                          \\
    \cargv{myheadings} & similar to the \cargv{headings} style but leaves
    the headers and footers empty allowing for them to be defined by
    the author. A description of how to do this is in
    \autoref{sec:fancy}.                                                  \\
    \bottomrule
  \end{tabular}
\end{table}

It is possible to change the page style of the current page with the command
\begin{lscommand}
  \csi{thispagestyle}[style: m]
\end{lscommand}

You may also control the style of the displayed page numbers. To change it use
the
\begin{lscommand}
  \csi{pagenumbering}[style: m]
\end{lscommand}
command, where \carg{style} is one of the styles presented in
\autoref{tb:numberings}.

\begin{table}
  \caption{Possible argument of the \csi{pagenumbering}
    command.}\label{tb:numberings}
  \begin{tabular}{@{}ll@{}}
    \toprule
    Style          & Description                                   \\
    \midrule
    \cargv{arabic} & Arabic numerals (1, 2, 3, \ldots)             \\
    \cargv{roman}  & Lowercase Roman numerals (i, ii, iii, \ldots) \\
    \cargv{Roman}  & Uppercase Roman numerals (I, II, III, \ldots) \\
    \cargv{alph}   & Lowercase Latin letters (a, b, c, \ldots)     \\
    \cargv{Alph}   & Uppercase Latin letters (A, B, C, \ldots)     \\
    \bottomrule
  \end{tabular}
\end{table}

If would like more control over the apparance of your headers and footers, refere to \autoref{sec:fancy} on \autopageref{sec:fancy}.

\subsection{Line Spacing}\index{line spacing}

When using custom fonts, you may find that the default line spacing does not
match your taste. You could redefine it using the \csi{fontsize} commands,
however, this will be overwritten once you use size changing commands described
in \autoref{sizes}.

To make such adjustments easier \LaTeX{} defines the
\begin{lscommand}
  \csi{linespread}[factor: m]
\end{lscommand}
command. Once used, each line skip will be multiplied by the \carg{factor}.
Note, that outside the preamble, you must follow it by \csi{selectfont} for the
changes to be visible.
\begin{example}
\linespread{0.9}\selectfont
If you want to save paper, set
the linespread to a value
below 1 to sacrifice a bit of
space between lines.

\linespread{1.1}\selectfont
On the other hand, if your
assignment is short a few
pages, setting it to a value
above 1 might just save you
some typing.
\end{example}

Using the \csi{linespread} command it's also possible to create an effect of
\enquote{one and a half} or \enquote{double} line spacing. Recall that the
default line spacing is around \qty{1.2}{em}. Thus if you want to make the
document use double line spacing you have to set the \carg{factor} to
\(\sfrac{2}{1.2} \approx 1.667\).

Note that setting the \csi{linespread} will change the line spacing
\emph{everywhere}---including footnotes, table of contents and floats. Doing so
is not always desirable, so another approach is to set the \csi{baselineskip}
length. This will only affect the line spacing of the main document text.
Similarly to the size changing commands it affects the whole paragraph.
\begin{example}
{\setlength{\baselineskip}{%
  1.5\baselineskip}
This paragraph is typeset with
the baseline skip set to 1.5 of
what it was before. Note the
par command at the end of the
paragraph.\par}

Here the line spacing returns
to normal, because line skip
changes are local to a group.
\end{example}

Yet another approach is to use a dedicated package, like \pai*{setspace} that
defines \csi{doublespacing} and similar commands. Note that in general you
should avoid using excessive line spacing, unless you are emulating an old
document look.

\subsection{Paragraph Formatting}\label{parsp}

In \LaTeX{}, there are two lengths influencing paragraph layout:
\csi{parindent} defines how much a paragraph is indented, while \csi{parskip}
is the amount of space inserted between paragraphs.
\begin{example}
\setlength{\parindent}{0pt}
\setlength{\parskip}{%
  \medskipamount}

On the web it is common to
separate paragraphs by some
space instead of indenting.

Like this.
\end{example}
Beware, that these lengths also affect the table of contents---its lines get
spaced more loosely. To avoid this, you might want to put the two commands
after the \cs{tableofcontents} command or to not use them at all, because
you'll find that most professional books use indenting and not spacing to
separate paragraphs.

If you want to indent a paragraph that is not indented, use \csi{indent} at the
beginning of the paragraph. Obviously, this will only have an effect when
\csi{parindent} is not set to zero. In continental Europe it is sometimes the
case that every paragraph should be indented, even after sections. To avoid
using ths \csi{indent} command everywhere, simply use the \pai*{indentfirst}
package in your preamble.
\begin{example}[standalone, paperheight=2.5cm, paperwidth=0.5\linewidth]
\usepackage{indentfirst}
% ...

\begin{document} %!hide
\section{Title}

The first paragraph is now
indented.
\end{document} %!hide
\end{example}

To create a non-indented paragraph, use \csi{noindent} as the first command of
the paragraph. This might come in handy when you start a document with body
text and not with a sectioning command.

\subsection{Page Layout}\label{sec:page_layout}
\begingroup
\LshortExampleSetup{
  standalone,
  template=empty,
  paperwidth=6cm,
}
As you have seen, \LaTeX{} allows you to specify the \wi{paper size} via
options in the \cs{documentclass} command. It then automatically picks the
right text \wi{margins}, but sometimes you may not be happy with the predefined
values. Naturally, you can customize them to your liking. But before you start
making the margins as narrow as possible to cram the text in, take a few
seconds to think. As with most things in \LaTeX, there are good reasons for the
page layout to be as it is.

Sure, compared to your off-the-shelf page of your favourite WYSIWYG editor, the
margins look awfully wide. But take a look at your favourite, professionally
printed book and count the number of characters on a standard text line. You
will find that most lines contain between 45 and 80 characters. Now do the same
on your \LaTeX{} page. You will find that the same relationship holds.
Empirical studies suggest that averaging around 66 characters per line creates
the optimal reading experience for readers. If you want to save space when
printing your document consider using \cargv{twocolumn} option or smaller page
sizes.

With that warning in place let us proceed to the proper introduction of
\pai*{geometry} package that allows you to easily customize the page dimensions
of your document. It is worth noting that simply including the package in your
preamble will result in considerably narrower margins, (the very thing we
warned you to avoid) so only use the package if you intend to set them
manually or use the \cargv{pass} option that disables most of the package
functions but retains the paper size adjustments.

As we have mentioned in \autoref{options}, simply setting \cargv{a5paper} (or
similar) option will only adjust margins of the document without changing the
paper dimensions themselves. The simplest way to fix that is to add the
\pai{geometry} package to your preamble. It will read the page size option
and adjust it accordingly. The package itself also supports many more page
sizes, such as \cargv{a0paper} or \cargv{a6paper}.
\begin{example}
\documentclass{article}

\usepackage[
  a6paper,
  landscape,
]{geometry}

\begin{document}
This document is typeset on
A6 paper in landscape mode.
\end{document}
\end{example}

If the predefined dimensions are not enough you can always set it directly
using \cargv{paperheight} and \cargv{paperwidth} keys.
\begin{example}
\documentclass{article} %!hide
\usepackage[
  paperwidth=6cm,
  paperheight=3cm
]{geometry}
% ...

\begin{document} %!hide
This page has dimensions of
6 by 3 centimetres.
\end{document} %!hide
\end{example}

As we have indicated, the package is also capable of adjusting the margins. The
simplest way to do so is to use the \cargv{left}, \cargv{right}, \cargv{top},
and \cargv{bottom} options, each of which sets the size of the respective
margin.
\begin{example}[paperheight=4cm]
\documentclass{article} %!hide
\usepackage[
  paperheight=\height, %!hide
  paperwidth=\width, %!hide
  top=0.5cm,
  bottom=1cm,
  left=1.5cm,
  right=2cm,
]{geometry}
% ...
%!hidebegin
\usepackage{blindtext}
\sloppy
\begin{document}
\blindtext
\end{document}
\end{example}
Note that by default the header and footer are not considered part of the
page body, so they may not fit on page if the margins are too narrow (like in
the example above). If you want to include them when considering margins use
the \cargv{includefoot} and/or \cargv{includehead} options.

One of the killer features of the \pai{geometry} package is its ability to
calculate the correct margin sizes based on other metrics. For example, we can
declare that we want the text to have a given width via \cargv{textwidth}
option and that each page should have a given number of lines via \cargv{lines}
option and the appropriate sizes of margins will be calculated automatically.
\begin{example}[paperheight=4cm]
\documentclass{article} %!hide
\usepackage[
  paperheight=\height, %!hide
  paperwidth=\width, %!hide
  textwidth=5cm,
  lines=3
]{geometry}
% ...
%!hidebegin
\usepackage{blindtext}
\sloppy
\begin{document}
\blindtext
\end{document}
\end{example}
There are many more options to influence the margins: specifying their ratios
(\cargv{ratio}), defining them to take up certain percent of available paper
size (\cargv{scale}), or including binding offset (\cargv{bindingoffset}).
Check out the package documentation~\cite{pack:geometry} for a full list with
examples.

A useful option for prototyping your layout is the \cargv{showframe} option
that draws frames around document body and margins to easier evaluate the
chosen layout.
\endgroup
\section{Fancy Headers}\label{sec:fancy}

\subsection{Basic commands}

%$$ Not sure if the overall conversational tone of this subsection is in keeping
%$$ with the bulk of the rest of the document.

The \pai*{fancyhdr} package provides a few simple commands that allow you to
customise the header and footer lines of your document. It defines an
additional page style \cargv{fancy} and a set of commands to customise it to
your liking. By default, it only adds a line separating the header from the page
body.
\begin{example}[standalone, paperheight=3cm]
\geometry{includefoot, includehead, headsep=.5em, footskip=1em} %!hide
%!showbegin %!hide
\documentclass{article}

%!showend %!hide
\usepackage{fancyhdr}
\pagestyle{fancy}

\begin{document}
\noindent %!hide
This statement is false.
\end{document}
\end{example}

The primary command of the package is
\begin{lscommand}
  \csi{fancyhf}[places: o, field: m]
\end{lscommand}
The \carg{places} is a comma separated list of places where the \carg{field} should be displayed. There are total of 12 different places and each is identified by a
combination of three letters:
\begin{itemize}
  \item The first specifies whether the location is in the header (\cargv{H}) or
        in the footer (\cargv{F}).
  \item The second letter specifies the location within the header or footer:
        \cargv{L} for left, \cargv{C} for centre, and \cargv{R} for right.
  \item The third letter defines whether the field should be printed on even
        (\cargv{E}) or odd (\cargv{O}) pages. If the document is not two sided,
        then all pages are treated as odd.
\end{itemize}
For example, the combination \cargv{FCE} identifies the centre part of the
footer on even pages. If any of the letters is omitted, then the identifier
points toward all positions specifiable by the omitted letter. For example,
\cargv{HR} is right side of the header on both odd and even pages.
\begin{example}[template=empty, standalone, paperheight=2.3cm, paperwidth=6cm, to_page=2,vertical_pages]
\documentclass[twoside]{article}

%!hidebegin
\usepackage[paperheight=\height,
    paperwidth=\width,
    margin=0.3cm,
    includefoot,
    includehead,
    headsep=.5em,
    footskip=1em
]{geometry}
%!hideend
\usepackage{fancyhdr}
\pagestyle{fancy}

\fancyhf[HCE]{A}
\fancyhf[L]{\emph{B}}
\fancyhf[FR]{\textbf{C}}
\fancyhf[HCO, HRE]{\textsl{D}}

\begin{document}
\noindent %!hide
The next statement is false.
The previous statement is true.
\end{document}
\end{example}

There are two additional commands, \csi{fancyhead} and \csi{fancyfoot}, that
work in the same way, except that they respectively assume \cargv{H} and
\cargv{F} in their \carg{places} argument, unless otherwise specified.
\begin{example}[standalone, paperheight=3.5cm]
\geometry{includefoot, includehead, headsep=.5em, footskip=1em} %!hide
\sloppy %!hide
\usepackage{csquotes} %!hide
\usepackage{fancyhdr}%!hide
\pagestyle{fancy}%!hide

\fancyhead[L]{A}
\fancyfoot[R]{B}

\begin{document}%!hide
\noindent %!hide
\enquote{Yields a falsehood when
appended to its own quotation}
yields a falsehood when appended
to its own quotation.
\end{document}%!hide
\end{example}

The lines drawn by the \pai{fancyhdr} package may also be customised. To change
%$$ Inserted comma after 'thickness'.
their thickness, redefine the
\begin{lscommand}
  \csi{headrulewidth} \\
  \csi{footrulewidth}
\end{lscommand}
macros to the desired size.
\begin{example}[standalone, paperheight=3cm, paperwidth=3cm]
\geometry{includefoot, includehead, headsep=.5em, footskip=1em} %!hide
\sloppy %!hide
\usepackage{fancyhdr} %!hide
\pagestyle{fancy} %!hide
\RenewDocumentCommand{\headrulewidth}{}{.2cm}
\RenewDocumentCommand{\footrulewidth}{}{.5cm}

\begin{document}%!hide
\noindent %!hide
Do not read this sentence.
\end{document}%!hide
\end{example}

By default, headers and footers are as long as the text on the page. If you want
to extend or shorten them, use the
\begin{lscommand}
  \csi{fancyhfoffset}[places: o, offset: m]
\end{lscommand}
Added comma after \csi{fancyhf}.
command. The \carg{places} argument is the same as in \csi{fancyhf}, except that
it cannot contain \cargv{C}.
\begin{example}[standalone, paperheight=3cm]
\geometry{includehead, includefoot, headsep=.5em, footskip=1em} %!hide
\sloppy %!hide
\usepackage{fancyhdr}%!hide
\pagestyle{fancy}%!hide
\fancyhfoffset[L]{-1cm}
\fancyhfoffset[R]{.2cm}

\begin{document}%!hide
\noindent %!hide
If this sentence is true,
then \(2 + 2 = 5\).
\end{document}%!hide
\end{example}
The command \csi{fancyheadoffset} and \csi{fancyfootoffset} are
used the same way as \csi{fancyf}, but they only modify header or footer respectively.

\subsection{Contents of the headers}

%$$ Not sure if the overall conversational tone of this subsection is in keeping
%$$ with the bulk of the rest of the document.

The default footer of the article class contains the current page number. To
use it inside the fancy header, simply use the command \csi{thepage}.
\begin{example}[standalone, paperheight=2.5cm, to_page=2, vertical_pages]
\geometry{includehead, includefoot, headsep=.5em, footskip=1em} %!hide
\sloppy %!hide
\usepackage{fancyhdr}%!hide
\pagestyle{fancy}%!hide
\fancyhf{Page~\thepage}

\begin{document}%!hide
\noindent %!hide
This statement is dedicated to
all statements that are not
dedicated to themselves.
\end{document}%!hide
\end{example}

It is often useful to have the header and footer contain information based on
the content of the page. These are called \enquote{marks} in \LaTeX{}
terminology. Before we talk about the default ones, let's consider how you can
define your own using the \pai{extramarks}\footnote{\pai{extramarks} is part
  of \pai*{fancyhdr}.} package.
\begin{lscommand}
  \csi{extramarks}[left:m, right:m] \\
  \csi{firstleftxmark} \\
  \csi{firstrightxmark} \\
  \csi{lastleftxmark} \\
  \csi{lastrightxmark}
\end{lscommand}
The \csi{extramarks} command sets the contents of \carg{left} and \carg{right}
marks. Then you can access these marks inside the headers by using the appropriate
command. The \texttt{first}- commands refer to the first mark occurring on the
page, while the \texttt{last}- refer to the last one.
\begin{example}[standalone, paperheight=4cm]
\geometry{includehead, includefoot, headsep=.5em, footskip=1em} %!hide
\sloppy %!hide
\usepackage{fancyhdr}%!hide
\usepackage{extramarks}
\pagestyle{fancy}%!hide

\fancyhead[L]{\firstleftxmark}
\fancyhead[R]{\lastleftxmark}
\fancyfoot[L]{\firstrightxmark}
\fancyfoot[R]{\lastrightxmark}

\begin{document} %!hide
\noindent %!hide
The second statement is false.
\extramarks{One}{2 is false}
The third statement is false.
\extramarks{Two}{3 is false}
The first statement is false.
\extramarks{Three}{1 is false}
\end{document} %!hide
\end{example}

Let us now look at the default marks defined by \LaTeX{}. After loading
\pai{extramarks}, these may be set and accessed similarly:
\begin{lscommand}
  \csi{markboth}[left:m, right:m] \\
  \csi{firstleftmark} \\
  \csi{firstrightmark} \\
  \csi{lastleftmark} \\
  \csi{lastrightmark}
\end{lscommand}
The only difference between these and the extra marks is that \LaTeX{} classes
automatically fill them. Hence, if you are not careful, you may lose your
content.\footnote{You may prevent this by setting the pagestyle to
  \cargv{myheadings} and then redefining it to use it with \pai{fancyhdr} as
  described later.}
For example, in the article class, \carg{left} is set by the \csi{section}
command, while \carg{right} is set by the \csi{subsection} command.
\begin{example}[standalone, paperheight=5cm]
\geometry{includehead, includefoot, headsep=.5em, footskip=1em} %!hide
\sloppy %!hide
\usepackage{fancyhdr}%!hide
\usepackage{extramarks}%!hide
\pagestyle{fancy}%!hide
\fancyhead[L]{\firstleftmark}
\fancyhead[R]{\lastleftmark}
\fancyfoot[L]{\firstrightmark}
\fancyfoot[R]{\lastrightmark}

\begin{document}%!hide
\section{First}
\subsection{Sub}
\section{Second}
\subsection{Sub}
\end{document}%!hide
\end{example}
Note that \csi{firstrightmark} is empty in the above example. This is caused by
the fact that \csi{section} commands set both marks, leaving the right one
empty.

You may have noticed that section titles are typeset in uppercase when provided
by -\texttt{mark} commands. To disable this use the \csi{nouppercase} command inside the
\csi{fancyhf} command.
\begin{example}[standalone, paperheight=3cm]
\geometry{includehead, includefoot, headsep=.5em, footskip=1em} %!hide
\sloppy %!hide
\usepackage{fancyhdr}%!hide
\usepackage{extramarks}%!hide
\pagestyle{fancy}%!hide
\fancyhead[R]{%
  \nouppercase{\firstleftmark}%
}

\begin{document}%!hide
\section{Today}
is opposite day.
\end{document}%!hide
\end{example}

\subsection{Advanced commands}

If you want even more control over section titles, you can redefine the
\csi{sectionmark}.\footnote{Or \csi{chaptermark}, \csi{subsectionmark} \ldots} The
command receives the section title as its first argument, while the section
number is available as \csi{thesection}.
\begin{example}[standalone, paperheight=4.5cm, paperwidth=4cm]
\geometry{includehead, includefoot, headsep=.5em, footskip=1em} %!hide
\sloppy %!hide
\usepackage{fancyhdr}%!hide
\usepackage{extramarks}%!hide
\pagestyle{fancy}%!hide
\fancyhead[R]{\firstleftmark}
\RenewDocumentCommand{\sectionmark}{m}{%
  \markboth{%
    Section no.\,\thesection: #1}{}%
}

\begin{document}%!hide
\section{Person}
There exists a person such that
if they are reading this then
everybody is reading this.
\end{document}%!hide
\end{example}
If you only want to change the right mark, use the \csi{markright} command.

By default, the field in the centre of the header or footer will expand equally
to the left and right. This is usually the desired behaviour, but in some
cases it may overlap with either left or right text, while still having some
space on the other side.
\begin{example}[standalone, paperheight=3cm]
\geometry{includehead, includefoot, headsep=.5em, footskip=1em} %!hide
\sloppy %!hide
\usepackage{fancyhdr}%!hide
\usepackage{extramarks}%!hide
\pagestyle{fancy}%!hide
\fancyhead[L]{\thepage}
\fancyhead[C]{\firstleftmark}
\fancyhead[R]{Jane Doe}

\begin{document}%!hide
\section{Section}
Section by Jane Doe
\end{document}%!hide
\end{example}
In this situation you may want to use the
\begin{lscommand}
  \csi{fancycenter}[distance:o, stretch:o, left:m, centre:m, right:m]
\end{lscommand}
command, which automatically shifts the centre toward the shorter text. The
\carg{distance} is the minimum distance by which the elements are always
surrounded (\cargv{1em}, by default). The \carg{stretch} controls the preference
for shifting the \carg{centre}; \cargv{1} means shift only when and as much
as necessary. Higher numbers will start shifting sooner and more aggressively.
The default is \cargv{3}. This command writes over the whole header\slash{}footer
space so it should only be put in one place (typically \cargv{C}) and other
places (\cargv{L,R}) should be empty.
\begin{example}[standalone, paperheight=3cm]
\geometry{includehead, includefoot, headsep=.5em, footskip=1em} %!hide
\sloppy %!hide
\usepackage{fancyhdr}%!hide
\usepackage{extramarks}%!hide
\pagestyle{fancy}%!hide
\fancyhead[L,R]{}
\fancyhead[C]{%
  \fancycenter%
    {\thepage}%
    {\firstleftmark}%
    {Jane Doe}%
}

\begin{document}%!hide
\section{Section}
Section by Jane Doe
\end{document}%!hide
\end{example}

You may want to present different headers or footers when the corresponding
page starts with a float or ends with a footnote. The \pai{fancyhdr} package
defines four commands that let you achieve this.
\begin{lscommand}
  \csi{iftopfloat}[true branch:m, false branch:m] \\
  \csi{ifbotfloat}[true branch:m, false branch:m] \\
  \csi{iffloatpage}[true branch:m, false branch:m] \\
  \csi{iffootnote}[true branch:m, false branch:m]
\end{lscommand}
Commands \csi{iftopfloat} and \csi{ifbotfloat} execute their \carg{true branch}
if a float sits at the top or bottom of the page (respectively).

Similarly, \csi{iffloatpage} checks whether
the page is a special float-only page, while \csi{iffootnote} checks for a
footnote at the bottom of the page.
\begin{example}[standalone, paperheight=4cm, to_page=2, vertical_pages]
\geometry{includehead, includefoot, headsep=.5em, footskip=1em} %!hide
\sloppy %!hide
\usepackage{fancyhdr}%!hide
\usepackage{extramarks}%!hide
\pagestyle{fancy}%!hide

\fancyhead[C]{%
  \iftopfloat{%
    A float is below me%
  } {}
}
\fancyfoot[C]{%
  \iffootnote{%
    A footnote is above me%
  } {%
    \thepage%
  }%
}

\begin{document}%!hide
\noindent %!hide
Ignore this footnote.%
\footnote{Read this.}
\begin{figure}[t]
  \centering
  A floating float.
  \caption{Hmm}
\end{figure}
\end{document}%!hide
\end{example}

The ruled lines in headers and footers are created by invoking \csi{headrule}
and \csi{footrule} commands. If you want finer control over the lines, consider
redefining these macros. Use \csi{headruleskip} and \csi{footruleskip} to raise
or lower them, if necessary.
\begin{example}[standalone, paperheight=4cm, paperwidth=3.2cm]
\geometry{includehead, includefoot, headsep=.5em, footskip=1em} %!hide
\sloppy %!hide
\usepackage{fancyhdr}%!hide
\usepackage{extramarks}%!hide
\pagestyle{fancy}%!hide
\RenewDocumentCommand{\headrule}{}{
  \rule{0.05\headwidth}{0.2cm}%
  \rule[0.1cm]{0.9\headwidth}{\headrulewidth}%
  \rule{0.05\headwidth}{0.2cm}%
}

\RenewDocumentCommand{\headruleskip}{}{-0.2cm}

\begin{document}%!hide
\noindent %!hide
My age is the first number not
nameable in under twenty words.
\end{document} %!hide
\end{example}

While working on a longer document, you may want to have several page styles for
different occasions. You may also notice that some commands (\csi{chapter} and
\csi{maketitle}, for example) change the page style to \cargv{plain}.
The solution to both of these problems is the
\begin{lscommand}
  \csi{fancypagestyle}[name:m, code:m]
\end{lscommand}
command. The \carg{name} is the name of page style to (re)define, while the %chktex 36
\carg{code} is the code to set the page style. All page styles declared this
way use the \cargv{fancy} page style as their basis, so if empty \carg{code} is
given they will match \cargv{fancy} exactly.
\begin{example}[standalone, paperheight=3cm]
\geometry{includehead, includefoot, headsep=.5em, footskip=1em} %!hide
\sloppy %!hide
\usepackage{fancyhdr}%!hide
\usepackage{extramarks}%!hide
\fancypagestyle{mine}{
  \fancyhead[L]{My style}
}
\pagestyle{mine}

\begin{document}%!hide
%!showbegin %!hide
Were you expecting a paradox here?
%!showend !hidebegin
\noindent
You may be disappointed.
\end{document}
\end{example}
%$$ I don't think I understand the above example.
%$$ The connection between 'paradox' on the left, and 'disappointed' on the 
%$$ right isn't obvious. Did I miss something?

\section{Boxes}\label{sec:boxes}
\LaTeX{} builds up its pages by pushing around boxes. At first, each
letter is a little box, which is then glued to other letters to form
words. These are again glued to other words, but with special glue,
which is elastic so that a series of words can be squeezed or
stretched as to exactly fill a line on the page.

I admit, this is a very simplistic version of what really happens, but the
point is that \TeX{} operates on glue and boxes. Letters are not the only
things that can be boxes. You can put virtually everything into a box,
including other boxes. Each box will then be handled by \LaTeX{} as if it
were a single letter.

In earlier chapters you encountered some boxes, although I did
not tell you. The \ei{tabular} environment and the \csi{includegraphics}, for
example, both produce a box. This means that you can easily arrange two
tables or images side by side. You just have to make sure that their
combined width is not larger than the text width.

You can also pack a paragraph of your choice into a box with either
the

\begin{lscommand}
  \csi{parbox}\verb|[|\emph{pos}\verb|]{|\emph{width}\verb|}{|\emph{text}\verb|}|
\end{lscommand}

\noindent command or the

\begin{lscommand}
  \verb|\begin{|\ei{minipage}\verb|}[|\emph{pos}\verb|]{|\emph{width}\verb|}| text
  \verb|\end{|\ei{minipage}\verb|}|
\end{lscommand}

\noindent environment. The \texttt{pos} parameter can take one of the letters
\texttt{c, t} or \texttt{b} to control the vertical alignment of the box,
relative to the baseline of the surrounding text. \texttt{width} takes
a length argument specifying the width of the box. The main difference
between a \ei{minipage} and a \csi{parbox} is that you cannot use all commands
and environments inside a \ei{parbox}, while almost anything is possible in
a \ei{minipage}.

While \csi{parbox} packs up a whole paragraph doing line breaking and
everything, there is also a class of boxing commands that operates
only on horizontally aligned material. We already know one of them;
it's called \csi{mbox}. It simply packs up a series of boxes into
another one, and can be used to prevent \LaTeX{} from breaking two
words. As boxes can be put inside boxes, these horizontal box packers
give you ultimate flexibility.

\begin{lscommand}
  \csi{makebox}\verb|[|\emph{width}\verb|][|\emph{pos}\verb|]{|\emph{text}\verb|}|
\end{lscommand}

\noindent \texttt{width} defines the width of the resulting box as
seen from the outside.\footnote{This means it can be smaller than the
  material inside the box. You can even set the
  width to 0pt so that the text inside the box will be typeset without
  influencing the surrounding boxes.}  Besides the length
expressions, you can also use \csi{width}, \csi{height}, \csi{depth}, and
\csi{totalheight} in the width parameter. They are set from values
obtained by measuring the typeset \emph{text}. The \emph{pos} parameter takes
a one letter value: \textbf{c}enter, flush\textbf{l}eft,
flush\textbf{r}ight, or \textbf{s}pread the text to fill the box.

The command \csi{framebox} works exactly the same as \csi{makebox}, but
it draws a box around the text.

The following example shows you some things you could do with
the \csi{makebox} and \csi{framebox} commands.

\begin{example}
\makebox[\textwidth]{%
    c e n t r a l}\par
\makebox[\textwidth][s]{%
    s p r e a d}\par
\framebox[1.1\width]{Guess I'm
    framed now!} \par
\hspace{1cm} %!hide
\framebox[0.8\width][r]{Bummer,
    I am too wide} \par
\framebox[1cm][l]{never
    mind, so am I}
Can you read this?
\end{example}

Now that we control the horizontal, the obvious next step is to go for
the vertical.\footnote{Total control is only to be obtained by
  controlling both the horizontal and the vertical \ldots}
No problem for \LaTeX{}. The

\begin{lscommand}
  \csi{raisebox}\verb|{|\emph{lift}\verb|}[|\emph{extend-above-baseline}\verb|][|\emph{extend-below-baseline}\verb|]{|\emph{text}\verb|}|
\end{lscommand}

\noindent command lets you define the vertical properties of a
box. You can use \csi{width}, \csi{height}, \csi{depth}, and
\csi{totalheight} in the first three parameters, in order to act
upon the size of the box inside the \emph{text} argument.

\begin{example}[examplewidth=0.45\linewidth]
\raisebox{0pt}[0pt][0pt]{\Large%
\textbf{Aaaa\raisebox{-0.3ex}{a}%
\raisebox{-0.7ex}{aa}%
\raisebox{-1.2ex}{r}%
\raisebox{-2.2ex}{g}%
\raisebox{-4.5ex}{h}}}
she shouted, but not even the next
one in line noticed that something
terrible had happened to her.
\end{example}

\section{Rules}\label{sec:rule}

A few pages back you may have noticed the command

\begin{lscommand}
  \csi{rule}\verb|[|\emph{lift}\verb|]{|\emph{width}\verb|}{|\emph{height}\verb|}|
\end{lscommand}

\noindent In normal use it produces a simple black box.

\begin{example}
\rule{3mm}{.1pt}%
\rule[-1mm]{5mm}{1cm}%
\rule{3mm}{.1pt}%
\rule[1mm]{1cm}{5mm}%
\rule{3mm}{.1pt}
\end{example}

\noindent This is useful for drawing vertical and horizontal
lines. The line on the title page, for example, has been created with a
\csi{rule} command.

\bigskip
\begin{FlushRight}
  The End.
\end{FlushRight}

%

% Local Variables:
% TeX-master: "lshort2e"
% mode: latex
% mode: flyspell
% End:

\appendix
\chapter{Installing \LaTeX}
\begin{intro}
Knuth published the source to \TeX{} back in a time when nobody knew
about OpenSource and/or Free Software. The License that comes with \TeX{}
lets you do whatever you want with the source, but you can only call the
result of your work \TeX{} if the program passes a set of tests Knuth has
also provided. This has lead to a situation where we have free \TeX{}
implementations for almost every Operating System under the sun. This chapter
will give some hints on what to install on Linux, Mac OS X and Windows, to
get a working \TeX{} setup.
\end{intro}

\section{What to Install}

For using \LaTeX{} on any computer system, you need several programs.

\begin{enumerate}

\item The \TeX{}/\LaTeX{} program for processing your \LaTeX{} source files
into typeset PDF or DVI documents.

\item A text editor for editing your \LaTeX{} source files. Some products even let
you start the \LaTeX{} program from within the editor.

\item A PDF/DVI viewer program for previewing and printing your
documents.

\item A program to handle \PSi{} files and images for inclusion into
your documents.

\end{enumerate}

For all platforms there are many programs that fit the requirements above.
Here we just tell about the ones we know, like and have some experience
with.

\section{\TeX{} on Mac OS X}

\subsection{Get a \TeX{} Distribution}

Just download \wi{MacTeX}. It is a
pre-compiled \LaTeX{} distribution for OS X. \wi{MacTeX} provides a full \LaTeX{}
installation plus a number of additional tools. Get Mac\TeX{} from
\url{http://www.tug.org/mactex/}.

If you are already using Macports or Fink for installing Unix software under
OS~X, install \LaTeX{} using these package managers. Macport users install
\LaTeX{} with \framebox{\texttt{port install texlive}},
Fink users use the command \framebox{\texttt{fink install texlive}}.

\subsection{Picking an Editor}

The most popular open source editor for \LaTeX{} on the mac seems to be
\TeX{}shop.  Get a copy from \url{http://www.uoregon.edu/~koch/texshop}. It
is also contained in the \wi{MacTeX} distribution.

Another fine editor is Texmaker. Apart from being a useful editor it has the
advantage of running on Windows, Mac and Unix/Linux equally well. Go to
\url{http://www.xm1math.net/texmaker} for further Information. Note there is
also a forked version of Texmaker called TexmakerX on
\url{http://texmakerx.sourceforge.net/} it promises additional functionality.

Recent \TeX Live distributions contain the \TeX{}works editor 
\url{http://texworks.org/} which is a multi-platform editor based on the \TeX{}Shop
design. Since \TeX{}works uses the Qt toolkit, it is available on any platform
supported by this toolkit (MacOS X, Windows, Linux.) 

\subsection{Treat yourself to \wi{PDFView}}

Use PDFView for viewing PDF files generated by \LaTeX{}, it integrates tightly
with your \LaTeX{} text editor. PDFView is an open-source application, available from the PDFView website on\\
\url{http://pdfview.sourceforge.net/}. After installing, open
PDFViews preferences dialog and make sure that the \emph{automatically reload
documents} option is enabled and that PDFSync support is set appropriately.

\section{\TeX{} on Windows}

\subsection{Getting \TeX{}}

First, get a copy of the excellent MiK\TeX\index{MiKTeX@MiK\TeX} distribution from\\
\url{http://www.miktex.org/}. It contains all the basic programs and files
required to compile \LaTeX{} documents.  The coolest feature in my eyes, is
that MiK\TeX{} will download missing \LaTeX{} packages on the fly and install them
magically while compiling a document. Alternatively you can also use
the TeXlive distribution which exists for Windows, Unix and Mac OS to
get your base setup going \url{http://www.tug.org/texlive/}.

\subsection{A \LaTeX{} editor}

\LaTeX{} is a programming language for text documents. \wi{TeXnicCenter}
uses many concepts from the programming-world to provide a nice and
efficient \LaTeX{} writing environment in Windows. Get your copy from\\
\url{http://www.texniccenter.org/}. TeXnicCenter integrates nicely with
MiKTeX. Version 2.0 of TeXnicCenter will support Unicode, the recent alpha version
seems to be quite stable.

Other excellent choice is the editor provided by the LEd project available
on \url{http://www.latexeditor.org}.

See the note on Texmaker in the Mac section above for a third choice.

Recent \TeX Live distributions contain the \TeX{}works Editor
\url{http://texworks.org/}. It supports Unicode and requires at least Windows XP.

\subsection{Document Preview}

You will most likely be using Yap for DVI preview as it gets installed with
MikTeX. For PDF you may want to look at Sumatra
PDF \url{http://blog.kowalczyk.info/software/sumatrapdf/}. I mention Sumatra PDF
because it lets you jump from any position in the pdf document back into
corresponding position in your source document.

\subsection{Working with graphics}

Working with high quality graphics in \LaTeX{} means that you have to use
\EPSi{} (eps) or PDF as your picture format. The program that helps you
deal with this is called \wi{GhostScript}. You can get it, together with its
own front-end \wi{GhostView}, from \url{http://www.cs.wisc.edu/~ghost/}.

If you deal with bitmap graphics (photos and scanned material), you may want
to have a look at the open source Photoshop alternative \wi{Gimp}, available
from \url{http://gimp-win.sourceforge.net/}.

\section{\TeX{} on Linux}

If you work with Linux, chances are high that \LaTeX{} is already installed
on your system, or at least available on the installation source you used to
setup. Use your package manager to install the following packages:

\begin{itemize}
\item texlive -- the base \TeX{}/\LaTeX{} setup.
\item emacs (with AUCTeX) -- an editor that integrates tightly with \LaTeX{} through the add-on AUCTeX package.
\item ghostscript -- a \PSi{} preview program.
\item xpdf and acrobat -- a PDF preview program.
\item imagemagick -- a free program for converting bitmap images.
\item gimp -- a free Photoshop look-a-like.
\item inkscape -- a free illustrator/corel draw look-a-like.
\end{itemize}

If you are looking for a more windows like graphical editing environment,
check out Texmaker or \TeX{}works. See the note in the Mac section above.

Most Linux distros insist on splitting up their \TeX{} environments into a
large number of optional packages, so if something is missing after your
first install, go check again.

\backmatter
%%%%%%%%%%%%%%%%%%%%%%%%%%%%%%%%%%%%%%%%%%%%%%%%%%%%%%%%%%%%%%%%%
% Contents: The Bibliography
% File: biblio.tex (lshort2e.tex)
% $Id$
%%%%%%%%%%%%%%%%%%%%%%%%%%%%%%%%%%%%%%%%%%%%%%%%%%%%%%%%%%%%%%%%%
\begin{thebibliography}{99}
\addcontentsline{toc}{chapter}{\bibname} 
\bibitem{manual} Leslie Lamport.  \newblock \emph{{\LaTeX:} A Document
    Preparation System}.  \newblock Addison-Wesley, Reading,
  Massachusetts, second edition, 1994, ISBN~0-201-52983-1.
  
\bibitem{texbook} Donald~E. Knuth.  \newblock \textit{The \TeX{}book,}
  Volume~A of \textit{Computers and Typesetting}, Addison-Wesley,
  Reading, Massachusetts, second edition, 1984, ISBN~0-201-13448-9.

\bibitem{companion} Frank Mittelbach, Michel Goossens, Johannes Braams,
  David Carlisle, Chris Rowley.  \newblock \emph{The {\LaTeX} Companion, (2nd Edition)}.  \newblock
  Addison-Wesley, Reading, Massachusetts, 2004, ISBN~0-201-36299-6.

\bibitem{graphicscompanion} Michel Goossens, Sebastian Rahtz and Frank
  Mittelbach.  \newblock \emph{The {\LaTeX} Graphics Companion}.  \newblock
  Addison-Wesley, Reading, Massachusetts, 1997, ISBN~0-201-85469-4.
 
\bibitem{local} Each \LaTeX{} installation should provide a so-called
  \emph{\LaTeX{} Local Guide}, which explains the things that are
  special to the local system.  It should be contained in a file called
  \texttt{local.tex}. Unfortunately, some lazy sysops do not provide such a
  document. In this case, go and ask your local \LaTeX{} guru for help.
 
\bibitem{usrguide} \LaTeX3 Project Team.  \newblock \emph{\LaTeXe~for
    authors}.  \newblock Comes with the \LaTeXe{} distribution as
  \texttt{usrguide.tex}.

\bibitem{clsguide} \LaTeX3 Project Team.  \newblock \emph{\LaTeXe~for
    Class and Package writers}.  \newblock Comes with the \LaTeXe{}
  distribution as \texttt{clsguide.tex}.

\bibitem{fntguide} \LaTeX3 Project Team.  \newblock \emph{\LaTeXe~Font
    selection}.  \newblock Comes with the \LaTeXe{} distribution as
  \texttt{fntguide.tex}.

\bibitem{graphics} D.~P.~Carlisle.  \newblock \emph{Packages in the
    `graphics' bundle}.  \newblock Comes with the `graphics' bundle as
  \texttt{grfguide.tex}, available from the same source your \LaTeX{}
  distribution came from.

\bibitem{verbatim} Rainer~Sch\"opf, Bernd~Raichle, Chris~Rowley.  
\newblock \emph{A New Implementation of \LaTeX's verbatim
  Environments}.
 \newblock Comes with the `tools' bundle as
  \texttt{verbatim.dtx}, available from the same source your \LaTeX{}
  distribution came from. 

\bibitem{cyrguide} Vladimir Volovich, Werner Lemberg and \LaTeX3 Project Team.                    
    \newblock \emph{Cyrillic languages support in \LaTeX}.                                        
    \newblock Comes with the \LaTeXe{} distribution as                                            
  \texttt{cyrguide.tex}.                                                                          

\bibitem{catalogue} Graham~Williams.  \newblock \emph{The TeX
    Catalogue} is a very complete listing of many \TeX{} and \LaTeX{}
    related packages.
  \newblock Available online from \CTAN|help/Catalogue/catalogue.html|
  
\bibitem{eps} Keith~Reckdahl.  \newblock \emph{Using EPS Graphics in
    \LaTeXe{} Documents}, which explains everything and much more than
  you ever wanted to know about EPS files and their use in \LaTeX{}
  documents.  \newblock Available online from
  \CTAN|info/epslatex.ps|

\bibitem{xy-pic} Kristoffer H. Rose.
  \newblock \emph{\Xy-pic User's Guide}.  \newblock
  Downloadable from CTAN with \Xy-pic distribution 
  
\bibitem{metapost} John D. Hobby.
  \newblock \emph{A User's Manual for \MP}. \newblock
  Downloadable from \url{http://cm.bell-labs.com/who/hobby/} 
  
\bibitem{unbound} Alan Hoenig.
  \newblock \emph{\TeX{} Unbound}. \newblock Oxford University Press, 1998,
    ISBN 0-19-509685-1; 0-19-509686-X (pbk.) 
  
\bibitem{ursoswald} Urs Oswald.  
    \newblock \emph{Graphics in \LaTeXe{}}, containing some Java source files for 
    generating arbitrary circles and ellipses within the \texttt{picture} environment,
    and \emph{\MP{} - A Tutorial}.
  \newblock Both downloadable from \url{http://www.ursoswald.ch}

\bibitem{pgfplot} Till Tantau.
  \newblock \emph{TikZ\&PGF Manual}.\newblock
  Download from \CTAN|graphics/pgf/base/doc/generic/pgf/pgfmanual.pdf|

% new items for XeLaTeX
\bibitem{polyglossia} Fran\c{c}ois Charette.                    
    \newblock \emph{Polyglossia: A Babel Replacement for \hologo{XeLaTeX}}.                                        
    \newblock Comes with the \TeX Live distribution as                                            
  \texttt{polyglossia.pdf}. (Type \texttt{texdoc polyglossia} on the command line.)

\bibitem{arabxetex} Fran\c{c}ois Charette.                    
    \newblock \emph{An Arab\TeX-like interface for typesetting languages
     in Arabic script with \hologo{XeLaTeX}}.                                        
    \newblock Comes with the \TeX Live distribution as                                            
  \texttt{arabxetex.pdf}. (Type \texttt{texdoc arabxetex} on the command line.)

\bibitem{fontspec} Will Robertson and Khaled Hosny.                    
    \newblock \emph{The \texttt{fontspec} package}.                                        
    \newblock Comes with the \TeX Live distribution as                                            
  \texttt{fontspec.pdf}. (Type \texttt{texdoc fontspec} on the command line.)
    
\bibitem{xgreek} Apostolos Syropoulos.                    
    \newblock \emph{The \texttt{xgreek} package}.                                        
    \newblock Comes with the \TeX Live distribution as                                            
  \texttt{xgreek.pdf}. (Type \texttt{texdoc xgreek} on the command line.)

\bibitem{bidi} Vafa Khalighi.                    
    \newblock \emph{The \texttt{bidi} package}.                                        
    \newblock Comes with the \TeX Live distribution as                                            
  \texttt{bidi.pdf}. (Type \texttt{texdoc bidi} on the command line.

\bibitem{xepersian} Vafa Khalighi.                    
    \newblock \emph{The \texttt{XePersian} package}.                                        
    \newblock Comes with the \TeX Live distribution as                                            
  \texttt{xepersian-doc.pdf}. (Type \texttt{texdoc xepersian} on the command line.

\bibitem{xecjk} Wenchang Sun.                    
    \newblock \emph{The \texttt{xeCJK} package}.                                        
    \newblock Comes with the \TeX Live distribution as                                            
  \texttt{xeCJK.pdf}. (Type \texttt{texdoc xecjk} on the command line.

\end{thebibliography}


%

% Local Variables:
% TeX-master: "lshort2e"
% mode: latex
% mode: flyspell
% End:

\refstepcounter{chapter}
\addcontentsline{toc}{chapter}{Index} 
\printindex
\end{document}





%

% Local Variables:
% TeX-master: "lshort2e"
% mode: latex
% mode: flyspell
% End:
