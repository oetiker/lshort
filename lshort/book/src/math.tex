%%%%%%%%%%%%%%%%%%%%%%%%%%%%%%%%%%%%%%%%%%%%%%%%%%%%%%%%%%%%%%%%
% Contents: Math typesetting with LaTeX
% $Id$
%
% Changes by Stefan M. Moser: 2008/10/22
%
% -Section 2: "Single Equations": added comment about preference of
%  equation* over \[
% -Replaced (almost) all examples with \[ by equation*
% -New section 4: "Single Equations that are Too Long: multline"
% -New section 5: "Multiple Equations"
% -Section 6: "Arrays and Matrices": made a full section and added
%  some material
% -Section 9: "Theorems, Lemmas, ...": added a subsection about proofs
%  with new material
%
% Other Changes:
% -in lshort.sty: 
%    *example environment adapted: changed in three places
%     \textwidth by \linewidth. This is necessary for
%     example-environment within a itemize-list.
%    *added \RequirePackage[retainorgcmds]{IEEEtrantools}
%
% THINGS TO DO:
% -adapt typesetting of new sections to rest of lshort, including all
%  the usual commands used so far. In particular, I guess we have to
%  get rid of the \verb-commands everywhere
% -include index-commands
%%%%%%%%%%%%%%%%%%%%%%%%%%%%%%%%%%%%%%%%%%%%%%%%%%%%%%%%%%%%%%%%%
 
\chapter{Typesetting Mathematical Formulae}

\begin{intro}
  Now you are ready! In this chapter, we will attack the main strength
  of \TeX{}: mathematical typesetting. But be warned, this chapter
  only scratches the surface. While the things explained here are
  sufficient for many people, don't despair if you can't find a
  solution to your mathematical typesetting needs here. It is highly likely
  that your problem is addressed in \AmS-\LaTeX{}.
\end{intro}
  


\section{The \texorpdfstring{\AmS}{AMS}-\LaTeX{} bundle}

If you want to typeset (advanced) \wi{mathematics}, you should
use \AmS-\LaTeX{}. The \AmS-\LaTeX{} bundle is a collection of packages and classes for
mathematical typesetting. We will mostly deal with the \pai{amsmath} package
which is a part of the bundle. \AmS-\LaTeX{} is produced by The \emph{\wi{American Mathematical Society}} 
and it is used extensively for mathematical typesetting. \LaTeX{} itself does provide
some basic features and environments for mathematics, but they are limited (or
maybe it's the other way around: \AmS-\LaTeX{} is \emph{unlimited}!) and
in some cases inconsistent. 

\AmS-\LaTeX{} is a part of the required distribution and is provided
with all recent \LaTeX{} distributions.\footnote{If yours is missing it, go to
  \CTAN|macros/latex/required/amslatex|.} In this chapter, we assume
  \pai{amsmath} is loaded in the preamble; \verb|\usepackage{amsmath}|.

\section{Single Equations}
  
There are two ways to typeset mathematical \wi{formulae}: in-line within a paragraph
(\emph{\wi{text style}}), or the paragraph can be broken to 
typeset it separately (\textit{\wi{display style}}). Mathematical \wi{equation}s 
\emph{within} a paragraph are entered \index{$@\texttt{\$}} %$
between \texttt{\$} and \texttt{\$}:
\begin{example}
Add $a$ squared and $b$ squared
to get $c$ squared. Or, using 
a more mathematical approach:
$a^2 + b^2 = c^2$
\end{example}
\begin{example}
\TeX{} is pronounced as 
$\tau\epsilon\chi$\\[5pt]
100~m$^{3}$ of water\\[5pt]
This comes from my $\heartsuit$
\end{example}

If you want your larger equations to be set apart
from the rest of the paragraph, it is preferable to \emph{display} them
rather than to break the paragraph apart.
To do this, you enclose them between \verb|\begin{|\ei{equation}\verb|}| and
\verb|\end{equation}|.\footnote{This is an \textsf{amsmath} command. If you don't
have access to the package for some obscure reason, you can use \LaTeX's own
\ei{displaymath} environment instead.} You can then \ci{label} an equation number and refer to
it somewhere else in the text by using the \ci{eqref} command. If you want to
name the equation something specific, you \ci{tag} it instead.
\begin{example}
Add $a$ squared and $b$ squared
to get $c$ squared. Or, using
a more mathematical approach
 \begin{equation}
   a^2 + b^2 = c^2
 \end{equation}
Einstein says
 \begin{equation}
   E = mc^2 \label{clever}
 \end{equation}
He didn't say
 \begin{equation}
  1 + 1 = 3 \tag{dumb}
 \end{equation}
This is a reference to 
\eqref{clever}. 
\end{example}

If you don't want \LaTeX{} to number the equations, use the starred
version of \texttt{equation} using an asterisk, \ei{equation*}, or even easier, enclose the
equation in \ci{[} and \ci{]}:\footnote{\index{equation!\textsf{amsmath}}
  \index{equation!\LaTeX{}}This is again from \textsf{amsmath}. Standard \LaTeX{}'s has only the \texttt{equation} environment without the star.}
\begin{example}
Add $a$ squared and $b$ squared
to get $c$ squared. Or, using
a more mathematical approach
 \begin{equation*}
   a^2 + b^2 = c^2
 \end{equation*}
or you can type less for the
same effect:
 \[ a^2 + b^2 = c^2 \]
\end{example}
While \ci{[} is short and sweet, it does not allow to switch between numbered and not numbered style as easily as
nicely \ei{equation} and \ei{equation*}.

Note the difference in typesetting style between \wi{text style} and \wi{display style}
equations: 
\begin{example}
This is text style: 
$\lim_{n \to \infty} 
 \sum_{k=1}^n \frac{1}{k^2} 
 = \frac{\pi^2}{6}$.
And this is display style:
 \begin{equation}
  \lim_{n \to \infty} 
  \sum_{k=1}^n \frac{1}{k^2} 
  = \frac{\pi^2}{6}
 \end{equation}
\end{example}

In text style, enclose tall or deep math expressions or sub
expressions in \ci{smash}. This makes \LaTeX{} ignore the height of
these expressions. This keeps the line spacing even.

\begin{example}
A $d_{e_{e_p}}$ mathematical
expression  followed by a
$h^{i^{g^h}}$ expression. As
opposed to a smashed 
\smash{$d_{e_{e_p}}$} expression 
followed by a
\smash{$h^{i^{g^h}}$} expression.
\end{example}

\subsection{Math Mode}

There are also differences between \emph{\wi{math mode}} and \emph{text mode}. For
example, in \emph{math mode}: 

\begin{enumerate}

\item \index{spacing!math mode} Most spaces and line breaks do not have any significance, as all spaces
are either derived logically from the mathematical expressions, or
have to be specified with special commands such as \ci{,}, \ci{quad} or
\ci{qquad} (we'll get back to that later, see section~\ref{sec:math-spacing}).
 
\item Empty lines are not allowed. Only one paragraph per formula.

\item Each letter is considered to be the name of a variable and will be
typeset as such. If you want to typeset normal text within a formula
(normal upright font and normal spacing) then you have to enter the
text using the \verb|\text{...}| command (see also section \ref{sec:fontsz} on
page \pageref{sec:fontsz}).

\end{enumerate}
\begin{example}
$\forall x \in \mathbf{R}:
 \qquad x^{2} \geq 0$
\end{example}
\begin{example}
$x^{2} \geq 0\qquad
 \text{for all }x\in\mathbf{R}$
\end{example}
 
Mathematicians can be very fussy about which symbols are used:
it would be conventional here to use the `\wi{blackboard bold}' font,
\index{bold symbols} which is obtained using \ci{mathbb} from the
package \pai{amssymb}.\footnote{\pai{amssymb} is not a part
  of the \AmS-\LaTeX{} bundle, but it is perhaps still a part of your \LaTeX{}
  distribution. Check your distribution
  or go to \texttt{CTAN:/fonts/amsfonts/latex/} to obtain it.}
\ifx\mathbb\undefined\else
The last example becomes
\begin{example}
$x^{2} \geq 0\qquad
 \text{for all } x 
 \in \mathbb{R}$
\end{example}
\fi
See Table~\ref{mathalpha} on page~\pageref{mathalpha} and
Table~\ref{mathfonts} on page~\pageref{mathfonts} for more math fonts.



\section{Building Blocks of a Mathematical Formula}

In this section, we describe the most important commands used in mathematical
typesetting. Most of the commands in this section will not require
\textsf{amsmath} (if they do, it will be stated clearly), but load it anyway.


\textbf{Lowercase \wi{Greek letters}} are entered as \verb|\alpha|,
 \verb|\beta|, \verb|\gamma|, \ldots, uppercase letters
are entered as \verb|\Gamma|, \verb|\Delta|, \ldots\footnote{There is no
  uppercase Alpha, Beta etc. defined in \LaTeXe{} because it looks the same as a 
  normal roman A, B\ldots{} Once the new math coding is done, things will
  change.} 

Take a look at Table~\ref{greekletters} on page~\pageref{greekletters} for a
list of Greek letters.
\begin{example}
$\lambda,\xi,\pi,\theta,
 \mu,\Phi,\Omega,\Delta$
\end{example}


\textbf{Exponents and Subscripts} can be specified using\index{exponent}\index{subscript}
the \verb|^|\index{^@\verb"|^"|} and the \verb|_|\index{_@\verb"|_"|} character.
Most math mode commands act only on the next character, so if you
want a command to affect several characters, you have to group them
together using curly braces: \verb|{...}|.

Table~\ref{binaryrel} on page \pageref{binaryrel} lists a lot of other binary
relations like $\subseteq$ and $\perp$.

\begin{example}
$p^3_{ij} \qquad
 m_\text{Knuth} \\[5pt]
 a^x+y \neq a^{x+y}\qquad 
 e^{x^2} \neq {e^x}^2$
\end{example}


The \textbf{\wi{square root}} is entered as \ci{sqrt}; the
$n^\text{th}$ root is generated with \verb|\sqrt[|$n$\verb|]|. The size of
the root sign is determined automatically by \LaTeX. If just the sign
is needed, use \verb|\surd|.

See other kinds of arrows like $\hookrightarrow$ and $\rightleftharpoons$ on
Table~\ref{tab:arrows} on page \pageref{tab:arrows}. 
\begin{example}
$\sqrt{x} \Leftrightarrow x^{1/2}
 \quad \sqrt[3]{2}
 \quad \sqrt{x^{2} + \sqrt{y}}
 \quad \surd[x^2 + y^2]$
\end{example}


\index{dots!three}
\index{vertical!dots}
\index{horizontal!dots}
While the \textbf{\wi{dot}} sign to indicate
the multiplication operation is normally left out, it is sometimes written
to help the eye in grouping a formula.
Use \ci{cdot} to typesets a single centered dot. \ci{cdots} is
three centered \textbf{\wi{dots}} while \ci{ldots} sets the dots on the
baseline. Besides that, there are \ci{vdots} for 
vertical and \ci{ddots} for \wi{diagonal dots}. There are more examples in 
section~\ref{sec:arraymat}.
\begin{example}
$\Psi = v_1 \cdot v_2
 \cdot \ldots \qquad 
 n! = 1 \cdot 2 
 \cdots (n-1) \cdot n$
\end{example}

The commands \ci{overline} and \ci{underline} create
\textbf{horizontal lines} directly over or under an expression:
\index{horizontal!line} \index{line!horizontal}
\begin{example}
$0.\overline{3} = 
 \underline{\underline{1/3}}$
\end{example}

The commands \ci{overbrace} and \ci{underbrace} create
long \textbf{horizontal braces} over or under an expression:
\index{horizontal!brace} \index{brace!horizontal} 
\begin{example}
$\underbrace{\overbrace{a+b+c}^6 
 \cdot \overbrace{d+e+f}^9}
 _\text{meaning of life} = 42$
\end{example}

\index{mathematical!accents} To add mathematical accents such as \textbf{small
arrows} or \textbf{\wi{tilde}} signs to variables, the commands
given in Table~\ref{mathacc} on page~\pageref{mathacc} might be useful.  Wide hats and
tildes covering several characters are generated with \ci{widetilde}
and \ci{widehat}. Notice the difference between \ci{hat} and \ci{widehat} and the placement of
\ci{bar} for a variable with subscript. The \wi{apostrophe} mark
\verb|'|\index{'@\verb"|'"|} gives a \wi{prime}:
% a dash is --
\begin{example}
$f(x) = x^2 \qquad f'(x) 
 = 2x \qquad f''(x) = 2\\[5pt]
 \hat{XY} \quad \widehat{XY}
 \quad \bar{x_0} \quad \bar{x}_0$
\end{example}


\textbf{Vectors}\index{vectors} are often specified by adding small
\wi{arrow symbols} on top of a variable. This is done with the
\ci{vec} command. The two commands \ci{overrightarrow} and
\ci{overleftarrow} are useful to denote the vector from $A$ to $B$:
\begin{example}
$\vec{a} \qquad
 \vec{AB} \qquad
 \overrightarrow{AB}$
\end{example}


Names of log-like functions are often typeset in an upright
font, and not in italics as variables are, so \LaTeX{} supplies the
following commands to typeset the most important function names:
\index{mathematical!functions}

\begin{tabular}{llllll}
\ci{arccos} &  \ci{cos}  &  \ci{csc} &  \ci{exp} &  \ci{ker}    & \ci{limsup} \\
\ci{arcsin} &  \ci{cosh} &  \ci{deg} &  \ci{gcd} &  \ci{lg}     & \ci{ln}     \\
\ci{arctan} &  \ci{cot}  &  \ci{det} &  \ci{hom} &  \ci{lim}    & \ci{log}    \\
\ci{arg}    &  \ci{coth} &  \ci{dim} &  \ci{inf} &  \ci{liminf} & \ci{max}    \\
\ci{sinh}   & \ci{sup}   &  \ci{tan}  & \ci{tanh}&  \ci{min}    & \ci{Pr}     \\
\ci{sec}    & \ci{sin} \\
\end{tabular}

\begin{example}
\begin{equation*}
  \lim_{x \rightarrow 0}
  \frac{\sin x}{x}=1
\end{equation*}
\end{example}

For functions missing from the list, use the \ci{DeclareMathOperator}
command. There is even a starred version for functions with limits.
This command works only in the preamble so the commented lines in the
example below must be put into the preamble.

\begin{example}
%\DeclareMathOperator{\argh}{argh}
%\DeclareMathOperator*{\nut}{Nut}
\begin{equation*}
  3\argh = 2\nut_{x=1}    
\end{equation*}
\end{example}

For the \wi{modulo function}, there are two commands: \ci{bmod} for the
binary operator ``$a \bmod b$'' and \ci{pmod}
for expressions
such as ``$x\equiv a \pmod{b}$:''
\begin{example}
$a\bmod b \\
 x\equiv a \pmod{b}$
\end{example}

A built-up \textbf{\wi{fraction}} is typeset with the
\ci{frac}\verb|{...}{...}| command. In in-line equations, the fraction is shrunk to
fit the line. This style is obtainable in display style with \ci{tfrac}. The
reverse, i.e.\ display style fraction in text, is made with \ci{dfrac}.
Often the slashed form $1/2$ is preferable, because it looks better
for small amounts of `fraction material:'
\begin{example}
In display style:
\begin{equation*}
  3/8 \qquad \frac{3}{8} 
  \qquad \tfrac{3}{8}
\end{equation*}
\end{example}

\begin{example}
In text style:
$1\frac{1}{2}$~hours \qquad
$1\dfrac{1}{2}$~hours
\end{example}
 
Here the \ci{partial} command for \wi{partial derivative}s is used:
\begin{example}
\begin{equation*} 
  \sqrt{\frac{x^2}{k+1}}\qquad
  x^\frac{2}{k+1}\qquad
  \frac{\partial^2f}
  {\partial x^2} 
\end{equation*}
\end{example}

To typeset \wi{binomial coefficient}s or similar structures, use
the command \ci{binom} from \pai{amsmath}:
\begin{example}
Pascal's rule is
\begin{equation*}
 \binom{n}{k} =\binom{n-1}{k}
 + \binom{n-1}{k-1}
\end{equation*}
\end{example}

For \wi{binary relations} it may be useful to stack symbols over each other.
\ci{stackrel}\verb|{#1}{#2}| puts the symbol given
in \verb|#1| in superscript-like size over \verb|#2| which
is set in its usual position.
\begin{example}
\begin{equation*}
 f_n(x) \stackrel{*}{\approx} 1
\end{equation*}
\end{example}

The \textbf{\wi{integral operator}} is generated with \ci{int}, the
\textbf{\wi{sum operator}} with \ci{sum}, and the \textbf{\wi{product operator}}
with \ci{prod}. The upper and lower limits are specified with~\verb|^|
and~\verb|_| like subscripts and superscripts:
\begin{example}
\begin{equation*}
\sum_{i=1}^n \qquad
\int_0^{\frac{\pi}{2}} \qquad
\prod_\epsilon
\end{equation*}
\end{example}

To get more control over the placement of indices in complex
expressions, \pai{amsmath} provides the \ci{substack} command:
\begin{example}
\begin{equation*}
\sum^n_{\substack{0<i<n \\ 
        j\subseteq i}}
   P(i,j) = Q(i,j)
\end{equation*}
\end{example}



\LaTeX{} provides all sorts of symbols for \textbf{\wi{braces}} and other
\textbf{\wi{delimiters}} (e.g.~$[\;\langle\;\|\;\updownarrow$).
Round and square braces can be entered with the corresponding keys and
curly braces with \verb|\{|, but all other delimiters are generated with
special commands (e.g.~\verb|\updownarrow|).
\begin{example}
\begin{equation*}
{a,b,c} \neq \{a,b,c\}
\end{equation*}
\end{example}

If you put \ci{left} in front of an opening delimiter and
\ci{right} in front of a closing delimiter, \LaTeX{} will automatically
determine the correct size of the delimiter. Note that you must close
every \ci{left} with a corresponding \ci{right}. If you
don't want anything on the right, use the invisible ``\ci{right.}'':
\begin{example}
\begin{equation*}
1 + \left(\frac{1}{1-x^{2}}
    \right)^3 \qquad 
\left. \ddagger \frac{~}{~}\right)
\end{equation*}
\end{example}

In some cases it is necessary to specify the correct size of a
mathematical delimiter\index{mathematical!delimiter} by hand,
which can be done using the commands \ci{big}, \ci{Big}, \ci{bigg} and
\ci{Bigg} as prefixes to most delimiter commands:
\begin{example}
$\Big((x+1)(x-1)\Big)^{2}$\\
$\big( \Big( \bigg( \Bigg( \quad
\big\} \Big\} \bigg\} \Bigg\} \quad
\big\| \Big\| \bigg\| \Bigg\| \quad
\big\Downarrow \Big\Downarrow 
\bigg\Downarrow \Bigg\Downarrow$
\end{example}
 For a list of all delimiters available, see Table~\ref{tab:delimiters} on page
\pageref{tab:delimiters}. 


\section{Single Equations that are Too Long: multline}
\index{long equations}
\label{sec:multline}

If an equation is too long, we have to wrap it somehow. Unfortunately,
wrapped equations are usually less easy to read than not wrapped
ones. To improve the readability, there are certain rules on how to do
the wrapping:
\begin{enumerate}
\item In general one should always wrap an equation \textbf{before} an
  equality sign or an operator.
\item A wrap before an equality sign is preferable to a wrap before
  any operator.
\item A wrap before a plus- or minus-operator is preferable to a wrap
  before a multiplication-operator.
\item Any other type of wrap should be avoided if ever possible.
\end{enumerate}
The easiest way to achieve such a wrapping is the use of the
\ei{multline} en\-vi\-ron\-ment:\footnote{The
  \texttt{multline}-environment is from \texttt{amsmath}.}
\begin{example}
\begin{multline}
  a + b + c + d + e + f 
  + g + h + i  
  \\
  = j + k + l + m + n 
\end{multline}
\end{example}
\noindent
The difference to the \ei{equation} environment is that an arbitrary
line-break (or also multiple line-breaks) can be introduced. This is
done by putting a \verb+\\+ on those places where the equation needs
to be wrapped. Similarly to \ei{equation*} there also exists a
\ei{multline*} version for preventing an equation number.

Often the
\ei{IEEEeqnarray} environment (see section~\ref{sec:IEEEeqnarray})
will yield better results.  Consider the following
situation:
\begin{example}
\begin{equation}
  a = b + c + d + e + f 
  + g + h + i + j 
  + k + l + m + n + o + p  
  \label{eq:equation_too_long}
\end{equation}
\end{example}
\noindent
Here it is actually the RHS that is too long to fit on one line. The
\ei{multline} environment creates the following output:
\begin{example}
\begin{multline}
  a = b + c + d + e + f 
  + g + h + i + j \\
  + k + l + m + n + o + p
\end{multline}
\end{example}

This is better than \eqref{eq:equation_too_long}, but
it has the disadvantage that the equality sign loses its natural
stronger importance with respect to the plus operator in front of
$k$. The better solution is provided by the
\ei{IEEEeqnarray} environment that will be discussed in detail in
Section~\ref{sec:IEEEeqnarray}.

\section{Multiple Equations}
\index{equation!multiple}
\label{sec:IEEEeqnarray}

In the most general situation we have a sequence of several
equalities that do not fit onto one line. Here we need to work with
vertical alignment in order to keep the array of equations in a nice
and readable structure.

Before we offer our suggestions on how to do this, we start with a few
bad examples that show the biggest drawbacks of some common solutions.


\subsection{Problems with Traditional Commands}
\label{sec:problems_traditional}

To group multiple equations the
\ei{align} environment\footnote{The \texttt{align}-environment can
  also be used to group several blocks of equations beside each other.
  Another excellent use case for
  \ei{IEEEeqnarray} environment. Try an argument like,
  \texttt{\{rCl+rCl\}}.} could be used:
\begin{example}
\begin{align}
  a & = b + c \\
  & = d + e
\end{align}
\end{example}

this approach fails once a single line is too long:
\begin{example}
\begin{align}
  a & = b + c \\
  & = d + e + f + g + h + i 
  + j + k + l \nonumber \\
  & + m + n + o \\
  & = p + q + r + s
\end{align}
\end{example}
\noindent
Here $+\:m$ should be below $d$ and not below the equality sign. Of
course, one could add some space (\verb+\hspace{...}+),
but this will never yield a precise arrangement (and is bad
style\ldots).

A better solution is offered by the \ei{eqnarray} environment:
\begin{example}
\begin{eqnarray}
  a & = & b + c \\
  & = & d + e + f + g + h + i 
  + j + k + l \nonumber \\
  && +\: m + n + o \\
  & = & p + q + r + s
\end{eqnarray}
\end{example}

This is still not optimal though:
\begin{itemize}
\item The spaces around the equality signs are too big.
  Particularly, they are \textbf{not} the same as in the
  \ei{multline} and \ei{equation} environments:
\begin{example}
\begin{eqnarray}
  a & = & a = a
\end{eqnarray}
\end{example}

\item The expression sometimes overlaps with the equation number even
  though there would be enough room on the left:
\begin{example}
\begin{eqnarray}
  a & = & b + c 
  \\
  & = & d + e + f + g + h^2 
  + i^2 + j 
  \label{eq:faultyeqnarray}
\end{eqnarray}
\end{example}

\item The environment offers a command \ci{lefteqn} that can
  be used when the LHS is too long:
\begin{example}
\begin{eqnarray}
  \lefteqn{a + b + c + d 
    + e + f + g + h}\nonumber\\
  & = & i + j + k + l + m 
  \\
  & = & n + o + p + q + r + s
\end{eqnarray}
\end{example}
again, this is not optimald as the RHS is too short and the array is
not properly centered:
\begin{example}
\begin{eqnarray}
  \lefteqn{a + b + c + d 
    + e + f + g + h} 
  \nonumber \\
  & = & i + j 
\end{eqnarray}
\end{example}
\end{itemize}

Fortunately there is a better way\ldots.

\subsection{IEEEeqnarray-Environment}
\label{sec:IEEEeqnarray_intro}

The \ei{IEEEeqnarray} environment is a very powerful command with
many options. Here, we will only introduce its basic
functionalities. For more information we refer to the
manual.\footnote{The official manual is called
  \texttt{IEEEtran\_HOWTO.pdf}. The part about \texttt{IEEEeqnarray}
  can be found in Appendix~F.}

First of all, in order to be able to use the
\ei{IEEEeqnarray} environment one needs to load the
package\footnote{This package can be found on CTAN.}
\pai{IEEEtrantools}. Include the following line in the header of
your document: \small
\begin{verbatim}
\usepackage[retainorgcmds]{IEEEtrantools}
\end{verbatim}
\normalsize

The strength of \ei{IEEEeqnarray} is the possibility of specifying
the number of \emph{columns} in the equation array. Usually, this
specification will be \verb+{rCl}+, \emph{i.e.}, three columns, the
first column right-justified, the middle one centered with a little
more space around it (therefore we specify capital \texttt{C} instead of
lower-case \texttt{c}) and the third column left-justified:
\begin{example}
\begin{IEEEeqnarray}{rCl}
  a & = & b + c 
  \\
  & = & d + e + f + g + h 
  + i + j + k \nonumber\\
  && +\: l + m + n + o 
  \\
  & = & p + q + r + s
\end{IEEEeqnarray}
\end{example}

any number of columns can be specified:
\verb+{c}+ will give only one column with all entries centered, or
\verb+{rCll}+ would add a fourth, left-justified column to use
for comments. Moreover, beside \texttt{l}, \texttt{c}, \texttt{r}, \texttt{L},
\texttt{C}, \texttt{R} for math mode entries there are also \texttt{s},
\texttt{t}, \texttt{u} for left, centered, and right text mode entries.
Additional space can be added with \texttt{.} and
\texttt{/} and \texttt{?} in increasing order.\footnote{For more spacing
  types refer to Section~\ref{sec:putting-qed-right}.}

Note the spaces around the equality signs in contrast to the space produced
by the \texttt{eqnarray} environment.

\subsection{Common Usage}
\label{sec:common-usage}

In the following we will describe how we use \texttt{IEEEeqnarray} to
solve the most common situations.
\begin{itemize}
\item If a line overlaps with the equation number as in
  \eqref{eq:faultyeqnarray}, the command 
\small
\begin{verbatim}
\IEEEeqnarraynumspace
\end{verbatim} 
\normalsize
  can be used: it has to be added in the corresponding line and makes
  sure that the whole equation array is shifted by the size of the
  equation numbers (the shift depends on the size of the number!):
  instead of
\begin{example}
\begin{IEEEeqnarray}{rCl}
  a & = & b + c 
  \\
  & = & d + e + f + g + h 
  + i + j + k 
  \\
  & = & l + m + n
\end{IEEEeqnarray}
\end{example}
  we get
\begin{example}
\begin{IEEEeqnarray}{rCl}
  a & = & b + c 
  \\
  & = & d + e + f + g + h 
  + i + j + k 
  \IEEEeqnarraynumspace\\
  & = & l + m + n.
\end{IEEEeqnarray}
\end{example}

\item If the LHS is too long, as a replacement for the faulty
  \ci{lefteqn} command, \texttt{IEEEeqnarray} offers the
  \ci{IEEEeqnarraymulticol} command which works in all situations:
\begin{example}
\begin{IEEEeqnarray}{rCl}
  \IEEEeqnarraymulticol{3}{l}{
    a + b + c + d + e + f 
    + g + h
  }\nonumber\\ \quad
  & = & i + j 
  \\
  & = & k + l + m
\end{IEEEeqnarray}
\end{example}
The usage is identical to the \ci{multicolumns} command in the
\texttt{tabular}-en\-vi\-ron\-ment. The first argument \verb+{3}+
specifies that three columns shall be combined to one which will be
left-justified \verb+{l}+.

Note that by adapting the \ci{quad} command one can easily adapt
the depth of the equation signs,\footnote{I think that one quad is the
  distance that looks good for most cases.} \emph{e.g.},
\begin{example}
\begin{IEEEeqnarray}{rCl}
  \IEEEeqnarraymulticol{3}{l}{
    a + b + c + d + e + f 
    + g + h
  }\nonumber\\ \qquad\qquad
  & = & i + j
  \\
  & = & k + l + m
\end{IEEEeqnarray}
\end{example}

\item If an equation is split into two or more lines, \LaTeX\
  interprets the first $+$ or $-$ as sign instead of operator.
  Therefore, it is necessary to add an additional space \ci{:}
  between the operator and the term: instead of
\begin{example}
\begin{IEEEeqnarray}{rCl}
  a & = & b + c 
  \\
  & = & d + e + f + g + h 
  + i + j + k \nonumber\\
  && + l + m + n + o 
  \\
  & = & p + q + r + s
\end{IEEEeqnarray}
\end{example}
  we should write
\begin{example}
\begin{IEEEeqnarray}{rCl}
  a & = & b + c 
  \\
  & = & d + e + f + g + h 
  + i + j + k \nonumber\\
  && +\: l + m + n + o 
  \\
  & = & p + q + r + s
\end{IEEEeqnarray}
\end{example}
  (Compare the space between $+$ and $l$!)
  
  \textbf{Attention:} \LaTeX\ is not completely silly: in certain
  situations like, \emph{e.g.}, in front of
  \begin{itemize}
  \item an operator name like \ci{log}, \ci{sin}, \ci{det},
    \ci{max}, \emph{etc.},
  \item an integral \ci{int} or sum \ci{sum},
  \item a bracket with adaptive size using \ci{left} and
      \ci{right} (this is in contrast to normal brackets or
    brackets with fixed size like \ci{big(} ),
  \end{itemize}
  a $+$ or $-$ cannot be a sign, but must be an operator. In those
  situations \LaTeX\ will add the correct spacing and no additional
  space is needed.
  \begin{itemize}
  \item[$\rhd$] \it Whenever you wrap a line, quickly check the result
    and verify that the spacing is correct!
  \end{itemize}

\item If a particular line should not have an equation number, the
  number can be suppressed using \ci{nonumber} (or
  \ci{IEEEnonumber}). If on such a line a label
  \verb+\label{eq:...}+ is defined, then this label is passed on
  further to the next equation number that is not suppressed. Put the labels right before the line-break
  \verb+\\+ or the end of the equation it belongs to. Apart from
  improving the readability of the source code this prevents a
  compilation error in the situation of a \ci{IEEEmulticol} command
  after the label-definition.
  
\item There also exists a *-version where all equation numbers are
  suppressed. In this case an equation number can be made to appear
  using the command \ci{IEEEyesnumber}:
\begin{example}
\begin{IEEEeqnarray*}{rCl}
  a & = & b + c \\
  & = & d + e \IEEEyesnumber\\
  & = & f + g
\end{IEEEeqnarray*}
\end{example}

\item Sub-numbers are also easily possible using 
  \ci{IEEEyessubnumber}:
\begin{example}
\begin{IEEEeqnarray}{rCl}
  a & = & b + c 
  \IEEEyessubnumber\\
  & = & d + e 
  \nonumber\\
  & = & f + g 
  \IEEEyessubnumber  
\end{IEEEeqnarray}
\end{example}
  
\end{itemize}



\section{Arrays and Matrices} \label{sec:arraymat}

To typeset \textbf{arrays}, use the \ei{array} environment. It works
somewhat similar to the \texttt{tabular} environment. The \verb|\\| command is
used to break the lines:
\begin{example}
  \begin{equation*}
    \mathbf{X} = \left( 
      \begin{array}{ccc}
        x_1 & x_2 & \ldots \\
        x_3 & x_4 & \ldots \\
        \vdots & \vdots & \ddots
      \end{array} \right)
  \end{equation*}
\end{example}

The \ei{array} environment can also be used to typeset \wi{piecewise function}s by
using a ``\verb|.|'' as an invisible \ci{right} delimiter:
\begin{example}
\begin{equation*}
  |x| = \left\{
    \begin{array}{rl}
      -x & \text{if } x < 0,\\
      0 & \text{if } x = 0,\\
      x & \text{if } x > 0.
    \end{array} \right.
\end{equation*}
\end{example}
The \ei{cases} environment from \textsf{amsmath} simplifies
the syntax, so it is worth a look:
\begin{example}
  \begin{equation*}
    |x| = 
    \begin{cases}
      -x & \text{if } x < 0,\\
      0 & \text{if } x = 0,\\
      x & \text{if } x > 0.
    \end{cases} 
\end{equation*}
\end{example}


Matrices\index{matrix} can also be typeset by \ei{array}, but
\pai{amsmath} provides a better solution using the different \ei{matrix}
environments. There are six versions with different delimiters: \ei{matrix}
(none), \ei{pmatrix} $($, \ei{bmatrix} $[$, \ei{Bmatrix} $\{$, \ei{vmatrix} $\vert$ and
\ei{Vmatrix} $\Vert$. You don't have to specify the number of columns as with
\ei{array}. The maximum number is 10, but it is customisable (though it is not
very often you need 10 columns!):
\begin{example}
\begin{equation*}
  \begin{matrix} 
    1 & 2 \\
    3 & 4 
  \end{matrix} \qquad
  \begin{bmatrix} 
    p_{11} & p_{12} & \ldots 
    & p_{1n} \\
    p_{21} & p_{22} & \ldots 
    & p_{2n} \\
    \vdots & \vdots & \ddots 
    & \vdots \\
    p_{m1} & p_{m2} & \ldots 
    & p_{mn} 
  \end{bmatrix}
\end{equation*}
\end{example}



\section{Spacing in Math Mode} \label{sec:math-spacing}

\index{math spacing} If the spacing within formulae chosen by \LaTeX{}
is not satisfactory, it can be adjusted by inserting special spacing
commands: \ci{,} for $\frac{3}{18}\:\textrm{quad}$
(\demowidth{0.166em}), \ci{:} for $\frac{4}{18}\: \textrm{quad}$
(\demowidth{0.222em}) and \ci{;} for $\frac{5}{18}\: \textrm{quad}$
(\demowidth{0.277em}).  The escaped space character \verb*|\ |
generates a medium sized space comparable to the interword spacing and
\ci{quad} (\demowidth{1em}) and \ci{qquad} (\demowidth{2em}) produce
large spaces. The size of a \ci{quad} corresponds to the width of the
character `M' of the current font. \verb|\!|\cih{"!} produces a
negative space of $-\frac{3}{18}\:\textrm{quad}$
($-$\demowidth{0.166em}).

Note that `d' in the differential is conventionally set in roman:
\begin{example}
\begin{equation*}
  \int_1^2 \ln x \mathrm{d}x 
  \qquad
  \int_1^2 \ln x \,\mathrm{d}x
\end{equation*}
\end{example}


In the next example, we define a new command \ci{ud} which produces
``$\,\mathrm{d}$'' (notice the spacing \demowidth{0.166em} before the
$\text{d}$), so we don't have to write it every time. The \ci{newcommand} is
placed in the preamble. %  More on
% \ci{newcommand} in section~\ref{} on page \pageref{}. To Do: Add label and
% reference to "Customising LaTeX" -> "New Commands, Environments and Packages"
% -> "New Commands".
\begin{example}
\newcommand{\ud}{\,\mathrm{d}}

\begin{equation*}
 \int_a^b f(x)\ud x 
\end{equation*}
\end{example}

If you want to typeset multiple integrals, you'll discover that the spacing
between the integrals is too wide. You can correct it using \ci{!}, but
\pai{amsmath} provides an easier way for fine-tuning
the spacing, namely the \ci{iint}, \ci{iiint}, \ci{iiiint}, and \ci{idotsint}
commands.

\begin{example}
\newcommand{\ud}{\,\mathrm{d}}

\begin{IEEEeqnarray*}{c}
  \int\int f(x)g(y) 
                  \ud x \ud y \\
  \int\!\!\!\int 
         f(x)g(y) \ud x \ud y \\
  \iint f(x)g(y)  \ud x \ud y 
\end{IEEEeqnarray*}
\end{example}

See the electronic document \texttt{testmath.tex} (distributed with
\AmS-\LaTeX) or Chapter 8 of \companion{} for further details.

\subsection{Phantoms}

When vertically aligning text using \verb|^| and \verb|_| \LaTeX{} is sometimes
just a little too helpful. Using the \ci{phantom} command you can
reserve space for characters that do not show up in the final output.
The easiest way to understand this is to look at an example:
\begin{example}
\begin{equation*}
{}^{14}_{6}\text{C}
\qquad \text{versus} \qquad
{}^{14}_{\phantom{1}6}\text{C}
\end{equation*}
\end{example}
If you want to typeset a lot of isotopes as in the example, the \pai{mhchem}
package is very useful for typesetting isotopes and chemical formulae too.


\section{Fiddling with the Math Fonts}\label{sec:fontsz}
Different math fonts are listed on Table~\ref{mathalpha} on page
\pageref{mathalpha}.
\begin{example}
 $\Re \qquad
  \mathcal{R} \qquad
  \mathfrak{R} \qquad
  \mathbb{R} \qquad $  
\end{example}
The last two require \pai{amssymb} or \pai{amsfonts}.

Sometimes you need to tell \LaTeX{} the correct font
size. In math mode, this is set with the following four commands:
\begin{flushleft}
\ci{displaystyle}~($\displaystyle 123$),
 \ci{textstyle}~($\textstyle 123$), 
\ci{scriptstyle}~($\scriptstyle 123$) and
\ci{scriptscriptstyle}~($\scriptscriptstyle 123$).
\end{flushleft}

If $\sum$ is placed in a fraction, it'll be typeset in text style unless you tell
\LaTeX{} otherwise:
\begin{example}
\begin{equation*}
 P = \frac{\displaystyle{ 
   \sum_{i=1}^n (x_i- x)
   (y_i- y)}} 
   {\displaystyle{\left[
   \sum_{i=1}^n(x_i-x)^2
   \sum_{i=1}^n(y_i- y)^2
   \right]^{1/2}}}
\end{equation*}    
\end{example}
Changing styles generally affects the way big operators and limits are displayed.

% This is not a math accent, and no maths book would be set this way.
% mathop gets the spacing right.


\subsection{Bold Symbols}
\index{bold symbols}

It is quite difficult to get bold symbols in \LaTeX{}; this is
probably intentional as amateur typesetters tend to overuse them.  The
font change command \verb|\mathbf| gives bold letters, but these are
roman (upright) whereas mathematical symbols are normally italic, and
furthermore it doesn't work on lower case Greek letters.
There is a \ci{boldmath} command, but \emph{this can only be used
outside math mode}. It works for symbols too, though:
\begin{example}
$\mu, M \qquad 
\mathbf{\mu}, \mathbf{M}$
\qquad \boldmath{$\mu, M$}
\end{example}

The package \pai{amsbsy} (included by \pai{amsmath}) as well as the
\pai{bm} from the \texttt{tools} bundle make this much easier as they include
a \ci{boldsymbol} command:

\begin{example}
$\mu, M \qquad
\boldsymbol{\mu}, \boldsymbol{M}$
\end{example}


\section{Theorems, Lemmas, \ldots}

When writing mathematical documents, you probably need a way to
typeset ``Lemmas'', ``Definitions'', ``Axioms'' and similar
structures.
\begin{lscommand}
\ci{newtheorem}\verb|{|\emph{name}\verb|}[|\emph{counter}\verb|]{|%
         \emph{text}\verb|}[|\emph{section}\verb|]|
\end{lscommand}
The \emph{name} argument is a short keyword used to identify the
``theorem''. With the \emph{text} argument you define the actual name
of the ``theorem'', which will be printed in the final document.

The arguments in square brackets are optional. They are both used to
specify the numbering used on the ``theorem''. Use  the \emph{counter}
argument to specify the \emph{name} of a previously declared
``theorem''. The new ``theorem'' will then be numbered in the same
sequence.  The \emph{section} argument allows you to specify the
sectional unit within which the ``theorem'' should get its numbers.

After executing the \ci{newtheorem} command in the preamble of your
document, you can use the following command within the document.
\begin{code}
\verb|\begin{|\emph{name}\verb|}[|\emph{text}\verb|]|\\
This is my interesting theorem\\
\verb|\end{|\emph{name}\verb|}|     
\end{code}

The \pai{amsthm} package (part of \AmS-\LaTeX) provides the 
\ci{theoremstyle}\verb|{|\emph{style}\verb|}|
command which lets you define what the theorem is all about by picking
from three predefined styles: \texttt{definition} (fat title, roman body),
\texttt{plain} (fat title, italic body) or \texttt{remark} (italic
title, roman body).

This should be enough theory. The following examples should
remove any remaining doubt, and make it clear that the
\verb|\newtheorem| environment is way too complex to understand.

% actually define things
\theoremstyle{definition} \newtheorem{law}{Law}
\theoremstyle{plain}      \newtheorem{jury}[law]{Jury}
\theoremstyle{remark}     \newtheorem*{marg}{Margaret}

First define the theorems:

\begin{verbatim}
\theoremstyle{definition} \newtheorem{law}{Law}
\theoremstyle{plain}      \newtheorem{jury}[law]{Jury}
\theoremstyle{remark}     \newtheorem*{marg}{Margaret}
\end{verbatim}

\begin{example}
\begin{law} \label{law:box}
Don't hide in the witness box
\end{law}
\begin{jury}[The Twelve]
It could be you! So beware and
see law~\ref{law:box}.\end{jury}
\begin{marg}No, No, No\end{marg}
\end{example}

The ``Jury'' theorem uses the same counter as the ``Law''
theorem, so it gets a number that is in sequence with
the other ``Laws''. The argument in square brackets is used to specify 
a title or something similar for the theorem.
\begin{example}
\newtheorem{mur}{Murphy}[section]

\begin{mur} If there are two or 
more ways to do something, and 
one of those ways can result in
a catastrophe, then someone 
will do it.\end{mur}
\end{example}

The ``Murphy'' theorem gets a number that is linked to the number of
the current section. You could also use another unit, for example chapter or
subsection.

If you want to customize your theorems down to the last dot, the
\pai{ntheorem} package offers a plethora of options.


\subsection{Proofs and End-of-Proof Symbol}
\label{sec:putting-qed-right}

The \pai{amsthm} package also provides the \ei{proof} environment.

\begin{example}
\begin{proof}
 Trivial, use
 \begin{equation*}
   E=mc^2.
 \end{equation*}
\end{proof}
\end{example}

With the command \ci{qedhere} you can move the `end of proof' symbol
around for situations where it would end up alone on a line.

\begin{example}
\begin{proof}
 Trivial, use
 \begin{equation*}
   E=mc^2. \qedhere
 \end{equation*}
\end{proof}
\end{example}

Unfortunately, this correction does not work for \texttt{IEEEeqnarray}:
\begin{example}
\begin{proof}
  This is a proof that ends
  with an equation array:
  \begin{IEEEeqnarray*}{rCl}
    a & = & b + c \\
    & = & d + e. \qedhere
  \end{IEEEeqnarray*}  
\end{proof}
\end{example}
\noindent
The reason for this is the internal structure of \texttt{IEEEeqnarray}:
it always puts two invisible columns at both sides of the array that
only contain a stretchable space. By this \texttt{IEEEeqnarray} ensures
that the equation array is horizontally centered. The
\ci{qedhere} command should actually be put \emph{outside} this
stretchable space, but this does not happen as these columns are
invisible to the user.

There is a very simple remedy. Define the stretching
explicitly!
\begin{example}
\begin{proof}
  This is a proof that ends
  with an equation array:
  \begin{IEEEeqnarray*}{+rCl+x*}
    a & = & b + c \\
    & = & d + e. & \qedhere
  \end{IEEEeqnarray*}  
\end{proof}
\end{example}
\noindent
Note that the \verb=+= in \verb={+rCl+x*}= denotes stretchable spaces, one
on the left of the equations (which, if not specified, will be done
automatically by \texttt{IEEEeqnarray}!) and one on the right of the
equations. But now on the right, \emph{after} the stretching column,
we add an empty column \verb=x=. This column will be only needed on
the last line when we will put the \ci{qedhere} command
there. Finally, we specify a \verb=*=. This is a null-space that
prevents \texttt{IEEEeqnarray} to add another unwanted \verb=+=-space!

In case of equation numbering, we have a similar problem. If you compare
\begin{example}
\begin{proof}
  This is a proof that ends
  with a numbered equation:
  \begin{equation}
    a = b + c.
  \end{equation}
\end{proof}
\end{example}
\noindent
with
\begin{example}
\begin{proof}
  This is a proof that ends
  with a numbered equation:
  \begin{equation}
    a = b + c. \qedhere
  \end{equation}
\end{proof}
\end{example}
\noindent
you notice that in the (correct) second version the $\Box$ is much
closer to the equation than in the first version.

Similarly, the correct way of putting the QED-symbol at the end of an
equation array is as follows:
\begin{example}
\begin{proof}
  This is a proof that ends
  with an equation array:
  \begin{IEEEeqnarray}{+rCl+x*}
    a & = & b + c \\
    & = & d + e. \\
    &&& \qedhere\nonumber
  \end{IEEEeqnarray}  
\end{proof}
\end{example}
\noindent
which contrasts with
\begin{example}
\begin{proof}
  This is a proof that ends
  with an equation array:
  \begin{IEEEeqnarray}{rCl}
    a & = & b + c \\
    & = & d + e.
  \end{IEEEeqnarray}  
\end{proof}
\end{example}


%

% Local Variables:
% TeX-master: "lshort"
% mode: latex
% mode: flyspell
% End:
