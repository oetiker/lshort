% The Not So Short Introduction to LaTeX
%
% Copyright (C) 1995--2022 Tobias Oetiker, Marcin Serwin, Hubert Partl,
% Irene Hyna, Elisabeth Schlegl and Contributors.
%
% This document is free software: you can redistribute it and/or modify it
% under the terms of the GNU General Public License as published by the Free
% Software Foundation, either version 3 of the License, or (at your option) any
% later version.
%
% This document is distributed in the hope that it will be useful, but WITHOUT
% ANY WARRANTY; without even the implied warranty of MERCHANTABILITY or FITNESS
% FOR A PARTICULAR PURPOSE.  See the GNU General Public License for more
% details.
%
% You should have received a copy of the GNU General Public License along with
% this document.  If not, see <https://www.gnu.org/licenses/>.

% !TEX root = ./lshort.tex
%%%%%%%%%%%%%%%%%%%%%%%%%%%%%%%%%%%%%%%%%%%%%%%%%%%%%%%%%%%%%%%%%
% Contents: Specialities of the LaTeX system
% $Id$
%%%%%%%%%%%%%%%%%%%%%%%%%%%%%%%%%%%%%%%%%%%%%%%%%%%%%%%%%%%%%%%%%

\chapter{Specialities}\label{specialities}
\begin{intro}
  When putting together a large document, \LaTeX{} will help with some special
  features like index generation, automatic linking to relevant pages and other
  things. A much more complete description of specialities and enhancements
  possible with \LaTeX{} can be found in the {\normalfont\manual{}} and
    {\normalfont\companion}.
\end{intro}

\section{Indexing}\label{sec:indexing}
A very useful feature of many books is their \wi{index}. With \LaTeX{}
and the support program \texttt{makeindex},\footnote{On systems not
  necessarily supporting
  filenames longer than 8~characters, the program may be called
  \texttt{makeidx}.} an index can be generated quite easily.  This
introduction will only explain the basic index generation commands.
For a more in-depth view, please refer to \companion.\index{makeindex
  program}\index{makeidx package}

To enable the indexing feature of \LaTeX{}, the \pai{makeidx} package
must be loaded in the preamble with
\begin{lscommand}
  \mintinline{latex}|\usepackage{makeidx}|
\end{lscommand}
\noindent and the special indexing commands must be enabled by putting
the
\begin{lscommand}
  \csi{makeindex}
\end{lscommand}
\noindent command in the preamble.

The content of the index is specified with
\begin{lscommand}
  \csi{index}[{{«\bs carg{key}»@«\bs carg{formatted entry}»}}: c]
\end{lscommand}
\noindent commands, where \carg{formatted entry} will appear in the index
and \carg{key} will be used for sorting.  The \carg{formatted entry} is
optional. If it is missing the \carg{key} will be used. You enter the index
commands at the points in the text that you want the final index entries to
point to. \autoref{index} explains the syntax with several examples.

\begin{table}[!tp]
  \centering
  \caption{Index Key Syntax Examples.}\label{index}
  \begin{tabular}{@{}lll@{}}
    \toprule
    Code                                   & Entry                 & Explanation            \\
    \midrule
    \rule{0pt}{1.05em}\verb|\index{hello}| & hello, 1              & Plain entry            \\
    \verb|\index{hello!Peter}|             & \hspace*{2ex}Peter, 3 & Subentry under `hello' \\
    \verb|\index{Sam@\emph{Sam}}|          & \emph{Sam}, 2         & Formatted entry        \\
    \verb|\index{Kaese@\emph{K\"ase}}|     & \emph{K\"ase}, 33     & Formatted entry        \\
    \verb.\index{ecole@\'ecole}.           & \'ecole, 4            & Formatted entry        \\
    \verb.\index{Jenny|emph}.              & Jenny, \emph{3}       & Formatted page number  \\
    \verb.\index{Joe@\emph{Joe}|emph}.     & \emph{Joe}, \emph{5}  & Formatted page number  \\
    \bottomrule
  \end{tabular}
\end{table}

When the input file is processed with \LaTeX{}, each \verb|\index|
command writes an appropriate index entry, together with the current
page number, to a special file. The file has the same name as the
\LaTeX{} input file, but a different extension (\verb|.idx|). This
\eei{.idx} file can then be processed with the \texttt{makeindex}
program:
\begin{lscommand}
  \texttt{makeindex} \emph{filename}
\end{lscommand}

The \texttt{makeindex} program generates a sorted index with the same base
file name, but this time with the extension \eei{.ind}. If now the
\LaTeX{} input file is processed again, this sorted index gets
included into the document at the point where \LaTeX{} finds
\begin{lscommand}
  \csi{printindex}
\end{lscommand}

The \pai{showidx} package that comes with \LaTeX{} prints out all
index entries in the left margin of the text. This is quite useful for
proofreading a document and verifying the index. Make sure to load the package
afer the \pai{hyperref} package. 

Note that the \csi{index} command can affect your layout if not used carefully.

\begin{chktexignore}
  \begin{example}
My Word \index{Word}. As opposed
to Word\index{Word}. Note the
position of the full stop.
\end{example}
\end{chktexignore}

Note that the texttt{makeindex} has no clue about characters outside the ASCII range. To
get the sorting correct, use the \verb|@| character as shown in the K\"ase
and \'ecole examples above.

\section{Fancy Headers}\label{sec:fancy}

\subsection{Basic commands}

%$$ Not sure if the overall conversational tone of this subsection is in keeping
%$$ with the bulk of the rest of the document.

The \pai*{fancyhdr} package provides a few simple commands that allow you to
customise the header and footer lines of your document. It defines an
additional page style \cargv{fancy} and a set of commands to customise it to
your liking. By default, it only adds a line separating the header from the page
body.
\begin{example}[standalone, paperheight=3cm]
\geometry{includefoot, includehead, headsep=.5em, footskip=1em} %!hide
%!showbegin %!hide
\documentclass{article}

%!showend %!hide
\usepackage{fancyhdr}
\pagestyle{fancy}

\begin{document}
\noindent %!hide
This statement is false.
\end{document}
\end{example}

The primary command of the package is
\begin{lscommand}
  \csi{fancyhf}[places: o, field: m]
\end{lscommand}
The \carg{places} is a comma separated list of places where the \carg{field} should be displayed. There are total of 12 different places and each is identified by a
combination of three letters:
\begin{itemize}
  \item The first specifies whether the location is in the header (\cargv{H}) or
        in the footer (\cargv{F}).
  \item The second letter specifies the location within the header or footer:
        \cargv{L} for left, \cargv{C} for centre, and \cargv{R} for right.
  \item The third letter defines whether the field should be printed on even
%$$ Inserted comma after 'sided'.
        (\cargv{E}) or odd (\cargv{O}) pages. If the document is not two sided,
        then all pages are treated as odd.
\end{itemize}
%$$ Changed 'So for example' to 'For example,'
For example, the combination \cargv{FCE} identifies the centre part of the
%$$ Inserted comma after 'omitted'.
footer on even pages. If any of the letters is omitted, then the identifier
%$$ Changed 'letter, so for example' to 'letter. For example,', to break up
%$$ a long sentence.
points toward all positions specifiable by the omitted letter. For example,
\cargv{HR} is right side of the header on both odd and even pages.
\begin{example}[template=empty, standalone, paperheight=2.3cm, paperwidth=6cm, to_page=2,vertical_pages]
\documentclass[twoside]{article}

%!hidebegin
\usepackage[paperheight=\height,
    paperwidth=\width,
    margin=0.3cm,
    includefoot,
    includehead,
    headsep=.5em,
    footskip=1em
]{geometry}
%!hideend
\usepackage{fancyhdr}
\pagestyle{fancy}

\fancyhf[HCE]{A}
\fancyhf[L]{\emph{B}}
\fancyhf[FR]{\textbf{C}}
\fancyhf[HCO, HRE]{\textsl{D}}

\begin{document}
\noindent %!hide
The next statement is false.
The previous statement is true.
\end{document}
\end{example}

%$$ Inserted commas after 'commands' and after '\csi{fancyfoot}'.
There are two additional commands, \csi{fancyhead} and \csi{fancyfoot}, that
%$$ Changed 'work the' to 'work in the'. Inserted comma after 'way'.
%$$ Changed 'assum ... respectively' to 'respectively assume ...'.
work in the same way, except that they respectively assume \cargv{H} and
%$$ Inserted comma after 'argument'. Changed 'specified differently' to
%$$ 'otherwise specified'.
\cargv{F} in their \carg{places} argument, unless otherwise specified.
\begin{example}[standalone, paperheight=3.5cm]
\geometry{includefoot, includehead, headsep=.5em, footskip=1em} %!hide
\sloppy %!hide
\usepackage{csquotes} %!hide
\usepackage{fancyhdr}%!hide
\pagestyle{fancy}%!hide

\fancyhead[L]{A}
\fancyfoot[R]{B}

\begin{document}%!hide
\noindent %!hide
\enquote{Yields a falsehood when
appended to its own quotation}
yields a falsehood when appended
to its own quotation.
\end{document}%!hide
\end{example}

The lines drawn by the \pai{fancyhdr} package may also be customised. To change
%$$ Inserted comma after 'thickness'.
their thickness, redefine the
\begin{lscommand}
  \csi{headrulewidth} \\
  \csi{footrulewidth}
\end{lscommand}
macros to the desired size.
\begin{example}[standalone, paperheight=3cm, paperwidth=3cm]
\geometry{includefoot, includehead, headsep=.5em, footskip=1em} %!hide
\sloppy %!hide
\usepackage{fancyhdr} %!hide
\pagestyle{fancy} %!hide
\RenewDocumentCommand{\headrulewidth}{}{.2cm}
\RenewDocumentCommand{\footrulewidth}{}{.5cm}

\begin{document}%!hide
\noindent %!hide
Do not read this sentence.
\end{document}%!hide
\end{example}

%$$ Inserted comma after 'default'.
By default, headers and footers are as long as the text on the page. If you want
%$$ Changed 'extend\slash{}shorten' to 'extend or shorten'.
%$$ Added comma after 'them'.
to extend\slash{}shorten them, use the
\begin{lscommand}
  \csi{fancyhfoffset}[places: o, offset: m]
\end{lscommand}
%$$ Added comma after \csi{fancyhf}.
command. The \carg{places} argument is the same as in \csi{fancyhf}, except that
it cannot contain \cargv{C}.
\begin{example}[standalone, paperheight=3cm]
\geometry{includehead, includefoot, headsep=.5em, footskip=1em} %!hide
\sloppy %!hide
\usepackage{fancyhdr}%!hide
\pagestyle{fancy}%!hide
\fancyhfoffset[L]{-1cm}
\fancyhfoffset[R]{.2cm}

\begin{document}%!hide
\noindent %!hide
If this sentence is true,
then \(2 + 2 = 5\).
\end{document}%!hide
\end{example}
%$$ Rewrote this sentence:
%$ Similarly to \csi{fancyf}, \csi{fancyheadoffset} and \csi{fancyfootoffset} are
%$ used the same way except they only modify header\slash{}footer.
The \csi{fancyheadoffset} and \csi{fancyfootoffset} commands are used in the
same manner as \csi{fancyf}, except that they only modify the header or footer.
%$$ Actually, that should probably be 'but' instead of 'except that'; there's
%$$ no 'except' in how they are used; that's the same, the difference is in the
%$$ result of that usage.

\subsection{Contents of the headers}

%$$ Not sure if the overall conversational tone of this subsection is in keeping
%$$ with the bulk of the rest of the document.

The default footer of the article class contains the current page number. To
%$$ Added comma after 'header'. Also 'simply' might be superfluous.
use it inside the fancy header, simply use the command \csi{thepage}.
\begin{example}[standalone, paperheight=2.5cm, to_page=2, vertical_pages]
\geometry{includehead, includefoot, headsep=.5em, footskip=1em} %!hide
\sloppy %!hide
\usepackage{fancyhdr}%!hide
\pagestyle{fancy}%!hide
\fancyhf{Page~\thepage}

\begin{document}%!hide
\noindent %!hide
This statement is dedicated to
all statements that are not
dedicated to themselves.
\end{document}%!hide
\end{example}

It is often useful to have the header and footer contain information based on
%$$ Changed 'contents' to 'content' (uncountable noun in this context).
the content of the page. These are called \enquote{marks} in \LaTeX{}
%$$ Added comma after 'ones'. Language and abbreviations are informal, here.
terminology. Before we talk about the default ones, let's consider how you can
define your own using the \pai{extramarks}\footnote{\pai{extramarks} is part
  of \pai*{fancyhdr}.} package.
\begin{lscommand}
  \csi{extramarks}[left:m, right:m] \\
  \csi{firstleftxmark} \\
  \csi{firstrightxmark} \\
  \csi{lastleftxmark} \\
  \csi{lastrightxmark}
\end{lscommand}
%$$ Changed 'is used to set' to just 'sets'.
The \csi{extramarks} command sets the contents of \carg{left} and \carg{right}
%$$ Changed 'Then you can' to 'You can then'.
marks. Then you can access these marks inside the headers by using the appropriate
command. The \texttt{first}- commands refer to the first mark occurring on the
%$$ Added 'commands' after \texttt{last}-, because still needs to be a plural.
page, while the \texttt{last}- refer to the last one.
\begin{example}[standalone, paperheight=4cm]
\geometry{includehead, includefoot, headsep=.5em, footskip=1em} %!hide
\sloppy %!hide
\usepackage{fancyhdr}%!hide
\usepackage{extramarks}
\pagestyle{fancy}%!hide

\fancyhead[L]{\firstleftxmark}
\fancyhead[R]{\lastleftxmark}
\fancyfoot[L]{\firstrightxmark}
\fancyfoot[R]{\lastrightxmark}

\begin{document} %!hide
\noindent %!hide
The second statement is false.
\extramarks{One}{2 is false}
The third statement is false.
\extramarks{Two}{3 is false}
The first statement is false.
\extramarks{Three}{1 is false}
\end{document} %!hide
\end{example}

Let us now look at the default marks defined by \LaTeX{}. After loading
%$$ Added comma after '\pai{extramarks}'.
\pai{extramarks}, these may be set and accessed similarly:
\begin{lscommand}
  \csi{markboth}[left:m, right:m] \\
  \csi{firstleftmark} \\
  \csi{firstrightmark} \\
  \csi{lastleftmark} \\
  \csi{lastrightmark}
\end{lscommand}
The only difference between these and the extra marks is that \LaTeX{} classes
%$$ Long sentence. Changed 'them, so' to 'them. Hence,'.
%$$ Added a comma after 'careful'.
%$$ Furthermore, does 'them' mean 'these' (as opposed to the extra marks)?
automatically fill them. Hence, if you are not careful, you may lose your
%$$ Removed whitespace between 'content.' and the footnote.
content.\footnote{You may prevent this by setting the pagestyle to
\cargv{myheadings} and then redefining it to use it with \pai{fancyhdr} as
described later.}
%$$ Added comma after 'for example'.
For example, in the article class, \carg{left} is set by the \csi{section}
command, while \carg{right} is set by the \csi{subsection} command.
\begin{example}[standalone, paperheight=5cm]
\geometry{includehead, includefoot, headsep=.5em, footskip=1em} %!hide
\sloppy %!hide
\usepackage{fancyhdr}%!hide
\usepackage{extramarks}%!hide
\pagestyle{fancy}%!hide
\fancyhead[L]{\firstleftmark}
\fancyhead[R]{\lastleftmark}
\fancyfoot[L]{\firstrightmark}
\fancyfoot[R]{\lastrightmark}

\begin{document}%!hide
\section{First}
\subsection{Sub}
\section{Second}
\subsection{Sub}
\end{document}%!hide
\end{example}
Note that \csi{firstrightmark} is empty in the above example. This is caused by
%$$ Changed 'set' to 'sets', to avoid the double plural with 'commands'.
the fact that \csi{section} commands set both marks, leaving the right one
empty.

You may have noticed that section titles are typeset in uppercase when provided
%$$ Consider "`marks'" or "``marks''" or "mark" or -\texttt{mark} instead of
%$$ "marks" (in 'marks commands', on the next line).
by marks commands. To disable this use the \csi{nouppercase} command inside the
\csi{fancyhf} command.
\begin{example}[standalone, paperheight=3cm]
\geometry{includehead, includefoot, headsep=.5em, footskip=1em} %!hide
\sloppy %!hide
\usepackage{fancyhdr}%!hide
\usepackage{extramarks}%!hide
\pagestyle{fancy}%!hide
\fancyhead[R]{%
  \nouppercase{\firstleftmark}%
}

\begin{document}%!hide
\section{Today}
is opposite day.
\end{document}%!hide
\end{example}


\subsection{Advanced commands}

%$$ Changed 'the section titles' to 'section titles,'.
%$$ Consider 'can' instead of 'may', the latter suggests permission to do so.
If you want even more control over section titles, you may redefine the
\csi{sectionmark}.\footnote{Or \csi{chaptermark}, \csi{subsectionmark} \ldots} The
command receives the section title as its first argument, while the section
number is available as \csi{thesection}.
\begin{example}[standalone, paperheight=4.5cm, paperwidth=4cm]
\geometry{includehead, includefoot, headsep=.5em, footskip=1em} %!hide
\sloppy %!hide
\usepackage{fancyhdr}%!hide
\usepackage{extramarks}%!hide
\pagestyle{fancy}%!hide
\fancyhead[R]{\firstleftmark}
\RenewDocumentCommand{\sectionmark}{m}{%
  \markboth{%
    Section no.\,\thesection: #1}{}%
}

\begin{document}%!hide
\section{Person}
There exists a person such that
if they are reading this then
everybody is reading this.
\end{document}%!hide
\end{example}
If you only want to change the right mark, use the \csi{markright} command.

By default, the field in the centre of the header or footer will expand equally
to the left and right. This is usually the desired behaviour, but in some
cases it may overlap with either left or right text, while still having some
space on the other side.
\begin{example}[standalone, paperheight=3cm]
\geometry{includehead, includefoot, headsep=.5em, footskip=1em} %!hide
\sloppy %!hide
\usepackage{fancyhdr}%!hide
\usepackage{extramarks}%!hide
\pagestyle{fancy}%!hide
\fancyhead[L]{\thepage}
\fancyhead[C]{\firstleftmark}
\fancyhead[R]{Jane Doe}

\begin{document}%!hide
\section{Section}
Section by Jane Doe
\end{document}%!hide
\end{example}
In this situation you may want to use the
\begin{lscommand}
  \csi{fancycenter}[distance:o, stretch:o, left:m, centre:m, right:m]
\end{lscommand}
command, which automatically shifts the centre toward the shorter text. The
\carg{distance} is the minimum distance by which the elements are always
surrounded (\cargv{1em}, by default). The \carg{stretch} controls the preference
for shifting the \carg{centre}; \cargv{1} means shift only when and as much
as necessary. Higher numbers will start shifting sooner and more aggressively.
The default is \cargv{3}. This command writes over the whole header\slash{}footer
space so it should only be put in one place (typically \cargv{C}) and other
places (\cargv{L,R}) should be empty.
\begin{example}[standalone, paperheight=3cm]
\geometry{includehead, includefoot, headsep=.5em, footskip=1em} %!hide
\sloppy %!hide
\usepackage{fancyhdr}%!hide
\usepackage{extramarks}%!hide
\pagestyle{fancy}%!hide
\fancyhead[L,R]{}
\fancyhead[C]{%
  \fancycenter%
    {\thepage}%
    {\firstleftmark}%
    {Jane Doe}%
}

\begin{document}%!hide
\section{Section}
Section by Jane Doe
\end{document}%!hide
\end{example}

You may want to present different headers or footers when the corresponding
page starts with a float or ends with a footnote. The \pai{fancyhdr} package
defines four commands that let you achieve this.
\begin{lscommand}
  \csi{iftopfloat}[true branch:m, false branch:m] \\
  \csi{ifbotfloat}[true branch:m, false branch:m] \\
  \csi{iffloatpage}[true branch:m, false branch:m] \\
  \csi{iffootnote}[true branch:m, false branch:m]
\end{lscommand}
Commands \csi{iftopfloat} and \csi{ifbotfloat} execute their \carg{true branch}
if a float sits at the top or bottom of the page (respectively).

Similarly, \csi{iffloatpage} checks whether
the page is a special float-only page, while \csi{iffootnote} checks for a
footnote at the bottom of the page.
\begin{example}[standalone, paperheight=4cm, to_page=2, vertical_pages]
\geometry{includehead, includefoot, headsep=.5em, footskip=1em} %!hide
\sloppy %!hide
\usepackage{fancyhdr}%!hide
\usepackage{extramarks}%!hide
\pagestyle{fancy}%!hide

\fancyhead[C]{%
  \iftopfloat{%
    A float is below me%
  } {}
}
\fancyfoot[C]{%
  \iffootnote{%
    A footnote is above me%
  } {%
    \thepage%
  }%
}

\begin{document}%!hide
\noindent %!hide
Ignore this footnote.%
\footnote{Read this.}
\begin{figure}[t]
  \centering
  A floating float.
  \caption{Hmm}
\end{figure}
\end{document}%!hide
\end{example}

The ruled lines in headers and footers are created by invoking \csi{headrule}
and \csi{footrule} commands. If you want finer control over the lines, consider
redefining these macros. Use \csi{headruleskip} and \csi{footruleskip} to raise
or lower them, if necessary.
\begin{example}[standalone, paperheight=4cm, paperwidth=3.2cm]
\geometry{includehead, includefoot, headsep=.5em, footskip=1em} %!hide
\sloppy %!hide
\usepackage{fancyhdr}%!hide
\usepackage{extramarks}%!hide
\pagestyle{fancy}%!hide
\RenewDocumentCommand{\headrule}{}{
  \rule{0.05\headwidth}{0.2cm}%
  \rule[0.1cm]{0.9\headwidth}{\headrulewidth}%
  \rule{0.05\headwidth}{0.2cm}%
}

\RenewDocumentCommand{\headruleskip}{}{-0.2cm}

\begin{document}%!hide
\noindent %!hide
My age is the first number not
nameable in under twenty words.
\end{document} %!hide
\end{example}

While working on a longer document, you may want to have several page styles for
different occasions. You may also notice that some commands (\csi{chapter} and
\csi{maketitle}, for example) change the page style to \cargv{plain}.
The solution to both of these problems is the
\begin{lscommand}
  \csi{fancypagestyle}[name:m, code:m]
\end{lscommand}
command. The \carg{name} is the name of page style to (re)define, while the %chktex 36
\carg{code} is the code to set the page style. All page styles declared this
way use the \cargv{fancy} page style as their basis, so if empty \carg{code} is
given they will match \cargv{fancy} exactly.
\begin{example}[standalone, paperheight=3cm]
\geometry{includehead, includefoot, headsep=.5em, footskip=1em} %!hide
\sloppy %!hide
\usepackage{fancyhdr}%!hide
\usepackage{extramarks}%!hide
\fancypagestyle{mine}{
  \fancyhead[L]{My style}
}
\pagestyle{mine}

\begin{document}%!hide
%!showbegin %!hide
Were you expecting a paradox here?
%!showend !hidebegin
\noindent
You may be disappointed.
\end{document}
\end{example}
%$$ I don't think I understand the above example.
%$$ The connection between 'paradox' on the left, and 'disappointed' on the 
%$$ right isn't obvious. Did I miss something?


\section{Installing Extra Packages}\label{sec:Packages}

Most \LaTeX{} installations come with a large set of pre-installed
style packages, but many more are available on the net. The main
place to look for style packages on the Internet is CTAN (\url{http://www.ctan.org/}).

Packages such as \pai{geometry}, \pai{hyphenat}, and many
others are typically made up of two files: a file with the extension
\texttt{.ins} and another with the extension \texttt{.dtx}. There
will often be a \texttt{readme.txt} with a brief description of the
package. You should of course read this file first.

In any event, once you have copied the package files onto your
machine, you still have to process them in a way that (a) tells your
\TeX\ distribution about the new style package and (b) gives you
the documentation.  Here's how you do the first part:

\begin{enumerate}
  \item Run \LaTeX{} on the \texttt{.ins} file. This will
        extract a \eei{.sty} file.
  \item Move the \eei{.sty} file to a place where your distribution
        can find it. Usually this is in your \texttt{\ldots/\emph{localtexmf}/tex/latex}
        subdirectory (Windows or OS/2 users should feel free to change the
        direction of the slashes).
  \item Refresh your distribution's file-name database. The command
        depends on the \LaTeX{} distribution you use:
        \TeXLive{} --- \texttt{texhash}; web2c --- \texttt{maktexlsr};
        MiK\TeX{} --- \texttt{initexmf -{}-update-fndb} or use the GUI\@.
\end{enumerate}

\noindent Now extract the documentation from the
\texttt{.dtx} file:

\begin{enumerate}
  \item Run \hologo{XeLaTeX} on the \texttt{.dtx} file.  This will generate a
        \texttt{.pdf} file. Note that you may have to run \hologo{XeLaTeX}
        several times before it gets the cross-references right.
  \item Check to see if \LaTeX\ has produced a \texttt{.idx} file
        among the various files you now have.
        If you do not see this file, then the documentation has no index. Continue
        with step~\ref{step:final}.
  \item In order to generate the index, type the following:\\
        \fbox{\texttt{makeindex -s gind.ist \textit{name}}}\\
        (where \textit{name} stands for the main-file name without any
        extension).
  \item Run \LaTeX\ on the \texttt{.dtx} file once again.\label{step:next}

  \item Last but not least, make a \texttt{.ps} or \texttt{.pdf}
        file to increase your reading pleasure.\label{step:final}

\end{enumerate}

Sometimes you will see that a \texttt{.glo}
(glossary) file has been produced. Run the following
command between
step~\ref{step:next} and~\ref{step:final}:

\noindent\texttt{makeindex -s gglo.ist -o \textit{name}.gls \textit{name}.glo}

\noindent Be sure to run \LaTeX\ on the \texttt{.dtx} one last
time before moving on to step~\ref{step:final}.

%%%%%%%%%%%%%%%%%%%%%%%%%%%%%%%%%%%%%%%%%%%%%%%%%%%%%%%%%%%%%%%%%
% Contents: Chapter on pdfLaTeX
% French original by Daniel Flipo 14/07/2004
%%%%%%%%%%%%%%%%%%%%%%%%%%%%%%%%%%%%%%%%%%%%%%%%%%%%%%%%%%%%%%%%%

\section{\LaTeX{} and PDF}\label{sec:pdftex}\index{PDF}
\begingroup
\LshortExampleSetup{hide_hyperref=false}

The initial release of \TeX{} predated the PDF format by nearly 16 years. The
original output files it produced---\eei{.dvi}s---were meant to be only
printed. Today many documents are never or seldom printed, we read them
directly on a screen. PDF format contains many improvements for viewing
documents like this but they are not implemented in core \LaTeX{}. These are
accessible via the \pai*{hyperref} package.

\subsection{Hypertext Links}\label{hyperlinks}

Hyperlinks are used to quickly jump around the document. The prime example of
using them is the table of contents, you don't have to manually scroll to a
given page---just click on a given chapter and you will be immediately
transported there. You already know that table of contents can be typeset using
the \csi{tableofcontents} command, but it doesn't contain any hyperlinks.

Luckily the process of updating your document is extremely easy: just add
\mintinline{latex}|\usepackage{hyperref}| as the \emph{last} package loaded in
your preamble. Doing so redefines internal \LaTeX{} commands to produce
hyperlinks.
\begin{example}
%!hidebegin
\hypersetup{
  hidelinks,
  pdfborder=0 0 1,
}
%!showbegin !hideend
% In preamble
\usepackage{hyperref}
%!showend !hide
% ...
The reference to this section
now looks like that:\footnote{
  Footnotes are also
  made into hyperlinks.}
Section~\ref{hyperlinks} is
on page~\pageref{hyperlinks}.
The hyperref bibliographic
entry is~\cite{pack:hyperref}.
\end{example}
By default links are marked by a red box around them. This box
is only visible when viewing the document on screen and will not be printed.
The boxes are, however, rather ugly, so you may want to add the \cargv{colorlinks}
option while loading the package.
\begin{example}
%!showbegin !hide
% In preamble
\usepackage[
  colorlinks
]{hyperref}
%!showend !hide
% ...
The reference to this section
now looks like that:
Section~\ref{hyperlinks} is
on page~\pageref{hyperlinks}.
\end{example}
This is the option used throughout this booklet and assumed in further
examples. While this makes the links more visually appealing, it has the
disadvantage that the coloured links will be printed. To marry the best of both
worlds you can use the \pai*{ocgx2} package with the option
\cargv{ocgcolorlinks} like this
\begin{code}  
\begin{minted}{latex}
\usepackage{hyperref}
\usepackage[ocglinks]{ocgx2}
\end{minted}
\end{code}
Be warned though that it is not well supported by popular PDF viewers. Another
option is to use option \cargv{hidelinks} that makes the links clickable but
does not distinguish them visually.

In addition to redefining internal \LaTeX{} command, \pai{hyperref} defines
some additional ones. To typeset URLs you can now use the \csi{url} command.
\begin{example}
\url{https://www.ctan.org/}
\end{example}
If you want to display different text for the clickable link you can use the
\begin{lscommand}
  \csi{href}[URL: m, text: m]
\end{lscommand}
command. Note that if you intend for the document to be useful when printed,
you have to provide the full URLs anyway.
\begin{example}
Packages can be found on
\href{https://www.ctan.org/}{
  CTAN}.
\end{example}

A similar command is
\begin{lscommand}
  \csi{hyperref}[marker: o, text: m]
\end{lscommand}
that allows to create a hyperlink within the document with a different text.
\begin{example}
\hyperref[hyperlinks]{
  Hyperlinks section}
describes the usage of
hyperlinks.
\end{example}
To avoid nesting hyperlinks inside one another, \pai{hyperref} provides starred
versions of \csi{ref} and \csi{pageref} commands that produce text without
hyperlinking them.
\begin{example}
\hyperref[hyperlinks]{
  Section~\ref*{hyperlinks}}
describes the usage of
hyperlinks.

This ref~\ref*{hyperlinks}
is not hyperlinked.
\end{example}

Throughout this booklet you have seen references such as
\enquote*{\autoref{hyperlinks} on \autopageref{hyperlinks}}. While this could
be achieved using the aforementioned \csi{hyperref} command, this usecase is so
common that \pai{hyperref} provides two commands that make it much easier:
\begin{lscommand}
  \csi{autoref}[marker: m] \\
  \csi{autopageref}[marker: m]
\end{lscommand}
Their usage is identical to the \csi{ref} and \csi{pageref} commands, but they
produce additional text based on the counter the \carg{marker} refers to.
\begin{example}
\RenewDocumentCommand{\chapterautorefname}{}{chapter} %!hide
\autoref{hyperlinks} in
\autoref{specialities} on
\autopageref{hyperlinks}.
\end{example}
The names are controlled by commands such as \cs{autorefchaptername} or
\cs{autorefsectionname}. See the \pai*{hyperref} documentation for a full list.

So far we have used the default colours for URLs, hyperlinks. These can be
changed using the \csi{hypersetup} command. It accepts a key value list
customising the appearance of links. Colours may be specified using
\cargv{linkcolor}, \cargv{citecolor} and \cargv{urlcolor}. If you have the
\pai{xcolor} package loaded, you may specify colours the same way as described
in \autoref{sec:colors}.
\begin{chktexignore}
\begin{example}
\hypersetup{
  urlcolor = pink,
  citecolor = purple,
  linkcolor = teal!50!yellow,
}
\url{https://www.ctan.org/} \\
\cite{pack:hyperref} \\
\autoref{hyperlinks}
\end{example}
\end{chktexignore}

You can also adjust the borders around the links. The basic key is
\cargv{pdfborder}. It accepts three numbers: horizontal corner radius, vertical
corner radius and border width.
\begin{example}
\hypersetup{pdfborder = 0 0 1}
\url{https://www.ctan.org/} \\
\hypersetup{pdfborder = 10 10 3}
\url{https://www.ctan.org/} \\
\hypersetup{pdfborder = 10 5 2}
\url{https://www.ctan.org/} \\
\hypersetup{pdfborder = 2 7 5}
\url{https://www.ctan.org/}
\end{example}
The colour of the boxes may be adjusted similarly to the text colours with the
\cargv{linkbordercolor}, \cargv{citebordercolor} and \cargv{urlbordercolor}
keys, assuming the \pai{xcolor} package is loaded.
\begin{chktexignore}
  \begin{example}
\hypersetup{
  pdfborder = 0 0 2,
  urlbordercolor = violet,
  citebordercolor = pink,
  linkbordercolor = teal,
}
\url{https://www.ctan.org/} \\
\cite{pack:hyperref} \\
\autoref{hyperlinks}
\end{example}
\end{chktexignore}

\subsection{Document Metadata}\label{sec:pdfmeta}

Another thing that may be adjusted with the \pai{hyperref} package is document
metadata. These are information about your document that are not visible in the
document itself, but may be used by your PDF viewer in various ways. For
example, the title of the document may be shown in its top window bar.

Additional information about your document may be set using \cargv{pdfinfo}
key. This key itself accepts a key value list of document properties.
\begin{minted}{latex}
\hypersetup{
  pdfinfo = {
    Title = Title of the Book,
    Author = {Us, Ourselves and We},
    Subject = Book creation with LaTeX,
    Creator = Our House,
    Keywords = {LaTeX, typesetting},
    Producer = LuaTeX,
  }
}
\end{minted}

You can also control the way your document presents itself when opening. For
example you can choose whether the bookmarks should be shown
(\cargv{bookmarksopen}), whether external links should be opened in new windows
(\cargv{pdfnewwindow}) whether the pages should initially fit the window
(\cargv{pdffitwindow}).

If you want to set the metadata without redefining internal \LaTeX{} commands
to produce links, you may pass \mintinline{latex}{implicit=false} to
\pai{hyperref} package options.
\endgroup

\subsection{Problems with Outline}

The \pai{hyperref} package automatically uses table of contents generated by
the document as a document outline for easier navigation. This may lead to some
problems if your section titles contain some non-text content (for example,
\enquote{\LaTeX}). If this is the case, then it will be ignored by the
\pai{hyperref}
package and the following warning will be reported
\begin{code}
\begin{verbatim}
Package hyperref Warning:
Token not allowed in a PDFDocEncoded string:
\end{verbatim}
\end{code}
To get around this problem you can use the
\begin{lscommand}
  \csi{texorpdfstring}[\bs TeX{} text: m, outline text: m]
\end{lscommand}
command. Its first argument, \carg{\TeX{} text}, is the text to be displayed
inside the document, while the second is the fallback for \pai{hyperref} to
use. An example would be to change
\begin{minted}{latex}
\section{\LaTeX{} is awesome!}
\end{minted}
to
\begin{minted}{latex}
\section{\texorpdfstring{\LaTeX}{LaTeX} is awesome!}
\end{minted}

While the above method may be necessary for some complicated math formulae
that need to be nicely printed in the outline, usually the replacement texts
are rather obvious for a given command. In this case you can use the
\begin{lscommand}
  \csi{pdfstringdefDisableCommands}[commands: m]
\end{lscommand}
command, which allows you to define general fallbacks for some commands. The
\carg{commands} argument is a list of redefinitions to be done when evaluating
outline titles. Commands may be redefined using the
\csi{RenewExpandableDocumentCommand} command. For a description of how to
redefine the commands see \autoref{sec:new_commands}. An example of using this
command would be
\begin{minted}{latex}
\pdfstringdefDisableCommands {
  \RenewExpandableDocumentCommand{\ldots}{}{...}
  \RenewExpandableDocumentCommand{\LaTeX}{}{LaTeX}
  \RenewExpandableDocumentCommand{\emph}{m}{*#1*}
}
\end{minted}

\section{Creating Presentations}\label{sec:beamer}
\begingroup
\LshortExampleSetup{
  paperheight=4.5cm,
  paperwidth=6cm,
  template=beamer,
  standalone,
}

\TeX{} and \LaTeX{} were primarily designed for creating text documents and
that is where they shine. Still, it is also possible to use them for creating
presentations, but these presentations will lack fancy animations and
transitions---they are just cleverly set up PDF files after all. But on the
plus side, you will be able to tap into the strengths of \LaTeX{} such as
logical markup, math typesetting and all the typesetting magic provided by the
various \LaTeX{} extension packages.

Historically there have been several ways to create presentations, for example
the standard \LaTeX{} \cli{slides} class or the \pai{powerdot} package. While
these are still supported, this booklet will focus on the \pai*{beamer} package
which is the most popular option these days. This section will only scratch the
surface of \pai{beamer}'s capabilities. For a more in-depth tutorial please see
the User Guide~\cite{pack:beamer} distributed with it.

\subsection{Basic Usage}

The \pai{beamer} package provides the \cli{beamer} class, which loads all the
necessary packages. The most basic environment is the \ei{frame} environment
which adds a single page to your presentation.\footnote{Other presentation
  software often uses the term \enquote{slide} instead of frame, but slides are
  a slightly different concept in \pai{beamer}.}
\begin{example}
%!showbegin !hide
\documentclass{beamer}
%!showend !hide

\begin{document}
\begin{frame}
  Small is beautiful.
\end{frame}
\end{document}
\end{example}
The full syntax of the \ei{frame} is actually
\begin{lscommand}
  \ei*{frame}[options: o, title: m, subtitle: m]
\end{lscommand}
Where \carg{options} is a key value list of options applied to the frame while
\carg{title} and \carg{subtitle} are displayed at the top of the slide. Despite
the curly brackets these are actually optional arguments and have been
omitted in the previous example.
\begin{example}
\begin{document} %!hide
\begin{frame}{%
  Interesting title}{%
  Even more interesting
  subtitle}
  
  Not so interesting contents.
\end{frame}
\end{document} %!hide
\end{example}
The title and subtitle can also be specified using the \csi{frametitle} and
\csi{framesubtitle} commands.

As in the standard classes you can specify title, author and date in the preamble
using the \csi{title}, \csi{author} and \csi{date} commands. The
\csi{maketitle} will then use theses settings to create a title frame.
\begin{example}
\author{Jane Doe}
\title{Interesting Title}
\date{\today}
% ...

\begin{document} %!hide
\maketitle
\end{document} %!hide
\end{example}
Additionally \csi{subtitle}, \csi{institute} and \csi{titlegraphic} are
available to add extra information to the title frame. The \csi{date} is often
repurposed to hold the conference name, since it is more informative than the
exact date. Some themes (discussed later) display these fields on each slide.
If you find that they are too long for such use, you may specify shorter
versions via the optional argument like this:
\begin{minted}{latex}
  \title[Short Title]{A Title That Is Too Long}
\end{minted}

Sectioning commands, such as \csi{section}, are meant to be used between frames
and define the structure of the presentation. Using them has no visible impact
on the frames themselves, but they allow for a  nice \csi{tableofcontents} with
properly hyperlinked entries.
\begin{example}
\begin{document} %!hide
\begin{frame}{Outline}
  \tableofcontents
\end{frame}

\section{A Section}
\begin{frame}
  % ...
\end{frame}
\section{A Longer Section}
% ...
\subsection{A Subsection}
% ...
%!hidebegin
\begin{frame}
  a dummy frame so the table of contents renders properly
\end{frame}
\end{document} %!hide
\end{example}

\cli{beamer} provides special commands for emphasising information on the slide:
the \csi{alert} command typesets its argument using a bright red colour, the
\ei{block} environment displays its contents with a title and separated from
the rest of the text.
\begin{example}
\begin{document} %!hide
\begin{frame} %!hide
Here we will talk about
\alert{emphasis}. It is a
\emph{very} interesting topic.
\begin{block}{Font Shape}
  \emph{Italic type} is often
  used.
\end{block}
\begin{block}{Colour}
  A distinct \alert{colour}
  also works.
\end{block}
\end{frame} %!hide
\end{document} %!hide
\end{example}
Beamer redefines many \pai{amsmath}\footnote{\pai{amsmath} is loaded
  automatically by \cli{beamer}.} environments, such as \ei{theorem} or
\ei{proof} to also produce blocks.

You will find that most of the \LaTeX{} commands and environments, such as
\cs{ref}s or \env{itemize}, work just as expected with \cli{beamer}. A notable
exception is the verbatim input and similar constructs described in
\autoref{sec:code_listings}. Using them requires you to pass the
\cargv{fragile} option to the frame options.
\begin{example}
\begin{document} %!hide
\begin{frame}[fragile]
  Inside \verb|fragile| frames
  you can use the verbatim
  freely.
  \begin{verbatim}
Hello!
  \end{verbatim}
\end{frame}
\end{document} %!hide
\end{example}

\subsection{Overlay Specification}

So far, the created frames were very static---they showed all of their contents
at once. It is, however, possible to show a frame piece-wise, in order to not
overwhelm an audience with lots of information.

The simplest way to do so is the \csi{pause} command. It splits the frame so
that the content before the pause is present on the previous page while the
rest of the frame is present on the next page. These partial frames are called
slides.
\begin{example}[
  vertical_mode,
  to_page=3,
  paperheight=2.8cm,
  paperwidth=3.7cm,
  examplewidth=.95\linewidth,
]
\begin{document} %!hide
\begin{frame}
  This will be shown first. \pause Then this. \pause
  And finally this.
\end{frame}
\end{document} %!hide
\end{example}
When presenting this will give an effect of revealing the sentences one by one.

If the \csi{pause} command is not enough for you, a more powerful variant
exist:
\begin{lscommand}
  \csi{onslide}[<«\bs carg{overlay specification}»>: c]
\end{lscommand}
The \carg{overlay specification} argument specifies at which slides should the
contents appear at. It may be a single number (slides start at 1), a range of
numbers (separated by \cargv{-}) or a comma delimited list of the two.
\begin{chktexignore}
\begin{example}[
  vertical_mode,
  to_page=3,
  paperheight=2.8cm,
  paperwidth=3.7cm,
  examplewidth=.95\linewidth,
]
\begin{document} %!hide
\begin{frame}
  \onslide<1> One. \onslide<2> Two. \onslide<1,3> One and
  three. \onslide<2-> Two onwards. \onslide<1-2> One to two.
\end{frame}
\end{document} %!hide
\end{example}
\end{chktexignore}
The \csi{pause} command uses \csi{onslide} internally together with a counter,
so these can be mixed together as desired, though care is required to set the
slide numbers correctly.

A similar command is \csi{uncover}. It accepts the same optional \carg{overlay
  specification} argument but only applies it to its mandatory argument.
\begin{chktexignore}
\begin{example}[
  vertical_mode,
  to_page=3,
  paperheight=2.8cm,
  paperwidth=3.7cm,
  examplewidth=.95\linewidth,
]
\begin{document} %!hide
\begin{frame}
  \uncover<1>{One.} \uncover<2>{Two.} \uncover<1,3>{One
    and three.} All.
\end{frame}
\end{document} %!hide
\end{example}
\end{chktexignore}
There are more commands that accept \carg{overlay specification} argument, and
many more preexisting commands and environments that are extended by
\pai{beamer} to support them. Refer to its User Guide~\cite{pack:beamer} for
more examples.

By default the contents that are not uncovered are not shown at all. If you
don't want to have a lot of empty space on your slides or you prefer not to
surprise your audience, you may adjust this behaviour to typeset the covered
text as transparent. To do so, use the \csi{setbeamercovered} with the
\cargv{transparent} option.
\begin{example}[
  vertical_mode,
  to_page=3,
  paperheight=2.8cm,
  paperwidth=3.7cm,
  examplewidth=.95\linewidth,
]
\setbeamercovered{transparent}
% ...

\begin{document} %!hide
\begin{frame}
  This will be shown first. \pause Then this. \pause
  And finally this.
\end{frame}
\end{document} %!hide
\end{example}
There are other possible arguments to these commands. For example, the
\cargv{dynamic} argument makes the text more transparent the later it is
shown.\footnote{The effect is amplified a bit in the below example, so it is
  more visible. The default variant works better when there are more
  \csi{pause}s.}
\begin{example}[
  vertical_mode,
  to_page=3,
  paperheight=2.8cm,
  paperwidth=3.7cm,
  examplewidth=.95\linewidth,
]
\setbeamercovered{dynamic}
\setbeamercovered{still covered={\opaqueness<1>{40}\opaqueness<2>{10}}} %!hide
% ...

\begin{document} %!hide
\begin{frame}
  This will be shown first. \pause Then this. \pause
  And finally this.
\end{frame}
\end{document} %!hide
\end{example}

\subsection{Customisation}

If the default appearance of the presentation is not to your liking,
\pai{beamer} provides a lot of options to customise it. The most basic commands
are \csi{usetheme} and \csi{usecolortheme}. These allow you to choose from a
predefined set of themes that alter the style and colours of the presentation.

\begin{example}
\usetheme{Madrid}
\usecolortheme{wolverine}
\author{Jane Doe}
\title{Title}
\date{Yesterday}
% ...

\begin{document} %!hide
\begin{frame}{Title}
  Normal text.
  \alert{Alerted text}.
  \begin{block}{Block}
    A block.
  \end{block}
\end{frame}
\end{document} %!hide
\end{example}
The full list of available themes can be viewed
at~\cite{AnotherBeamerThemeMatrix}.

The \csi{usefonttheme} command can be used in a similar fashion. By default the
text in frames is typeset using sans serif font. This choice makes sense,
because it is easier to read on lower resolution projectors, but is less
relevant if you use high-resolution display for presentation. You can pass
\cargv{serif} option to switch to default \LaTeX{} font.
\begin{example}
\usefonttheme{serif}
% ...

\begin{document} %!hide
\begin{frame}{Is serif better?}
  Often repeated assertion is
  that serif fonts are easier
  to read on paper, however, 
  this is not scientifically
  confirmed.
\end{frame}
\end{document} %!hide
\end{example}

More granular customisation is also possible using the
\begin{lscommand}
  \csi{setbeamerfont}[element: m, attributes: m] \\
  \csi{setbeamercolor}[element: m, attributes: m]
\end{lscommand}
commands. The \carg{element} argument is the element to which the customised
font or colour should be applied, while the \carg{attributes} specify the
attributes such as \cargv{size}, \cargv{shape} or \cargv{family} for fonts, and
\cargv{fg} (foreground) or \cargv{bg} (background) for colours. Fonts are
further discussed in \autoref{sec:fontspec}, while colours are touched upon in
\autoref{sec:colors}.
\begin{example}
\setbeamercolor{frametitle}{
  fg=red,
  bg=lime,
}
\setbeamerfont{block title}{
  series=\bfseries,
  family=\ttfamily,
}
% ...

\begin{document} %!hide
\begin{frame}{Frame Title}
  Normal text.
  \begin{block}{Block title}
    More text.
  \end{block}
\end{frame}
\end{document} %!hide
\end{example}

The default math font of \LaTeX{} is meant to typeset it in serif fonts. While
\pai{beamer} tries to adapt it by using the sans serif variables, not all
symbols exist in sans serif version (for example, Greek letters), which looks
silly at times. It is best to switch to a sans serif math font and matching
text font if your presentation is math heavy. Switching math fonts is described
further in \autoref{sec:math_fonts}.
\begin{example}
\usepackage{unicode-math}
\setsansfont{Fira Sans}
\setmathfont{Fira Math}
\setoperatorfont{\mathsf}
% ...

\begin{document} %!hide
\begin{frame} %!hide
With these fonts Greek
variables don't look silly
next to the Latin ones.
\[ x + b = \chi + \beta \]
\[ \lim_{x \to 0}
   \frac{\sin(x)}{x} = 1 \]
\end{frame} %!hide
\end{document} %!hide
\end{example}

By default, \pai{beamer} typesets the slides onto \(4 : 3\) pages. If your
projector uses other aspect ratio, you may prefer to change it accordingly. You
may do so by passing the \cargv{aspectratio} to the \cs{documentclass} options.
It accepts a single integer that encodes the desired ratio, for example
\cargv{169} is interpreted as \(16:9\), while \cargv{54} is interpreted as
\(5:4\). Wider ratios are especially useful if you want to include a sidebar
with the table of contents into your slides.
\begin{example}[paperwidth=8cm]
%!showbegin !hide
\documentclass[
  aspectratio=169
]{beamer}
%!showend !hide

\begin{document}
\begin{frame}
  Wide is beautiful.
\end{frame}
\end{document}
\end{example}

\subsection{Handouts}

One of the killer features of \pai{beamer} compared to other presentation
software is the ability to easily create handouts directly from your source
files. The simplest way to do so is to add the option \cargv{handout} to the
\cs{documentclass} command. Doing so will make the \cli{beamer} class ignore
all of the \csi{pause}s and similar commands to produce documents with fewer
pages.
\begin{example}[template=empty]
\documentclass[handout]{beamer}

% !hidebegin
\geometry{
  paperheight=\height,
  paperwidth=\width,
}

\setbeamersize{
  text margin left=0.6cm,
  text margin right=0.6cm,
}
% !hideend
\begin{document}
\begin{frame}
  All of the \pause text is on
  \pause a single slide, \pause
  even though \pause pauses
  are present.
\end{frame}
\end{document}
\end{example}
This mode is also useful while creating the presentation to preview the frames
as a whole without seeking the last slide each time.

Internally this effect is accomplished by using overlay specifications with
modes. The default mode is \cargv{beamer}, while \cargv{handout} class option
switches to \cargv{handout} mode. You can specify modes explicitly within
overlay specification by passing \cargv{\carg{mode}:\carg{spec}}. Several mode
specifications can be specified by separating them using \cargv{|}~(vertical
bar). If no mode is specified,
\cargv{beamer} mode is assumed.
\begin{example}[
  template=empty, 
  vertical_mode,
  to_page=2,
  paperheight=3cm,
  paperwidth=4cm,
  examplewidth=.7\linewidth,
]
\documentclass[handout]{beamer}

% !hidebegin
\geometry{
  paperheight=\height,
  paperwidth=\width,
}

\setbeamersize{
  text margin left=0.6cm,
  text margin right=0.6cm,
}
% !hideend
\begin{document}
\begin{frame}
  Some text \pause with pauses.
  \onslide<beamer: 3-| handout: 2-> This text
  will be on a separate slide, even in handout.
\end{frame}
\end{document}
\end{example}
Often you will want to show some parts only in one mode, for example to provide
additional commentary in handouts, or to hide things that only make sense
during presentation. Doing so is easily achieved by using a special zeroth
slide that is not included when compiling the document. An example of doing so
is presented in \autoref{lst:zero_slides}.
\begin{listing}
  \begin{example}[
    template=empty, 
    vertical_mode,
    to_page=2,
    paperheight=3cm,
    paperwidth=4cm,
    examplewidth=.7\linewidth,
  ]
\documentclass[handout]{beamer}

% !hidebegin
\geometry{
  paperheight=\height,
  paperwidth=\width,
}

\setbeamersize{
  text margin left=0.6cm,
  text margin right=0.6cm,
}
% !hideend
\begin{document}
\begin{frame}
  A frame.
\end{frame}
\begin{frame}<handout: 0>
  A frame only visible in presentation.
\end{frame}
\begin{frame}<0>
  A frame only visible in handout.
\end{frame}
\end{document}
\end{example}
  \caption{An example of using zeroth slides to hide content in presentation and
    in handout.}\label{lst:zero_slides}
\end{listing}
\endgroup

While handouts created this way are better compared to full blown presentation,
they are still rather uneconomical, since the slide structure---with titles,
lots of empty space and bright colours---is used. The \pai{beamer} package
provides another way to create handouts---the \cargv{article} mode which, as
the name suggests, will typeset the document in a fashion similar to a regular
article. Due to much starker differences between layouts in these modes this
may require more tweaking compared to the previous mode, but the result will be
usually much better suited for an actual printed handout.

Switching to the \cargv{article} mode is a bit different compared to other
\pai{beamer} modes. Instead of passing class option, you simply replace the
class altogether for an \cli{article} and use \pai{beamerarticle} package in
your preamble. See \autoref{lst:article_mode} for an example. The
\pai{beamerarticle} package defines all of the \pai{beamer} commands and
environments, so that they will produce sensible output in the article.
\begin{listing}
  \begin{lined}{\textwidth}
    \begin{example}[standalone]
%!showbegin !hide
% \documentclass{beamer}
\documentclass{article}
%!showend !hide

\usepackage{beamerarticle}

\title{Handouts}
\author{Me and Myself}

\begin{document}
\maketitle

\begin{frame}
  Handouts are \alert{very}
  important.
\end{frame}

\begin{frame}
  Almost as important as the
  presentations.
\end{frame}
\end{document}
\end{example}
  \end{lined}
  \caption{An example of using \pai{beamer} in article
    mode.}\label{lst:article_mode}
\end{listing}

Because the concept of slides is not present in an article,\footnote{To be more
  precise, only contents of the first slide are typeset.} the overlay
specifications will be generally useless. However the special zeroth slide
works normally and can be similarly used to hide content from a given mode.
\begin{example}[standalone, paperheight=3cm]
%!hidebegin
\usepackage{beamerarticle}

\begin{document}
%!hideend
\begin{frame}<0>
  Additional information for
  the article.
  \uncover<article: 2->{This
    won't print.}
\end{frame}

\begin{frame}<article: 0>
  Additional information for
  the presentation.
\end{frame}
\end{document} %!hide
\end{example}

Because the overlay specification is often used to provide contents for a
single mode, there exist another way of specifying it for that purpose---if you
provide only the mode name, this will be the same as setting all other modes to
zeroth slide and leaving the first slide for the specified mode. This usually
results in a more readable code.
\begin{example}[standalone, paperheight=3cm]
%!hidebegin
\usepackage{beamerarticle}

\begin{document}
%!hideend
\begin{frame}<article>
  Content for the article.
\end{frame}

\begin{frame}<beamer>
  Content for the presentation.
\end{frame}
\end{document} %!hide
\end{example}
