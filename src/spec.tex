% The Not So Short Introduction to LaTeX
%
% Copyright (C) 1995--2022 Tobias Oetiker, Marcin Serwin, Hubert Partl,
% Irene Hyna, Elisabeth Schlegl and Contributors.
%
% This document is free software: you can redistribute it and/or modify it
% under the terms of the GNU General Public License as published by the Free
% Software Foundation, either version 3 of the License, or (at your option) any
% later version.
%
% This document is distributed in the hope that it will be useful, but WITHOUT
% ANY WARRANTY; without even the implied warranty of MERCHANTABILITY or FITNESS
% FOR A PARTICULAR PURPOSE.  See the GNU General Public License for more
% details.
%
% You should have received a copy of the GNU General Public License along with
% this document.  If not, see <https://www.gnu.org/licenses/>.

% !TEX root = ./lshort.tex
%%%%%%%%%%%%%%%%%%%%%%%%%%%%%%%%%%%%%%%%%%%%%%%%%%%%%%%%%%%%%%%%%
% Contents: Specialities of the LaTeX system
% $Id$
%%%%%%%%%%%%%%%%%%%%%%%%%%%%%%%%%%%%%%%%%%%%%%%%%%%%%%%%%%%%%%%%%

\chapter{Specialities}\label{specialities}
\begin{intro}
  When putting together a large document, \LaTeX{} will help with some special
  features like index generation, automatic linking to relevant pages and other
  things. A much more complete description of specialities and enhancements
  possible with \LaTeX{} can be found in the {\normalfont\manual{}} and
    {\normalfont\companion}.
\end{intro}

\section{Indexing}\label{sec:indexing}
A very useful feature of many books is their \wi{index}. With \LaTeX{}
and the support program \texttt{makeindex},\footnote{On systems not
  necessarily supporting
  filenames longer than 8~characters, the program may be called
  \texttt{makeidx}.} an index can be generated quite easily.  This
introduction will only explain the basic index generation commands.
For a more in-depth view, please refer to \companion.\index{makeindex
  program}\index{makeidx package}

To enable their indexing feature of \LaTeX{}, the \pai{makeidx} package
must be loaded in the preamble with
\begin{lscommand}
  \mintinline{latex}|\usepackage{makeidx}|
\end{lscommand}
\noindent and the special indexing commands must be enabled by putting
the
\begin{lscommand}
  \csi{makeindex}
\end{lscommand}
\noindent command in the preamble.

The content of the index is specified with
\begin{lscommand}
  \csi{index}[{{«\bs carg{key}»@«\bs carg{formatted entry}»}}: c]
\end{lscommand}
\noindent commands, where \carg{formatted entry} will appear in the index
and \carg{key} will be used for sorting.  The \carg{formatted entry} is
optional. If it is missing the \carg{key} will be used. You enter the index
commands at the points in the text that you want the final index entries to
point to. \autoref{index} explains the syntax with several examples.

\begin{table}[!tp]
  \centering
  \caption{Index Key Syntax Examples.}\label{index}
  \begin{tabular}{@{}lll@{}}
    \toprule
    Code                                   & Entry                 & Explanation            \\
    \midrule
    \rule{0pt}{1.05em}\verb|\index{hello}| & hello, 1              & Plain entry            \\
    \verb|\index{hello!Peter}|             & \hspace*{2ex}Peter, 3 & Subentry under `hello' \\
    \verb|\index{Sam@\emph{Sam}}|          & \emph{Sam}, 2         & Formatted entry        \\
    \verb|\index{Kaese@\emph{K\"ase}}|     & \emph{K\"ase}, 33     & Formatted entry        \\
    \verb.\index{ecole@\'ecole}.           & \'ecole, 4            & Formatted entry        \\
    \verb.\index{Jenny|emph}.              & Jenny, \emph{3}       & Formatted page number  \\
    \verb.\index{Joe@\emph{Joe}|emph}.     & \emph{Joe}, \emph{5}  & Formatted page number  \\
    \bottomrule
  \end{tabular}
\end{table}

When the input file is processed with \LaTeX{}, each \verb|\index|
command writes an appropriate index entry, together with the current
page number, to a special file. The file has the same name as the
\LaTeX{} input file, but a different extension (\verb|.idx|). This
\eei{.idx} file can then be processed with the \texttt{makeindex}
program:

\begin{lscommand}
  \texttt{makeindex} \emph{filename}
\end{lscommand}
The \texttt{makeindex} program generates a sorted index with the same base
file name, but this time with the extension \eei{.ind}. If now the
\LaTeX{} input file is processed again, this sorted index gets
included into the document at the point where \LaTeX{} finds
\begin{lscommand}
  \csi{printindex}
\end{lscommand}

The \pai{showidx} package that comes with \LaTeX{} prints out all
index entries in the left margin of the text. This is quite useful for
proofreading a document and verifying the index.

Note that the \csi{index} command can affect your layout if not used carefully.

\begin{chktexignore}
  \begin{example}
My Word \index{Word}. As opposed
to Word\index{Word}. Note the
position of the full stop.
\end{example}
\end{chktexignore}

\texttt{makeindex} has no clue about characters outside the ASCII range. To
get the sorting correct, use the \verb|@| character as shown in the K\"ase
and \'ecole examples above.

\section{Fancy Headers}\label{sec:fancy}

\subsection{Basic commands}

The \pai*{fancyhdr} package provides a few simple commands that allow you to
customise the header and footer lines of your document. It defines an
additional page style \cargv{fancy} and a set of commands to customise it to our liking. By
default it only adds a line separating the header from the page body.
\begin{example}[standalone, paperheight=3cm]
\geometry{includefoot, includehead, headsep=.5em, footskip=1em} %!hide 
%!showbegin %!hide
\documentclass{article}

%!showend %!hide
\usepackage{fancyhdr}
\pagestyle{fancy}

\begin{document}
\noindent %!hide
This statement is false.
\end{document}
\end{example}

The primary command of the package is
\begin{lscommand}
  \csi{fancyhf}[places: o, field: m]
\end{lscommand}
The \carg{places} is a comma separated list of places where the \carg{field} should be displayed. There are total of 12 different places and each is identified by a
combination of three letters:
\begin{itemize}
  \item The first specifies whether the location is in the header (\cargv{H}) or in the
        footer (\cargv{F}).
  \item The second letter specifies the location within the header\slash{}footer: \cargv{L} for left, \cargv{C} for centre and
        \cargv{R} for right.
  \item The third letter defines whether the field should be printed on even
        (\cargv{E}) or odd (\cargv{O}) pages. If the document is not two sided then
        all pages are treated as odd.
\end{itemize}
So for example the combination \cargv{FCE} identifies the centre part of the
footer on even pages. If any of the letters is omitted then the identifier
points toward all positions specifiable by the omitted letter, so for example
\cargv{HR} is right side of the header on both odd and even pages.
\begin{example}[template=empty, standalone, paperheight=2.3cm, paperwidth=6cm, to_page=2,vertical_pages]
\documentclass[twoside]{article}

%!hidebegin
\usepackage[paperheight=\height,
    paperwidth=\width,
    margin=0.3cm,
    includefoot,
    includehead,
    headsep=.5em,
    footskip=1em
]{geometry}
%!hideend
\usepackage{fancyhdr} 
\pagestyle{fancy}

\fancyhf[HCE]{A}
\fancyhf[L]{\emph{B}}
\fancyhf[FR]{\textbf{C}}
\fancyhf[HCO, HRE]{\textsl{D}}

\begin{document}
\noindent %!hide
The next statement is false.
The previous statement is true.
\end{document}
\end{example}

There are two additional commands \csi{fancyhead} and \csi{fancyfoot} that work
the same way except that they assume \cargv{H} and \cargv{F} respectively
in their \carg{places} argument unless specified differently.
\begin{example}[standalone, paperheight=3.5cm]
\geometry{includefoot, includehead, headsep=.5em, footskip=1em} %!hide
\sloppy %!hide
\usepackage{csquotes} %!hide
\usepackage{fancyhdr}%!hide
\pagestyle{fancy}%!hide

\fancyhead[L]{A}
\fancyfoot[R]{B}

\begin{document}%!hide
\noindent %!hide
\enquote{Yields a falsehood when
appended to its own quotation}
yields a falsehood when appended
to its own quotation.
\end{document}%!hide
\end{example}

The lines drawn by the \pai{fancyhdr} package may also be customised. To change
their thickness redefine the
\begin{lscommand}
  \csi{headrulewidth} \\
  \csi{footrulewidth}
\end{lscommand}
macros to the desired size.
\begin{example}[standalone, paperheight=3cm, paperwidth=3cm]
\geometry{includefoot, includehead, headsep=.5em, footskip=1em} %!hide
\sloppy %!hide
\usepackage{fancyhdr} %!hide
\pagestyle{fancy} %!hide
\RenewDocumentCommand{\headrulewidth}{}{.2cm}
\RenewDocumentCommand{\footrulewidth}{}{.5cm}

\begin{document}%!hide
\noindent %!hide
Do not read this sentence.
\end{document}%!hide
\end{example}

By default headers and footers are as long as the text on the page. If you want
to extend\slash{}shorten them use the
\begin{lscommand}
  \csi{fancyhfoffset}[places: o, offset: m]
\end{lscommand}
command. The \carg{places} argument is the same as in \csi{fancyhf} except that
it cannot contain \cargv{C}.
\begin{example}[standalone, paperheight=3cm]
\geometry{includehead, includefoot, headsep=.5em, footskip=1em} %!hide
\sloppy %!hide
\usepackage{fancyhdr}%!hide
\pagestyle{fancy}%!hide
\fancyhfoffset[L]{-1cm}
\fancyhfoffset[R]{.2cm}

\begin{document}%!hide
\noindent %!hide
If this sentence is true,
then \(2 + 2 = 5\).
\end{document}%!hide
\end{example}
Similarly to \csi{fancyf}, \csi{fancyheadoffset} and \csi{fancyfootoffset} are
used the same way except they only modify header\slash{}footer.

\subsection{Contents of the headers}

The default footer of the article class contains the current page number. To
use it inside the fancy header simply use the command \csi{thepage}.
\begin{example}[standalone, paperheight=2.5cm, to_page=2, vertical_pages]
\geometry{includehead, includefoot, headsep=.5em, footskip=1em} %!hide
\sloppy %!hide
\usepackage{fancyhdr}%!hide
\pagestyle{fancy}%!hide
\fancyhf{Page~\thepage}

\begin{document}%!hide
\noindent %!hide
This statement is dedicated to
all statements that are not
dedicated to themselves. 
\end{document}%!hide
\end{example}

It is often useful to have the header and footer contain information based on
the contents of the page. These are called \enquote{marks} in \LaTeX{}
terminology. Before we talk about the default ones let's consider how you can
define your own using the \pai{extramarks}\footnote{\pai{extramarks} is part
  of \pai*{fancyhdr}.} package.
\begin{lscommand}
  \csi{extramarks}[left:m, right:m] \\
  \csi{firstleftxmark} \\
  \csi{firstrightxmark} \\
  \csi{lastleftxmark} \\
  \csi{lastrightxmark}
\end{lscommand}
The \csi{extramarks} command is used to set the contents of \carg{left} and \carg{right}
marks. Then you can access these marks inside the headers by using the appropriate
command. The \texttt{first}- commands refer to the first mark occurring on the
page, while the \texttt{last}- refer to the last one.
\begin{example}[standalone, paperheight=4cm]
\geometry{includehead, includefoot, headsep=.5em, footskip=1em} %!hide
\sloppy %!hide
\usepackage{fancyhdr}%!hide
\usepackage{extramarks}
\pagestyle{fancy}%!hide

\fancyhead[L]{\firstleftxmark}
\fancyhead[R]{\lastleftxmark}
\fancyfoot[L]{\firstrightxmark}
\fancyfoot[R]{\lastrightxmark}

\begin{document} %!hide
\noindent %!hide
The second statement is false.
\extramarks{One}{2 is false}
The third statement is false.
\extramarks{Two}{3 is false}
The first statement is false.
\extramarks{Three}{1 is false}
\end{document} %!hide
\end{example}

Let us now look at the default marks defined by \LaTeX{}. After loading
\pai{extramarks} these may be set and accessed similarly:
\begin{lscommand}
  \csi{markboth}[left:m, right:m] \\
  \csi{firstleftmark} \\
  \csi{firstrightmark} \\
  \csi{lastleftmark} \\
  \csi{lastrightmark}
\end{lscommand}
The only difference between these and the extra marks is that \LaTeX{} classes
automatically fill them, so if you are not careful you may lose your content.
\footnote{You may prevent this by setting the pagestyle to \cargv{myheadings}
  and then redefining it to use it with \pai{fancyhdr} as described later.}
For example in the article class, \carg{left} is set by the \csi{section}
command, while \carg{right} is set by the \csi{subsection} command.
\begin{example}[standalone, paperheight=5cm]
\geometry{includehead, includefoot, headsep=.5em, footskip=1em} %!hide
\sloppy %!hide
\usepackage{fancyhdr}%!hide
\usepackage{extramarks}%!hide
\pagestyle{fancy}%!hide
\fancyhead[L]{\firstleftmark}
\fancyhead[R]{\lastleftmark}
\fancyfoot[L]{\firstrightmark}
\fancyfoot[R]{\lastrightmark}

\begin{document}%!hide
\section{First}
\subsection{Sub}
\section{Second}
\subsection{Sub}
\end{document}%!hide
\end{example}
Note that \csi{firstrightmark} is empty in the above example. This is caused by
the fact that \csi{section} commands sets both marks, leaving the right one
empty.

You may have noticed that section titles are typeset in uppercase when provided by
marks commands. To disable this use the \csi{nouppercase} command inside the \csi{fancyhf}
command.
\begin{example}[standalone, paperheight=3cm]
\geometry{includehead, includefoot, headsep=.5em, footskip=1em} %!hide
\sloppy %!hide
\usepackage{fancyhdr}%!hide
\usepackage{extramarks}%!hide
\pagestyle{fancy}%!hide
\fancyhead[R]{%
  \nouppercase{\firstleftmark}%
}

\begin{document}%!hide
\section{Today}
is opposite day.
\end{document}%!hide
\end{example}

\subsection{Advanced commands}

If you want even more control over the section titles you may redefine the
\csi{sectionmark}.\footnote{Or \csi{chaptermark}, \csi{subsectionmark} \ldots} The
command receives the section title as its first argument, while the section
number is available as \csi{thesection}.
\begin{example}[standalone, paperheight=4.5cm, paperwidth=4cm]
\geometry{includehead, includefoot, headsep=.5em, footskip=1em} %!hide
\sloppy %!hide
\usepackage{fancyhdr}%!hide
\usepackage{extramarks}%!hide
\pagestyle{fancy}%!hide
\fancyhead[R]{\firstleftmark}
\RenewDocumentCommand{\sectionmark}{m}{%
  \markboth{%
    Section no.\,\thesection: #1}{}%
}

\begin{document}%!hide
\section{Person}
There exists a person such that
if they are reading this then
everybody is reading this.
\end{document}%!hide
\end{example}
If you only want to change the right mark, use the \csi{markright} command.

By default the field in the centre of the header\slash{}footer will expand equally
to the left and to the right. This is usually the desired behaviour but in some
cases it may overlap with either left or right text, while still having some
space on the other side.
\begin{example}[standalone, paperheight=3cm]
\geometry{includehead, includefoot, headsep=.5em, footskip=1em} %!hide
\sloppy %!hide
\usepackage{fancyhdr}%!hide
\usepackage{extramarks}%!hide
\pagestyle{fancy}%!hide
\fancyhead[L]{\thepage}
\fancyhead[C]{\firstleftmark}
\fancyhead[R]{Jane Doe}

\begin{document}%!hide
\section{Section}
Section by Jane Doe
\end{document}%!hide
\end{example}
In this situation you may want to use the
\begin{lscommand}
  \csi{fancycenter}[distance:o, stretch:o, left:m, centre:m, right:m]
\end{lscommand}
command that automatically shifts the centre toward the shorter text. The
\carg{distance} is the minimal distance that the elements are always surrounded
with (\cargv{1em} by default), while the \carg{stretch} controls the preference
for shifting the \carg{centre}: \cargv{1} means shift only when- and as much
as necessary, higher numbers will start shifting sooner and more aggressively,
default is \cargv{3}. This command writes over the whole header\slash{}footer
space so it should only be put in one place (typically \cargv{C}) and other
places (\cargv{L,R}) should be empty.
\begin{example}[standalone, paperheight=3cm]
\geometry{includehead, includefoot, headsep=.5em, footskip=1em} %!hide
\sloppy %!hide
\usepackage{fancyhdr}%!hide
\usepackage{extramarks}%!hide
\pagestyle{fancy}%!hide
\fancyhead[L,R]{}
\fancyhead[C]{%
  \fancycenter%
    {\thepage}%
    {\firstleftmark}%
    {Jane Doe}%
}

\begin{document}%!hide
\section{Section}
Section by Jane Doe
\end{document}%!hide
\end{example}

You may want to present different headers\slash{}footers when the corresponding
page starts with a float or ends with a footnote. The \pai{fancyhdr} package
defines four commands that let you achieve this.
\begin{lscommand}
  \csi{iftopfloat}[true branch:m, false branch:m] \\
  \csi{ifbotfloat}[true branch:m, false branch:m] \\
  \csi{iffloatpage}[true branch:m, false branch:m] \\
  \csi{iffootnote}[true branch:m, false branch:m]
\end{lscommand}
\csi{iftopfloat}/\csi{ifbotfloat} execute \carg{true branch} if a float sits at
the top\slash{}bottom of the page. Similarly \csi{iffloatpage} checks whether
the page is special float-only page and \csi{iffootnote} checks if footnote is
at the bottom of the page.
\begin{example}[standalone, paperheight=4cm, to_page=2, vertical_pages]
\geometry{includehead, includefoot, headsep=.5em, footskip=1em} %!hide
\sloppy %!hide
\usepackage{fancyhdr}%!hide
\usepackage{extramarks}%!hide
\pagestyle{fancy}%!hide

\fancyhead[C]{%
  \iftopfloat{%
    A float is below me%
  } {}
}
\fancyfoot[C]{%
  \iffootnote{%
    A footnote is above me%
  } {%
    \thepage%
  }%
}

\begin{document}%!hide
\noindent %!hide
Ignore this footnote.%
\footnote{Read this.}
\begin{figure}[t]
  \centering
  A floating float.
  \caption{Hmm}
\end{figure}
\end{document}%!hide
\end{example}

The lines in headers and footers are created by invoking \csi{headrule} and
\csi{footrule} commands. If you want finer control over the lines consider
redefining these macros. Use \csi{headruleskip} and \csi{footruleskip} to raise
or lower them if necessary.
\begin{example}[standalone, paperheight=4cm, paperwidth=3.2cm]
\geometry{includehead, includefoot, headsep=.5em, footskip=1em} %!hide
\sloppy %!hide
\usepackage{fancyhdr}%!hide
\usepackage{extramarks}%!hide
\pagestyle{fancy}%!hide
\RenewDocumentCommand{\headrule}{}{
  \rule{0.05\headwidth}{0.2cm}%
  \rule[0.1cm]{0.9\headwidth}{\headrulewidth}%
  \rule{0.05\headwidth}{0.2cm}%
}

\RenewDocumentCommand{\headruleskip}{}{-0.2cm}

\begin{document}%!hide
\noindent %!hide
My age is the first number not
nameable in under twenty words.
\end{document} %!hide
\end{example}

While working on a longer document you may want to have several page styles for
different occasions. You may also notice that some commands (for example
\csi{chapter}, \csi{maketitle}) change the page style to \cargv{plain} whether
you want it or not. The solution to both of these problems is the
\begin{lscommand}
  \csi{fancypagestyle}[name:m, code:m]
\end{lscommand}
command. The \carg{name} is the name of page style to (re)define, while the %chktex 36
\carg{code} is the code to set up the page style. All page styles declared this
way use the \cargv{fancy} page style as its basis, so if empty \carg{code} is
given they will match \cargv{fancy} exactly.
\begin{example}[standalone, paperheight=3cm]
\geometry{includehead, includefoot, headsep=.5em, footskip=1em} %!hide
\sloppy %!hide
\usepackage{fancyhdr}%!hide
\usepackage{extramarks}%!hide
\fancypagestyle{mine}{
  \fancyhead[L]{My style}
}
\pagestyle{mine}

\begin{document}%!hide
%!showbegin %!hide
Were you expecting a paradox here?
%!showend !hidebegin
\noindent
You may be disappointed.
\end{document}
\end{example}

\section{Installing Extra Packages}\label{sec:Packages}

Most \LaTeX{} installations come with a large set of pre-installed
style packages, but many more are available on the net. The main
place to look for style packages on the Internet is CTAN (\url{http://www.ctan.org/}).

Packages such as \pai{geometry}, \pai{hyphenat}, and many
others are typically made up of two files: a file with the extension
\texttt{.ins} and another with the extension \texttt{.dtx}. There
will often be a \texttt{readme.txt} with a brief description of the
package. You should of course read this file first.

In any event, once you have copied the package files onto your
machine, you still have to process them in a way that (a) tells your
\TeX\ distribution about the new style package and (b) gives you
the documentation.  Here's how you do the first part:

\begin{enumerate}
  \item Run \LaTeX{} on the \texttt{.ins} file. This will
        extract a \eei{.sty} file.
  \item Move the \eei{.sty} file to a place where your distribution
        can find it. Usually this is in your \texttt{\ldots/\emph{localtexmf}/tex/latex}
        subdirectory (Windows or OS/2 users should feel free to change the
        direction of the slashes).
  \item Refresh your distribution's file-name database. The command
        depends on the \LaTeX{} distribution you use:
        \TeX{}live --- \texttt{texhash}; web2c --- \texttt{maktexlsr};
        MiK\TeX{} --- \texttt{initexmf -{}-update-fndb} or use the GUI\@.
\end{enumerate}

\noindent Now extract the documentation from the
\texttt{.dtx} file:

\begin{enumerate}
  \item Run \hologo{XeLaTeX} on the \texttt{.dtx} file.  This will generate a
        \texttt{.pdf} file. Note that you may have to run \hologo{XeLaTeX}
        several times before it gets the cross-references right.
  \item Check to see if \LaTeX\ has produced a \texttt{.idx} file
        among the various files you now have.
        If you do not see this file, then the documentation has no index. Continue
        with step~\ref{step:final}.
  \item In order to generate the index, type the following:\\
        \fbox{\texttt{makeindex -s gind.ist \textit{name}}}\\
        (where \textit{name} stands for the main-file name without any
        extension).
  \item Run \LaTeX\ on the \texttt{.dtx} file once again.\label{step:next}

  \item Last but not least, make a \texttt{.ps} or \texttt{.pdf}
        file to increase your reading pleasure.\label{step:final}

\end{enumerate}

Sometimes you will see that a \texttt{.glo}
(glossary) file has been produced. Run the following
command between
step~\ref{step:next} and~\ref{step:final}:

\noindent\texttt{makeindex -s gglo.ist -o \textit{name}.gls \textit{name}.glo}

\noindent Be sure to run \LaTeX\ on the \texttt{.dtx} one last
time before moving on to step~\ref{step:final}.

%%%%%%%%%%%%%%%%%%%%%%%%%%%%%%%%%%%%%%%%%%%%%%%%%%%%%%%%%%%%%%%%%
% Contents: Chapter on pdfLaTeX
% French original by Daniel Flipo 14/07/2004
%%%%%%%%%%%%%%%%%%%%%%%%%%%%%%%%%%%%%%%%%%%%%%%%%%%%%%%%%%%%%%%%%

\section{\LaTeX{} and PDF}\label{sec:pdftex}\index{PDF}
\begingroup
\LshortExampleSetup{hide_hyperref=false}
The initial release of \TeX{} predated the PDF format by nearly 16 years. The
original output files it produced---\eei{.dvi}s---were not meant to be read
on screen, only printed. Today many documents are never or seldom
printed, we read them directly on a screen. PDF format contains many
improvements to this but they are not implemented in core \LaTeX{}. These are
accessible via the \pai*{hyperref} package.

\subsection{Hypertext Links}\label{hyperlinks}

Hyperlinks are used to quickly jump around the document. The prime example of
using them is the table of contents, you don't have to manually scroll to a
given page---just click on a given chapter and you will be immediately
transported there. You already know that table of contents can be typeset using
the \csi{tableofcontents} command, but it doesn't contain any hyperlinks.

Luckily the process of updating your document is extremely easy: just add
\mintinline{latex}|\usepackage{hyperref}| as the \emph{last} package loaded in
your preamble. It redefines many \LaTeX{} commands to produce hyperlinked text.

Simply including the \pai{hyperref} package in your preamble redefines
internal \LaTeX{} commands to produce hyperlinks.
\begin{example}
%!hidebegin
\hypersetup{
  hidelinks,
  pdfborder=0 0 1,
}
%!showbegin !hideend
% In preamble
\usepackage{hyperref}
%!showend !hide
% ...
The reference to this section
now looks like that:\footnote{
  Footnotes are also
  made into hyperlinks.}
Section~\ref{hyperlinks} is
on page~\pageref{hyperlinks}.
The hyperref bibliographic
entry is~\cite{pack:hyperref}.
\end{example}
By default links are marked by a red box around them. This box
is only visible when viewing the document on screen and will not be printed.
The boxes are, however, rather ugly, so you may want to add the \cargv{colorlinks}
option while loading the package.
\begin{example}
%!showbegin !hide
% In preamble
\usepackage[
  colorlinks
]{hyperref}
%!showend !hide
% ...
The reference to this section
now looks like that:
Section~\ref{hyperlinks} is
on page~\pageref{hyperlinks}.
\end{example}
This is the option used throughout this booklet and assumed in further
examples. While this makes the links more visually appealing, it has the
disadvantage that the coloured links will be printed. To marry the best of both
worlds you can use the \pai*{ocgx2} package with the option
\cargv{ocgcolorlinks} like this
\begin{code}  
\begin{minted}{latex}
\usepackage{hyperref}
\usepackage[ocglinks]{ocgx2}
\end{minted}
\end{code}
Be warned though that it is not well supported by popular PDF viewers. Another
option is to use option \cargv{hidelinks} that makes the links clickable but
does not distinguish them visually.

In addition to redefining internal \LaTeX{} command, \pai{hyperref} defines
some additional ones. To typeset URLs you can now use the \csi{url} command.
\begin{example}
\url{https://www.ctan.org/}
\end{example}
If you want to display different text for the clickable link you can use the
\begin{lscommand}
  \csi{href}[URL: m, text: m]
\end{lscommand}
command. Note that if you intend for the document to be useful when printed,
you have to provide the full URLs anyway.
\begin{example}
Packages can be found on
\href{https://www.ctan.org/}{
  CTAN}.
\end{example}

A similar command is
\begin{lscommand}
  \csi{hyperref}[marker: o, text: m]
\end{lscommand}
that allows to create a hyperlink within the document with a different text.
\begin{example}
\hyperref[hyperlinks]{
  Hyperlinks section}
describes the usage of
hyperlinks.
\end{example}
To avoid nesting hyperlinks inside one another, \pai{hyperref} provides starred
versions of \csi{ref} and \csi{pageref} commands that produce text without
hyperlinking them.
\begin{example}
\hyperref[hyperlinks]{
  Section~\ref*{hyperlinks}}
describes the usage of
hyperlinks.

This ref~\ref*{hyperlinks}
is not hyperlinked.
\end{example}

Throughout this booklet you have seen references such as
\enquote*{\autoref{hyperlinks} on \autopageref{hyperlinks}}. While this could
be achieved using the aforementioned \csi{hyperref} command, this usecase is so
common that \pai{hyperref} provides two commands that make it much easier:
\begin{lscommand}
  \csi{autoref}[marker: m] \\
  \csi{autopageref}[marker: m]
\end{lscommand}
Their usage is identical to the \csi{ref} and \csi{pageref} commands, but they
produce additional text based on the counter the \carg{marker} refers to.
\begin{example}
\RenewDocumentCommand{\chapterautorefname}{}{chapter} %!hide
\autoref{hyperlinks} in
\autoref{specialities} on
\autopageref{hyperlinks}.
\end{example}
The names are controlled by commands such as \cs{chaptername} or
\cs{sectionname}. See the \pai*{hyperref} documentation for a full list.

So far we have used the default colours for URLs, hyperlinks. These can be
changed using the \csi{hypersetup} command. It accepts a key value list
customising the appearance of links. Colours may be specified using
\cargv{linkcolor}, \cargv{citecolor} and \cargv{urlcolor}. If you have the
\pai{xcolor} package loaded, you may specify colours the same way as described
in \autoref{sec:colors}.
% TODO: Add colors section.
\begin{chktexignore}
\begin{example}
\hypersetup{
  urlcolor = pink,
  citecolor = purple,
  linkcolor = teal!50!yellow,
}
\url{https://www.ctan.org/} \\
\cite{pack:hyperref} \\
\autoref{hyperlinks}
\end{example}
\end{chktexignore}

You can also adjust the borders around the links. The basic key is
\cargv{pdfborder}. It accepts three numbers: horizontal corner radius, vertical
corner radius and border width.
\begin{example}
\hypersetup{pdfborder = 0 0 1}
\url{https://www.ctan.org/} \\
\hypersetup{pdfborder = 10 10 3}
\url{https://www.ctan.org/} \\
\hypersetup{pdfborder = 10 5 2}
\url{https://www.ctan.org/} \\
\hypersetup{pdfborder = 2 7 5}
\url{https://www.ctan.org/}
\end{example}
The colour of the boxes may be adjusted similarly to the text colours with the
\cargv{linkbordercolor}, \cargv{citebordercolor} and \cargv{urlbordercolor}
keys, assuming the \pai{xcolor} package is loaded.
\begin{chktexignore}
  \begin{example}
\hypersetup{
  pdfborder = 0 0 2,
  urlbordercolor = violet,
  citebordercolor = pink,
  linkbordercolor = teal,
}
\url{https://www.ctan.org/} \\
\cite{pack:hyperref} \\
\autoref{hyperlinks}
\end{example}
\end{chktexignore}

\subsection{Document Metadata}\label{sec:pdfmeta}

Another thing that may be adjusted with the \pai{hyperref} package is document
metadata. These are information about your document that are not visible in the
document itself, but may be used by your PDF viewer in various ways. For
example, the title of the document may be shown in its top window bar.

Additional information about your document may be set using \cargv{pdfinfo}
key. This key itself accepts a key value list of document properties.
\begin{minted}{latex}
\hypersetup{
  pdfinfo = {
    Title = Title of the Book,
    Author = {Us, Ourselves and We},
    Subject = Book creation with LaTeX,
    Creator = Our House,
    Keywords = {LaTeX, typesetting},
    Producer = LuaTeX,
  }
}
\end{minted}

You can also control the way your document presents itself when opening. For
example you can choose whether the bookmarks should be shown
(\cargv{bookmarksopen}), whether external links should be opened in new windows
(\cargv{pdfnewwindow}) whether the pages should initially fit the window
(\cargv{pdffitwindow}).

If you want to set the metadata without redefining internal \LaTeX{} commands
to produce links, you may pass \mintinline{latex}{implicit=false} to
\pai{hyperref} package options.
\endgroup

\subsection{Problems with Outline}

The \pai{hyperref} package automatically uses table of contents generated by
the document as a document outline for easier navigation. This may lead to some
problems if your section titles contain some non-text content (for example,
\enquote{\LaTeX}). If this is the case, then it will be ignored by the
\pai{hyperref}
package and the following warning will be reported
\begin{code}
\begin{verbatim}
Package hyperref Warning:
Token not allowed in a PDFDocEncoded string:
\end{verbatim}
\end{code}
To get around this problem you can use the
\begin{lscommand}
  \csi{texorpdfstring}[\bs TeX{} text: m, outline text: m]
\end{lscommand}
command. Its first argument, \carg{\TeX{} text}, is the text to be displayed
inside the document, while the second is the fallback for \pai{hyperref} to
use. An example would be to change
\begin{minted}{latex}
\section{\LaTeX{} is awesome!}
\end{minted}
to
\begin{minted}{latex}
\section{\texorpdfstring{\LaTeX}{LaTeX} is awesome!}
\end{minted}

While the above method may be necessary for some complicated math formulae
that need to be nicely printed in the outline, usually the replacement texts
are rather obvious for a given command. In this case you can use the
\begin{lscommand}
  \csi{pdfstringdefDisableCommands}[commands: m]
\end{lscommand}
command, which allows you to define general fallbacks for some commands. The
\carg{commands} argument is a list of redefinitions to be done when evaluating
outline titles. Commands may be redefined using the
\csi{RenewExpandableDocumentCommand} command. For a description of how to
redefine the commands see \autoref{sec:new_commands}. An example of using this
command would be
\begin{minted}{latex}
\pdfstringdefDisableCommands {
  \RenewExpandableDocumentCommand{\ldots}{}{...}
  \RenewExpandableDocumentCommand{\LaTeX}{}{LaTeX}
  \RenewExpandableDocumentCommand{\emph}{m}{*#1*}
}
\end{minted}

\section{Creating Presentations}%
\label{sec:beamer}
\secby{Daniel Flipo}{Daniel.Flipo@univ-lille1.fr}
You can present the results of your scientific work on a blackboard,
with transparencies, or directly from your laptop using some
presentation software.

\wi{pdf\LaTeX} combined with the \pai{beamer} class allows you to
create presentations in PDF, looking much like something you might be
able to generate with LibreOffice or PowerPoint if you had a very good day, but much
more portable because PDF readers are available on many more
systems.

The \pai{beamer} class uses \pai{graphicx}, \pai{color} and
\pai{hyperref} with options adapted to screen presentations.
%La figure~\ref{fig:pdfscr} contient un exemple de fichier minimal
%compiler avec \wi{pdf\LaTeX} et le
%rsultat produit.

% cran captur par ImageMagick (man ImageMagick) fonction  import
% et convertie en jpg toujours par ImageMagick.

\begin{listing}
  \begin{minted}{latex}
\documentclass[10pt]{beamer}
\mode<beamer>{%
  \usetheme[hideothersubsections,
            right,width=22mm]{Goettingen}
}

\title{Simple Presentation}
\author[D. Flipo]{Daniel Flipo}
\institute{U.S.T.L. \& GUTenberg}
\titlegraphic{\includegraphics[width=20mm]{USTL}}
\date{2005}

\begin{document}

\begin{frame}<handout:0>
  \titlepage
\end{frame}

\section{An Example}

\begin{frame}
  \frametitle{Things to do on a Sunday Afternoon}
  \begin{block}{One could \ldots}
    \begin{itemize}
      \item walk the dog\dots \pause
      \item read a book\pause
      \item confuse a cat\pause
    \end{itemize}
  \end{block}
  and many other things
\end{frame}
\end{document}
\end{minted}
  \caption{Sample code for the \pai{beamer} class}%
  \label{lst:code-beamer}
\end{listing}

When you compile the code presented in \autoref{lst:code-beamer}
with \wi{pdf\LaTeX} you get a PDF file with a title page and a second page
showing several items that will be revealed one at a time as you step
though your presentation.

One of the advantages of the beamer class is that it produces a PDF
file that is directly usable without first going through a \PSi{}
stage like \pai{prosper} or requiring additional post processing like
presentations created with the \pai{ppower4} package.

With the \pai{beamer} class you can produce several versions (modes) of your
document from the same input file. The input file may contain special
instructions for the different modes in angular brackets. The
following modes are available:
\begin{description}
  \item[beamer] for the presentation PDF
    discussed above.
  \item[trans] for transparencies.
  \item[handout] for the printed version.
\end{description}
The default mode is \texttt{beamer}, change it by setting a
different mode as a global option, like
\mintinline{latex}|\documentclass[10pt,handout]{beamer}| to print the handouts for
example.

The look of the screen presentation depends on the theme you choose. Pick one of the themes shipped with the beamer class or
create your own. See the beamer class documentation in
\texttt{beameruserguide.pdf} for more information on this.

Let's have a closer look at the code in \autoref{lst:code-beamer}.

For the screen version of the presentation \mintinline{latex}|\mode<beamer>| we
have chosen the \cargv{Goettingen} theme to show a navigation panel
integrated into the table of contents. The options allow us to choose the
size of the panel (22~mm in this case) and its position (on the right
side of the body text). The option \cargv{hideothersubsections}, shows
the chapter titles, but only the subsections of the present
chapter. There are no special settings for \mintinline{latex}|\mode<trans>| and
\mintinline{latex}|\mode<handout>|. They appear in their standard layout.

The commands \mintinline{latex}|\title{}|, \mintinline{latex}|\author{}|, \mintinline{latex}|\institute{}|,
and\\ \mintinline{latex}|\titlegraphic{}| set the content of the title page. The
optional arguments of \mintinline{latex}|\title[]{}| and \mintinline{latex}|\author[]{}|
let you specify a special version of the title and the author
name to be displayed on the panel of the \cargv{Goettingen} theme.

The titles and subtitles in the panel are created with normal
\cs{section} and \cs{subsection} commands that you place
\emph{outside} the \ei{frame} environment.

The tiny navigation icons at the bottom of the screen also allow to
navigate the document. Their presence is not dependent on the theme
you choose.

The contents of each slide or screen has to be placed inside a
\ei{frame} environment. There is an optional argument in angular
brackets (\mintinline{latex}|<| and \mintinline{latex}|>|), it allows us to suppress a particular
frame in one of the versions of the presentation. In the example the
first page would not be shown in the handout version due to the
\mintinline{latex}|<handout:0>| argument.

It is highly recommended to set a title for each slide apart from the
title slide. This is done with the command \mintinline{latex}|\frametitle{}|. If a
subtitle is necessary use the \ei{block} environment as shown
in the example. Note that the sectioning commands \mintinline{latex}|\section{}|
and \mintinline{latex}|\subsection{}| do not produce output on the slide proper.

The command \mintinline{latex}|\pause| in the itemize environment lets you reveal
the items one by one. For other presentation effects check out the
commands \mintinline{latex}|\only|, \mintinline{latex}|\uncover|, \mintinline{latex}|\alt| and
\mintinline{latex}|\temporal|. In many place it is possible to use angular brackets to
further customise the presentation.

In any case make sure to read through the beamer class documentation
\texttt{beameruserguide.pdf} to get a complete picture of what is in
store for you. This package is being actively developed, check out their website
to get the latest information. (\href{http://latex-beamer.sourceforge.net/}{http://latex-beamer.sourceforge.net/})

% Local Variables:
% TeX-master: "lshort2e"
% mode: latex
% mode: flyspell
% End:
