% The Not So Short Introduction to LaTeX
%
% Copyright (C) 1995--2022 Tobias Oetiker, Marcin Serwin, Hubert Partl,
% Irene Hyna, Elisabeth Schlegl and Contributors.
%
% This document is free software: you can redistribute it and/or modify it
% under the terms of the GNU General Public License as published by the Free
% Software Foundation, either version 3 of the License, or (at your option) any
% later version.
%
% This document is distributed in the hope that it will be useful, but WITHOUT
% ANY WARRANTY; without even the implied warranty of MERCHANTABILITY or FITNESS
% FOR A PARTICULAR PURPOSE.  See the GNU General Public License for more
% details.
%
% You should have received a copy of the GNU General Public License along with
% this document.  If not, see <https://www.gnu.org/licenses/>.

% !TEX root = ./lshort.tex
%%%%%%%%%%%%%%%%%%%%%%%%%%%%%%%%%%%%%%%%%%%%%%%%%%%%%%%%%%%%%%%%%
% Contents: Who contributed to this Document
% $Id$
%%%%%%%%%%%%%%%%%%%%%%%%%%%%%%%%%%%%%%%%%%%%%%%%%%%%%%%%%%%%%%%%%
\thispagestyle{empty}
\begin{small}
  \noindent Copyright \copyright{} 1995--2022 Tobias~Oetiker, Marcin~Serwin,
  Hubert~Partl, Irene~Hyna, Elisabeth~Schlegl and Contributors.

  This document is free software: you can redistribute it and/or modify it under
  the terms of the GNU General Public License as published by the Free Software
  Foundation, either version 3 of the License, or (at your option) any later
  version.

  This document is distributed in the hope that it will be useful, but
  \emph{without any warranty}; without even the implied warranty of
  \emph{merchantability} or \emph{fitness for a particular purpose}.  See
  the GNU General Public License for more details.

  You should have received a copy of the GNU General Public License along with
  this document.  If not, see \url{https://www.gnu.org/licenses/}.
  \vspace{-5pt}
  \begin{flushright}
    \href{https://www.gnu.org/licenses/gpl-3.0.html}{%
      \includegraphics[width=2cm]{images/gpl3.pdf}}\hspace*{10pt}
  \end{flushright}

  \bigskip\noindent The illustration on the first page was created for this booklet by Roman Schmid
  (\href{https://github.com/bummzack}{@bummzack} on GitHub). It is dedicated to the
  public domain via CC0 1.0 Universal
  Public Domain Dedication. This means that you can use it freely in your own
  work without any licensing issues. See
  \url{https://creativecommons.org/publicdomain/zero/1.0/} for more details.
  \vspace{-15pt}
  \begin{flushright}
    \href{https://creativecommons.org/publicdomain/zero/1.0/}{%
      \includegraphics[width=2cm]{images/cc-zero.pdf}}\hspace*{10pt}
  \end{flushright}

  \bigskip\noindent
  Additionally, the code examples (text written using \texttt{monospace font}
  in this document) are dedicated to the public domain via CC0 1.0 Universal
  Public Domain Dedication. This means that you can use them freely in your own
  work without any licensing issues. See
  \url{https://creativecommons.org/publicdomain/zero/1.0/} for more details.
  \vspace{-15pt}
  \begin{flushright}
    \href{https://creativecommons.org/publicdomain/zero/1.0/}{%
      \includegraphics[width=2cm]{images/cc-zero.pdf}}\hspace*{10pt}
  \end{flushright}

  \vspace{\stretch{1}}
  \noindent The full text of the GNU General Public License can be found in
  \autoref{gplfull} on \autopageref{gplfull}.
\end{small}
\cleardoublepage
