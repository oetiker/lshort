% The Not So Short Introduction to LaTeX
%
% Copyright (C) 1995--2022 Tobias Oetiker, Marcin Serwin, Hubert Partl,
% Irene Hyna, Elisabeth Schlegl and Contributors.
%
% This document is free software: you can redistribute it and/or modify it
% under the terms of the GNU General Public License as published by the Free
% Software Foundation, either version 3 of the License, or (at your option) any
% later version.
%
% This document is distributed in the hope that it will be useful, but WITHOUT
% ANY WARRANTY; without even the implied warranty of MERCHANTABILITY or FITNESS
% FOR A PARTICULAR PURPOSE.  See the GNU General Public License for more
% details.
%
% You should have received a copy of the GNU General Public License along with
% this document.  If not, see <https://www.gnu.org/licenses/>.

%%%%%%%%%%%%%%%%%%%%%%%%%%%%%%%%%%%%%%%%%%%%%%%%%%%%%%%%%%%%%%%%%
% Contents: Who contributed to this Document
% $Id$
%%%%%%%%%%%%%%%%%%%%%%%%%%%%%%%%%%%%%%%%%%%%%%%%%%%%%%%%%%%%%%%%%

% Because this introduction is the reader's first impression, I have
% edited very heavily to try to clarify and economize the language.
% I hope you do not mind! I always try to ask "is this word needed?"
% in my own writing but I don't want to impose my style on you...
% but here I think it may be more important than the rest of the book.
% --baron

\chapter{Preface}

\LaTeX{}~\cite{manual} is a typesetting system that is very
suitable for producing scientific and mathematical documents of high
typographical quality. It is also suitable for producing all
sorts of other documents, from simple letters to complete books.
\LaTeX{} uses \TeX{}~\cite{texbook} as its formatting engine.

This short introduction will teach you all you need to get going with \LaTeX.
Refer to~\cite{manual,companion} for a complete description of the \LaTeX{}
system.

\bigskip
\noindent This introduction is split into 7 chapters:
\begin{description}
  \item[\autoref{chap:basics}] tells you about the basic structure of
    \LaTeX{} documents. You will also learn a bit about the history of
    \LaTeX{}. After reading this chapter, you should have a rough understanding
    of how \LaTeX{} works.
  \item[\autoref{chap:realworld}] goes into the details of typesetting your
    documents. It explains most of the essential \LaTeX{} commands and
    environments. After reading this chapter, you will be able to write your
    first documents, with itemized lists, tables, graphics and floating bodies.
  \item[\autoref{chap:math}] explains how to typeset formulae with \LaTeX.
    Many examples demonstrate how to use one of \LaTeX{}'s main strengths. At
    the end of the chapter are tables listing all mathematical symbols
    available in \LaTeX{}.
  \item[\autoref{chap:bibliography}] shows how to create bibliographies for
    your publications.
  \item[\autoref{specialities}] explains the secrets of indexes and some finer
    points about creating PDFs.
  \item[\autoref{chap:graphics}] shows how to use \LaTeX{} for creating
    graphics. Instead of drawing a picture with some graphics program, saving
    it to a file and then including it into \LaTeX{}, you describe the picture
    and have \LaTeX{} draw it for you.
  \item[\autoref{chap:custom}] contains some potentially dangerous
    information about how to alter the standard document layout produced by
    \LaTeX{}. It will tell you how  to change things such that the beautiful
    output of \LaTeX{} turns ugly or stunning, depending on your abilities.
\end{description}
\bigskip
\noindent It is important to read the chapters in order---the book is
not that big, after all. Be sure to carefully read the examples,
because a lot of the information is in the
examples placed throughout the book.

\bigskip
\noindent \LaTeX{} is available for most computers, from the PC and Mac, to
large UNIX and VMS systems. On many university computer clusters, you will
find that a \LaTeX{} installation is available, ready to use.
Information on how to access
the local \LaTeX{} installation should be provided in the \guide. If
you have problems getting started, ask the person who gave you this
booklet. The scope of this document is \emph{not} to tell you how to
install and set up a \LaTeX{} system, but to teach you how to write
your documents so that they can be processed by~\LaTeX{}.

\bigskip
\noindent If you need to get hold of any \LaTeX{} related material,
have a look at one of the Comprehensive \TeX{} Archive Network
(CTAN) sites. The homepage is at
\url{http://www.ctan.org}.

You will find other references to CTAN throughout the book, especially
pointers to software and documents you might want to download. Instead
of writing down complete URLs, I just wrote \texttt{CTAN:} followed by
whatever location within the CTAN tree you should go to.

If you want to run \LaTeX{} on your own computer, take a look at what
is available from \CTAN|systems|.

\vspace{\stretch{1}}
\noindent If you have ideas for something to be
added, removed or altered in this document, please let me know. I am
especially interested in feedback from \LaTeX{} novices about which
bits of this intro are easy to understand and which could be explained
better.

\bigskip
\begin{verse}
  \contrib{Tobias Oetiker}{tobi@oetiker.ch}%
  \noindent{OETIKER+PARTNER AG\\Aarweg 15\\4600 Olten\\Switzerland}
\end{verse}
\vspace{\stretch{1}}
\noindent The current version of this document is available on\\
\CTAN|info/lshort|

\endinput

%

% Local Variables:
% TeX-master: "lshort2e"
% mode: latex
% mode: flyspell
% End:
