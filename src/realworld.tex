% The Not So Short Introduction to LaTeX
%
% Copyright (C) 1995--2022 Tobias Oetiker, Marcin Serwin, Hubert Partl,
% Irene Hyna, Elisabeth Schlegl and Contributors.
%
% This document is free software: you can redistribute it and/or modify it
% under the terms of the GNU General Public License as published by the Free
% Software Foundation, either version 3 of the License, or (at your option) any
% later version.
%
% This document is distributed in the hope that it will be useful, but WITHOUT
% ANY WARRANTY; without even the implied warranty of MERCHANTABILITY or FITNESS
% FOR A PARTICULAR PURPOSE.  See the GNU General Public License for more
% details.
%
% You should have received a copy of the GNU General Public License along with
% this document.  If not, see <https://www.gnu.org/licenses/>.

% !TEX root = ./lshort.tex
%%%%%%%%%%%%%%%%%%%%%%%%%%%%%%%%%%%%%%%%%%%%%%%%%%%%%%%%%%%%%%%%%
% Contents: Typesetting Part of LaTeX Introduction
%%%%%%%%%%%%%%%%%%%%%%%%%%%%%%%%%%%%%%%%%%%%%%%%%%%%%%%%%%%%%%%%%

\chapter{Real World \LaTeX{}}\label{chap:realworld}

\begin{intro}
  After reading the previous chapter, you should have some general idea
  about \LaTeX{}. This chapter
  will fill in the remaining structure you will need to know in order
  to produce real world documents.
\end{intro}

\section{Line Breaking and Page Breaking}

\subsection{Justified Paragraphs}

Books are often typeset with each line having the same length.
\LaTeX{} inserts the necessary \wi{line break}s and spaces between words
by optimising the contents of a whole paragraph. If necessary, it
also hyphenates words that would not fit comfortably on a line.
How the paragraphs are typeset depends on the document class.
Normally the first line of a paragraph is indented, and there is no
additional space between two paragraphs. Refer to \autoref{parsp}
for more information.

In special cases it might be necessary to order \LaTeX{} to break a
line.
\begin{lscommand}
  \csi{\bs}[length:o] or \csi{newline}
\end{lscommand}
starts a new line without starting a new paragraph. The optional \carg{length}
argument adds additional space after the line.

\begin{lscommand}
  \csi{\bs*}[length:o]
\end{lscommand}
additionally prohibits a page break after the forced line break.

\begin{lscommand}
  \csi{newpage}
\end{lscommand}
starts a new page.

\begin{lscommand}
  \csi{linebreak}[n:o] \\
  \csi{nolinebreak}[n:o] \\
  \csi{pagebreak}[n:o] \\
  \csi{nopagebreak}[n:o]
\end{lscommand}
suggest places where a break may (or may not) happen.
They enable the author to influence their
actions with the optional argument \emph{n}, which can be set to a number
between zero and four. By setting \emph{n} to a value below 4, you leave
\LaTeX{} the option of ignoring your command if the result would look very
bad. Do not confuse these \enquote{break} commands with the \enquote{new}
commands. Even when you give a \enquote{break} command, \LaTeX{} still tries to
even out the right border of the line and the total length of the page. As
described in the next paragraph, this can lead to unpleasant gaps in your text.

If you really want to start a \enquote{new line} or a \enquote{new page}, then
use the corresponding command. Guess their names!
\begin{example}[examplewidth=0.4\linewidth]
  Start a new line here,\\
  and also here but make
  it bigger.\\[1cm]
  Do a linebreak here\linebreak
  and maybe here, \linebreak[1]
  but no pressure.
  Linebreak here\linebreak[3]
  would be \emph{really} cool.
\end{example}

\LaTeX{} always tries to produce the best line breaks possible. If it
cannot find a way to break the lines in a manner that meets its high
standards, it lets one line stick out on the right of the paragraph.
\LaTeX{} then complains (\enquote{\wi{overfull hbox}}) while processing the
input file. This happens most often when \LaTeX{} cannot find a
suitable place to hyphenate a word.\footnote{Although \LaTeX{} gives
  you a warning when that happens (\texttt{Overfull \bs{}hbox}) and displays the
  offending line, such lines are not always easy to find. If you use
  the option \texttt{draft} in the \csi{documentclass} command, these
  lines will be marked with a thick black line on the right margin.}
Instruct \LaTeX{} to lower its standards a little by giving
the \csi{sloppy} command. It prevents such over-long lines by
increasing the inter-word spacing---even if the final output is not
optimal.  In this case a warning (``\wi{underfull hbox}'') is given to
the user.  In most such cases the result doesn't look very good. The
command \csi{fussy} brings \LaTeX{} back to its default behaviour.

\subsection{Hyphenation}\label{hyph}

\LaTeX{} hyphenates words whenever necessary. If the hyphenation
algorithm does not find the correct hyphenation points,
remedy the situation by using the following commands to tell \TeX{}
about the exception.

The command
\begin{lscommand}
  \csi{hyphenation}[word list: m]
\end{lscommand}
causes the words listed in the argument to be hyphenated only at the points
marked by \enquote*{\texttt{-}}.  The argument of the command should only contain
words built from normal letters, or rather glyphs that are considered to be
normal letters by \LaTeX{}. The hyphenation hints are stored for the language
that is active when the hyphenation command occurs. This means that if you
place a hyphenation command into the preamble of your document, it will
influence the English language hyphenation. If you place the command after the
\begin{chktexignore}
  \mintinline{latex}|\begin{document}|
\end{chktexignore}
and you are using a package for national language
support, like \pai{polyglossia}, then the hyphenation hints will be active in
the language activated through \pai{polyglossia}.

The example below allows the word \enquote{locomotion} to be hyphenated, as
well as \enquote{Locomotion}, and it prevents \enquote{FORTRAN},
\enquote{Fortran} and \enquote{fortran} from being hyphenated at all.  No
special characters or symbols are allowed in the argument.
\begin{code}
\csi{hyphenation}[FORTRAN Lo-co-mo-tion:vm]
\end{code}

The command \csi{-} inserts a discretionary hyphen into a word. This also
becomes the only point hyphenation is allowed in this word. This command is
especially useful for words containing special characters, because \LaTeX{}
does not automatically hyphenate them.
\begin{example}
I think this is: su\-per\-%
cal\-i\-frag\-i\-lis\-tic\-%
ex\-pi\-al\-i\-do\-cious
\end{example}

Several words can be kept together on one line with the command
\begin{lscommand}
  \csi{mbox}[text: m]
\end{lscommand}
\noindent It causes its argument to be kept together under all circumstances.

\begin{example}
My phone number will
change soon. It will
be \mbox{0116 291 2319}.

The parameter
\mbox{[filename]} should
contain the name of the file.
\end{example}

\section{Ready-Made Strings}

In some of the examples on the previous pages, you have seen
some very simple \LaTeX{} commands for typesetting special
text strings:
\begin{center}
  \begin{tabular}{@{}lll@{}}
    \toprule
    Command      & Example   & Description               \\
    \midrule
    \csi{today}  & \today    & Current date              \\
    \csi{TeX}    & \TeX{}    & Your favourite typesetter \\
    \csi{LaTeX}  & \LaTeX{}  & The Name of the Game      \\
    \csi{LaTeXe} & \LaTeXe{} & The current incarnation   \\
    \bottomrule
  \end{tabular}
\end{center}

\section{Dashes and Hyphens}

\LaTeX{} knows four kinds of \wi{dash}es. You access three of them with
different numbers of consecutive dashes. The fourth sign is actually not a dash
at all---it is the mathematical minus sign. It is typed automatically when
inside math mode, as described later in \autoref{chap:math}.
\begin{example}
daughter-in-law, X-rated\\
pages 13--67\\
yes---or no?
\end{example}

The names for these dashes are: `-' \wi{hyphen}\index{-}, `--'
\wi{en-dash}\index{--} and `---' \wi{em-dash}\index{---}. Hyphens are used when
writing compound words (and inserted automatically by \LaTeX{} when splitting a
single word), en-dashes are used for writing a range of numbers, and em-dashes
are used to mark an interruption in speech or an abrupt change of thought.

\section{Slash (/)}\index{Slash}

In order to typeset a slash between two words, one can simply type, for example,
\texttt{read/write}, but this makes \LaTeX{} treat the two words as one.
Hyphenation is disabled by the slash character (see \autoref{hyph}), so there
may be \enquote*{overfull} errors.  To overcome this, use \csi{slash}.
For example, type
\begin{code}
  \mintinline{latex}|read\slash write|
\end{code}
which allows hyphenation.  Normal
\enquote*{\texttt{/}} characters may be still used for inline fractions.
The typesetting of units, such as \unit[per-mode = symbol]{\mebi\byte\per\s},
will be described in \autoref{sec:units}.

\section{Ellipsis (\ldots)}

On a typewriter, a \wi{comma} or a \wi{period} takes the same amount of
space as any other letter. In book printing, these characters occupy
only a little space and are set very close to the preceding letter.
Therefore, entering \enquote*{\wi{ellipsis}} by just typing three
dots would produce the wrong result. Instead, there is a special
command for these dots. It is called
\begin{lscommand}
  \csi{ldots} (low dots)\index{...@\ldots}
\end{lscommand}

\begin{chktexignore}
  \begin{example}
Not like this ...
but like this:\\
New York, Tokyo,
Budapest, \ldots
\end{example}
\end{chktexignore}

\section{Ligatures}

Some letter combinations are typeset not just by setting the
different letters one after the other, but by actually using special
symbols.
\begin{code}
{\large ff fi fl ffi \ldots}\quad
instead of\quad {\large f\mbox{}f f\mbox{}i f\mbox{}l f\mbox{}f\mbox{}i \ldots}
\end{code}
These so-called \wi{ligature}s can be prohibited by inserting an \mintinline{latex}|\mbox{}|
between the two letters in question. This might be necessary with
words built from two words.

\begin{example}
\Large %!hide
Not shelfful\\
but shelf\mbox{}ful
\end{example}

\section{Abstract}

In scientific publications, it is customary to start with an abstract which
gives the reader a quick overview of what to expect. \LaTeX{} provides the
\ei{abstract} environment for this purpose. Normally \ei{abstract} is used
in documents typeset with the article document class.

\begin{example}[standalone, paperheight=3.5cm]
\begin{document}%!hide
\begin{abstract}
  This paper will talk about
  abstracts.
\end{abstract}

Abstracts are very important \ldots
\end{document}%!hide
\end{example}

\section{International Language Support}\label{sec:polyglossia}%
\index{international}
\secby{Axel Kielhorn}{A.Kielhorn@web.de}

When you write documents in \wi{language}s other than English, there are three
areas where \LaTeX{} has to be configured appropriately:

\begin{enumerate}
  \item All automatically generated text strings (\enquote{Table of Contents},
        \enquote{List of Figures}, \ldots) have to be adapted to the new
        language.
  \item \LaTeX{} needs to know the hyphenation rules for the current language.
  \item Language-specific typographic rules. For example, in French there is a
        mandatory space before each colon character (:).
\end{enumerate}

The package \pai*{polyglossia} is a replacement for the venerable \pai{babel}
package. It takes care of the hyphenation patterns and automatically-generated
text strings in your documents. Polyglossia works only with the \hologo{XeTeX}
and \hologo{LuaTeX} engines, so if you are using \hologo{pdfTeX} you'll have to
stick with \pai{babel}.

\subsection{Entering Characters}

Early versions of \TeX{} were only intended to support the English language,
and thus assumed that source files were ASCII encoded; that is, in a format
that only supports the Latin alphabet plus few symbols common on computers. With
the popularization of \LaTeX{} outside the English-speaking world, the situation
has slowly improved. These days \TeX{} engines speak UTF-8 natively, which
considerably eases the usage of \LaTeX{} when typesetting non-English
documents. Thus, inputting non-English characters is as easy as:
\begin{chktexignore}
\begin{example}
  Hôtel, naïve, élève, \\
  smørrebrød, ¡Señorita!, \\
  Schönbrunner Schloß, Żółć
\end{example}
\end{chktexignore}

If your keyboard does not have all the special characters, you can also enter
\wi{accent}s and \wi{special character}s via dedicated commands.
The commands for adding accents to characters are listed in \autoref{accents}.
Keep in mind that while older \TeX{} engines worked by combining accents with existing
letters manually, nowadays they simply look up the correct glyph in the font.
This means that accents and characters can no longer be freely combined, and
output is produced only if the font supports the combination. The Example above would
look like this if entered on an english keyboard:
\begin{chktexignore}
\begin{example}
  H\^otel, na\"\{i}ve, \'el\`ve, \\
  sm\{o}rrebr\{o}d, !`Se\~norita!, \\
  Sch\"onbrunner Schlo\ss, \.Z\'o\l\'c
\end{example}
\end{chktexignore}

\begin{table}
  \caption{Accents and Special Characters.}\label{accents}
  \begin{tabular}{@{}*3{ll@{\qquad}}ll@{}}
    \toprule
    Code         & Result       & Code          & Result       & Code & Result    & Code & Result \\
    \midrule
    \mstA{\`o}   & \mstA{\'o}   & \mstA{\^o}    & \mstA{\~o}                                      \\
    \mstA{\=o}   & \mstA{\.o}   & \mstA{\"o}    & \mstB{\c}{c}                                    \\[6pt]
    \mstB{\u}{o} & \mstB{\v}{o} & \mstB{\H}{o}  & \mstB{\k}{a}                                    \\
    \mstB{\d}{o} & \mstB{\b}{o} & \mstB{\t}{oo} & \mstB{\r}{o}                                    \\[6pt]
    \mstA{\oe}   & \mstA{\OE}   & \mstA{\ae}    & \mstA{\AE}                                      \\
    \mstA{\aa}   & \mstA{\AA}   &               &              &      &                           \\[6pt]
    \mstA{\o}    & \mstA{\O}    & \mstA{\l}     & \mstA{\L}                                       \\
    \mstA{\i}    & \mstA{\j}    & !`            & \verb|!`|    & ?`   & \verb|?`|                 \\ %chktex 26
    \bottomrule
  \end{tabular}%
  \index{dotless \i{} and \j}\index{Scandinavian letters}%
  \index{ae@\ae}\index{umlaut}\index{grave}\index{acute}%
  \index{oe@\oe}\index{aa@\aa}
\end{table}

The Unicode standard defines over \num{100 000} glyphs for use across many
languages and specialized disciplines. Currently, no font implements the whole
set, so if you are working with a lot of non-Latin characters you will, sooner
or later, find one that is not supported by your currently selected font. In
such cases, \LaTeX{} will not typeset the requested character, and prints a
\enquote{Missing character} warning in its log. Because such warnings are easy
to overlook, we recommend you put
\begin{minted}{latex}
\tracinglostchars=3
\end{minted}
somewhere in the preamble of the document. This command will elevate the
warnings to error status, so they will be much harder to miss.

\subsection{Polyglossia Usage}

Writing in different languages is easy. Just load the \pai{polyglossia} package
and specify the languages in the preamble using
\begin{lscommand}
  \csi{setdefaultlanguage}[options: o, language: m] \\
  \csi{setotherlanguage}[options: o, language: m]
\end{lscommand}
where \carg{language} is either the name of the language to use, such as
\cargv{gaelic} or \cargv{japanese}, or its BCP-47 tag such as \cargv{de-CH} or
\cargv{ru-luna1918}. When specifying language via its name you may pass
additional \carg{options} to decide between its variants, for example
\cargv{variant=british} or \cargv{script=Arabic}. Note that when loading a
language, either via tag or with options, all variants are loaded and the
tag\slash{}options specified are just the defaults. For a full list of
supported languages and variants, see the \pai{polyglossia} package
documentation.

To write a paragraph in German, you can use the \ei{german} environment:

\begin{example}
%!showbegin !hide
% In the preamble
\setdefaultlanguage{english}
\setotherlanguage{german}
% ...
%!showend !hide
\year=3022 \month=1 \day=19 %!hide
Today is not \today.

\begin{german}
  Heute ist nicht \today.
\end{german}
\end{example}

You can also use \ei{lang} environment, which accepts the \carg{language} as its
first argument. This is especially useful if you prefer to specify the language
via a BCP-47 tag, since environment names cannot contain the
\enquote*{\texttt{-}} symbol.
\begin{example}
\year=3022 \month=1 \day=19 %!hide
\begin{lang}{de-AT}
  Heute ist nicht \today.
\end{lang}
\end{example}

You can also pass \carg{options} to the environment.
\begin{example}
\year=3023 \month=8 \day=17 %!hide
\begin{german}[
  script=blackletter
]
  Heute ist nicht \today.
\end{german}
\end{example}

If you just need a word or short phrase in a foreign language, you can use
either the
\begin{lscommand}
  \csi{text«\bs carg*{language}»}[options:o, text:m]
\end{lscommand}
or the
\begin{lscommand}
  \csi{textlang}[options:o, language:m, text:m]
\end{lscommand}
command:
\begin{example}[examplewidth=0.8\linewidth, vertical_mode]
In Austria they write
\textlang[variant=austrian]{de}{Jänner}
instead of \textgerman{Januar}. And in the olden
days \textlang{de}{August} was written
\textgerman[script=blackletter]{Auguſt}.
\end{example}

For languages that are not very different from English, the results are not
very impressive. We get correct hyphenation, and some context-aware commands,
such as \csi{today}, adjust their output. However, the further we stray from
English, the more useful these commands become.

Sometimes the font used in the main document does not contain glyphs required
by the second language. The default Latin Modern font, for example, does
not contain Cyrillic letters. The solution is to define a custom font that will
be used for that language.

To set the fonts, use the preamble command
\begin{lscommand}
  \csi{newfontfamily}[familyname:M, options:o, font:m]
\end{lscommand}
from the \pai{fontspec} package,\footnote{\pai{fontspec} is loaded automatically
  by \pai{polyglossia}, so you don't need to add it to the preamble of your
  document.} which is described in much more detail in
\autoref{sec:fontspec}.
When changing the text language, \pai{polyglossia} checks whether a font family
named \cargv{\textit{language}font} exists, and switches to that if it's
available.

If you are happy with the default Latin Modern font, you may want to try the
\enquote{CMU} font, which contains some additional Cyrillic or Greek glyphs
while looking nearly identical to Latin Modern.\footnote{Both fonts are
  actually based on the Computer Modern font, designed by Donald Knuth for the
  first \TeX{} versions.}
If you have the CMU font installed in your system (unlikely), or if you are
using \hologo{LuaLaTeX}, then you can simply write
\begin{minted}{latex}
  \newfontfamily\greekfont{CMU Serif}
\end{minted}
If you are using \hologo{XeLaTeX} and don't have the font installed, then you
need to specify the font by its file name.
\begin{minted}{latex}
  \newfontfamily\greekfont[
    Extension=.otf, UprightFont=*rm, ItalicFont=*ti,
    BoldFont=*bx, BoldItalicFont=*bi,
  ]{Cm}
\end{minted}

With the appropriate fonts loaded, you can now write:
\begin{example}[examplewidth=0.8\linewidth, vertical_mode,noextend]
%!showbegin !hide
% In the preamble (LuaLaTeX)
\setotherlanguage{greek}
\newfontfamily\greekfont{CMU Serif}
\setotherlanguage{russian}
\newfontfamily\russianfont{CMU Serif}
% ...
%!showend !hide

\begin{chktexignore} %!hide
\textrussian{Правда} is a Russian newspaper. %!hide
\textlang{greek}{ἀλήθεια} is truth or disclosure in %!hide
%!showbegin %!hide
\textrussian{«\fontspec{cmuntt.otf}Правда»} is a Russian newspaper.
\textlang{greek}{«\fontspec{cmuntt.otf}ἀλήθεια»} is truth or disclosure in
%!showend %!hide
philosophy.
\end{chktexignore} %!hide
\end{example}

\LaTeX{} actually uses three different fonts for typesetting documents. The
above commands only set the default serif font used for the body text. The
other fonts are \textsf{sans serif} (used in presentation slides) and
\texttt{monospace} (used when displaying code). In order to adjust the fonts
used for a particular language, define the \cargv{\textit{language}fontsf}
family for sans serif, and \cargv{\textit{language}fonttt} for monospace. For
example, to define CMU as the font for Greek in all three fonts when using
\hologo{LuaLaTeX}, you would write
\begin{minted}{latex}
  \newfontfamily\greekfont{CMU Serif}
  \newfontfamily\greekfontsf{CMU Sans Serif}
  \newfontfamily\greekfonttt{CMU Typewriter Text}
\end{minted}
\pagebreak[3]
or in \hologo{XeLaTeX}
\begin{minted}{latex}
  \newfontfamily\greekfont[
    Extension=.otf, UprightFont=*rm, ItalicFont=*ti,
    BoldFont=*bx, BoldItalicFont=*bi,
  ]{cmun}
  \newfontfamily\greekfontsf[
    Extension=.otf, UprightFont=*ss, ItalicFont=*si,
    BoldFont=*sx, BoldItalicFont=*so,
  ]{cmun}
  \newfontfamily\greekfonttt[
    Extension=.otf, UprightFont=*btl, ItalicFont=*bto,
    BoldFont=*tb, BoldItalicFont=*tx,
  ]{cmun}
\end{minted}

The commands above redefine the fonts for languages used as \enquote*{other} in
polyglossia. If you want to influence the font of the main document language,
use \csi{setmainfont}, \csi{setsansfont} and \csi{setmonofont}. These work the
same way, but without the \carg{familyname} argument.
\begin{minted}{latex}
  \setmainfont{CMU Serif}
  \setsansfont{CMU Sans Serif}
  \setmonofont{CMU Typewriter Text}
\end{minted}

If you want to learn even more about fonts, have a look at
\autoref{sec:fontspec}.

\subsection{Right to Left (RTL) languages}

Some languages are written left to right, others are written right to left
(RTL). In order to support RTL languages, \pai{polyglossia} needs the
  \pai*{bidi} package.\footnote{\pai{bidi}
  supports only the \hologo{XeTeX} engine. If you use \hologo{LuaTeX},
  \pai{polyglossia} uses the \pai*{luabidi} package, but keep in mind that it
  is much more limited than \pai{bidi}.}
The \pai{bidi} package should be the last package you load; even after
\pai{hyperref}, which is otherwise usually the last package. (Since
\pai{polyglossia}
loads \pai{bidi}, this means that \pai{polyglossia} should be the last package
loaded.)

To typeset Arabic, you can use the Iran Nastaliq font~\cite{font:IranNastaliq}
developed by the SCICT and updated by Mohammad Saleh Souzanchi.
\begin{example}[standalone, template=empty, paperwidth=0.35\linewidth, paperheight=4cm]
\documentclass{article}

\usepackage{polyglossia}
\setdefaultlanguage{arabic}
%!hidebegin
\usepackage[paperheight=\height,paperwidth=\width,margin=1cm,includefoot]{geometry}
\defaultfontfeatures[Iran Nastaliq] {
  Path=src/font-IranNastaliq/WebFonts/,
  Extension=.ttf,
  UprightFont=IranNastaliq-Web,
  Script=Arabic,
}
%!hideend
\newfontfamily\arabicfont{Iran Nastaliq}

\begin{document}
%!showbegin !hide
«\textarabic{هذا نص عربي}»
%!showend !hide
هذا نص عربي % !hide
\end{document}
\end{example}

The package \pai*{xepersian}\footnote{Works only with
  \hologo{XeLaTeX}.}\index{Persian} offers support for the Persian language. It
supplies Persian \LaTeX-commands that allows you to enter commands like
\csi{section} in Persian, which makes this really attractive to native speakers.
\pai{xepersian} is the only package that supports kashida\index{kashida} with
\hologo{XeLaTeX}. A package for Syriac, which uses a similar algorithm, is under
development.

The \pai*{arabxetex} package supports several languages with
an Arabic script:
\begin{itemize}
  \item arab (Arabic)\index{Arabic}
  \item persian\index{Persian}
  \item urdu\index{Urdu}
  \item sindhi\index{Sindhi}
  \item pashto\index{Pashto}
  \item ottoman (turk)\index{Ottoman}\index{Turkish}
  \item kurdish\index{Kurdish}
  \item kashmiri\index{Kashmiri}
  \item malay (jawi)\index{Malay}\index{Jawi}
  \item uighur\index{Uighur}
\end{itemize}
It offers a font mapping that enables \hologo{XeLaTeX} to process input
using the Arab\TeX{} ASCII transcription.

There is no package available for Hebrew\index{Hebrew}, because none is needed.
The Hebrew support in \pai{polyglossia} should be sufficient, but you do need a
suitable font with real Unicode Hebrew. An extension to the default Latin
Modern font, called
\citetitle{font:NewComputerModern}~\cite{font:NewComputerModern}, adds, among
other things, a full Hebrew character set. It's distributed with \TeXLive, so
there is a good chance you already have it. Another Hebrew font, available under
the SIL Open Font License, is Ezra SIL~\cite{font:ezrasil}.

\subsection{Chinese, Japanese and Korean (CJK)}%
\index{Chinese}\index{Japanese}\index{Korean}

The package \pai*{xeCJK}\footnote{Works only with
  \hologo{XeLaTeX}.} takes care of font selection and
punctuation for these languages.

\section{Simple Commands}\label{sec:simple_commands}

You may find yourself retyping the same name\slash{}logo\slash{}complicated text
in multiple places of your documents. This is tedious and bears the risk of
introducing typos. Brand logos also often have special casing rules, such as
all uppercase or starting with a lowercase letter even at the beginning of the
sentence.

We have already seen a solution for this problem---commands. For example, to
refer to \LaTeX{}, which uses a rather complicated logo within text, you don't
need to insert pictures or manually space letters---just use the \csi{LaTeX}
command. In \LaTeX{}, you can define your own commands to replace
complicated sequences producing given text.

For example, let's imagine that you are writing an article about Polish
notation. In doing so, you may have to refer to the the inventor of the
notation, Jan Łukasiewicz. His last name may not be easy to remember or
type for someone who does not speak Polish. It also starts with letter
\enquote*{Ł}, which may not be easily accessible on your keyboard, so
you will have to use \csi{L}, making the code less readable.

\LaTeX{} enables you to define your own commands using the
\begin{lscommand}
  \csi{NewDocumentCommand}[name:M, {{}}:c, definition:m]
\end{lscommand}
command. The \carg{name} is the name of the command that you will use in the
code, while the \carg{definition} is the output it will produce. The second
argument is left empty for now, but will be explored in
\autoref{sec:new_commands}.
\begin{example}
\NewDocumentCommand{\lukas}{}{%
  Łukasiewicz%
}

Polish notation (also known as
\lukas{} notation, due to its
inventor, Jan \lukas) is a
mathematical notation \ldots{}
\end{example}
In the example above, we have used \ai{\%} comments to avoid introducing spaces.
If you need a refresher on this technique, refer to \autoref{sec:comments}.

%$ I am missing the commercial minus glyph (three below)
While the \enquote*{Ł} character is available in \LaTeX{} from the \csi{L}
command, this is not the case for all unicode characters. For example, the
default font contains a glyph for the commercial minus sign (⁒), used in some
German speaking countries, but \LaTex{} does not define a command to enter it.
If you want to avoid continuously copy-pasting it into your code, you can set
up a command to typeset it for you.
\begin{example}
\NewDocumentCommand{%
  \comminus}{}{⁒}

Now I can type ⁒ or \comminus.
\end{example}
%$ I am missing the commercial minus glyph (three above)

Yet another use is to define a command for a simple string you may
want to change later. For example, let's imagine you are writing a novel, but
are not so sure about your protagonist's name. You can define a command for it,
and in the event you change your mind, you need only modify it
in one place instead of hunting down every single instance
throughout the book.
\begin{example}
\NewDocumentCommand{%
  \hero}{}{Launcelot}

\hero{} unsheathed his sword.
He  had a bad feeling about
this place from the start.
\end{example}

The \csi{NewDocumentCommand} checks if a command with the given name is already
defined, and raises an error if it is. This means you can freely use it
to define your own commands without worrying about breaking internal
commands used by \LaTeX{} or its packages. But sometimes, redefining existing
commands is actually what you want. For example, many packages allow
customisation in this way. In this, case you should use
\csi{RenewDocumentCommand}. It works the same way as the former command, but
raises an error if a command with the given name is \emph{not} already defined.
\begin{example}
\RenewDocumentCommand{%
  \ldots}{}{...}

I like my ellipsis looking
uglier \ldots
\end{example}

These commands can be used with nearly all features of \LaTeX{}. Although
we have only dealt with text here, the same commands may be used to insert
pictures or mathematical expressions. They are especially useful in the latter
case, since mathematical symbols are often repurposed to mean different things,
and consistency of their usage is very important.

Note that these commands are actually much more powerful than described here.
Their full capabilities are discussed in \autoref{sec:new_commands}.

\section{The Space Between Words}

To produce a straight right-margin in its output, \LaTeX{} inserts varying
amounts of space between words. It inserts slightly more space at
the end of a sentence, as this makes the text more readable.  \LaTeX{}
assumes that sentences end with periods, question marks or exclamation
marks. If a period follows an uppercase letter, this is not taken as the end
of a sentence, since periods after uppercase letters normally occur in
abbreviations.

Any exception from these assumptions has to be specified by the author. A
backslash in front of a space generates a space that will not be enlarged. A
tilde~\enquote*{\ai{\~}} character generates a non-breaking space (prohibits a
line break). The command \csi{@} in front of a period specifies that this
period terminates a sentence even when it follows an uppercase letter.
\csih{@}\index{tilde@tilde (\verb.~.)}% chktex 18
\index{., space after}


\begin{example}
Mr.~Smith was happy to see her\\
cf.~Fig.~5\\
I like BASIC\@. What about you?
\end{example}

The additional space after periods can be disabled with the command
\begin{lscommand}
  \csi{frenchspacing}
\end{lscommand}
which tells \LaTeX{} \emph{not} to insert more space after a period than after
an ordinary character. This is very common in non-English languages
(\pai{polyglossia} sets it automatically, based on the main language).

\section{Titles, Chapters, and Sections}

To help the reader find his or her way through your work, you should
divide it into chapters, sections, and subsections.  \LaTeX{} supports
this with special commands that take the section title as their
argument.  It is up to you to use them in the correct order.

The following sectioning commands are available for the
\texttt{article} class: \nopagebreak

\begin{lscommand}
  \csi{section}[title: m]\\
  \csi{subsection}[title: m]\\
  \csi{subsubsection}[title: m]\\
  \csi{paragraph}[title: m]\\
  \csi{subparagraph}[title: m]
\end{lscommand}

If you want to split your document into parts without influencing the
section or chapter numbering use
\begin{lscommand}
  \csi{part}[title: m]
\end{lscommand}

When you work with the \texttt{report} or \texttt{book} class,
an additional top-level sectioning command becomes available:
\begin{lscommand}
  \csi{chapter}[title: m]
\end{lscommand}

As the \texttt{article} class does not know about chapters, it is quite easy
to add articles as chapters to a book.
The spacing between sections, the numbering, and the font size of the
titles will be set automatically by \LaTeX.

\LaTeX{} creates a table of contents by taking the section headings
and page numbers from the last compile cycle of the document. The command
\begin{lscommand}
  \csi{tableofcontents}
\end{lscommand}
\noindent expands to a table of contents at the place it is issued. A new
document has to be compiled twice to get a correct \wi{table of contents}.
Sometimes it might be necessary to compile the document a third time. \LaTeX{}
will tell you when this is necessary.

All sectioning commands listed above also exist as starred versions. This
generates section headings that do not show up in the table of contents and are
not numbered. The command \mintinline{latex}|\section{Help}|, for example,
would become \mintinline{latex}|\section*{Help}|.

Normally, the section headings show up in the table of contents exactly
as they are entered in the text. Sometimes this is not possible,
because the heading is too long to fit into the table. The
entry for the table of contents can then be specified as an
optional argument in front of the actual heading.

\begin{example}[standalone, paperwidth=6cm,paperheight=5cm]
\begin{document}%!hide
\tableofcontents

\section[Title for the
  table of contents]{A long
  and especially boring title,
  shown in the text}
\end{document}%!hide
\end{example}

The \wi{title} of the whole document is generated by issuing a
\begin{lscommand}
  \csi{maketitle}
\end{lscommand}
command. The contents of the title have to be defined by the commands
\begin{lscommand}
  \csi{title}[title: m], \csi{author}[author: m]
  and optionally \csi{date}[date: m]
\end{lscommand}
before calling \mintinline{latex}|\maketitle|. In the argument to \csi{author},
you can supply several names separated by \csi{and} commands.

An example of some of the commands mentioned above can be found in
\autoref{document} on \autopageref{document}.

Apart from the sectioning commands explained above, \LaTeX{} defines four
additional commands for use with the \texttt{book} class. They are useful for
dividing your publication. The commands alter chapter headings and page
numbering to work as you would expect in a book:
\begin{description}
  \item[\csi{frontmatter}] should be the very first command after the start of
    the document body
    \begin{chktexignore}
  (\mintinline{latex}|\begin{document}|).
\end{chktexignore}
    It will switch page
    numbering to Roman numerals and sections will be non-enumerated as if you were
    using the starred sectioning commands (for example
    \mintinline{latex}|\chapter*{Preface}|) but the sections will still show up in
    the table of contents.
  \item[\csi{mainmatter}] comes right before the first chapter of
    the book. It turns on Arabic page numbering and restarts the page
    counter.
  \item[\csi{appendix}] marks the start of additional material in your book.
    After this command chapters will be numbered with letters. Unlike others
    this method can also be used inside articles. In this case, it changes the
    numbering of sections.
  \item[\csi{backmatter}] should be inserted before the very last items
    in your book, such as the bibliography and the index. In the standard
    document classes, this has no visual effect.
\end{description}

\section{Cross References}\label{sec:crossref}

In books, reports and articles, there are often
\wi{cross-references} to figures, tables and special segments of text.
\LaTeX{} provides the following commands for cross referencing
\begin{lscommand}
  \csi{label}[marker: m], \csi{ref}[marker: m]
  and \csi{pageref}[marker: m]
\end{lscommand}
where \carg{marker} is an identifier chosen by the user. \LaTeX{} replaces
\csi{ref} by the number of the section, subsection, figure, table, or theorem
after which the corresponding \csi{label} command was issued. \csi{pageref}
prints the page number of the page where the \csi{label} command
occurred.\footnote{Note that these commands are not aware of what they refer
  to. \csi{label} just saves the last automatically generated number.} As with
section titles and page numbers for the table of contents, the numbers from the
previous compile cycle are used.

\begin{example}[examplewidth=0.4\textwidth]
A reference to this subsection%
\label{sec:this} looks like:
\enquote{see Section~\ref{sec:this}
on page~\pageref{sec:this}.}
\end{example}

\section{Footnotes}
With the command
\begin{lscommand}
  \csi{footnote}[footnote text: m]
\end{lscommand}
a footnote is printed at the foot of the current page.  Footnotes should always
be put after the word or sentence they refer to. Footnotes referring to a
sentence or part of it should therefore be put after the comma or period.

\begin{example}
Footnotes\footnote{This is
  a footnote.} are often used
by people using \LaTeX.
\end{example}

\section{Lists}

The \ei{itemize} environment is suitable for simple lists, the
\ei{enumerate} environment for enumerated lists, and the
\ei{description} environment for descriptions.
\csih{item}

\begin{example}
\begin{enumerate}
\item You can nest the list
environments to your taste:
\begin{itemize}
\item But it might start to
look silly.
\item[-] With a dash.
\end{itemize}
\item Therefore remember:
\begin{description}
\item[Stupid] things will not
become smart because they are
in a list.
\item[Smart] things, though,
can be presented beautifully
in a list.
\end{description}
\end{enumerate}
\end{example}

\section{Non-Justified Text}\label{sec:ragged}

By default, \LaTeX{} justifies text inside a document. While this is desirable
for the main text there may be occasions where you want the text to be left- or
\wi{right-aligned}\index{left-aligned}. This is where the \pai*{ragged2e}
package is useful. It defines environments
\begin{lscommand}
  \ei{FlushLeft} \\
  \ei{FlushRight} \\
  \ei{Center}
\end{lscommand}
which allow you to achieve left-aligned, right-aligned and centred text.

\begin{example}
\begin{FlushLeft}
  This text is\\ left-aligned.
  \LaTeX{} is not trying to make
  each line the same length.
\end{FlushLeft}
\end{example}

\begin{example}
\begin{FlushRight}
  This text is right-\\aligned.
  \LaTeX{} is not trying to make
  each line the same length.
\end{FlushRight}
\end{example}

\begin{example}
\begin{Center}
  At the centre\\of the earth
\end{Center}
\end{example}

\section{Quotations}\label{sec:csquotes}

You should \emph{not} use the \enquote*{\texttt{"}} character for %chktex 18
\wi{quotation marks}\index{""@\texttt{""}} as you would in a WYSIWYG program.
In publishing, there are special opening and closing
quotation marks. In \LaTeX{}, the \pai*{csquotes} package provides the
\begin{lscommand}
  \csi{enquote}[text: m]
\end{lscommand}
command that automatically encloses the \carg{text} within the right
quotes.
\begin{example}
\enquote{Please press
  the \enquote{x} key.}
\end{example}

If you want to skip directly to the inner form of quotation, use the starred
version of the command:
\begin{example}
The \enquote*{x} key is here.
\end{example}

\subsection{Formal Quotes}

The above commands are useful when you want to use, for example, scare quotes.
If you are actually quoting someone else, it is better to use the
\csi{textquote} command.
It accepts an optional argument with the source of the quote.
\begin{example}
\textquote[A.~Einstein]{Why
  is it nobody understands
  me and everybody likes me?}
\end{example}

For longer quotations you may want to use the \ei{displayquote} environment.
\begin{example}[examplewidth=0.5\linewidth]
As Einstein once said
\begin{displayquote}[%
  Albert Einstein]
  Whoever undertakes to
  set himself up as a judge
  of Truth and Knowledge is
  shipwrecked by the laughter
  of the gods.
\end{displayquote}
\end{example}

In academic writing, there may be strict rules present when quotes should be
typeset inline, and when they should be typeset in display style, based on their
length. This can be achieved automatically with the \csi{blockquote} command.
It will scan the text inside and typeset the quote accordingly.
\begin{example}[examplewidth=0.5\textwidth]
\csdisplaytrue%!hide
\blockquote[Me]{A short quote}

\blockquote[Also Me]{A very long
  quote that spans over multiple
  lines and should be typeset in
  display quote style
  according to my publisher.}
\end{example}
By default, \csi{blockquote} will switch to display quote style if the quote
spans more than three lines or contains more than one paragraph. The number of
lines may be changed using the \cargv{threshold} package options. If you prefer
to count words instead of lines, you may change the \cargv{thresholdtype} from
lines to words and set \cargv{threshold} accordingly. If you want to ignore
paragraphs, and rely only on automatic line\slash{}word counting, set the
\cargv{parthreshold} to false.

The quote attribution shown in the previous examples is especially useful when
combined with bibliography commands described in
\autoref{chap:bibliography}.

\subsection{Foreign Quotes}

The \pai{csquotes} package has excellent integration with \pai{polyglossia}.
Thus, simply changing the document's main language automatically adapts the
quotation marks used.
\begin{example}
%!showbegin !hide
% In preamble
\setmainlanguage{french}
% ...
%!showend !hide
\setquotestyle{french} %!hide

\textquote[Antoine de
  Saint Exupéry]{Toute
  nation est égoïste.
  Toute nation considère
  son égoïsme comme sacré.}
\end{example}

If you load the package with the \cargv{autostyle} option, it will also adapt
the quotation marks to the language of the surrounding text.
\begin{example}
%!showbegin !hide
% In preamble
\usepackage[autostyle]{csquotes}
% ...
%!showend !hide

\enquote{English quote}

\begin{german}
  \enquote{Deutsches zitat}
\end{german}
\end{example}

If frequently quoting in foreign languages, you may get tired of first
switching to a given language with a polyglossia command, and then using
\csi{textquote}. In order to avoid this, \pai{csquotes} defines an additional
\csi{foreignquote} command, which takes the language as its first language.
\begingroup
\setmonofont{cmuntt.otf}
\begin{example}
As Dostoevsky once wrote
\foreignquote{russian}{%
  Мир спасёт красота}.
\end{example}
\endgroup
The commands \csi{foreigntextquote} and \csi{foreignblockquote}, and the
environment\ei{foreigndisplayquote}, are also available, and work like those
described in the previous section.

Some languages use multiple variants of quotes. To change the variant
used by the given language, add option \cargv{\carg{language}=\carg{variant}}
when loading the \pai{csquotes} package.
\begin{example}
%!showbegin !hide
% In preamble
\usepackage[
  autostyle,
  german=guillemets,
]{csquotes}
% ...
%!showend !hide

\setquotestyle[guillemets]{german} %!hide
\foreignquote{german}{%
  Deutsches zitat}
\end{example}

\subsection{Long Quotations and Poetry}
\LaTeX{} provides two additional environments: \ei{quotation} and \ei{verse}.
The \texttt{quotation} environment is useful for
longer quotes running over several paragraphs, because it indents the first line
of each paragraph. The \texttt{verse} environment is useful for poems where the
line breaks are important. The lines are separated by issuing a \csi{\bs} at the
end of a line, and an empty line after each verse. Note that these environments
are not defined by \pai{csquotes}, so they do not accept the optional
attribution argument, and are not context sensitive.

\begin{example}
I wanted to quote my
favourite speech:
\begin{quotation}
  This is a very
  long speech.

  It spans over multiple
  paragraphs.
\end{quotation}
\end{example}

\begin{example}
I know only one English
poem by heart. It is
about Humpty Dumpty.
\begin{verse}
  Humpty Dumpty sat
  on a wall:\\
  Humpty Dumpty had
  a great fall.\\
  All the King's horses
  and all the King's men\\
  Couldn't put Humpty
  together again.
\end{verse}
\end{example}

\section{Code Listings}\label{sec:code_listings}

When writing about \LaTeX{}, or other programming languages, you often need to
insert short code snippets like this \verb|\LaTeX{}|. While you could escape
all the characters, this would quickly become rather tiresome, especially for
longer pieces of code where correct spacing is crucial for readability
(multiple spaces being collapsed by \LaTeX{} to singles).
We will present three solutions to this problem.

\subsection{Verbatim}\label{sec:verbatim}

\LaTeX{} itself comes with the \csi{verb} command. It is unusual in that it does
not use groups for getting its first argument. Instead, you pass the argument
between a chosen delimiter that can be any character except a letter, \verb|*|
or space. For example, to use \cargv{|} as the delimiter (as is typical),
you type
\begin{lscommand}
  \csi{verb}[|«\bs carg{text}»|: c]
\end{lscommand}
Any text that is enclosed between these delimiters will be directly printed, as
if typed on an old typewriter (in a monospaced font), with all spaces, and without any \LaTeX{} commands
being executed.
\begin{example}
Use \verb|\LaTeX{}| to
print \LaTeX.

We normally use \verb+|+ but
when it's not available we
use \verb|+| to delimit the
argument of the \verb|\verb|
command.
\end{example}

The starred version of the \csi{verb} command replaces spaces with the
\enquote*{\textvisiblespace{}} symbol.

\begin{chktexignore}
\begin{example}
\verb*|\TeX user| will
produce \TeX user.

To get proper spacing use
\verb*|\TeX{} user|.

Several spaces are treated
by \LaTeX{} as one space,
so \verb*|a     b| will
produce a     b.
\end{example}
\end{chktexignore}

For longer text, you may want to use the \ei{verbatim} environment from the
\pai*{verbatim} package.
\begin{example}
\begin{verbatim}
\documentclass{article}

\begin{document}
Small is beautiful.
\end{document}
\end{verbatim}
\end{example}
Its starred version will draw the \enquote*{\textvisiblespace}
symbol instead of spaces.

For code snippets, it may be useful to store them in separate files,
outside of the \LaTeX{} document. This allows you to edit them in your editor of
choice, and makes the \LaTeX{} code less cluttered. To accomplish this, use the
\csi{verbatiminput}[file: m] command from the \pai{verbatim} package.\footnote{This
  and the following examples use code snippets that can be found at \url{https://github.com/oetiker/lshort/tree/master/src/examples}.}
\begin{example}[examplewidth=0.55\linewidth, vertical_mode]
\verbatiminput{hello.c}
\end{example}
\csi{verbatiminput*} also exists and works exactly as you would predict.

The \texttt{verbatim} environment, and the \csi{verb} and \csi{verbatiminput}
commands, may not be used within parameters of other commands, so the below code
will result in an error.
\begin{minted}{latex}
Text\footnote{\verb|abc| cannot be used here.}
\end{minted}
If this is a problem for you, look at the \pai*{fancyvrb} package.

\subsection{The \ei{listings} Package}

While the \ei{verbatim} environment is fine for small snippets, they are very
crude and look basic. In order to make them a bit more fancy, you can use the
\pai*{listings} package. It defines the
\begin{lscommand}
  \csi{lstinline} command that works like \csi{verb} \\
  \ei{lstlisting} environment that works like \ei{verbatim} \\
  \csi{lstinputlisting} command that works like \csi{verbatiminput}
\end{lscommand}
\begin{example}[examplewidth=0.6\linewidth, vertical_mode]
\lstinline|\LaTeX{}|

\begin{lstlisting}
Here is some \LaTeX{} code.
\end{lstlisting}

\lstinputlisting{hello.c}
\end{example}
%$ There are really three separate examples here. I guess they don't quite fit
%$ side-by-side, but the equivalent example in the minted section manages it.
%$ The side-by-side layout makes it easier to visually associate each of the
%$ three inputs with their respective outputs.

%$ See commit notes. Short version is that this is in a serif font, and
%$ maybe the \ttfamily setting should be made now rather than much later.
As you can see, by default it isn't much different from the \ei{verbatim
  environment}. The big difference is that every command accepts optional
argument that allows you to customise the output. The argument accepts a
comma-delimited key-value list. For example, to automatically highlight the
code, we may pass the \cargv{language} key.
\begin{example}[examplewidth=0.6\linewidth, vertical_mode]
\lstinputlisting[language=C]{hello.c}
\end{example}
If a language has several dialects, you may specify which inside square brackets
before the language name. You must enclose the value in curly brackets.
\begin{example}[examplewidth=0.8\linewidth, vertical_mode]
\lstinputlisting[language={[LaTeX]TeX}]{hello.tex}
\end{example}
When presenting code, it is often that we want to point to a specific line. To
save the reader from counting which line we mean, it is possible to print line
numbers by using the \cargv{numbers} key. Its possible values are \cargv{none},
\cargv{left} and \cargv{right}.
\begin{example}[examplewidth=0.8\linewidth, vertical_mode]
\lstset{xleftmargin=1.2em}%!hide
\lstinputlisting[language=C, numbers=left]{hello.c}
\end{example}
The appearance of numbers may be further customised with the following keys:
\begin{description}
  \item[\cargv{firstnumber}] allows you to specify the first number. Besides
    numbers, this key can also contain two special values: \cargv{last} and
    \cargv{auto} (the default). \cargv{last} continues numbering from the last
    listing, while \cargv{auto} continues it from the last listing with the
    same \cargv{name} argument, or starts over if no \cargv{name} is present.
  \item[\cargv{stepnumber}] prints only every \(n\)-th number. For
    example, if you pass \cargv{stepnumber=7}, then only line numbers \(1, 8,
    15, 22, \ldots\) will be printed.
  \item[\cargv{numberblanklines}] controls whether or not line numbers are
    printed on empty lines. Either \cargv{true} or \cargv{false}.
  \item[\cargv{numberstyle}] allows you to customise the font for printing
    numbers. It accepts switch commands, as described in \autoref{sec:fontsize}.
\end{description}
\begin{example}[examplewidth=0.61\linewidth]
\lstinputlisting[
  language=C,
  numbers=left,
  stepnumber=2,
  numberblanklines=false,
  firstnumber=4,
  xleftmargin=1.2em, %!hide
  numberstyle=\tiny,
]{hello.c}
\end{example}

If you have a long file and want to show it piecewise, you can use the
\cargv{firstline} and \cargv{lastline} keys.
\begin{example}[examplewidth=0.6\linewidth]
The \lstinline|main|
function body
consists of:
\lstinputlisting[
  language=C,
  firstline=5,
  lastline=6,
]{hello.c}
\end{example}

If you are trying to typeset long lines, you may order the \pai{listings} to
automatically break them by using the \cargv{breaklines} key.
\begin{example}[examplewidth=0.6\linewidth]
\lstinputlisting[
  language=python,
  numbers=left,
  xleftmargin=1.2em, %!hide
  breaklines,
]{factorial.py}
\end{example}
You can customise the indentation width with the \cargv{breakindent} key. It
may be useful to indicate that a line break has occurred by using the
\cargv{prebreak} and \cargv{postbreak} keys, which print their value
respectively before and after an artificial line break.
\autoref{lst:prepostbreak} shows an example of this.
\begin{listing}
  \begin{example}[examplewidth=0.6\linewidth, vertical_mode]
\lstinputlisting[
  language=python,
  numbers=left,
  xleftmargin=1.2em, %!hide
  breaklines,
  breakindent=1cm,
  prebreak=\textrightarrow,
  postbreak=\textleftarrow,
]{factorial.py}
\end{example}
  \caption{An example of marking the artificial linebreaks in \ei{listing}
    package.}\label{lst:prepostbreak}
\end{listing}

One of the neat features of the \pai{listings} package is that you can evaluate
\LaTeX{} code inside the listing. The easiest way is to pass the
\cargv{texcl} key, which enables \LaTeX{} syntax within comments.
\begin{example}
\begin{lstlisting}[
  texcl,
  language=haskell,
]
-- I can use \LaTeX{} here.
-- The cost is in \texteuro{}
cost x = show (
    foldr (+) 0 x
  ) ++ " eur"
\end{lstlisting}
\end{example}

This feature is especially useful in combination with the \csi{label} command.
We can use it to point to a specific line of code without hardcoding it into the
document.
\begin{example}
\begin{lstlisting}[
  texcl,
  numbers=left,
  xleftmargin=1.2em, %!hide
  language=haskell,
]
cost x = show (
    foldr (+) 0 x
  ) ++ " eur" -- \label{concat}
\end{lstlisting}
The \lstinline|++| in Haskell
as seen in line~\ref{concat}
means string concatenation.
\end{example}
%$ It's such a downside, consider combining these two examples. All you really
%$ need is the first example changed to use escapeinside.
This has the downside that it introduces empty comments that are just there to
include label. To fix this we may use the \cargv{escapeinside} key that accepts
sets the two delimiters between which \LaTeX{} code can be typed. The
delimiters themselves will not be printed.
\begin{example}
  \begin{lstlisting}[
  escapeinside={(*}{*)},
  numbers=left,
  xleftmargin=1.2em, %!hide
  language=haskell,
]
cost x = show ( (*\label{show}*)
    foldr (+) 0 x
  ) ++ " eur" (*(\texteuro)*)
\end{lstlisting}
%$ The looks strange, because the 'show' is syntax-highlighted in the listing,
%$ but not in the \lstinline (so they don't match).
The \lstinline|show| in Haskell
as seen in line~\ref{show}
converts its argument to a string.
\end{example}


%$ This should probably come first in the section, before worrying about
%$ cross-references, step-numbers, and dialects.
If you do not like the default style of listings, you may customise them using
the
\begin{lscommand}
  \cargv{basicstyle}, \cargv{keywordstyle}, \cargv{identifierstyle} and
  \cargv{commentstyle}
\end{lscommand}
keys. Note that you will need to use some font changing commands, which
you will learn about in \autoref{sec:fontsize}.
\begin{example}[examplewidth=0.55\linewidth]
\lstinputlisting[
  language=SQL,
  basicstyle=\ttfamily,
  commentstyle=\color{gray},
  keywordstyle=\itshape,
]{employee.sql}
\end{example}

If you intend to use similar options for many listings in the document, you can
use the \csi{lstset} command to set them beforehand. In this way, you will avoid
repeating them every time you want to typeset some code.

\begin{example}[examplewidth=0.55\linewidth]
\lstset{
  language=C,
  numbers=left,
  xleftmargin=1.2em, %!hide
  breaklines,
  basicstyle=\ttfamily,
  prebreak=\textrightarrow,
}

\lstinputlisting{hello.c}

The statement
\lstinline|int x = 1;|
assigns 1 to variable
\lstinline|x|.
\end{example}

\subsection{The \pai{minted} Package}

While the \pai{listings} package allows you to typeset pretty decent looking
code snippets, you may still find it lacking in some respects. The most obvious
problem is the syntax highlighting: you may find that your language of choice
is not supported. You can write a custom style for the language, but it relies
on a simple keyword search, so this is an option only for simple languages.

If this is a problem for you, it may be worth considering the \pai*{minted}
package. It relies on the Pygments~\cite{pygments} program to highlight code.
In contrast to the \pai{listings} package, this program uses pushdown automata
to colour the code, which may produce much better results. Moreover, the
predefined styles and language support are much richer than those offered by
the \pai{listings} package. This is the package that is used to colour examples
throughout this booklet.

Due to the fact that the \pai{minted} package relies on an external program,
there are two things that you need to do before using it:
\begin{enumerate}
  \item Ensure that the Pygments program is installed. If you are using a
        \Unix{} system it is probably enough to execute\\
        \verb|pip install Pygments|\\
        You can find more information at their official website:
        \citeurl{pygments} and in the \pai*{minted} package documentation.

  \item
        Add the \verb|--shell-escape| flag when the compiling \LaTeX{} document,
        for example:\\
        \verb|xelatex --shell-escape document.tex|
\end{enumerate}
It is important to understand what the cryptic flag mentioned in the second
point actually means. It allows the \LaTeX{} document to call any external
program via the shell. Let me repeat this to make the point clear:
\emph{Enabling this flag allows the \LaTeX{} document to call any program via
the shell}. This means that a maliciously crafted document could, for example,
delete all files from your computer. Only enable this flag if the document comes
from a trusted source.

Once you meet these requirements, you can use the package in a manner similar to
the previous approach with \pai{listings}. It provides
\begin{lscommand}
  \csi{mintinline}[options:o, language:m, code:m] \\
  \csi{inputminted}[options:o, language:m, code:m] \\
  \ei*{minted}[options:o, language:m]
\end{lscommand}
In contrast to \pai{listings}, the \carg{language} is a required parameter.
The command \csi{mintinline} works with both normal groups and custom
delimiters.
\begin{example}[examplewidth=0.56\linewidth]
\mintinline{latex}|\LaTeX{}|

\begin{minted}{python}
x = 0
for i in range(10):
  x += 2**i
print(f"Result: {x}")
\end{minted}

\inputminted{rust}{age.rs}
\end{example}

\pai{minted}'s capabilities are comparable to those of the \pai{listings}
package. Some of the options have synonyms that makes their usage identical to
the \pai{listings} package. For example, to enable line numbers you can use
either the \cargv{linenos} key (which is a boolean value) or the \cargv{numbers}
key (which accepts \cargv{left}, \cargv{right}, \cargv{none} or \cargv{both}).
\begin{example}[examplewidth=0.56\linewidth]
\inputminted[
  linenos,
  xleftmargin=1.2em, %!hide
]{css}{review.css}
\end{example}
You can change the line numbering using \cargv{firstnumber} and
\cargv{stepnumber}. Additionally you may highlight some lines using the
\cargv{highlightlines} key to draw attention to them. Note that
\cargv{highlightlines} uses the transformed line numbers.
\begin{chktexignore}
\begin{example}[examplewidth=0.56\linewidth]
\inputminted[
  linenos,
  xleftmargin=1.2em, %!hide
  firstnumber=5,
  stepnumber=3,
  highlightlines={6, 10-12}
]{coq}{em.v}
\end{example}
\end{chktexignore}

As in the \pai{listings} package, \cargv{breaklines} enables automatic
breaking of long lines, however it will only break the lines at whitespace
characters. If you want the breaks to occur anywhere, use the
\cargv{breakanywhere} key. The breaking behaviour may be
fine-tuned by using the keys \cargv{breakbytoken}, \cargv{breakbytokenanywhere},
\cargv{breakbefore} and \cargv{breakafter}.
\begin{example}[examplewidth=0.55\linewidth]
\inputminted[
  linenos,
  xleftmargin=1.2em, %!hide
  breaklines,
]{haskell}{quicksort.hs}
\end{example}
As you can see, by default it inserts \enquote{\(\hookrightarrow\)} to indicate
that an automatic line break occurred. You can customise the printed symbols
with the \cargv{breaksymbolleft} and \cargv{breaksymbolright} keys.

In order to use \LaTeX{} inside the comments, you can use either the
\cargv{texcl} or its synonym \cargv{texcomments}. You can also add custom
\LaTeX{} escape delimiters with \cargv{escapeinside}, but these must be single
tokens (unlike in \pai{listings}).
\begin{example}
\begin{minted}[
  linenos,
  xleftmargin=1.2em, %!hide
  texcomments,
  escapeinside=||,
]{html}
<!DOCTYPE html>
<html>
  <head> |\label{head}|
    <title>On |\LaTeX{}|</title>
  </head>
  <body>
    <div class="review">
      <!-- \LaTeX{} review -->
      <p>It's awesome!</p>
    </div>
  </body>
</html>
\end{minted}
The \mintinline{html}{<head>}
element in line~\ref{head}
contains page metadata.
\end{example}

The Pygments program comes with numerous predefined styles for typesetting your
code. In order to choose between them use the \cargv{style} key.
\begin{example}
\inputminted[
  style=bw,
]{bash}{echo.sh}
\inputminted[
  style=sas,
]{bash}{echo.sh}
\inputminted[
  style=xcode,
]{bash}{echo.sh}
\end{example}

To avoid having to set the options for every listing and code snippet,
you can use \csi{setminted} to define the default values.
\begin{example}
\setminted{
  linenos,
  xleftmargin=1.2em, %!hide
  breaklines,
  style=xcode,
}
\inputminted{c++}{goodbye.cpp}
\end{example}

\section{Tables}\label{sec:tables}

In \LaTeX{}, the environment to typeset tables (and more) is called
\ei{tabular}. While it can be used on its own, the resulting table layouts look
extremely old fashioned. We recommend adding the \pai*{booktabs} package, which
provides several commands that let you typeset beautiful modern looking tables.

Note that the \pai*{booktabs} package does not modify the tabular environment,
it just adds some extra commands for spacing and adding lines to your tables.
If you write  for a publication which has other requirements, like not using
any extra packages, simply do not use the extra commands from \pai*{booktabs}.

To achieve some more advanced table layouts, you will also learn about the
\pai*{longtable}, \pai*{array} and \pai*{multirow} packages in this section.
They are so useful, that you'll probably want to put them all into the preamble
of your document right away.

In addition to a description of its commands, the \pai{booktabs}
documentation~\cite{pack:booktabs} contains guidelines on typesetting
professional-looking tables. We have taken these guidelines to heart when
writing this booklet. You will find some of them in this section.

\subsection{Basic Tables}

The environment \ei{tabular} has the following form
\begin{lscommand}
  \ei*{tabular}[pos:o, colspec:m]
\end{lscommand} %
The \carg{colspec} argument, which stands for column specifiers, defines the
format of the columns in the table. Use an \cargv{l} for a column of
left-aligned text, \cargv{r} for right-aligned text, and \cargv{c} for centred
text. Within the environment, use \ai{\&} to begin the next cell within the
current row, and \csi{\bs} to begin the next row.
\begin{example}
\begin{tabular}{lcr}
  left & centre & right \\
  1    & 2      & 3     \\
\end{tabular}
\end{example}
Note that the text inside the cells will not be wrapped. If you want the column
to contain justified text with line breaks, use the \cargv{p\{\carg{width}\}}
column specifier, where \carg{width} is the width of the column.
\begin{example}
\begin{tabular}{lp{3cm}}
  left & Very long paragraph
         that gets broken into
         multiple lines. \\
  1    & Another one,
         but shorter. \\
\end{tabular}
\end{example}

The optional \carg{pos} argument specifies the vertical position of the table
with respect to the baseline of the text. There are three possible alignments:
\cargv{c} centres the table (the default), \cargv{t} matches the baseline of
the top row, and \cargv{b} matches the baseline of the bottom row.
\begin{example}[examplewidth=0.85\linewidth, vertical_mode]
text
\begin{tabular}{ll} 1 & 2 \\ 3 & 4 \\ \end{tabular}
text
\begin{tabular}[t]{ll} 1 & 2 \\ 3 & 4 \\ \end{tabular}
text
\begin{tabular}[b]{ll} 1 & 2 \\ 3 & 4 \\ \end{tabular}
text
\end{example}

What we have seen so far, allows aligning some items in rows and columns,
but real tables need visible headings. To insert them, use the commands
\csi{toprule}, \csi{midrule} and \csi{bottomrule} from the \pai{booktabs}
package.
All of these accept an optional argument that specifies their thickness, but
usually the default settings are just fine. See
\autoref{lst:booktabssimple} for an example of using these commands.
\begin{listing}
  \begin{example}[examplewidth=0.6\linewidth, vertical_mode]
\begin{tabular}{lcl}
  \toprule
  Alignment & Letter & Niceness  \\
  \midrule
  Left      & l      & Very nice \\
  Centre    & c      & Very nice \\
  Right     & r      & Very nice \\
  \bottomrule
\end{tabular}
\end{example}
  \caption{A simple table using the \ei{booktabs} commands.}\label{lst:booktabssimple}
\end{listing}

With these commands you are already able to produce simple, yet nicely looking
tables. You may be are wondering why we have not mentioned how to add vertical
lines between the columns. Well, we did not mention them because the first rule
of producing professional-looking tables is that you \emph{must not} use
vertical lines.

%$ Someone should tell my journal editors; double-lines are mandatory.
The second rule is to never use double lines such as these shown in
\autoref{lst:doublerule}.
\begin{listing}
  \begin{chktexignore}
  \begin{example}[examplewidth=0.6\linewidth, vertical_mode]
\begin{tabular}{lll}
  \toprule[0.1cm]
  \toprule
  Person      & Face & Table    \\
  \midrule
  \midrule
  Me          & :(   & Not nice \\
  You         & :[   & Awful    \\
  Your reader & :<   & Terrible \\
  \bottomrule
  \bottomrule[0.1cm]
\end{tabular}
\end{example}
\end{chktexignore}
  \caption{An anti-example of using double rules inside a table.}\label{lst:doublerule}
\end{listing}
The default lines are already of different weight in order to signify
their meaning. Using more lines than is necessary means cluttering the space
without adding information. If you stick to these two rules, your tables will
look quite good already.

To make the tables look even more sleek, you may want to remove the padding in
the first and last column. To control the space between the columns use the
\cargv{@\{\carg{sep}\}} column specifier, where \carg{sep} is either text or
space. Space of arbitrary length can be inserted using the \csi{hspace}[width:
  m] command. The contents of the \carg{sep} arguments will be put between cells
in the relevant column.
\begin{example}[examplewidth=0.3\linewidth]
\begin{tabular}{
  @{a} c @{\hspace{1cm}} c @{|} c @{ b}
}
  1 & 2 & 3 \\
  4 & 5 & 6\\
  7 & 8 & 9\\
\end{tabular}
\end{example}
If you leave the \carg{sep} empty it will suppress the padding between the
columns. See \autoref{lst:sleek} for an example.
\begin{listing}
  \begin{chktexignore}
  \begin{example}[examplewidth=0.6\linewidth, vertical_mode]
\begin{tabular}{@{}lll@{}}
  \toprule
  Person      & Face & Table       \\
  \midrule
  Me          & :)   & Nice        \\
  You         & :]   & Sleek       \\
  Your reader & :>   & Informative \\
  \bottomrule
\end{tabular}
\end{example}
\end{chktexignore}
  \caption{An example of using \cargv{@\{\}} column specifier to suppress outer separators.}\label{lst:sleek}
\end{listing}

Sometimes it may make sense to group columns under one heading. To achieve this,
we need two things: a way to group multiple heading columns, and to have a line
covering the same columns (sitting underneath the merged heading cell). Use the
\begin{lscommand}
  \csi{multicolumn}[ncols:m, colspec:m, text:m]
\end{lscommand}
command to merge the cells. The \carg{ncols} argument indicates how many
columns should the \carg{text} span, while the \carg{colspec} is the column
specification for the new content---the same as when starting a \ei{tabular}
environment. See \autoref{lst:multicolumn} for an example.
\begin{listing}
  \begin{chktexignore}
  \begin{example}[examplewidth=0.8\linewidth, vertical_mode]
\begin{tabular}{@{}lll@{}}
  \toprule
  Person      & Face & Table                      \\
  \midrule
  Me          & :)   & Nice                       \\
  You         & :]   & Sleek                      \\
  Your reader & \multicolumn{2}{c}{Not available} \\
  \bottomrule
\end{tabular}
\end{example}
\end{chktexignore}
  \caption{An example of using the \csi{multicolumn} command in a table.}\label{lst:multicolumn}
\end{listing}

To get a horizontal line to span multiple columns, use the command
\begin{lscommand}
  \csi{cmidrule}[dim:o, trim:p, a-b:m]
\end{lscommand}
While the command uses nonstandard syntax for \carg{trim}, this is just an
optional argument with different pair of delimiters.\footnote{This syntax
  allows specification of the second optional argument without the first,
  which would be impossible if square brackets were used for both.} The
\carg{dim} allows us to specify the thickness of the line. The \carg{trim}
argument accepts any combination of \cargv{r}, \cargv{r\{\carg{dim}\}},
\cargv{l} or \cargv{l\{\carg{dim}\}}. This allows you trim the rule from right
or left, either by the package default or the specified \carg{dim}. It is
usually recommended to trim the rules from the side where they touch other
columns. The command's only required argument is the span of columns
to draw the line over (\carg{a-b}). See \autoref{lst:cmidrule} for an example.
\begin{listing}
  \begin{chktexignore}
  \begin{example}[examplewidth=0.85\linewidth, vertical_mode]
\begin{tabular}{@{}lll@{}}
  \toprule
              & \multicolumn{2}{c}{Reaction}      \\
  \cmidrule(l){2-3}
  Person      & Face & Exclamation                \\
  \midrule
  Me          & :)   & Nice                       \\
  You         & :]   & Sleek                      \\
  Your reader & \multicolumn{2}{c}{Not available} \\
  \bottomrule
\end{tabular}
\end{example}
\end{chktexignore}
  \caption{An example of using the \csi{cmidrule} command inside a table.}\label{lst:cmidrule}
\end{listing}

If you need a cell to span multiple rows instead of columns, then you have to
use package \pai{multirow}. Its main command is a little more complicated
\begin{lscommand}
  \csi{multirow}[vpos:o, nrows:m, width:m, vmove:o, text:m]
\end{lscommand}
The arguments are
\begin{description}
  \item[\carg{vpos}] is the vertical position of the text within the cell.
    Can be either \cargv{c} for centre (the default), \cargv{t} for top or
    \cargv{b} for bottom.
  \item[\carg{nrows}] is the number of rows for the new cell to span.
  \item[\carg{width}] is the width of the cell. Apart from a regular length,
    you can also pass two special arguments here: \cargv{*} for the natural
    length of the text and \cargv{=} for the same width as the column (this
    only makes sense if the column width was specified, for example via
    \cargv{p\{3cm\}}).
  \item[\carg{vmove}] allows adjusting the position of the text if it sits too
    low or too high.
  \item[\carg{text}] is the text to be put in the cell.
\end{description}
Unlike with \csi{multicolumn} you still have to write all the cells in remaining
rows, but they should be empty. See \autoref{lst:multirow} for an example.
\begin{listing}
  \begin{chktexignore}
  \begin{example}[examplewidth=0.85\linewidth, vertical_mode]
\begin{tabular}{@{}lll@{}}
  \toprule
              & \multicolumn{2}{c}{Reaction}      \\
  \cmidrule(l){2-3}
  Person      & Face            & Exclamation     \\
  \midrule
  \multirow[t]{2}{*}{VIPs} & :) & Nice            \\
              & :]              & Sleek           \\
  Others      & \multicolumn{2}{c}{Not available} \\
  \bottomrule
\end{tabular}
\end{example}
\end{chktexignore}
  \caption{An example of using \csi{multirow} inside a table.}\label{lst:multirow}
\end{listing}

In general it is not a good idea to use the \csi{multirow} command for showing
values common to multiple rows. Usually repeating the values in question makes
the table more readable.

When typesetting numerical data in the table, you may want to align it by
decimal point. A way to do this (and more) is described in
\autoref{sec:sitables}. The section also contains some guidelines about
typesetting numerical data in tables.

\subsection{Long Tables}

Material typeset with the tabular environment always stays together on one
page. This poses a problem for especially long tables. If your table is not
very wide, you may get away with it by typesetting its rows side by side. You
should then put a bigger space between those to indicate that these are
separate; either by using \cargv{@\{...\}} or by putting an empty column between
those as shown in \autoref{lst:rowsnext}.
\begin{listing}
  \begin{example}[vertical_mode, examplewidth=0.6\linewidth]
\begin{tabular}{@{}cllcl@{}}
  \toprule
  Digit & Word  && Digit & Word  \\
  \midrule
  0     & Zero  && 5     & Five  \\
  1     & One   && 6     & Six   \\
  2     & Two   && 7     & Seven \\
  3     & Three && 8     & Eight \\
  4     & Four  && 9     & Nine  \\
  \bottomrule
\end{tabular}
\end{example}
  \caption{An example of producing table with rows next to each other.}\label{lst:rowsnext}
\end{listing}

This approach will obviously not work for \emph{really} long tables. This is
where the \pai{longtable} package shines. It defines a \ei{longtable}
environment that works in a similar way to the \ei{tabular} environment, but
allows page breaks inside.

\begin{lscommand}
  \ei*{longtable}[align:o, colspec:m]
\end{lscommand}
In contrast to \ei{tabular}, \ei{longtable} always starts a new paragraph
and is centred by default. Thus the optional argument \carg{align} specifies
whether the table should be centred, on the left, or on the right. Use
\cargv{c}, \cargv{l} and \cargv{r} to specify this. The \carg{colspec} argument
is the same as in the \ei{tabular} environment. See \autoref{lst:longtable}
for an example.

\begin{listing}
  \begin{lined}{0.8\linewidth}
    \begin{example}[standalone, to_page=2, paperwidth=4.5cm, paperheight=3.5cm, vertical_pages]
%!hidebegin
\geometry{margin=0.1cm}
\usepackage{longtable}
\usepackage{booktabs}
\begin{document}
%!hideend
\begin{longtable}{cl}
  \toprule
  Number & Word \\
  \midrule
  0      & Zero \\
  1      & One  \\
  2      & Two  \\
  % ...
%!hidebegin
  3      & Three \\
  4      & Four  \\
  5      & Five  \\
  6      & Six   \\
  7      & Seven \\
  8      & Eight \\
  9      & Nine  \\
%!hideend
  10     & Ten  \\
  \bottomrule
\end{longtable}
\end{document} %!hide
\end{example}
  \end{lined}
  \caption{An example of using the \ei{longtable} environment.}\label{lst:longtable}
\end{listing}

Note that the page breaks are always placed between rows. If you have a tall
row, thanks to a \cargv{p\{...\}} specification, it will not broken (as seen in
\autoref{lst:longtablepars}).
\begin{listing}
  \begin{lined}{12cm}
    \begin{example}[standalone, to_page=3, paperwidth=5.5cm, paperheight=2cm, vertical_pages]
%!hidebegin
\usepackage{longtable}
\usepackage{booktabs}
\begin{document}
%!hideend
\begin{longtable}{
  cp{2cm}
}
  \toprule
  Number & Words              \\
  \midrule
  0      & Zero               \\
  1      & Multi paragraph
           text.

           It makes the table
           cell very tall.    \\
  2      & Two                \\
  \bottomrule
\end{longtable}
\end{document} %!hide
\end{example}
  \end{lined}

  \caption{An anti-example of using \ei{longtable} assuming it will break
    the paragraphs inside the cells.}\label{lst:longtablepars}
\end{listing}

The \ei{longtable} environment also defines some additional commands that end
table rows. One of them is \csi{textbackslash*} which, similarly to its
use in a paragraph, prohibits a page break after the row. Other useful commands
are \csi{endhead}, \csi{endfirsthead}, \csi{endfoot} and \csi{endlastfoot}. If
you end a row with them, they will set the preceding rows as headers and
footers of the table. The \csi{endhead} and \csi{endfoot} commands put the
headers and footers on every page, while \csi{endfirsthead} and
\csi{endlastfoot} put them on the first and last page. An example of using these
commands is presented in \autoref{lst:longtablerunning}.
\begin{listing}
  \begin{example}[
    standalone,
    to_page=2,
    paperwidth=4.3cm,
    paperheight=5cm,
    examplewidth=8.7cm,
    vertical_mode,
  ]
%!hidebegin
\usepackage{longtable}
\usepackage{booktabs}
\begin{document}
%!hideend
\begin{longtable}{cl}
  \toprule
  \multicolumn{2}{c}{V.~Important Table} \\
  Number & Word                           \\
  \midrule \endfirsthead

  \toprule
  \multicolumn{2}{c}{VIT (continued)}    \\
  Number & Word                          \\
  \midrule \endhead

  \midrule
  \multicolumn{2}{c}{Not the end}        \\
  \bottomrule \endfoot

  \midrule
  \multicolumn{2}{c}{The end of VIT}     \\
  \bottomrule \endlastfoot

  0      & Zero   \\
  1      & One    \\
  % ...
%!hidebegin
  2      & Two     \\
  3      & Three \\
  4      & Four  \\
  5      & Five  \\
  6      & Six   \\
  7      & Seven \\
  8      & Eight \\
  9      & Nine  \\
  %!hideend
  10     & Ten    \\
\end{longtable}
\end{document} %!hide
\end{example}
  \caption{An example of \ei{longtable} with running headers and
    footers.}\label{lst:longtablerunning}
\end{listing}

Note that while you can use \csi{multicolumn} and \csi{multirow} normally within
the \ei{longtable} environment, the table may get very complicated. \LaTeX{}
will then need several passes to properly calculate the column widths.

\subsection{Advanced Tables and Non-Tables}

In regular text, you can specify the amount of space between two lines as an
optional argument to the \csi{\textbackslash} command. Can the same be done
within a table? It turns out that it can, but there is one caveat. If you use
it together with a \cargv{p\{...\}} column, the resulting space will be
different depending on column order.
\begin{example}[standalone, paperheight=3cm, paperwidth=0.57\linewidth]
\begin{document} %!hide
\noindent %!hide
\begin{tabular}{lp{1cm}}
  1 & 2\newline x \\[1cm]
  3 & 4           \\
\end{tabular}
\begin{tabular}{p{1cm}l}
  2\newline x & 1 \\[1cm]
  4           & 3 \\
\end{tabular}
\end{document} %!hide
\end{example}
This counterintuitive behaviour is caused by the fact that the space to add is
calculated based on the last column. In order to prevent that, simply add
\mintinline{latex}|\usepackage{array}| to your preamble. Since there are no
downsides to this, it is recommended to always use this package when starting a
new document, to avoid breaking tables that rely on the original behaviour
later. The \pai{array} package also defines some additional column specifiers.

The \cargv{p\{...\}} specifier allows the insertion of text with line breaks
into the table. The text always starts from the top though, so the \pai{array}
package defines two additional specifiers: \cargv{m\{...\}} for vertically
centred text, and \cargv{b\{...\}} for text placed vertically from the bottom.
\begin{example}[examplewidth=0.57\linewidth]
\begin{tabular}{
  lp{1.4cm}m{1.4cm}b{1.4cm}
}
  Cell                 &
  Top matches cell.    &
  Centre matches cell. &
  Bottom matches cell. \\
\end{tabular}
\end{example}

When a column must be formatted a certain way, it is inconvenient to put
the same commands in every cell. Moreover, if you decide that something needs to
be changed about the formatting, then you would have to edit every cell
individually. To avoid that, the \pai{array} package defines
\cargv{>\{\carg{cmds}\}} and \cargv{<\{\carg{cmds}\}} column specifiers which
can be used to put some code before and after a column.
\begin{example}
\begin{tabular}{
  l
  >{\begin{em}}l<{\end{em}}
}
  text & emphasised text \\
  text & emphasised text \\
  text & emphasised text \\
\end{tabular}
\end{example}

% TODO: Find a useful environment that still does this.
\begin{comment}
Some environments change the meaning of the \csi{\textbackslash} command thus
making it unavailable to go to the next row. In such occasions you may use the
\csi{tabularnewline} command.
\begin{example}
\begin{tabular}{
  >{\begin{FlushLeft}}
    p{2cm}
  <{\end{FlushLeft}}
  >{\begin{FlushRight}}
    p{2cm}
  <{\end{FlushRight}}
}
  This cell will
    be flushed left. &
  This cell will
    be flushed right.
  \tabularnewline

  This \\ is newline &
  Here \\ too
  \tabularnewline
\end{tabular}
\end{example}
\end{comment}

While using \cargv{>\{...\}} and \cargv{<\{...\}} comes in handy if we only have
one such column, it quickly becomes inconvenient when many columns of the same
type appear in any tables. In that case, it may be desirable to use the command
\begin{lscommand}
  \csi{newcolumntype}[newcolspec:m, definition:m]
\end{lscommand}
where \carg{newcolspec} is a single letter used for specification of columns,
and the \carg{definition} is what should be inserted in the table when using
it.
\begin{example}
\newcolumntype{e}{
  >{\begin{em}\begin{FlushLeft}}
  p{2.5cm}
  <{\end{FlushLeft}\end{em}}
}
\begin{tabular}{ee}
  This cell will
    be flushed left
    and emphasised. &
  This cell will
    be flushed left
    and emphasised.
  \tabularnewline

  This cell will
    be flushed left
    and emphasised. &
  This cell will
    be flushed left
    and emphasised.
  \tabularnewline
\end{tabular}
\end{example}

%$ Several paragraphs back, we were told we \emph{must not} use vertical lines.
In order to create a tic-tac-toe grid, you need to put some vertical and
horizontal lines of the same width. In order to do so, you may use the
\cargv{|}~(vertical bar) column specifier for vertical rules and \csi{hline} for
horizontal rules.
\begin{chktexignore}
  \begin{example}
\begin{tabular}{c|c|c}
  O & X & O \\
  \hline
  X & X & O \\
  \hline
  X & O & X \\
\end{tabular}
\end{example}
\end{chktexignore}
In case your grid gets more complicated, you may need the \csi{cline}[a-b: m] %chktex 8
command, which draws a horizontal line spanning only the specified rows.
\begin{chktexignore}
  \begin{example}[examplewidth=0.35\linewidth]
\begin{tabular}{|cccc|}
  \cline{4-4}
  a & a & \multicolumn{1}{c|}{a} & b \\
  \cline{2-3}
  a & \multicolumn{1}{|c}{b} & b & b \\
  \cline{2-2}
  a & a & \multicolumn{1}{|c}{b} & b \\
  \cline{1-2}
\end{tabular}
\end{example}
\end{chktexignore}

\section{Including Graphics and Images}\label{sec:images}

\LaTeX{} itself does not provide facilities to include images. In order to so,
it is necessary to use the \pai*{graphicx} package. It provides the
\begin{lscommand}
  \csi{includegraphics}[options: o, file: m]
\end{lscommand}
command that can be used to include the \carg{file} in the
document.\footnote{In the subsequent examples, we are using images from the
  \pai*{mwe} package that should be installed with your \LaTeX{} distribution.} %$ Nothing below a footnote
\begin{example}
Here is an image:
%!showbegin !hide
\includegraphics{example-image}
%!showend !hide
\includegraphics[width=3cm]{example-image}%!hide
\end{example}
The file may be a PDF, PNG or JPEG\@. If you omit the extension, then
\LaTeX{} will try the following in order: \eei{.pdf}, \eei{.png}, \eei{.jpg},
\eei{.mps}, \eei{.jpeg}, \eei{.jbig2}, \eei{.jb2}. Note that if your filename
contains dots (other than the extension dot), then its basename has to be put in
curly braces like so:
\begin{code}
  \mintinline{latex}|\includegraphics{{image.with.dots}.jpg}|
\end{code}

The \carg{options} parameter is a comma delimited list of keys and values. The
most basic keys are \cargv{width} and \cargv{height}. If only one is supplied,
then \LaTeX{} will calculate the other to keep the aspect ratio of the
original image.
\begin{example}[examplewidth=0.85\linewidth, vertical_mode]
\includegraphics[width=2cm]{example-image}
\includegraphics[height=1cm]{example-image}
\includegraphics[width=2cm, height=1cm]{example-image}
\end{example}

You may also pass the \cargv{keepaspectratio} key. This will change the meaning
of \cargv{width} and \cargv{height} keys to the \emph{maximum} possible width
and height.
\begin{example}[examplewidth=0.6\linewidth]
\includegraphics[
  width=2cm,
  height=2cm,
]{example-image}
\includegraphics[
  width=2cm,
  height=2cm,
  keepaspectratio,
]{example-image}
\end{example}

Instead of specifying width or height, you may also pass the \cargv{scale}
argument to scale it by a specific amount. Passing negative values will
reflect the image.
\begin{example}[examplewidth=0.6\linewidth]
\includegraphics[
  scale=0.1,
]{example-image}
\includegraphics[
  scale=0.3,
]{example-image}
\includegraphics[
  scale=-0.15,
]{example-image}
%!hidebegin
\vspace{1.3cm}
\end{example}

If you want to rotate the picture, you can pass the \cargv{angle} key. By
default, the image is rotated about its bottom-left corner. You can change the
anchor point by specifying the \cargv{origin} key. Its value may contain one or
two of the following specifications: \cargv{l} for left, \cargv{r} for right,
\cargv{c} for centre, \cargv{t} for top, \cargv{b} for bottom, and \cargv{B} for
baseline (useful with \TikZ{} pictures as described in
\autoref{chap:graphics}). So, for example, \cargv{lc} will rotate the image
about the centre of its left edge, while \cargv{c} will rotate it about its
centre.
\begin{example}
Text \includegraphics[
  scale=0.1, %!hide
  angle=45,
]{example-image}
Text \includegraphics[
  scale=0.1, %!hide
  angle=180,
]{example-image}  \\
Text \includegraphics[
  scale=0.1, %!hide
  angle=180,
  origin=c,
]{example-image}
Text \includegraphics[
  scale=0.1, %!hide
  angle=180,
  origin=lt,
]{example-image}
\end{example}

If you want to show only a part of the picture, you may specify the
\cargv{clip} key and set the \cargv{viewport} key to required part of the
image. The \cargv{viewport} takes as its value four numbers: the first two
specify the bottom-left point of a rectangle, while the last two specify the
top-right point of rectangle.
\begin{example}
\includegraphics[
  clip,
  viewport=3cm 3cm 8.5cm 5.5cm,
]{example-image}
\end{example}

If you are including a PDF file, \csi{includegraphics} will only include its
first page. If you want to specify a different page, pass it to the \cargv{page}
key.
\begin{example}
\includegraphics[
  width=1.5cm,
]{example-image-a4-numbered}
\includegraphics[
  width=1.5cm,
  page=2,
]{example-image-a4-numbered}
\includegraphics[
  width=1.5cm,
  page=11,
]{example-image-a4-numbered}
\end{example}

If you pass the \cargv{draft} option then, instead of the actual images, only
their names are printed, to save on file size and processing time. This key will
be inherited from the \csi{documentclass} options, if specified there.
\begin{example}
\includegraphics[
  scale=0.3,
  draft,
]{example-image}
\end{example}

\section{Floating Bodies}
If you try writing a longer text with tables and images, you will soon notice a
problem: they cannot be broken across multiple pages (except for
\pai{longtable}s). If you simply include them in normal text flow, this will
lead to a lot of empty space if the image is large and the text preceding it
would be to close to the bottom of the page. In \autoref{lst:floatrationale}
the first page is half empty.
\begin{listing}
  \begin{lined}{11cm}
    \begin{example}[standalone, paperheight=4cm, to_page=2, vertical_pages]
\usepackage{graphicx} %!hide
\usepackage{ragged2e} %!hide
\geometry{includefoot} %!hide
\begin{document}%!hide
Here is some text
to take up some space.

Below this text is an image.

\begin{Center}
  \includegraphics[
    scale=0.2
  ]{example-image}
\end{Center}

Here is more text.
\end{document}%!hide
\end{example}
  \end{lined}
  \caption{An anti-example of half-empty page due to a large, non-floated
    image.}\label{lst:floatrationale}
\end{listing}

The solution to this problem is to \enquote*{float} any figure or table that
does not fit on the current page to a later page, while filling the
current page with body text. \LaTeX{} offers two environments for
\wi{floating bodies}; one for tables and  one for figures.  To
take full advantage of these two environments, it is important to
understand approximately how \LaTeX{} handles floats internally.
Otherwise, floats may become a major source of frustration, because
\LaTeX{} never puts them where you want them to be.

Let's first have a look at the commands \LaTeX{} supplies
for floats.
Any material enclosed in a \ei{figure} or \ei{table} environment will
be treated as floating matter.
\begin{lscommand}
  \ei*{figure}[placement specifier:o] \\
  \ei*{table}[placement specifier:o]
\end{lscommand}
The optional \carg{placement specifier} parameter is used to tell \LaTeX{}
about the locations to which the float is allowed to be moved.  A
\carg{placement specifier} is constructed by building a string of
\emph{float-placing permissions}. See \autoref{tab:permiss} for a description
of available specifiers. Note that the order of the specifiers does not matter.

\begin{table}
  \begin{minipage}{\textwidth}
    \centering
    \caption{Float Placing Permissions.}\label{tab:permiss}
    \begin{tabular}{@{}cp{8cm}@{}}
      \toprule
      Spec       & Permission to place the float \ldots             \\
      \midrule
      \texttt{h} & \emph{here} at the very place in the text
      where it occurred.  This is useful mainly for small floats.   \\
      \texttt{t} & at the \emph{top} of a page                      \\
      \texttt{b} & at the \emph{bottom} of a page                   \\
      \texttt{p} & on a special \emph{page} containing only floats. \\
      \texttt{!} & without considering most of the
      internal parameters\footnote{Such as the
        maximum number of floats allowed  on one page.},
      which could otherwise stop this
      float from being placed.                                      \\
      \bottomrule
    \end{tabular}
  \end{minipage}
\end{table}

For example, a table float could be started with the following line:
\begin{code}
\mintinline{latex}|\begin{table}[!hbp]|
\end{code}
The \wi{placement specifier} \cargv{[!hbp]} allows \LaTeX{} to
place the table right here (\cargv{h}) or at the bottom (\cargv{b})
of some page
or on a special floats page (\cargv{p}), and all this even if it does not
look that good (\cargv{!}). If no placement specifier is given, the standard
classes assume \cargv{[tbp]}.

\LaTeX{} will place every float it encounters according to the placement
specifier supplied by the author. If a float cannot be placed on the current
page it is deferred either to the figures queue or the tables
queue.\footnote{These are FIFO---`first in first out'---queues!}  When a new
page is started, \LaTeX{} first checks if it is possible to fill a special
\enquote*{float page} with floats from the queues. If this is not possible, the
first float on each queue is treated as if it had just occurred in the text:
\LaTeX{} tries again to place it according to its respective placement
specifiers (except \cargv{h}, which is no longer possible).  Any new floats
occurring in the text get placed into the appropriate queues. \LaTeX{} strictly
maintains the original order of appearance for each type of float. That's why a
figure that cannot be placed pushes all subsequent figures to the end of the
document. Therefore:
\begin{quote}
  If \LaTeX{} is not placing the floats as you expected,
  it is often only one float jamming one of the two float queues.
\end{quote}

While it is possible to give \LaTeX{}  single-location placement specifiers,
this causes problems.  If the float does not fit in the location specified it
becomes stuck, blocking subsequent floats. In particular, you should never,
ever use the [h] option---it is so bad that in more recent versions of \LaTeX,
it is automatically replaced by [ht].

Using floats, \autoref{lst:floatrationale} could be made better by
rewriting it like \autoref{lst:floatexample}. As you can see there,
\LaTeX{} proceeded to typeset the remaining text and delayed inserting the image
until the next page.
\begin{listing}
  \begin{lined}{11cm}
    \begin{example}[standalone, paperheight=4cm, to_page=2, vertical_pages]
\usepackage{graphicx} %!hide
\geometry{includefoot} %!hide
\begin{document}%!hide
Here is some text
to take up some space.

Below this text is an image.

\begin{figure}
  \centering
  \includegraphics[
    scale=0.2
  ]{example-image}
\end{figure}

Here is more text.
\end{document}%!hide
\end{example}
  \end{lined}
  \caption{An example of using \ei{figure} float to achieve dynamic image
    placement inside a document.}\label{lst:floatexample}
\end{listing}

Having explained the difficult bit, there are some more things to mention about
the \ei{table} and \ei{figure} environments. Use the
\begin{lscommand}
  \csi{caption}[short caption: o, full caption: m]
\end{lscommand}
command to define a caption for the float. A running number, and
the string \enquote{Figure} or \enquote{Table}, will be added by \LaTeX.

The two commands
\begin{lscommand}
  \csi{listoffigures} and \csi{listoftables}
\end{lscommand}
operate similar to the \csi{tableofcontents} command, printing a list of
figures or tables, respectively.  These lists will display the whole caption,
so if you tend to use long captions you must have a shorter version of the
caption for the lists. This is accomplished by entering the short version in
brackets after the \csi{caption} command.
\begin{code}
\mintinline{latex}|\caption[Short]{LLLLLoooooonnnnnggggg}|
\end{code}

Use \csi{label} and \csi{ref} to create a reference to a float within
your text. Note that the \csi{label} command must come \emph{after} the
\csi{caption} command, since you want it to reference the number of the
caption. \autoref{lst:floatrefs} presents an example usage of these
commands.
\begin{listing}
  \begin{lined}{11cm}
    \begin{example}[standalone, paperheight=4.5cm, to_page=2, vertical_pages]
\usepackage{graphicx} %!hide
\geometry{includefoot} %!hide
\begin{document}%!hide
\listoffigures
\vspace{.3cm}%!hide

Here is some text describing
the Figure~\ref{figure}.

\begin{figure}
  \centering
  \includegraphics[
    scale=0.2
  ]{example-image}
  \caption[A figure]{An
    interesting figure}%
  \label{figure}
\end{figure}

Here is more text.
\end{document}%!hide
\end{example}
  \end{lined}
  \caption{An example of using \ei{figure} float together with references and
    list of figures.}\label{lst:floatrefs}
\end{listing}

Under certain circumstances it might be necessary to use the
\begin{lscommand}
  \csi{clearpage} or even the \csi{cleardoublepage}
\end{lscommand}
command. It orders \LaTeX{} to immediately place all floats remaining in the
queues and then start a new page. \csi{cleardoublepage} even goes to a new
right-hand page.

\subsection{Other Floats}

If you want to have other float lists besides \enquote{Table} and
\enquote{Figure}, use the \pai*{newfloat} package. It provides the
\begin{lscommand}
  \csi{DeclareFloatingEnvironment}[options: o, name: m]
\end{lscommand}
For example, the \mintinline{latex}|\DeclareFloatingEnvironment{cat}| will
define a new environment, \mintinline{latex}|cat|, which can be used like
the \ei{figure} and \ei{table} environments, and command
\mintinline{latex}|\listofcats|, which works in the same way as
\csi{listoftables}. You can customise the floats using the
\carg{options} to change the printed captions (\cargv{name}), list title
(\cargv{listname}) or default placement (\cargv{placement}).

\subsection{The \ei{longtable} Environment}

Floating bodies occupy only a single page. Thus, putting a \ei{longtable} into
one does not make sense. Still, you may want to have the \ei{longtable} listed
in the list of tables with some caption. The \pai{longtable} package defines
its own \csi{caption} command, which you can use inside the environment. Its use
is similar to the standard \csi{caption} command, except that it is treated as a
row. If you pass an empty optional argument to the \csi{caption} command, it
will typeset normally but it won't be put in the list of tables, which is useful
if you want to have a running caption. See \autoref{lst:longrunningcaption}, for
an example.

\begin{listing}
  \begin{lined}{0.8\linewidth}
    \begin{example}[standalone,
    to_page=2,
    paperwidth=4.1cm,
    paperheight=4cm,
    vertical_pages,
]
%!hidebegin
\usepackage{longtable}
\usepackage{booktabs}

\begin{document}
%!hideend
\begin{longtable}{@{}cl@{}}
  \caption{Numbers}       \\
  \toprule
  Number & Word           \\
  \midrule \endfirsthead

  \caption[]{(continued)} \\
  \toprule
  Number & Word           \\
  \midrule \endhead

  \bottomrule \endfoot
  \bottomrule \endlastfoot

  0      & Zero   \\
  % ...
%!hidebegin
  1      & One    \\
  2      & Two     \\
  3      & Three \\
  4      & Four  \\
  5      & Five  \\
  6      & Six   \\
  7      & Seven \\
  8      & Eight \\
  9      & Nine  \\
  10     & Ten    \\
  11     & Eleven \\
  12     & Twelve \\
  %!hideend
\end{longtable}
\end{document} %!hide
\end{example}
  \end{lined}
  \caption{An example of \ei{longtable} with running caption.}\label{lst:longrunningcaption}
\end{listing}

\section{Big Projects}
When working on big documents, you might want to split the input file
into several parts. \LaTeX{} has two commands that help you do this.

Use the
\begin{lscommand}
  \csi{include}[filename: m]
\end{lscommand}
command in the document body to insert the contents of another file named
\emph{filename.tex}. Note that \LaTeX{} will start a new page before processing
the material input from \emph{filename.tex}.

The second command can be used in the preamble. It allows you to
instruct \LaTeX{} to only input some of the \mintinline{latex}|\include|d % chktex 27
files.
\begin{lscommand}
  \csi{includeonly}[{{«\bs carg{filename}», «\bs carg{filename}», ...}}: c]
\end{lscommand}
After this command is executed in the preamble of the document, only
\csi{include} commands for the filenames that are listed in the
argument of the \csi{includeonly} command will be executed.

The \csi{include} command starts typesetting the included text on a new
page. This is helpful when you use \csi{includeonly}, because the
page breaks will not move, even when some \csi{include} files are omitted.
Sometimes this might not be desirable. In this case, use the
\begin{lscommand}
  \csi{input}[filename: m]
\end{lscommand}
command. It simply includes the file specified.
No flashy suits, no strings attached.

To make \LaTeX{} quickly check your document, use the \pai{syntonly}
package. This makes \LaTeX{} skim through your document, only checking for
proper syntax and usage of commands, but doesn't produce any (PDF) output.
As \LaTeX{} runs faster in this mode, you may save yourself valuable time.
Usage is very simple:

\begin{minted}{latex}
  \usepackage{syntonly}
  \syntaxonly
\end{minted}
When you want to produce pages, just comment out the second line
(by adding a percent sign).
