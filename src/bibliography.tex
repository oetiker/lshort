% The Not So Short Introduction to LaTeX
%
% Copyright (C) 1995--2022 Tobias Oetiker, Marcin Serwin, Hubert Partl,
% Irene Hyna, Elisabeth Schlegl and Contributors.
%
% This document is free software: you can redistribute it and/or modify it
% under the terms of the GNU General Public License as published by the Free
% Software Foundation, either version 3 of the License, or (at your option) any
% later version.
%
% This document is distributed in the hope that it will be useful, but WITHOUT
% ANY WARRANTY; without even the implied warranty of MERCHANTABILITY or FITNESS
% FOR A PARTICULAR PURPOSE.  See the GNU General Public License for more
% details.
%
% You should have received a copy of the GNU General Public License along with
% this document.  If not, see <https://www.gnu.org/licenses/>.

% !TEX root = ./lshort.tex
\chapter{Bibliographies}\label{chap:bibliography}
\begin{intro}
  When writing articles or books concerned with some topic you will often need
  to reference other books and articles to point out where some information
  might be found. You have already seen this done multiple times throughout
  this book. Doing this by hand would be tedious, so \LaTeX{} comes with an
  option to manage the bibliographic data of our document.
\end{intro}

This chapter will describe two approaches to bibliography management in \LaTeX{}:
\begin{description}
  \item[\ei{thebibliography} environment] which is suitable for rather small
    bibliographies.
  \item[\pai{biblatex} with \hologo{biber}] which is an advanced bibliography
    management system that is suitable for books and publications with extensive
    bibliographies. These systems also make it easy to format all bibliography
    entries in exactly the style required by the publisher.
\end{description}

\section{\ei{thebibliography} environment}

Produce a \wi{bibliography} with the \ei{thebibliography}
environment.  Each entry starts with
\begin{lscommand}
  \csi{bibitem}[label: o, marker: m]
\end{lscommand}
The \carg{marker} is then used to cite the book, article, or paper
within the document.
\begin{lscommand}
  \csi{cite}[marker: m]
\end{lscommand}
If you do not use the \carg{label} option, the entries will get enumerated
automatically.  The parameter after the \verb|\begin{thebibliography}|
command defines how much space to reserve for the number of labels. In the example below,
\cargv{\{99\}} tells \LaTeX{} to expect that none of the bibliography item numbers will be wider
than the number 99.
\begin{example}[standalone, paperwidth=6cm, paperheight=4cm]
\begin{document} %!hide
Partl~\cite{pa} has
proposed that \ldots
\begin{thebibliography}{99}
\bibitem{pa} H.~Partl:
\emph{German \TeX},
TUGboat Volume~9, Issue~1 (1988)
\end{thebibliography}
\end{document} %!hide
\end{example}

In order to get the citations typeset properly, two passes of \LaTeX{} compiler
are needed similar to how tables of contents are done.

This approach has the advantage that it is simple and entirely self-contained---no external packages and programs are needed since it is supported out
of the box by \LaTeX{}. There are however many problems:
\begin{itemize}
  \item Each entry has to be formatted manually, which may lead to inconsistent
        styling. Moreover, simple changes in format will require editing each entry by
        hand.
  \item There is no logical/structural markup to imply which part of
        bibliographic entry is author, title, journal, etc.
  \item No sorting is performed. If you want the entries to be processed in
        citation order, you have to ensure this manually.
  \item If you want to use different citation style than numeric, you have to
        manually set all the labels.
\end{itemize}
For all projects that contain more than few citations \pai{biblatex} is
strongly recommended.

% For larger projects, you might want to check out the Bib\TeX{}
% program. Bib\TeX{} is included with most \TeX{} distributions. It
% allows you to maintain a bibliographic database and then extract the
% references relevant to things you cited in your paper. The visual
% presentation of Bib\TeX{}-generated bibliographies is based on a style-sheets concept that allows you to create bibliographies following
% a wide range of established designs.

\section{\pai{biblatex} with \hologo{biber}}

The \pai*{biblatex} package uses a separate program, \pai*{biber}, for
bibliography management. \hologo{biber} is a successor of the popular
\hologo{BibTeX} program. Its main advantage is UTF-8 support, which was lacking
in the original \hologo{BibTeX}. The database format is largely backward
compatible, so on many sites you can find ``BibTeX entry'' or ``Export Bibtex
Citation'', which will be useful when compiling your own citation database.

\subsection{Database files}
The bibliographic database is stored in special \eei{.bib} files with their own
syntax. A single bibliographic entry is of the form
\begin{lscommand}
  \texttt{@}\carg{entry type}\texttt{\{}\carg{marker},\\
  \hspace*{1em}\carg{field1}\texttt{ = \{}\carg{value1}\texttt{\}},\\
  \hspace*{1em}\carg{field2}\texttt{ = \{}\carg{value2}\texttt{\}},\\
  \hspace*{1em}\carg{field3}\texttt{ = \{}\carg{value3}\texttt{\}},\\
  \hspace*{3em}\(\vdots\)\\
  \texttt{\}}
\end{lscommand}
The fields required depend on the bibliography style you choose,
the default style of \pai{biblatex} will accept
the following \carg{entry type}s:
\begin{description}
  \item[\cargv{article}] for articles from journals or other periodicals. Important fields are
    \cargv{author}, \cargv{title}, \cargv{journaltitle},
    \cargv{date}, \cargv{url}, \cargv{doi}.
  \item[\cargv{book}] for single-volume book. With the fields
    \cargv{author}, \cargv{title}, \cargv{date}, \cargv{publisher},
    \cargv{volume}.
  \item[\cargv{online}] for accessed online resources. With
    \cargv{author}, \cargv{title}, \cargv{date}, \cargv{url}, \cargv{urldate}.
  \item[\cargv{manual}] for technical documentation. With
    \cargv{author}, \cargv{title}, \cargv{date}, \cargv{url}, \cargv{version}.
  \item[\cargv{misc}] for entries that do not fit any of the predefined
    categories. With \cargv{author}, \cargv{title},
    \cargv{date}, \cargv{howpublished}, \cargv{note}.
\end{description}
Note that the above list is far from exhaustive both in terms of \carg{entry
  type}s and the field names mentioned. For a full list check the \pai{biblatex}
manual or the style you are using.

Each \eei{.bib} file will typically contain multiple entries. See
\autoref{lst:bibfile} for an example.

\begin{listing}[htp]
  \begin{lined}{\textwidth}
    \inputminted{bibtex}{src/examples/example.bib}
  \end{lined}
  \caption[An example of bibliography database]{An example bibliography
    database for \hologo{biber} (\eei{.bib} file)}\label{lst:bibfile}
\end{listing}

\subsection{Using \pai{biblatex}}

In order to use biblatex, the following three commands are necessary.

The \pai{biblatex} package must be loaded with the
\begin{lscommand}
  \csi{usepackage}[options:o, biblatex:vm]
\end{lscommand}
command. The \carg{options} are comma delimited key value pairs that allow to
customise the behaviour of biblatex package. Some of them will be explained
later.

The command
\begin{lscommand}
  \csi{addbibresource}[options:o, file:m]
\end{lscommand}
must be put in the preamble. The \carg{file} is the name of the \eei{.bib} file
containing bibliographic entries. You may also specify a remote location here,
but then you must also put \cargv{location=remote} in the \carg{options}.

Finally the
\begin{lscommand}
  \csi{printbibliography}[options:o]
\end{lscommand}
typesets the loaded bibliography. The \carg{options} may be used to alter the
title or filter the entries included.

Processing files requires three \LaTeX{} passes in addition to the biber
command. A typical command line run may look like this:
\begin{code}
\begin{minted}{console}
$ xelatex document.tex
$ biber document
$ xelatex document.tex
$ xelatex document.tex
\end{minted}
\end{code}
The first \LaTeX{} pass extracts the citation data from the document which are
then read by \hologo{biber}. In the second pass, the bibliography is placed in the
document and the third pass finally typesets the document with all the correct citations.

\autoref{lst:biblatexfull} is a complete example that shows how the bits
work together. In this and all the following examples, the \texttt{example.bib}
file is assumed to be the same as in the \autoref{lst:bibfile}.
\begin{listing}
  \begin{example}[standalone,
  template=empty,
  biber,
  paperwidth=10cm,
  paperheight=7.5cm,
  vertical_mode,
  examplewidth=0.9\linewidth,
]
\documentclass{article}

\usepackage[paperheight=\height,paperwidth=\width,margin=0.3cm]{geometry} %!hide
\usepackage{biblatex}
\addbibresource{example.bib}

\begin{document}
\setlength{\parindent}{0pt} %!hide
\ldots{} Recently I was learning to use Bib\LaTeX{}
from~\cite{lshort}. It seems very useful. \ldots

\ldots{} which was already shown by Mrs.~Curie
in~\cite{curie}. \ldots

\ldots{} this can be easily explained by
the fact that Einstein was a time
traveller~\cite{dream}. \ldots

\printbibliography
\end{document}
\end{example}
  \caption{An example of using \pai{biblatex} to manage references in
    an article}\label{lst:biblatexfull}
\end{listing}

\subsection{Controlling the bibliography}

The default sorting order is \emph{N}ame (author\slash{}editor), \emph{T}itle,
\emph{Y}ear or \cargv{nty} in short. This order can be changed by passing
\cargv{sorting=}\carg{order} as a package option. For example if you want to sort
entries in chronological order simply pass the \cargv{ynt} option. Other options
allow sorting by citation order (\cargv{none}) and by the number of citations
(\cargv{count}).
\begin{example}[standalone,
  biber,
  paperwidth=6.5cm,
  paperheight=5.5cm,
]
% In preamble
\usepackage[
  sorting=ynt
]{biblatex}

% ...
%!hidebegin
\sloppy
\addbibresource{example.bib}

\begin{document}
\nocite{*}
\printbibliography[heading=none]
\end{document}
\end{example}

While the default citation style is \cargv{numeric}, this can be easily changed by
setting the preferred style via the \carg{style} option. Choose from
\cargv{alphabetic},
\cargv{authoryear}, \cargv{authortitle} or \cargv{verbose}. Some styles come in
several variations.
\begin{example}[standalone,
  biber,
  paperwidth=6.5cm,
  paperheight=7.2cm,
]
% In preamble
\usepackage[
  style=alphabetic
]{biblatex}

% ...

%!hidebegin
\addbibresource{example.bib}
\sloppy

\begin{document}
%!hideend
\setlength{\parindent}{0pt} %!hide
The validity of~\cite{dream}
as a scientific source was
recently called into question
though it correctly claims
that polonium was first
described in~\cite{curie}.
%!hidebegin
\printbibliography
\end{document}
\end{example}

In contrast to the \ei{thebibliography} environment, the \pai{biblatex}'s
\csi{printbibliography} command only prints entries that were referenced in the
document. If you want to print entries not mentioned in the document, you may
use the \csi{nocite}[marker: m] command. It will insert invisible citations, thus
instructing \pai{biblatex} to put them in the bibliography. The special value
\cargv{*} can be passed as a \carg{marker} if you want to print all entries in
the database.
\begin{example}[standalone,
  biber,
  paperwidth=7cm,
  paperheight=7cm,
]
%!hidebegin
\usepackage{biblatex}
\addbibresource{example.bib}
\begin{document}
\setlength{\parindent}{0pt}
%!hideend
I've only
cited~\cite{lshort}.

\nocite{*}
\printbibliography
%!hidebegin
\end{document}
\end{example}

If your bibliography gets really large it may make sense to split it into
several parts. This can be done by passing filter options to the
\csi{printbibliography} command. Available filters include \cargv{type},
\cargv{category}, \cargv{keyword}. It may be useful to also change the titles of
different categories using \cargv{title} option.

\begin{example}[standalone,
  biber,
  paperwidth=6.5cm,
  paperheight=6.5cm,
]
%!hidebegin
\usepackage{biblatex}
\addbibresource{example.bib}
\begin{document}
\setlength{\parindent}{0pt}
\nocite{*}
%!hideend
\printbibliography[
  type=book,
  title=Books I've referenced
]

\printbibliography[
  keyword=unreliable,
  title=Don't trust those
]
%!hidebegin
\end{document}
\end{example}

By default, the bibliography does not appear in the table of contents. This is
because it is using starred version of \csi{chapter}\slash\csi{section}
(depending on the class) to generate its heading. In order to change the default we
may pass the option \cargv{heading} to \csi{printbibliography}. Available
options include
\begin{description}
  \item[\cargv{bibliography}] the default, starred version of heading
  \item[\cargv{bibintoc}] starred version of heading, will appear in table of contents
  \item[\cargv{subbibliography}] will drop the bibliography one level in
    hierarchy (\csi{section*} instead of \csi{chapter*} and so on)
  \item[\cargv{bibnumbered}] will use non starred version for heading, thus
    numbering it and appearing in table of contents
  \item[\cargv{subbibintoc}] will drop the bibliography one level and put it in
    table of contents
  \item[\cargv{subbibnumbered}] will drop the bibliography one level and use
    non starred version of heading
  \item[\cargv{none}] will not print the heading
\end{description}

\begin{example}[
  standalone,
  biber,
  paperwidth=7cm,
  paperheight=7.5cm,
]
%!hidebegin
\usepackage{biblatex}
\addbibresource{example.bib}
\begin{document}
\setlength{\parindent}{0pt}
\nocite{curie}
%!hideend
\tableofcontents
\section{Important section}

% ...

\section{Other}

\printbibliography[
  heading=subbibnumbered,
  title=Bibliography,
]

% ...

%!hidebegin
\end{document}
\end{example}

As you may have noticed, when entries have many authors, then not all of
them will be printed; instead ``et al.'' shows up at the end of the author list. This behaviour
may be controlled via \cargv{maxnames} (default $3$) and \cargv{minnames}
(default $1$) options. If the number of names is greater than \carg{maxnames}
then it will be shortened to \carg{minnames} and ``et al.'' will be added.

\begin{example}[standalone,
  biber,
  paperwidth=7cm,
  paperheight=3cm,
]
\usepackage[
  maxnames=4,
]{biblatex}
%!hidebegin
\addbibresource{example.bib}
\sloppy
\begin{document}
\nocite{lshort}
\printbibliography[heading=none]

\end{document}
\end{example}

\begin{example}[standalone,
  biber,
  paperwidth=7cm,
  paperheight=3cm,
]
\usepackage[
  minnames=2,
]{biblatex}
%!hidebegin
\addbibresource{example.bib}
\sloppy
\begin{document}
\nocite{lshort}
\printbibliography[heading=none]

\end{document}
\end{example}

\subsection{Citing commands}

Until now we have only used the basic \csi{cite} command. \pai{biblatex} extends
the command set to allow extra control with
citations.

Most of the citing commands (\csi{cite} included) allow inserting notes
around citations:
\begin{lscommand}
  \csi{cite}[pre:o, post:o, marker:m]
\end{lscommand}
Other standard citation commands include \csi{parencite} (citation in
parentheses), \csi{footcite} (in footnotes), \csi{textcite} (citation intended to
be subject in a sentence), \csi{smartcite} (context dependent).

\begin{example}[standalone,
  biber,
  paperwidth=5cm,
  paperheight=5cm,
]
%!hidebegin
\usepackage[style=alphabetic]{biblatex}
\usepackage{csquotes}
\addbibresource{example.bib}
\sloppy
\begin{document}
%!hideend
Indicate page in
citation~\cite[25]{lshort}. 
This citation is in
parentheses~\parencite{curie}.
Footnote cite~\footcite{dream}.
Smart \smartcite[See][78]{lshort}.
Again\footnote{Smart
\smartcite[12--56]{dream}.}.
\enquote{\Textcite{curie}
  was an important paper.
}
%!hidebegin
\end{document}
\end{example}

Another set of commands allows us to extract specific information from
the bibliographic entry, and use the information directly in the text. This is includes commands such as \csi{citeauthor},
\csi{citetitle}, \csi{citeyear}, \csi{citedate}, \csi{citeurl}.

\begin{example}[standalone,
  biber,
  paperwidth=5cm,
  paperheight=4cm,
]
%!hidebegin
\usepackage{biblatex}
\addbibresource{example.bib}
\sloppy
\begin{document}
\noindent
%!hideend
\Citetitle{lshort} is a book by
\citeauthor{lshort}. Latest version
was released in \citeyear{lshort},
or to be more precise on
\citedate{lshort}. It is available
at \citeurl{lshort}.
%!hidebegin
\end{document}
\end{example}

If you want to be less tied to a specific style, the \csi{autocite}
command follows the citation style specified in the package options.

\begin{example}[standalone,
  biber,
  paperwidth=5cm,
  paperheight=4cm,
]
\usepackage[
  style=verbose,
  autocite=footnote,
]{biblatex}

% ...

%!hidebegin
\addbibresource{example.bib}
\sloppy
\begin{document}
\noindent
%!hideend
This is auto citation
\autocite{curie}.
%!hidebegin
\end{document}
\end{example}

\begin{example}[standalone,
  biber,
  paperwidth=5cm,
  paperheight=2cm,
]
\usepackage[
  style=authoryear,
  autocite=inline,
]{biblatex}

% ...
%!hidebegin
\addbibresource{example.bib}
\sloppy
\begin{document}
\noindent
This is auto citation \autocite{curie}.
\end{document}
\end{example}

\subsection{More about entries}

\hologo{biber} uses ``and'' as a separator in author entries. To prevent
this behaviour, enclose ``and'' in curly brackets
\begin{minted}{bibtex}
@book{kru,
  author = {Kruger {and} sons}
}
\end{minted}
The same trick may be useful when \hologo{biber} changes capitalisation, even
though it shouldn't.

When writing about certain subject it often happens that the same author or
publishing company released several books. In order to reuse the information in
several entries in the \eei{.bib} file, a special entry \cargv{xdata} is available. It may be used
like this

\begin{example}[standalone,
  biber,
  paperwidth=5cm,
  paperheight=7cm,
  minted_language=bibtex,
]
%!hidebegin
\begin{filecontents}{example2.bib} 
%!hideend
@xdata{krugers,
  author    = {Kruger {and} sons},
  publisher = {Krugers Inc.},
  location  = {Paris}
}

@book{kru21,
  title = {Why is \LaTeX{} so hard?},
  year  = {2021},
  xdata = {krugers}
}

@book{kru22,
  title = {\LaTeX{} is awesome!},
  year  = {2022},
  xdata = {krugers}
}
\end{filecontents} %!hidebegin
\usepackage{biblatex}
\addbibresource{example2.bib}
\sloppy
\begin{document}
\nocite{*}
\noindent
\printbibliography
\end{document}
\end{example}
